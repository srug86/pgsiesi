% Clase
\documentclass[11pt,a4paper,spanish,twoside]{report}

% Órdenes auxiliares
\input{inc/includes.tex}

% Árboles de directorios
%\usepackage{dirtree}

% Encabezado y pie de página
\encabezado
\setcounter{secnumdepth}{3} 
\begin{document}

% Silabación extra
\hyphenation{
a-sig-na-tu-ras
au-to-ma-ti-za-rá
ca-tá-lo-go
ca-rre-ra
cons-truc-ción
co-rres-pon-de
diag-nos-tico
fi-na-li-za-ción
ge-ne-ra-ción
in-fe-rior
man-te-ni-mien-to
me-dian-te
per-so-nal
pro-ce-di-mien-tos
pro-por-cio-na-rá
pu-bli-ca-da
re-qui-si-tos
res-pecto
u-su-a-rios
vi-lla-rre-al
}


% Portada
\portada{Planificación y Gestión de\\Sistemas de Información}
{Práctica 1}{Elaboración de un plan de proyecto\\utilizando Microsoft Project}
{Sergio de la Rubia García-Carpintero\\Miguel Millán Sánchez-Grande\\
  Luis Muñoz Villarreal\\Alicia Serrano Sánchez\\
  Juan Miguel Torres Triviño}{26 de Mayo de 2010}

% Licencia
\licencia{Sergio de la Rubia García-Carpintero, Miguel Millán Sánchez-Grande,
  Luis Muñoz Villarreal, Alicia Serrano Sánchez, Juan Miguel Torres Triviño}

\chapter*{Ficha de trabajo}
\begin{description}
\item[Código] P2
\item[Fecha] 26 de Abril de 2010
\item[Título] Estimación del esfuerzo de desarrollo de un Software utilizando
USC COCOMO II
\end{description}

\begin{table}[!ht]
  \centering
  \begin{tabular}{lp{5cm}c}
    \multicolumn{3}{l}{\Large \textbf{Equipo} G4} \\ \\
    \multicolumn{1}{c}{\emph{Apellidos y nombre}} & 
    \multicolumn{1}{c}{\emph{Firma}} & \emph{Puntos} \\
    \hline \\
    de la Rubia García-Carpintero, Sergio & & 6 \\ \\
    Millán Sánchez-Grande, Miguel         & & 6 \\ \\
    Muñoz Villarreal, Luis                & & 6 \\ \\
    Serrano Sánchez, Alicia               & & 6 \\ \\
    Torres Triviño, Juan Miguel           & & 6 \\ \\
    \hline
  \end{tabular}
\end{table}

% Índices
\tableofcontents
\listoffigures
%\listoftables

%% INICIO DEL DOCUMENTO %%%%%%%%%%%%%%%%%%%%%%%%%%%%%%%%%%%%%%%%%%%%%%%%%
\chapter{Ajustes de tamaño aplicados}

Para la estimación del tamaño de los tres módulos de los que se compone el
proyecto se va a utilizar el método de \emph{puntos función}. Como lenguaje
de implementación del proyecto se va a utilizar \emph{php} (lenguaje
orientado a objetos). Además, se ha considerado que el porcentaje de código que
se va a desechar cuando finalice la implementación va a ser \emph{0\%}.

Para rellenar las tablas del formulario de entrada de estimaciones de tamaño
en \emph{puntos función} se han utilizado los valores calculados en las tablas
de \emph{PFSA} del trabajo 5.

\section{Gestión de usuarios}
La \emph{figura \ref{CodGesUs}} muestra las estimaciones en puntos función sin
ajustar del módulo de gestión de usuario, extraídas de las estimaciones
realizadas en el trabajo 5, y el equivalente total en tamaño de líneas de
código calculado por COCOMO II.

\imagen{CodGesUs.png}{8}{Tamaño en líneas de código del módulo de gestión de
usuarios}{CodGesUs}

Como puede verse, el tamaño total equivalente en líneas de código del módulo
de gestión de usuarios es de \textbf{1440}.

\section{Gestión de cursos}
La \emph{figura \ref{CodGesCu}} muestra las estimaciones en puntos función sin
ajustar del módulo de gestión de usuario, extraídas de las estimaciones
realizadas en el trabajo 5, y el equivalente total en tamaño de líneas de
código calculado por COCOMO II.

\imagen{CodGesCu.png}{8}{Tamaño en líneas de código del módulo de gestión de
cursos}{CodGesCu}

Como puede verse, el tamaño total equivalente en líneas de código del módulo
de gestión de cursos es de \textbf{2656}.

\section{Interfaz}
La \emph{figura \ref{CodInt}} muestra las estimaciones en puntos función sin
ajustar del módulo de la interfaz, extraídas de las estimaciones realizadas en
el trabajo 5, y el equivalente total en tamaño de líneas de código calculado
por COCOMO II.

\imagen{CodInt.png}{8}{Tamaño en líneas de código del módulo de gestión de
cursos}{CodInt}

Como puede verse, el tamaño total equivalente en líneas de código del módulo
de la interfaz es de \textbf{2208}.

\chapter{Multiplicadores y estimaciones de esfuerzo}
\section{Gestión de usuarios}
\subsection{Post Architecture Model}
Para el Post Architecture Model, a partir de ahora PAM, en el valor de Effort
Adjustment Factor (EAF), se va a describir los valores referentes a producto,
plataforma, personal, proyecto y usuario. 
\subsubsection{Factores de producto}
Se refieren a las restricciones y requerimientos sobre el producto a
desarrollar.
\begin{description}
\item[RELY] Mide la confiabilidad del producto software. Se le asigna un 
valor \textbf{bajo} que implica unas mínimas pérdidas al usuario,
fácilmente recuperables.

\item[DATA] Mide el tamaño de la base de datos, es decir, la relación entre
el tamaño de las bases de datos y el tamaño del código en líneas de código. Se
le asigna un valor \textbf{normal}, lo que significa que la relación se
encuentra entre cien y mil. Tendrá un aumento del \textbf{50\%}. 

\item[CPLX] Mide la complejidad del producto. Se le asigna un
valor \textbf{normal}, esto conlleva que use anidamientos sencillos, rutinas
estándares de matemática y estadística, el uso simple de algunos dispositivos
y consultas y actualizaciones complejas. Tendrá un aumento del 
\textbf{25\%}.

\item[RUSE] Requerimientos de reusabilidad. Tiene un valor \textbf{normal},
lo que significa que tendrá partes reusables dentro del mismo
proyecto. Tendrá un aumento del \textbf{25\%}.

\item[DOCU] Corresponde con la documentación de las diferentes etapas del
ciclo de vida software. A este factor se le asigna un valor \textbf{alto},
esto significa que las necesidades del ciclo de vida están cubiertas
ampliamente.

\end{description}

\subsubsection{Factores de plataforma}
Analizan la complejidad de la plataforma subyacente.
\begin{description}
\item[PVOL] Mide la volatilidad de la plataforma y representa la frecuencia de
  los cambios en la plataforma. Se le asigna un valor \textbf{bajo}, que
  corresponde con un cambio principal cada doce meses y un cambio menor todos
  los meses.

\item[STOR] Mide la restricción del almacenamiento principal. Es una función
  que representa el grado de restricción del almacenamiento principal
  impuesto sobre un sistema software. A este factor se le asigna un valor
  \textbf{normal}, es decir, un uso del 50\% del porcetanje total de
  almacenamiento o menor. Tendrá un aumento del \textbf{50\%}.

\item[TIME] Restricción del tiempo de ejecución. Se le asigna un valor
  \textbf{alto}, que equivale a un 70\% del tiempo de ejecución disponible.
\end{description}

\subsubsection{Factores de personal}
Estos factores están referidos al nivel de habilidad que posee el equipo de
desarrollo.
\begin{description}
\item[ACAP] Mide la capacidad del analista. Se le asigna un valor
\textbf{alto}, que equivale al 75\% de su capacidad.
\item[PCAP] Mide la capacidad del programador. Se le asigna un valor
\textbf{alto}, que equivale al 75\% de su capacidad.
\item[PCON] Mide el grado de permanencia anual del personal afectado en un
proyecto. Se le asigna un valor \textbf{bajo}, que equivale a un 24\% de
permanencia por año.
\item[AEXP] Mide el nivel de experiencia en el desarrollo de aplicaciones
similares. Se le asigna un valor \textbf{normal}, que equivale a una
experiencia de 1 año. Tendrá un aumento del \textbf{25\%}.
\item[PEXP] Mide el reconocimiento de nuevas y potentes plataformas. Se le
asigna un valor \textbf{normal}, que equivale a una experiencia de 1
año. Tendrá un aumento del \textbf{25\%}. 
\item[LTEX] Mide el nivel de experiencia del equipo en el uso del lenguaje y
herramientas a emplear. Se le asigna un valor \textbf{alto}, que
equivale a un de 3 años de experiencia.
\end{description}

\subsubsection{Factores del proyecto}
Se refieren a las condiciones y restricciones bajo las cuales se lleva a cabo
el proyecto.
\begin{description}
\item[TOOL] Mide el uso de herramientas software. Se le asigna un valor
\textbf{alto}, que significa que se cuenta con herramientas robustas y
maduras, integradas moderadamente.
\item[SITE] Mide dos factores: la ubicación espacial y las comunicaciones.
Se le asigna un valor \textbf{muy alto}, que significa que el desarrollo
se producirá en un mismo edificio o complejo y las comunicaciones serán
electrónicas de banda ancha y ocasionalmente por videoconferencia.
\end{description}

\subsection{Early Design Model}
Para el Early Design Model, a partir de ahora EDM, se van a diferenciar los
valores de esfuerzo de seis multiplicadores.

\begin{description}
\item[PERS] Mide la capacidad del personal. Combina los valores ACAP, PCAP y
PCON, que se ven en el modelo \emph{post architecture}. Se le asigna un
valor \textbf{alto}. Esto implica que la suma promedio de los factores
ACAP, PCAP y PCON es aproximadamente de \emph{10,11}, el percentil combinado
de ACAP y PCAP es del 65\%; y el grado de permanencia anual del personal es
del \emph{9\%}.

\item[RCPX] Mide la fiabilidad y complejidad del producto. Combina los
factores RELY, DATA, CPLX y DOCU del modelo \emph{post architecture}. Se le
asigna un valor \textbf{normal}, con un aumento del \textbf{50\%}. Esto
implica que el valor promedio de la suma de los factores RELY, DATA, CPLX y
DOCU está en torno a \emph{12}, el énfasis en la confiabilidad de la
documentación es \emph{básico}, la complejidad del producto es
\emph{moderada} y el tamaño de la base de datos también es también
\emph{moderada}.

\item[RUSE] Mide la reutilización requerida. Equivale al factor de esfuerzo
del modelo \emph{post architecture} (RUSE). Se le asigna un valor
\textbf{normal} con un aumento del \textbf{25\%}.

\item[PDIF] Mide la dificultad de la plataforma. Combina los factores TIME, 
STOR y PVOL del modelo \emph{post architecture} . Se le asigna un valor
\textbf{normal} con un aumento del \textbf{25\%}. Esto implica que el valor 
promedio de la suma de los factores TIME, STOR y PVOL está en torno a \emph{9},
la volatilidad de la plataforma es \emph{estable} y la restricción de tiempo y
almacenamiento es \emph{menor o igual que el 50\%}.

\item[PREX] Mide la experiencia del personal. Combina los factores AEXP, PEXP y
LTEX del modelo \emph{post architecture} . Se le asigna un valor 
\textbf{normal} con un aumento del \textbf{25\%}. Esto implica que el valor 
promedio de la suma de los factores AEXP, PEXP y LTEX está en torno a \emph{9}
y la experiencia en el lenguaje y herramientas es de \emph{1 año}.

\item[FCIL] Mide los medios disponibles. Combina los factores TOOL y SITE del 
modelo \emph{post architecture} . Se le asigna un valor \textbf{alto}. Esto 
implica que el valor promedio de la suma de los factores TOOL y SITE está en 
entre \emph{7 y 8}, el soporte de TOOL es \emph{bueno} y la condición de 
multisitio es \emph{fuerte soporte para el desarrollo}.

\end{description}

\section{Gestión de cursos}
\subsection{Post Architecture Model}
\subsubsection{Factores de producto}
Se refieren a las restricciones y requerimientos sobre el producto a
desarrollar.
\begin{description}
\item[RELY] Mide la confiabilidad del producto software. Se le asigna un 
valor \textbf{bajo}, que implica unas mínimas pérdidas al usuario,
fácilmente recuperables.

\item[DATA] Mide el tamaño de la base de datos, es decir, la relación entre
el tamaño de las bases de datos y el tamaño del código en líneas de código. Se
le asigna un valor \textbf{normal}, lo que significa que la relación se
encuentra entre cien y mil. Tendrá un aumento del \textbf{75\%}. 

\item[CPLX] Mide la complejidad del producto. Se le asigna un
valor \textbf{nominal}, esto conlleva que use anidamientos sencillos, rutinas
estándares de matemática y estadística, el uso simple de algunos dispositivos
y consultas y actualizaciones complejas. Tendrá un aumento del 
\textbf{25\%}.

\item[RUSE] Requerimientos de reusabilidad. Tiene un valor \textbf{normal},
lo que significa que tendrá partes reusables dentro del mismo
proyecto. Tendrá un aumento del \textbf{25\%}.

\item[DOCU] Corresponde con la documentación de las diferentes etapas del
ciclo de vida software. A este factor se le asigna un valor de \textbf{alto},
esto significa que las necesidades del ciclo de vida están cubiertas
ampliamente.

\end{description}

\subsubsection{Factores de plataforma}
Analizan la complejidad de la plataforma subyacente.
\begin{description}
\item[PVOL] Mide la volatilidad de la plataforma y representa la frecuencia de
  los cambios en la plataforma. Se le asigna un valor \textbf{bajo}, que
  corresponde con un cambio principal cada doce meses y un cambio menor todos
  los meses.

\item[STOR] Mide la restricción del almacenamiento principal. Es una función
  que representa el grado de restricción del almacenamiento principal
  impuesto sobre un sistema de software. A este factor se le asigna un valor
  \textbf{normal}, es decir, un uso del 50\% del porcetanje total de
  almacenamiento o menor. Tendrá un aumento del \textbf{50\%}.

\item[TIME] Restricción del tiempo de ejecución. Se le asigna un valor
  \textbf{alto}, que equivale a un 70\% del tiempo de ejecución disponible.
\end{description}

\subsubsection{Factores de personal}
Estos factores están referidos al nivel de habilidad que posee el equipo de
desarrollo.
\begin{description}
\item[ACAP] Mide la capacidad del analista. Se le asigna un valor
\textbf{alto}, que equivale al 75\% de su capacidad.
\item[PCAP] Mide la capacidad del programador. Se le asigna un valor
\textbf{alto}, que equivale al 75\% de su capacidad.
\item[PCON] Mide el grado de permanencia anual del personal afectado en un
proyecto. Se le asigna un valor \textbf{bajo}, que equivale a un 24\% de
permanencia por año.
\item[AEXP] Mide el nivel de experiencia en el desarrollo de aplicaciones
similares. Se le asigna un valor \textbf{normal}, que equivale a una
experiencia de 1 año. Tendrá un aumento del \textbf{25\%}.
\item[PEXP] Mide el reconocimiento de nuevas y potentes plataformas. Se le
asigna un valor \textbf{normal}, que equivale a una experiencia de 1
año. Tendrá un aumento del \textbf{25\%}. 
\item[LTEX] Mide el nivel de experiencia del equipo en el uso del lenguaje y
herramientas a emplear. Se le asigna un valor \textbf{alto} que equivale a una
experiencia de 3 años.
\end{description}

\subsubsection{Factores del proyecto}
Se refieren a las condiciones y restricciones bajo las cuales se lleva a cabo
el proyecto.
\begin{description}
\item[TOOL] Mide el uso de herramientas software. Se le asigna un valor
\textbf{alto}, que significa que se cuenta con herramientas robustas y
maduras, integradas moderadamente.
\item[SITE] Mide dos factores: la ubicación espacial y las comunicaciones.
Se le asigna un valor de \textbf{muy alto}, que significa que el desarrollo
se producirá en un mismo edificio o complejo y las comunicaciones serán
electrónicas de banda ancha y ocasionalmente por videoconferencia.
\end{description}

\subsection{Early Design Model}
\begin{description}
\item[PERS] Mide la capacidad del personal. Combina los valores ACAP, PCAP y
PCON, que se ven en el modelo \emph{post architecture}. Se le asigna un
valor \textbf{muy alto}. Esto implica que la suma promedio de los factores
ACAP, PCAP y PCON es aproximadamente de \emph{11,12}, el percentil combinado
de ACAP y PCAP es del 75\% y el grado de permanencia anual del personal es
del \emph{5\%}.

\item[RCPX] Mide la fiabilidad y complejidad del producto. Combina los
factores RELY, DATA, CPLX y DOCU del modelo \emph{post architecture}. Se le
asigna un valor \textbf{normal}, con un aumento del \textbf{50\%}. Esto
implica que el valor promedio de la suma de los factores RELY, DATA, CPLX y
DOCU está en torno a \emph{12}, el énfasis en la confiabilidad de la
documentación es \emph{básico}, la complejidad del producto es
\emph{moderada} y el tamaño de la base de datos también es también
\emph{moderada}.

\item[RUSE] Mide la reutilización requerida. Equivale al factor de esfuerzo
del modelo \emph{post architecture} (RUSE). Se le asigna un valor
\textbf{normal} con un aumento del \textbf{25\%}.

\item[PDIF] Mide la dificultad de la plataforma. Combina los factores TIME, 
STOR y PVOL del modelo \emph{post architecture} . Se le asigna un valor
\textbf{normal}, con un aumento del \textbf{50\%}. Esto implica que el valor 
promedio de la suma de los factores TIME, STOR y PVOL está en torno a \emph{9},
la volatilidad de la plataforma es \emph{estable} y la restricción de tiempo y
almacenamiento es \emph{menor o igual que el 50\%}.

\item[PREX] Mide la experiencia del personal. Combina los factores AEXP, PEXP y
LTEX del modelo \emph{post architecture} . Se le asigna un valor 
\textbf{normal}, con un aumento del \textbf{25\%}. Esto implica que el valor 
promedio de la suma de los factores AEXP, PEXP y LTEX está en torno a \emph{9}
y la experiencia en el lenguaje y herramientas es de \emph{1 año}.

\item[FCIL] Mide los medios disponibles. Combina los factores TOOL y SITE del 
modelo \emph{post architecture} . Se le asigna un valor \textbf{alto}. Esto 
implica que el valor promedio de la suma de los factores TOOL y SITE está en 
entre \emph{7 y 8}, el soporte de TOOL es \emph{bueno} y la condición de 
multisitio es \emph{fuerte soporte para el desarrollo}.

\end{description}
\section{Interfaz}
\subsection{Post Architecture Model}
\subsubsection{Factores de producto}
Se refieren a las restricciones y requerimientos sobre el producto a
desarrollar.
\begin{description}
\item[RELY] Mide la confiabilidad del producto software. Se le asigna un 
valor \textbf{bajo}, que implica unas mínimas pérdidas al usuario,
fácilmente recuperables.

\item[DATA] Mide el tamaño de la base de datos, es decir, la relación entre
el tamaño de las bases de datos y el tamaño del código en líneas de código. Se
le asigna un valor \textbf{normal}, lo que significa que la relación se
encuentra entre cien y mil. Tendrá un aumento del \textbf{25\%}. 

\item[CPLX] Mide la complejidad del producto. Se le asigna un
valor \textbf{nominal}, esto conlleva que use anidamientos sencillos, rutinas
estándares de matemática y estadística, el uso simple de algunos dispositivos
y consultas y actualizaciones complejas. Tendrá un aumento del 
\textbf{25\%}.

\item[RUSE] Requerimientos de reusabilidad. Tiene un valor \textbf{normal},
lo que significa que tendrá partes reusables dentro del mismo
proyecto. Tendrá un aumento del \textbf{25\%}.

\item[DOCU] Corresponde con la documentación de las diferentes etapas del
ciclo de vida software. A este factor se le asigna un valor de \textbf{alto},
esto significa que las necesidades del ciclo de vida están cubiertas
ampliamente.

\end{description}

\subsubsection{Factores de plataforma}
Analizan la complejidad de la plataforma subyacente.
\begin{description}
\item[PVOL] Mide la volatilidad de la plataforma y representa la frecuencia de
  los cambios en la plataforma. Se le asigna un valor \textbf{bajo}, que
  corresponde con un cambio principal cada doce meses y un cambio menor todos
  los meses.

\item[STOR] Mide la restricción del almacenamiento principal. Es una función
  que representa el grado de restricción del almacenamiento principal
  impuesto sobre un sistema de software. A este factor se le asigna un valor
  \textbf{normal}, es decir, un uso del 50\% del porcetanje total de
  almacenamiento o menor. Tendrá un aumento del \textbf{25\%}.

\item[TIME] Restricción del tiempo de ejecución. Se le asigna un valor
  \textbf{alto}, que equivale a un 70\% del tiempo de ejecución disponible.
\end{description}

\subsubsection{Factores de personal}
Estos factores están referidos al nivel de habilidad que posee el equipo de
desarrollo.
\begin{description}
\item[ACAP] Mide la capacidad del analista. Se le asigna un valor
\textbf{bajo}, que equivale al 35\% de su capacidad.
\item[PCAP] Mide la capacidad del programador. Se le asigna un valor
\textbf{alto}, que equivale al 75\% de su capacidad.
\item[PCON] Mide el grado de permanencia anual del personal afectado en un
proyecto. Se le asigna un valor \textbf{bajo}, que equivale a un 24\% de
permanencia por año.
\item[AEXP] Mide el nivel de experiencia en el desarrollo de aplicaciones
similares. Se le asigna un valor \textbf{normal}, que equivale a una
experiencia de 1 año. Tendrá un aumento del \textbf{50\%}.
\item[PEXP] Mide el reconocimiento de nuevas y potentes plataformas. Se le
asigna un valor \textbf{normal}, que equivale a una experiencia de 1
año. Tendrá un aumento del \textbf{25\%}. 
\item[LTEX] Mide el nivel de experiencia del equipo en el uso del lenguaje y
herramientas a emplear. Se le asigna un valor \textbf{alto}, que equivale a una
experiencia de 3 años.
\end{description}

\subsubsection{Factores del proyecto}
Se refieren a las condiciones y restricciones bajo las cuales se lleva a cabo
el proyecto.
\begin{description}
\item[TOOL] Mide el uso de herramientas software. Se le asigna un valor
\textbf{alto}, que significa que se cuenta con herramientas robustas y
maduras, integradas moderadamente.
\item[SITE] Mide dos factores: la ubicación espacial y las comunicaciones.
Se le asigna un valor de \textbf{muy alto}, que significa que el desarrollo
se producirá en un mismo edificio o complejo y las comunicaciones serán
electrónicas de banda ancha y ocasionalmente por videoconferencia.
\end{description}

\subsection{Early Design Model}

\begin{description}
\item[PERS] Mide la capacidad del personal. Combina los valores ACAP, PCAP y
PCON, que se ven en el modelo \emph{post architecture}. Se le asigna un
valor \textbf{bajo}. Esto implica que la suma promedio de los factores
ACAP, PCAP y PCON es aproximadamente de \emph{7,8}, el percentil combinado
de ACAP y PCAP es del 45\%; y el grado de permanencia anual del personal es
del \emph{20\%}.

\item[RCPX] Mide la fiabilidad y complejidad del producto. Combina los
factores RELY, DATA, CPLX y DOCU del modelo \emph{post architecture}. Se le
asigna un valor \textbf{normal} con un aumento del \textbf{25\%}. Esto
implica que el valor promedio de la suma de los factores RELY, DATA, CPLX y
DOCU está en torno a \emph{12}, el énfasis en la confiabilidad de la
documentación es \emph{básico}, la complejidad del producto es
\emph{moderada} y el tamaño de la base de datos también es también
\emph{moderada}.

\item[RUSE] Mide la reutilización requerida. Equivale al factor de esfuerzo
del modelo \emph{post architecture} (RUSE). Se le asigna un valor
\textbf{normal} con un aumento del \textbf{25\%}.

\item[PDIF] Mide la dificultad de la plataforma. Combina los factores TIME, 
STOR y PVOL del modelo \emph{post architecture} . Se le asigna un valor
\textbf{normal} con un aumento del \textbf{25\%}. Esto implica que el valor 
promedio de la suma de los factores TIME, STOR y PVOL está en torno a \emph{9},
la volatilidad de la plataforma es \emph{estable} y la restricción de tiempo y
almacenamiento es \emph{menor o igual que el 50\%}.

\item[PREX] Mide la experiencia del personal. Combina los factores AEXP, PEXP y
LTEX del modelo \emph{post architecture} . Se le asigna un valor 
\textbf{normal} con un aumento del \textbf{25\%}. Esto implica que el valor 
promedio de la suma de los factores AEXP, PEXP y LTEX está en torno a \emph{9}
y la experiencia en el lenguaje y herramientas es de \emph{1 año}.

\item[FCIL] Mide los medios disponibles. Combina los factores TOOL y SITE del 
modelo \emph{post architecture} . Se le asigna un valor \textbf{alto}. Esto 
implica que el valor promedio de la suma de los factores TOOL y SITE está en 
entre \emph{7 y 8}, el soporte de TOOL es \emph{bueno} y la condición de 
multisitio es \emph{fuerte soporte para el desarrollo}.

\end{description}

\chapter{Justificaciones del proyecto}
\section{Factores de escala}
Los factores de escala para el proyecto son los siguientes:
\begin{description}
\item[PREC] Mide el grado de experiencia previa. Se le asigna un valor de
\textbf{bajo}, esto implica que el entendimiento organizacional de los
objetivos del producto es \emph{general}, la experiencia en el trabajo con 
software relacionado es \emph{moderada}, el desarrollo concurrente de nuevo
hardware y procedimientos operacionales es \emph{abundante}, y la necesidad
de innovación en el procesamiento de datos, aquitectura y algoritmos es
\emph{considerable}.

\item[FLEX] Mide el nivel de exigencia en el cumplimiento de requerimientos
y plazos. Se le asigna un valor \textbf{nominal}, esto implica que la
necesidad de conformar requerimientos preestablecidos es \emph{considerable},
la necesidad de conformar especificaciones externas de interfase también es
\emph{considerable} y el estímulo por terminación temprana es \emph{medio}.

\item[RESL] Mide aspectos relacionados con los ítems de riesgo crítico y el
modo de abordarlos. Se le asigna un valor \textbf{alto}. Esto implica que:
\begin{itemize}
\item La planificación de la administración de riesgo, identificando los
ítems de riesgo y estableciendo hitos de control para su solución por medio
de la revisión del diseño del producto (PDR) es \emph{general}.
\item El cronograma, presupuesto e hitos internos especificados en el PDR,
compatibles con el plan de administración de riesgo, es también \emph{general}.
\item El porcentaje del cronograma dedicado a la definición de la arquitectura
de acuerdo a los objetivos generales del producto es aproximadamente del
\emph{25\%}.
\item El porcentaje de arquitecturas de software disponibles para el proyecto
es aproximadamente del \emph{80\%}.
\item Y las herramientas disponibles para resolver ítems de riesgo,
desarrollando y verificando las especulaciones de arquitecturas, son
\emph{buenas}.
\end{itemize}

\item[TEAM] Mide el grado de cohesión del equipo. Se le asigna un valor
\textbf{alto}. Esto implica que la compatibilidad entre los objetivos y
culturas de los integrantes del equipo es \emph{considerable}, la habilidad
y predisposición para conciliar objetivos es también \emph{considerable}, la
experiencia en el trabajo en grupo es \emph{básica} y la visión compartida
de objetivos y compromisos, es también \emph{básica}.

\item[PMAT] Mide la madurez del proceso. Según la evaluación CMM se le asigna
un valor \textbf{1 - mitad superior}, que corresponde a un nivel \emph{bajo}
para PMAT.
\end{description}

\section{Estimaciones de las variables calculadas}
\subsection{Post Architecture Model}
En la tabla \ref{Tab:EstPAM} se muestran las estimaciones pesimista,
optimista y más probable de las variables calculadas por el software.
\begin{table}[!h]
  \centering
  \begin{tabular}{p{4.5cm}|c|c|b{2cm}<{\centering}}
    %\cline{2-4}
    & \textbf{Pesimista} & \textbf{Más probable} & \textbf{Optimista}\\
    \hline \hline
    Esfuerzo & 20,1 & 16,1 & 12,8 \\ \hline
    Duración & 9,5 & 8,9 & 8,3 \\ \hline
    Productividad & 314 & 392,5 & 490,7 \\ \hline
    Coste & 41822,71 & 33458,17 & 26766,53 \\ \hline
    Coste por instrucción & 6,6 & 5,3 & 4,2 \\ \hline
    Personal & 2,1 & 1,8 & 1,6 \\ \hline
  \end{tabular}
  \caption{Estimaciones del modelo PAM} \label{Tab:EstPAM}
\end{table}

\subsection{Early Design Model}
En la tabla \ref{Tab:EstEDM} se muestran las estimaciones pesimista,
optimista y más probable de las variables calculadas por el software.
\begin{table}[!h]
  \centering
  \begin{tabular}{p{4.5cm}|c|c|b{2cm}<{\centering}}
    %\cline{2-4}
    & \textbf{Pesimista} & \textbf{Más probable} & \textbf{Optimista}\\
    \hline \hline
    Esfuerzo & 30,7 & 20,4 & 13,7 \\ \hline
    Duración & 10,9 & 9,6 & 8,4 \\ \hline
    Productividad & 205,6 & 308,5 & 460,4 \\ \hline
    Coste & 63868,11 & 42578,74 & 28527,76 \\ \hline
    Coste por instrucción & 10,1 & 6,8 & 4,5 \\ \hline
    Personal & 2,8 & 2,1 & 1,6 \\ \hline
  \end{tabular}
  \caption{Estimaciones del modelo EDM} \label{Tab:EstEDM}
\end{table}

\chapter{Informes}
\section{Early Design Model}
\section{Post Architecture Model}

\end{document}
