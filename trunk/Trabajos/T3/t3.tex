% Clase
\documentclass[11pt,a4paper,spanish,twoside]{report}

% Órdenes auxiliares
\input{inc/includes.tex}

% Encabezado y pie de página
\encabezado

\begin{document}

% Silabación extra
\hyphenation{
a-sig-na-tu-ras
au-to-ma-ti-za-rá
ca-tá-lo-go
ca-rre-ra
cons-truc-ción
co-rres-pon-de
diag-nos-tico
fi-na-li-za-ción
ge-ne-ra-ción
in-fe-rior
man-te-ni-mien-to
me-dian-te
per-so-nal
pro-ce-di-mien-tos
pro-por-cio-na-rá
pu-bli-ca-da
re-qui-si-tos
res-pecto
u-su-a-rios
vi-lla-rre-al
}


% Portada
\portada{Planificación y Gestión de\\Sistemas de Información}
{Trabajo 3}{Calendario del proyecto}
{Sergio de la Rubia García-Carpintero\\Miguel Millán Sánchez-Grande\\
  Luis Muñoz Villarreal\\Alicia Serrano Sánchez\\
  Juan Miguel Torres Triviño}{26 de Abril de 2010}

% Licencia
\licencia{Sergio de la Rubia García-Carpintero, Miguel Millán Sánchez-Grande,
  Luis Muñoz Villarreal, Alicia Serrano Sánchez, Juan Miguel Torres Triviño}

\chapter*{Ficha de trabajo}
\begin{description}
\item[Código] T3
\item[Fecha] 26 de Abril de 2010
\item[Título] Calendario del proyecto
\end{description}

\begin{table}[!ht]
  \centering
  \begin{tabular}{lp{5cm}c}
    \multicolumn{3}{l}{\Large \textbf{Equipo} G4} \\ \\
    \multicolumn{1}{c}{\emph{Apellidos y nombre}} & 
    \multicolumn{1}{c}{\emph{Firma}} & \emph{Puntos} \\
    \hline \\
    de la Rubia García-Carpintero, Sergio & & 6 \\ \\
    Millán Sánchez-Grande, Miguel         & & 6 \\ \\
    Muñoz Villarreal, Luis                & & 6 \\ \\
    Serrano Sánchez, Alicia               & & 6 \\ \\
    Torres Triviño, Juan Miguel           & & 6 \\ \\
    \hline
  \end{tabular}
\end{table}

% Índices
\tableofcontents
\listoftables
\listoffigures

%% INICIO DEL DOCUMENTO %%%%%%%%%%%%%%%%%%%%%%%%%%%%%%%%%%%%%%%%%%%%%%%%%

\chapter*{Introducción}
En este trabajo se realiza la elaboración del calendario del proyecto
mediante el uso de la técnica PERT (Program Evaluation and Review Technique),
la cual proporciona un método para realizar una estimación de la duración
total del proyecto a partir de las actividades, su secuencia y la estimación
ponderada de la duración media de estas. 

Para la estimación de los tiempos PERT se aplica la técnica DELPHI, reuniendo
a 5 expertos en sistemas de información, de tal modo que por su nivel de
formación y grado de 
conocimiento puedan aportar ideas y puntos de vistas diferentes al problema
en cuestión, con el fin de obtener juicios coherentes y enriquecidos con
respecto al problema. El criterio de convergencia utilizado es el dado, que
consiste en que una votación de expertos se acepta como final cuando todas
las valoraciones se encuentran en un rango acotado que corresponde
con la media más menos una vez y media la desviación típica

\chapter{Técnica DELPHI}
Mediante la técnica DELPHI se intenta obtener un consenso lo más fiable
posible del grupo de expertos.
    
\section{Tablas de tiempos DELPHI}
En las tablas \ref{Tab:DELPHIana}, \ref{Tab:DELPHIdis}, \ref{Tab:DELPHIimp} y
\ref{Tab:DELPHIpru}, de las páginas \pageref{Tab:DELPHIana},
\pageref{Tab:DELPHIdis} y \pageref{Tab:DELPHIimp}; se
puede observar la evolución seguida a la hora de calcular los tiempos
DELPHI. Cada tabla contiene el identificador de cada tarea, la iteración en
la que se encuentra, las valoraciones de los distintos expertos, así como los
valores mínimo (que corresponde con el tiempo optimista), medio (que
correspondeo con la media aritmética) y máximo (que es el tiempo pesimista de
cada tarea), y finalmente, si se cumple el criterio de convergencia o no.

\begin{table}[!h]
\centering
  \begin{tabular}{|c|c||c|c|c|c|c||c|c|c||c|}
    \hline
    \textbf{T} & \textbf{I} & \textbf{1} &
    \textbf{2} & \textbf{3} & \textbf{4} & \textbf{5} & \textbf{m}
    &\textbf{$\bar{x}$} &\textbf{M} & \textbf{C}\\
    \hline \hline

    \multirow{3}{*}{1.1} 
    & 1 & 4 & 5 & 3 & 6 & 4 & 3 & 4'4 & 6 & N \\
    & 2 & 4 & 5 & 3 & 4 & 5 & 3 & 4'2 & 5 & N \\
    &\textbf{3} &\textbf {5} &\textbf {5} &\textbf {4} &\textbf {4}
    &\textbf{5} & \textbf{4} & \textbf{4'6} &\textbf{5} & \textbf{S}\\
    \hline

    \multirow{3}{*}{1.2} 
    & 1 & 5 & 5 & 4 & 7 & 5 & 4 & 5'2 & 7 & N \\
    & 2 & 6 & 5 & 4 & 6 & 5 & 4 & 5'2 & 6 & N \\
    &\textbf{3} & \textbf{6} & \textbf{5} & \textbf{4} & \textbf{6} &
    \textbf{4} & \textbf{4} & \textbf{5} & \textbf{6} & \textbf{S} \\    
    \hline

    \multirow{5}{*}{1.3} 
    & 1 & 2 & 3 & 10 & 3 & 5 & 2 & 4'6 & 10 & N \\
    & 2 & 2 & 3 & 8  & 3 & 5 & 2 & 4'2 & 8  & N \\
    & 3 & 3 & 3 & 6  & 3 & 5 & 3 & 4   & 6  & N \\
    & 4 & 3 & 3 & 6  & 4 & 5 & 3 & 4'2 & 6  & N \\
    & \textbf{5} & \textbf{3} & \textbf{3} & \textbf{5} & \textbf{4} &
    \textbf{5} & \textbf{3} & \textbf{4} & \textbf{5} & \textbf{S} \\
    \hline

    \multirow{7}{*}{1.4.1} 
    & 1 & 2 & 4 & 7 & 4 & 3 & 2 & 4   & 7 & N \\
    & 2 & 4 & 4 & 3 & 7 & 4 & 3 & 4'4 & 7 & N \\
    & 3 & 4 & 4 & 4 & 7 & 5 & 4 & 4'8 & 7 & N \\
    & 4 & 4 & 4 & 4 & 5 & 6 & 4 & 4'6 & 6 & N \\
    & 5 & 4 & 5 & 4 & 5 & 6 & 4 & 4'8 & 6 & N \\
    & 6 & 4 & 6 & 5 & 5 & 5 & 4 & 5   & 6 & N \\
    & \textbf{7} & \textbf{5} & \textbf{5} & \textbf{5} & \textbf{5} &
    \textbf{5} & \textbf{5} & \textbf{5} & \textbf{5} & \textbf{S} \\
    \hline

    \multirow{2}{*}{1.4.2} & 1 & 2 & 2 & 4 & 3 & 2 & 2 & 2'6 & 4 & N\\
    & \textbf{2} & \textbf{2} & \textbf{2} & \textbf{3} & \textbf{3} &
    \textbf{2} & \textbf{2} & \textbf{2'4} & \textbf{3} & \textbf{S} \\ 
    \hline

    \multirow{3}{*}{1.4.3} 
    & 1 & 2 & 2 & 2 & 2 & 1 & 1 & 1'8 & 2 & N \\
    & 2 & 2 & 2 & 2 & 2 & 3 & 2 & 2'2 & 3 & N \\
    & \textbf{3} & \textbf{2} & \textbf{2} & \textbf{2} & \textbf{2} &
    \textbf{2} & \textbf{2} & \textbf{2} & \textbf{2} & \textbf{S} \\ 
    \hline

    \multirow{5}{*}{1.5} 
    & 1 & 9   & 5   & 4   & 6 & 5   & 4   & 5'8 & 9   & N \\
    & 2 & 8   & 5'5 & 4'5 & 6 & 5'5 & 4'5 & 5'9 & 8   & N \\
    & 3 & 7'5 & 5'5 & 4'5 & 6 & 5'5 & 4'5 & 5'8 & 7'5 & N \\
    & 4 & 5'5 & 6   & 5'5 & 6 & 5   & 5   & 5'6 & 6   & N \\
    & \textbf{5} & \textbf{5'5} & \textbf{6} & \textbf{5'5} & \textbf{6} &
    \textbf{6} & \textbf{5'5} & \textbf{5'8} & \textbf{6} & \textbf{S} \\ 
    \hline

    \multirow{5}{*}{1.6} 
    & 1 & 4 & 10 & 2   & 3 & 5   & 2   & 4'8 & 10 & N \\
    & 2 & 6 & 8  & 3   & 5 & 6   & 3   & 5'6 & 8  & N \\
    & 3 & 5 & 8  & 3'5 & 5 & 6   & 3'5 & 5'5 & 8  & N \\
    & 4 & 5 & 8  & 6   & 4 & 5'5 & 4   & 5'7 & 8  & N \\
    & \textbf{5} & \textbf{5} & \textbf{7} & \textbf{6} & \textbf{4} &
    \textbf{5'5} & \textbf{4} & \textbf{5'5} & \textbf{7} & \textbf{S} \\ 
    \hline

    \multirow{2}{*}{1.7} & 1 & 4 & 5 & 3 & 2 & 2 & 2 & 3'2 & 5 & N \\
    & \textbf{2} & \textbf{4} & \textbf{4'5} & \textbf{4} & \textbf{2} &
    \textbf{2} & \textbf{2} & \textbf{3'3} & \textbf{4'5} & \textbf{S} \\ 
    \hline

    \multirow{11}{*}{1.8} 
    & 1  & 4   & 3   & 6   & 3   & 2   & 2   & 3'6 & 6   & N \\
    & 2  & 4   & 3   & 5'5 & 3   & 2'5 & 2'5 & 3'6 & 5'5 & N \\
    & 3  & 4   & 3'5 & 5'5 & 3   & 2'5 & 2'5 & 3'7 & 5'5 & N \\
    & 3  & 4   & 3'5 & 5'5 & 2'5 & 3   & 2'5 & 3'7 & 5'5 & N \\
    & 5  & 4   & 3'5 & 5   & 2'5 & 3   & 2'5 & 3'6 & 5   & N \\
    & 6  & 4   & 3'5 & 5   & 3   & 3   & 3   & 3'7 & 5   & N \\
    & 7  & 4   & 3   & 5   & 4   & 4   & 3   & 4   & 5   & N \\
    & 8  & 4   & 3'5 & 5   & 4   & 4   & 3'5 & 4'1 & 5   & N \\
    & 9  & 4'5 & 4   & 5   & 4'5 & 4   & 4   & 4'4 & 5   & N \\
    & 10 & 4   & 4   & 4'5 & 4   & 4   & 4   & 4'1 & 4'5 & N \\
    & \textbf{11} & \textbf{4'5} & \textbf{4} & \textbf{4'5} & \textbf{4} &
    \textbf{4} & \textbf{4} & \textbf{4'2} & \textbf{4'5} & \textbf{S} \\ 
    \hline

    1.9 & \textbf{1} & \textbf{1} & \textbf{1} & \textbf{1} &
    \textbf{1} & \textbf{1} & \textbf{1} & \textbf{1} & \textbf{1} &
    \textbf{S} \\
    \hline
  \end{tabular}
  \caption{\textbf{DELPHI} de tareas de \textbf{análisis}} 
  \label{Tab:DELPHIana}
\end{table}

\begin{table}[!h]
\centering
  \begin{tabular}{|c|c||c|c|c|c|c||c|c|c||c|}
    \hline
    \textbf{T} & \textbf{I} & \textbf{1} &
    \textbf{2} & \textbf{3} & \textbf{4} & \textbf{5} & \textbf{m}
    &\textbf{$\bar{x}$} &\textbf{M} & \textbf{C}\\
    \hline \hline

    2.1 & \textbf{1} & \textbf{6} & \textbf{3} & \textbf{6} &
    \textbf{5} & \textbf{3} & \textbf{3} & \textbf{4'6} & \textbf{6} &
    \textbf{S}\\
    \hline

    2.2 & \textbf{1} & \textbf{6} & \textbf{6} & \textbf{7} &
    \textbf{2} & \textbf{2} & \textbf{2} & \textbf{4'6} & \textbf{7} &
    \textbf{S}\\ 
    \hline

    \multirow{3}{*}{2.3.1}
    & 1 & 3 & 11 & 15 & 8 & 6 & 3 & 8'6 & 15 & N \\
    & 2 & 3 & 11 & 14 & 8 & 8 & 3 & 8'8 & 14 & N \\
    & \textbf{3} & \textbf{3} & \textbf{11} & \textbf{14} & \textbf{7} &
    \textbf{7} & \textbf{3} & \textbf{8'4} & \textbf{14} & \textbf{S} \\
    \hline

    \multirow{2}{*}{2.3.2} & 1 & 4 & 7 & 3 & 5 & 2 & 2 & 4.2 & 7 & N \\
    & \textbf{2} & \textbf{5} & \textbf{6} & \textbf{3} & \textbf{6} &
    \textbf{2} & \textbf{2} & \textbf{4'4} & \textbf{6} & \textbf{S} \\
    \hline

    \multirow{3}{*}{2.3.3}
    & 1 & 4 & 5 & 4 & 5 & 2 & 2 & 4   & 5 & N \\
    & 2 & 3 & 5 & 4 & 5 & 2 & 2 & 3'8 & 5 & N \\
    & \textbf{3} & \textbf{3} & \textbf{5} & \textbf{4} & \textbf{5} &
    \textbf{2'5} & \textbf{2'5} & \textbf{3'9} & \textbf{5} & \textbf{S} \\
    \hline

    2.4.1 & \textbf{1} & \textbf{5} & \textbf{2} & \textbf{5} &
    \textbf{4} & \textbf{2} & \textbf{2} & \textbf{3'6} & \textbf{5} &
    \textbf{S} \\ 
    \hline

    \multirow{3}{*}{2.4.2}
    & 1 & 4 & 8 & 4 & 3 & 1 & 1 & 4   & 8 & N \\
    & 2 & 5 & 7 & 5 & 3 & 1 & 1 & 4'2 & 7 & N \\
    & \textbf{3} & \textbf{5} & \textbf{7} & \textbf{5} & \textbf{3} &
    \textbf{2} & \textbf{2} & \textbf{4'4} & \textbf{7} & \textbf{S} \\ 
    \hline

    \multirow{4}{*}{2.4.3}
    & 1 & 5 & 3 & 4 & 1 & 3 & 1 & 3'2 & 5 & N \\
    & 2 & 5 & 4 & 5 & 2 & 3 & 2 & 3'8 & 5 & N \\
    & 3 & 4 & 4 & 5 & 2 & 3 & 2 & 3'6 & 5 & N \\
    & \textbf{4} & \textbf{4} & \textbf{4} & \textbf{5} & \textbf{2'5} &
    \textbf{3} & \textbf{2'5} & \textbf{3'7} & \textbf{5} & \textbf{S} \\ 
    \hline

    \multirow{2}{*}{2.5}& 1 & 1 & 2 & 1 & 1 & 3 & 1 & 1'6 & 3 & N \\
    & \textbf{2} & \textbf{1} & \textbf{2} & \textbf{1} & \textbf{1} &
    \textbf{2} & \textbf{1} & \textbf{1'4} & \textbf{2} & \textbf{S} \\ 
    \hline

    \multirow{3}{*}{2.6}
    & 1 & 4 & 12 & 5 & 4 & 3 & 3 & 5'6 & 12 & N \\
    & 2 & 4 & 11 & 8 & 4 & 3 & 3 & 6 & 11 & N \\
    & \textbf{3} & \textbf{4} & \textbf{9} & \textbf{8} & \textbf{4} &
    \textbf{3} & \textbf{3} & \textbf{5'6} & \textbf{9} & \textbf{S} \\
    \hline

    \multirow{2}{*}{2.7}& 1 & 2 & 4 & 2 & 2 & 5 & 2 & 3 & 5 & N \\
    & \textbf{2} & \textbf{2} & \textbf{4} & \textbf{2} & \textbf{2} &
    \textbf{4} & \textbf{2} & \textbf{2'8} & \textbf{4} & \textbf{S} \\
    \hline

    2.8 & \textbf{1} & \textbf{3} & \textbf{6} & \textbf{5} &
    \textbf{2} & \textbf{2} & \textbf{2} & \textbf{3'6} & \textbf{6} &
    \textbf{S} \\ 
    \hline

    \multirow{4}{*}{2.9}
    & 1 & 4 & 20 & 3 & 4 & 3 & 3 & 6'8 & 20 & N \\
    & 2 & 6 & 15 & 4 & 4 & 5 & 4 & 6'8 & 15 & N \\
    & 3 & 6 & 13 & 6 & 6 & 8 & 6 & 7'8 & 13 & N \\
    & \textbf{4} & \textbf{6} & \textbf{11} & \textbf{6} & \textbf{6} &
    \textbf{10} & \textbf{6} & \textbf{7'8} & \textbf{11} & \textbf{S} \\ 
    \hline

    2.10 & \textbf{1} & \textbf{2} & \textbf{1} & \textbf{1} &
    \textbf{2} & \textbf{1} & \textbf{1} & \textbf{1'4} & \textbf{2} &
    \textbf{S} \\
    \hline
  \end{tabular}
  \caption{\textbf{DELPHI} de tareas de \textbf{diseño}}
  \label{Tab:DELPHIdis}
\end{table}

\begin{table}[!h]
\centering
  \begin{tabular}{|c|c||c|c|c|c|c||c|c|c||c|}
    \hline 
    \textbf{T} & \textbf{I} & \textbf{1} &
    \textbf{2} & \textbf{3} & \textbf{4} & \textbf{5} & \textbf{m}
    &\textbf{$\bar{x}$} &\textbf{M} & \textbf{C}\\    
    \hline \hline

    \multirow{2}{*}{3.1}& 1 & 2 & 3 & 3 & 2 & 4 & 2 & 2'8 & 4 & N \\
    & \textbf{2} & \textbf{2} & \textbf{4} & \textbf{4} & \textbf{2} &
    \textbf{4} & \textbf{2} & \textbf{3'2} & \textbf{4} & \textbf{S} \\
    \hline

    3.2 & \textbf{1} & \textbf{14} & \textbf{8} & \textbf{7} &
    \textbf{5} & \textbf{12} & \textbf{5} & \textbf{9'2} & \textbf{14} &
    \textbf{S} \\ 
    \hline

    \multirow{3}{*}{3.3}
    & 1 & 18 & 5 & 10 & 3 & 6 & 3 & 8'4 & 18 & N \\
    & 2 & 15 & 7 & 11 & 5 & 7 & 5 & 9   & 15 & N \\
    & \textbf{3} & \textbf{15} & \textbf{9} & \textbf{13} & \textbf{7} &
    \textbf{8} & \textbf{7} & \textbf{10'4} & \textbf{15} & \textbf{S} \\
    \hline

    \multirow{3}{*}{3.4}
    & 1 & 7 & 3 & 2 & 2 & 3 & 2 & 3'4 & 7 & N \\
    & 2 & 5 & 3 & 3 & 3 & 4 & 3 & 3'6 & 5 & N \\
    & \textbf{3} & \textbf{4} & \textbf{3} & \textbf{4} & \textbf{3} &
    \textbf{4} & \textbf{3} & \textbf{3'6} & \textbf{4} & \textbf{S} \\
    \hline

    3.5 & \textbf{1} & \textbf{7} & \textbf{3} & \textbf{6} &
    \textbf{5} & \textbf{4} & \textbf{3} & \textbf{5} &\textbf{7} &
    \textbf{S} \\ 
    \hline

  \end{tabular}
  \caption{\textbf{DELPHI} de tareas de \textbf{implementación}}
  \label{Tab:DELPHIimp}
\end{table}

\begin{table}[!h]
\centering
  \begin{tabular}{|c|c||c|c|c|c|c||c|c|c||c|}
    \hline
    \textbf{T} & \textbf{I} & \textbf{1} &
    \textbf{2} & \textbf{3} & \textbf{4} & \textbf{5} & \textbf{m}
    &\textbf{$\bar{x}$} &\textbf{M} & \textbf{C}\\    
    \hline \hline

    \multirow{4}{*}{4.1}
    & 1 & 5 & 3 & 4 & 2   & 4 & 2   & 3'6 & 5 & N \\
    & 2 & 5 & 4 & 4 & 2'5 & 4 & 2'5 & 3'9 & 5 & N \\
    & 3 & 5 & 4 & 4 & 3   & 4 & 3   & 4   & 5 & N \\
    & \textbf{5} & \textbf{4'5} & \textbf{4'5} & \textbf{3'5} & \textbf{3'5}
    & \textbf{4} & \textbf{3'5} & \textbf{4} & \textbf{4'5} & \textbf{S} \\
    \hline
    
    \multirow{2}{*}{4.2} & 1 & 5 & 7 & 5 & 3 & 3 & 3 & 4.6 & 7 & N \\
    & \textbf{2} & \textbf{5} & \textbf{6} & \textbf{6} & \textbf{3} &
    \textbf{3} & \textbf{3} & \textbf{4'6} & \textbf{6} & \textbf{S} \\
    \hline

    \multirow{3}{*}{4.3}
    & 1 & 6 & 4 & 10 & 3 & 3 & 3 & 5.2 & 10 & N \\
    & 2 & 6 & 4 & 8  & 3 & 4 & 3 & 5   & 8  & N \\
    & \textbf{3} & \textbf{7} & \textbf{4} & \textbf{8} & \textbf{3} &
    \textbf{4} & \textbf{3} & \textbf{5'2} & \textbf{8} & \textbf{S} \\
    \hline

    4.4 & \textbf{1} & \textbf{3} & \textbf{2} & \textbf{1} &
    \textbf{3} & \textbf{1} & \textbf{1} & \textbf{2} & \textbf{3} &
    \textbf{S} \\
    \hline
  \end{tabular}
  \caption{\textbf{DELPHI} de tareas de \textbf{pruebas}}
  \label{Tab:DELPHIpru}
\end{table} 

\section{Tabla resumen de DELPHI}
Las tablas \ref{Tab:rDELPHIana}, \ref{Tab:rDELPHIdis}, \ref{Tab:rDELPHIimp} y
\ref{Tab:rDELPHIpru}, de las páginas \pageref{Tab:rDELPHIana} y
\pageref{Tab:rDELPHIdis}; reprensentan los resultados abreviados de aplicar
la técnica DELPHI. Cada tabla se corresponde con las tareas de cada etapa de
elaboración del software. Cada fila de la tabla contiene las tareas, el
tiempo mínimo, el medio y el máximo.

\begin{table}[!h]
\centering
  \begin{tabular}{|c||p{8cm}||c|c|c|}
    \hline
    \textbf{Id} & \textbf{Tarea} & \textbf{m} & 
    \textbf{$\bar{x}$} &\textbf{M} \\
    \hline \hline
    1.1 & Definición del sistema & 4 & 4'6 & 5\\ 
    \hline
    1.2 & Establecimiento de requisitos & 4 & 5 & 6\\
    \hline
    1.3 & Identificación de subsistemas & 3 & 4 & 5\\
    \hline
    1.4.1 & Elaboración del modelo conceptual y lógica de datos & 5 & 5 & 5\\
    \hline
    1.4.2 & Normalización & 2 & 2'4 & 3 \\
    \hline
    1.4.3 & Especificación de necesidades de carga inicial & 2 & 2 & 2\\
    \hline
    1.5 & Elaboración del modelo de procesos & 5'5 & 5'8 & 6\\
    \hline
    1.6 & Definición de interfaz de usuario & 4 & 5'5 & 7\\
    \hline
    1.7 & Análisis de consistencia y especificación de requisitos & 2 & 3'3 &
    4.5\\
    \hline
    1.8 & Especificación del plan de pruebas & 4 & 4'3 & 4'5\\
    \hline
    1.9 & Aprobación del análisis del SI & 1 & 1 & 1\\
    \hline
  \end{tabular}
  \caption{Resumen: \textbf{DELPHI} de tareas de \textbf{análisis}}
  \label{Tab:rDELPHIana}
\end{table}

\begin{table}[!h]
\centering
  \begin{tabular}{|c||p{8cm}||c|c|c|}
    \hline
    \textbf{Id} & \textbf{Tarea} & \textbf{m} & 
    \textbf{$\bar{x}$} &\textbf{M} \\
    \hline \hline
    2.1 & Definición de la arquitectura del sistema & 3 & 4'6 & 6 \\
    \hline
    2.2 & Diseño de arquitectura de soporte & 2 & 4'6 & 7 \\
    \hline
    2.3.1 & Diseño de módulos del sistema & 3 & 8'4 & 14 \\
    \hline
    2.3.2 & Diseño de comunicación entre módulos & 2 & 4'4 & 6 \\
    \hline
    2.3.3 & Revisión de la interfaz de usuario &2'5 & 3'9 & 5 \\
    \hline
    2.4.1 & Diseño del modelo físico de datos & 2 & 3'6 & 5 \\
    \hline
    2.4.2 & Especificación de los caminos de acceso a los datos & 2 & 4'4 & 7\\
    \hline
    2.4.3 & Especificación de la distribución de datos & 2'5 & 3'7 & 5 \\
    \hline
    2.5 & Verificación y aceptación de la arquitectura del sistema & 1 & 1'4
    & 2 \\
    \hline
    2.6 & Generación y especificación de construcción & 3 & 5'6 & 9 \\
    \hline
    2.7 & Diseño de migración y carga inicial de datos & 2 & 2'8 & 4 \\
    \hline
    2.8 & Especificación técnica del plan de prueba & 2 & 3'6 & 6 \\
    \hline
    2.9 & Establecimiento de requisitos de implantación & 6 & 7'8 & 11 \\
    \hline
    2.10 & Aprobación de diseño y SI & 1 & 1'4 & 2 \\
    \hline
  \end{tabular}
  \caption{Resumen: \textbf{DELPHI} de tareas de \textbf{diseño}}
  \label{Tab:rDELPHIdis}
\end{table}

\begin{table}[!h]
\centering
  \begin{tabular}{|c||p{8cm}||c|c|c|}
    \hline
    \textbf{Id} & \textbf{Tarea} & \textbf{m} & 
    \textbf{$\bar{x}$} &\textbf{M} \\
    \hline \hline
    3.1 & Preparación del entorno de generación y construcción & 2 & 3'2  & 4 \\
    \hline
    3.2 & Generación del código de los componentes y los procedimientos & 5 &
    9'2 & 14\\
    \hline
    3.3 & Elaboración del manual de usuario & 7 & 10'4 & 15\\
    \hline
    3.4 & Definición de la formación de los usuarios finales & 3 & 3'6 & 4 \\
    \hline
    3.5 & Construcción de los componentes y procedimientos de carga inicial
    de datos & 3 & 5 & 7\\
    \hline
  \end{tabular}
  \caption{Resumen: \textbf{DELPHI} de tareas de \textbf{implementación}}
  \label{Tab:rDELPHIimp}
\end{table}

\begin{table}[!h]
\centering
  \begin{tabular}{|c||p{8cm}||c|c|c|}
    \hline
    \textbf{Id} & \textbf{Tarea} & \textbf{m} & 
    \textbf{$\bar{x}$} &\textbf{M} \\
    \hline \hline
    4.1 & Ejecución de las pruebas unitarias & 3'5 & 4 & 4'5\\
    \hline
    4.2 & Ejecución de las pruebas de integración & 3 & 4'6 & 6\\
    \hline
    4.3 & Ejecución de las pruebas del sistema & 3 & 5'2 & 8\\
    \hline
    4.4 & Aprobación del SI & 1 & 2 & 3 \\
    \hline
  \end{tabular}
  \caption{Resumen: \textbf{DELPHI} de tareas de \textbf{pruebas}}
  \label{Tab:rDELPHIpru}
\end{table}


\chapter{Técnica PERT}
Técnica que permite realizar una estimación de la duración total de un
proyecto a partir de la secuencia de actividades y de una estimación
ponderada de la duración media de cada una. 

\section{Cálculos de los tiempos PERT}
Las tablas \ref{Tab:PERTana}, \ref{Tab:PERTdis}, \ref{Tab:PERTimp} y
\ref{Tab:PERTpru}, de las páginas \pageref{Tab:PERTana}, \pageref{Tab:PERTdis} y
\pageref{Tab:PERTpru}; reprensentan los resultados de aplicar las fórmulas de la
técnica PERT, donde cada tabla se corresponde con las actividades de cada
etapa de elaboración del software. Cada fila de la tabla contiene las
actividades, el tiempo PERT y varianza.

\begin{table}[!h]
  \centering
  \begin{tabular}{|c||p{5.3cm}||c|c|c||c|c|}
    \hline
    \textbf{Id} & \textbf{Tarea} & \textbf{m} & 
    \textbf{$\bar{x}$} &\textbf{M} & \textbf{PERT} & \textbf{Var}\\
    \hline \hline
    1.1 & Definición del sistema & 4 & 4.6 & 5 & 4.57 & 0.17\\ 
    \hline
    1.2 & Establecimiento de requisitos & 4 & 5 & 6 & 5 & 0.33\\
    \hline 
    1.3 & Identificación de subsistemas & 3 & 4 & 5 & 4 & 0.33\\
    \hline
    1.4.1 & Elaboración del modelo conceptual y lógica de datos & 5 & 5 & 5 &
    5 & 0\\
    \hline
    1.4.2 & Normalización & 2 & 2.4 & 3 & 2.43 & 0.17\\
    \hline
    1.4.3 & Especificación de necesidades de carga inicial & 2 & 2 & 2 & 2 & 0\\
    \hline
    1.5 & Elaboración del modelo de procesos & 5.5 & 5.8 & 6 & 5.78 & 0.08\\
    \hline
    1.6 & Definición de interfaz de usuario & 4 & 5.5 & 7 & 5.5 & 0.5\\
    \hline
    1.7 & Análisis de consistencia y especificación de requisitos & 2 & 3.3
    & 4.5 & 3.28 & 0.42\\
    \hline
    1.8 & Especificación del plan de pruebas & 4 & 4.3 & 4.5 & 4.22 & 0.08\\
    \hline
    1.9 & Aprobación del análisis del SI & 1 & 1 & 1 & 1 & 0\\
    \hline
  \end{tabular}
  \caption{\textbf{PERT} de tareas de \textbf{análisis}} 
  \label{Tab:PERTana}
\end{table}

\begin{table}[!h]
\centering
  \begin{tabular}{|c||p{5.3cm}||c|c|c||c|c|}
    \hline
    \textbf{Id} & \textbf{Tarea} & \textbf{m} & 
    \textbf{$\bar{x}$} &\textbf{M} & \textbf{PERT} & \textbf{Var}\\
    \hline \hline
    2.1 & Definición de la arquitectura del sistema & 3 & 4.6 & 6 & 4.57 & 0.5\\
    \hline
    2.2 & Diseño de arquitectura de soporte & 2 & 4.6 & 7 & 4.57 & 0.83\\
    \hline
    2.3.1 & Diseño de módulos del sistema & 3 & 8.4 & 14 & 8.43 & 1.83\\
    \hline
    2.3.2 & Diseño de comunicación entre módulos & 2 & 4.4 & 6 & 4.27 & 0.67\\
    \hline
    2.3.3 & Revisión de la interfaz de usuario &2.5 & 3.9 & 5 & 3.85 & 0.42\\
    \hline
    2.4.1 & Diseño del modelo físico de datos & 2 & 3.6 & 5 & 3.57 & 0.5\\
    \hline
    2.4.2 & Especificación de los caminos de acceso a los datos & 2 & 4.4 &
    7 & 4.43 & 0.83\\
    \hline
    2.4.3 & Especificación de la distribución de datos & 2.5 & 3.7 & 5 & 3.72
    & 0.42\\
    \hline
    2.5 & Verificación y aceptación de la arquitectura del sistema & 1 & 1.4
    & 2 & 1.43 & 0.17\\
    \hline
    2.6 & Generación y especificación de construcción & 3 & 5.6 & 9 & 5.73 & 1\\
    \hline
    2.7 & Diseño de migración y carga inicial de datos & 2 & 2.8 & 4 & 2.87 &
    0.33\\
    \hline
    2.8 & Especificación técnica del plan de prueba & 2 & 3.6 & 6 & 3.73 &
    0.67 \\
    \hline
    2.9 & Establecimiento de requisitos de implantación & 6 & 7.8 & 11 & 8.03
    & 0.83\\
    \hline
    2.10 & Aprobación de diseño y SI & 1 & 1.4 & 2 & 1.43 & 0.17\\
    \hline
  \end{tabular}
  \caption{\textbf{PERT} de tareas de \textbf{diseño}}
  \label{Tab:PERTdis}
\end{table}

\begin{table}[!h]
\centering
  \begin{tabular}{|c||p{5.3cm}||c|c|c||c|c|}
    \hline
    \textbf{Id} & \textbf{Tarea} & \textbf{m} & 
    \textbf{$\bar{x}$} &\textbf{M} & \textbf{PERT} & \textbf{Var}\\
    \hline \hline
    3.1 & Preparación del entorno de generación y construcción & 2 & 3.2  & 4
    & 3.13 & 0.33\\
    \hline
    3.2 & Generación del código de los componentes y los procedimientos & 5 &
    9.2 & 14 & 9.3 & 1.5\\
    \hline
    3.3 & Elaboración del manual de usuario & 7 & 10.4 & 15 & 10.6 & 1.33\\
    \hline
    3.4 & Definición de la formación de los usuarios finales & 3 & 3.6 & 4
    &3.57 & 0.17\\
    \hline
    3.5 & Construcción de los componentes y procedimientos de carga inicial
    de datos & 3 & 5 & 7 & 5 & 0.67\\
    \hline
  \end{tabular}
  \caption{\textbf{PERT} de tareas de \textbf{implementación}}
  \label{Tab:PERTimp}
\end{table}

\begin{table}[!h]
\centering
  \begin{tabular}{|c||p{5.3cm}||c|c|c||c|c|}
    \hline
    \textbf{Id} & \textbf{Tarea} & \textbf{m} & 
    \textbf{$\bar{x}$} &\textbf{M} & \textbf{PERT} & \textbf{Var}\\
    \hline \hline
    4.1 & Ejecución de las pruebas unitarias & 3.5 & 4 & 4.5 & 4 & 0.17\\
    \hline
    4.2 & Ejecución de las pruebas de integración & 3 & 4.6 & 6 & 4.57 & 0.5\\
    \hline
    4.3 & Ejecución de las pruebas del sistema & 3 & 5.2 & 8 & 5.3 & 0.83\\
    \hline
    4.4 & Aprobación del SI & 1 & 2 & 3 & 2 & 0.33\\
    \hline
  \end{tabular}
  \caption{\textbf{PERT} de para tareas de \textbf{pruebas}}
  \label{Tab:PERTpru}
\end{table}

\section{Cálculos de tiempos Early y Late}
En la tabla \ref{Tab:tearly}, de la página \pageref{Tab:tearly}, se
reprensentan los resultados de aplicar las
fórmulas de la técnica PERT para tiempos Early y Late, donde cada tabla se
corresponde con los sucesos de cada etapa de elaboración del software. Cada
fila de la tabla contiene los sucesos, el tiempo Early y Late.

\begin{table}[!h]
\centering
  \begin{tabular}{|c|c|c|}
    \hline
    \textbf{Sucesos} & \textbf{Early} & \textbf{Late} \\
    \hline \hline
    1  & 0     & 0     \\
    \hline
    2  & 4,57  & 4,57  \\
    \hline
    3  & 9,57  & 9,57  \\
    \hline
    4  & 14,57 & 14,57 \\
    \hline
    5  & 17    & 17    \\
    \hline
    6  & 19    & 19    \\
    \hline
    7  & 23,22 & 23,22 \\
    \hline
    8  & 24,22 & 24,22 \\
    \hline
    9  & 28,79 & 28,79 \\
    \hline
    10 & 37,22 & 37,22 \\
    \hline
    11 & 32,36 & 37,19 \\
    \hline
    12 & 41,49 & 41,49 \\
    \hline
    13 & 36,79 & 41,62 \\
    \hline
    14 & 45,34 & 41,34 \\
    \hline
    15 & 46,77 & 46,77 \\
    \hline
    16 & 54,8  & 54,8  \\
    \hline
    17 & 56,23 & 56,23 \\
    \hline
    18 & 59,36 & 59,36 \\
    \hline
    19 & 68,66 & 68,66 \\
    \hline
    20 & 73,96 & 73,96 \\
    \hline
    21 & 75,96 & 75,96 \\
    \hline
  \end{tabular}
  \caption{Tiempos Early y Late de los sucesos}
  \label{Tab:tearly}
\end{table}

\section{Cálculo de las holguras}
Las tablas \ref{Tab:HOLana}, \ref{Tab:HOLdis}, \ref{Tab:HOLimp} y
\ref{Tab:HOLpru}, de las páginas \pageref{Tab:HOLana} y \pageref{Tab:HOLpru};
reprensentan los resultados de aplicar las fórmulas de la
técnica PERT para el cálculo de las holguras, donde cada tabla se corresponde
con las actividades de cada etapa de elaboración del software. Cada fila de la
tabla contiene las actividades, el tiempo PERT, la holgura total, la holgura
libre y la holgura independiente.

\begin{table}[!h]
  \centering
  \begin{tabular}{|c||c||c|c|c|}
    \hline
    \textbf{Actividad} & \textbf{PERT} & \textbf{Total} & \textbf{Libre}
    & \textbf{Indep.}\\
    \hline \hline
    A1.1   & 4.57 & 0    & 0    & 0    \\ 
    \hline
    A1.2   & 5    & 0    & 0    & 0    \\
    \hline 
    A1.3   & 4    & 5.43 & 5.43 & 5.43 \\
    \hline
    A1.4.1 & 5    & 0    & 0    & 0    \\
    \hline
    A1.4.2 & 2.43 & 0    & 0    & 0    \\
    \hline
    A1.4.3 & 2    & 0    & 0    & 0    \\
    \hline
    A1.5   & 5.78 & 3.65 & 3.65 & 3.65 \\
    \hline
    A1.6   & 5.5  & 3.93 & 3.93 & 3.93 \\
    \hline
    A1.7   & 3.28 & 0.94 & 0.94 & 0.94 \\
    \hline
    A1.8   & 4.22 & 0    & 0    & 0    \\
    \hline
    A1.9   & 1    & 0    & 0    & 0    \\
    \hline
  \end{tabular}
  \caption{\textbf{Holgura} de tareas de \textbf{análisis}} 
  \label{Tab:HOLana}
\end{table}

\begin{table}[!h]
\centering
  \begin{tabular}{|c||c||c|c|c|}
    \hline
     \textbf{Actividad} & \textbf{PERT} & \textbf{Total} & \textbf{Libre}
    & \textbf{Indep.}\\
    \hline \hline
    D2.1 & 4.57 & 0 & 0 & 0\\
    \hline
    D2.2 & 4.57 & 11.98 & 11.98 & 11.98\\
    \hline
    D2.3.1 & 8.43 & 0 & 0 & 0\\
    \hline
    D2.3.2 & 4.27 & 0 & 0 & 0\\
    \hline
    D2.3.3 & 3.85 & 0 & 0 & 0\\
    \hline
    D2.4.1 & 3.57 & 4.86 & 0 & 0\\
    \hline
    D2.4.2 & 4.43 & 4.83 & 0 & -4.83\\
    \hline
    D2.4.3 & 3.72 & 4.83 & 4.83 & 0\\
    \hline
    D2.5 & 1.43 & 0 & 0 & 0\\
    \hline
    D2.6 & 5.73 & 2.3 & 2.3 & 2.3\\
    \hline
    D2.7 & 2.87 & 5.16 & 5.16 & 5.16\\
    \hline
    D2.8 & 3.73 & 4.3 & 4.3 & 4.3\\
    \hline
    D2.9 & 8.03 & 0 & 0 & 0\\
    \hline
    D2.10 & 1.43 & 0 & 0 & 0\\
    \hline
  \end{tabular}
  \caption{\textbf{Holgura} de tareas de \textbf{diseño}}
  \label{Tab:HOLdis}
\end{table}

\begin{table}[!h]
\centering
  \begin{tabular}{|c||c||c|c|c|}
    \hline
     \textbf{Actividad} & \textbf{PERT} & \textbf{Total} & \textbf{Libre}
    & \textbf{Indep.}\\
    \hline \hline
    I3.1 & 3.13 & 0 & 0 & 0\\
    \hline
    I3.2 & 9.3 & 0 & 0 & 0\\
    \hline
    I3.3 & 10.6 & 1.93 & 1.93 & 1.93\\
    \hline
    I3.4 & 3.57 & 8.96 & 8.96 & 8.96\\
    \hline
    I3.5 & 5 & 7.53 & 7.53 & 7.53\\
    \hline
  \end{tabular}
  \caption{\textbf{Holgura} de tareas de \textbf{implementación}}
  \label{Tab:HOLimp}
\end{table}

\begin{table}[!h]
\centering
  \begin{tabular}{|c||c||c|c|c|}
    \hline
     \textbf{Actividad} & \textbf{PERT} & \textbf{Total} & \textbf{Libre}
    & \textbf{Indep.}\\
    \hline \hline
    P4.1 & 4 & 1.3 & 1.3 & 1.3\\
    \hline
    P4.2 & 4.57 & 0.73 & 0.73 & 0.73\\
    \hline
    P4.3 & 5.3 & 0 & 0 & 0\\
    \hline
    P4.4 & 2 & 0 & 0 & 0\\
    \hline
  \end{tabular}
  \caption{\textbf{Holgura} de tareas de \textbf{pruebas}}
  \label{Tab:HOLpru}
\end{table}

\section{Determinación de los caminos críticos}
Tras realizar la tabla de holguras y el grafo observamos que el camino está
formado por las siguientes actividades:
\begin{itemize}
\item En la etapa de análisis: A1.1, A1.2, A1.4.1, A1.4.2, A1.4.3, A1.8,
  A4.3, A1.9 
\item En la etapa de diseño: D2.1, D2.3.1, D2.3.2, D2.3.3, D2.5, D2.9, D2.10 
\item En la etapa de implementación: I3.1, I3.2
\item En la etapa de pruebas: P4.3, P4.4
\end{itemize}

\chapter{Calendario}
Para la realización de este calendario seguiremos la técnica PERT, ésta
permite la determinación de las fechas, tanto de comienzo como de
finalización. El calendario definitivo muestra las fechas tempranas y tardías
de cada fecha de comienzo y finalización.


\section{Tabla de dependencias}
Las dependencias corresponden a la secuencia que seguimos para realizar las
secuencias. En la tabla \ref{Tab:tabdep}, de la página \pageref{Tab:tabdep},
que posteriormente adjuntamos podemos
ver las correspondientes a nuestro proyecto y que han sido definidas
atendiendo a la planificación.

\begin{table}[!h]
  \centering
  \begin{tabular}{|l||l|}
    \hline
    \textbf{Actividad} & \textbf{Precedentes}\\
    \hline \hline
    A1.1   & -                        \\
    A1.2   & A1.1                     \\
    A1.3   & A1.2                     \\
    A1.4.1 & A1.2                     \\
    A1.4.2 & A1.4.1                   \\
    A1.4.3 & A1.4.2                   \\
    A1.5   & A1.2                     \\
    A1.6   & A1.2                     \\
    A1.7   & A1.3, A1.4.3, A1.5, A1.6 \\
    A1.8   & A1.3, A1.4.3, A1.5, A1.6 \\
    A1.9   & A1.7, A1.8               \\
    \hline
    D2.1   & A1.9                     \\
    D2.2   & D2.1                     \\
    D2.3.1 & D2.1                     \\
    D2.3.2 & D2.3.1                   \\
    D2.3.3 & D2.3.2                   \\
    D2.4.1 & D2.1                     \\
    D2.4.2 & D2.4.1                   \\
    D2.4.3 & 2.4.2                    \\
    D2.5   & D2.2, D2.3.3, D2.4.3     \\
    D2.6   & D2.5                     \\
    D2.7   & D2.5                     \\
    D2.8   & D2.5                     \\
    D2.9   & D2.5                     \\
    D2.10  & D2.6, D2.7, D2.8, D2.9   \\
    \hline
    I3.1   & D2.10                    \\
    I3.2   & I3.1                     \\
    I3.3   & I3.1                     \\
    I3.4   & I3.1                     \\
    I3.5   & I3.1                     \\
    \hline
    P4.1   & I3.2, I3.3, I3.4, I3.5   \\
    P4.2   & I3.2, I3.3, I3.4, I3.5   \\
    P4.3   & I3.2, I3.3, I3.4, I3.5   \\
    P4.4   & P4.1, P4.2, P4.3, P4.4   \\
    \hline
    
  \end{tabular}
  \caption{Dependencias entre actividades} \label{Tab:tabdep}
\end{table}

\section{Grafo de actividades}
Empleando la tabla de dependencias, los tiempos Early y Late; y las holguras,
se han elaborado los siguientes grafos de actividades:

\begin{itemize}
\item En las figuras \ref{gELana}, \ref{gELdis}, \ref{gELimp} y \ref{gELpru},
  que se encuentran en las páginas \pageref{gELana} y \pageref{gELpru}, se
  pueden ver las etapas del grafo de actividades con los tiempos Early y
  Late.
\item En segundo lugar, en las figuras \ref{gCCana}, \ref{gCCdis},
  \ref{gCCimp} y \ref{gCCpru}, que están en las páginas \pageref{gELana} y
  \pageref{gELpru}, vemos las etapas del grafo de actividades sin tiempos
  Early y Late, pero en el que podemos ver señalado en rojo el camino crítico
  formado a partir de todas la actividades que tiene holgura total 0. En este
  caso, puede observarse que existe un único camino crítico.
\end{itemize}

\imagen{gELana}{17.5}{Grafo Early-Late de la etapa de análisis}{gELana}
\imagen{gELdis}{15.5}{Grafo Early-Late de la etapa de diseño}{gELdis}
\imagen{gELimp}{8}{Grafo Early-Late de la etapa de implementación}{gELimp}
\imagen{gELpru}{8}{Grafo Early-Late de la etapa de pruebas}{gELpru}
\imagen{gCCana}{13}
{Camino crítico del grafo de actividades de la etapa de análisis}{gCCana}
\imagen{gCCdis}{13}
{Camino crítico del grafo de actividades de la etapa de diseño}{gCCdis}
\imagen{gCCimp}{8}
{Camino crítico del grafo de actividades de la etapa de implementación}{gCCimp}
\imagen{gCCpru}{8}
{Camino crítico del grafo de actividades de la etapa de pruebas}{gCCpru}

\section{Tabla de tiempos de comienzo y finalizacion}
Las tablas \ref{Tab:CALana}, \ref{Tab:CALdis}, \ref{Tab:CALimp} y
\ref{Tab:CALpru}, de las páginas \pageref{Tab:CALana}, \pageref{Tab:CALdis} y
\pageref{Tab:CALpru}; reprensentan los resultados de aplicar las fórmulas de la
técnica PERT para cálculo de fechas, donde cada tabla se corresponde con las
actividades de cada etapa de elaboración del software. Cada fila de la
tabla contiene las actividades, la fecha de comienzo temprana, la fecha
de comienzo tardía, la fecha de finalización temprana y la fecha de
finalización tardía.

\begin{table}[!h]
  \centering
  \begin{tabular}{|c||p{5.3cm}||c|c|c|c|}
    \hline
    \textbf{Id} & \textbf{Tarea} & \textbf{FCE} & \textbf{FCL} &
    \textbf{FFE} & \textbf{FFL}\\
    \hline \hline
    1.1 & Definición del sistema & 0 & 0 & 4,57 & 4,57 \\
    \hline
    1.2 & Establecimiento de requisitos & 4,57 & 4,57 & 9,57  & 9,57 \\
    \hline
    1.3 & Identificación de subsistemas  & 9,57 & 15 & 13,57 & 19 \\
    \hline
    1.4.1 & Elaboración del modelo conceptual y lógica de datos & 9,57 &
    9,57 & 14,57 & 14,57 \\
    \hline
    1.4.2 & Normalización  & 14,57 & 14,57 & 17 & 17 \\
    \hline
    1.4.3 & Especificación de necesidades de carga inicial  & 17 & 17 & 19
    & 19 \\ 
    \hline
    1.5 & Elaboración del modelo de procesos & 9,57  & 13,22 & 15,35 & 19 \\
    \hline
    1.6 & Definición de interfaz de usuario  & 9,57 & 13,5 & 15,07 & 19 \\
    \hline
    1.7 & Análisis de consistencia y especificación de requisitos & 19 &
    19,94 & 22,55 & 23,22\\ 
    \hline
    1.8 & Especificación del plan de pruebas & 19 & 19 & 23,22 & 23,22\\
    \hline
    1.9 & Aprobación del análisis del SI & 23,22 & 23,22 & 24,22 & 24,22 \\
    \hline
  \end{tabular}
  \caption{\textbf{Calendario} de tareas de \textbf{análisis}}
  \label{Tab:CALana}
\end{table}
    
\begin{table}[!h]
  \centering
  \begin{tabular}{|c||p{5.3cm}||c|c|c|c|}
    \hline
    \textbf{Id} & \textbf{Tarea} & \textbf{FCE} & \textbf{FCL} &
    \textbf{FFE} & \textbf{FFL}\\
    \hline \hline
    2.1 & Definición de la arquitectura del sistema & 24,22 & 24,22 & 28,79
    & 28,79 \\ 
    \hline
    2.2 & Diseño de arquitectura de soporte & 28,79 & 40,77 & 33,36 & 45,34 \\
    \hline
    2.3.1 & Diseño de módulos del sistema  & 28,79 & 28,79 & 37,22 & 37,22 \\
    \hline
    2.3.2 & Diseño de comunicación entre módulos & 37,22 & 37,22 & 41,49  &
    41,49  \\ 
    \hline
    2.3.3 & Revisión de la interfaz de usuario  & 41,49 & 41,49 & 45,34 &
    45,34 \\ 
    \hline
    2.4.1 & Diseño del modelo físico de datos  & 28,79 & 33,65 & 32,36 &
    37,19 \\ 
    \hline
    2.4.2 & Especificación de los caminos de acceso a los datos & 32,36
    &37,19  & 36,79  & 41,62 \\ 
    \hline
    2.4.3 & Especificación de la distribución de datos  & 36,79 & 41,62
    &45,34  & 45,34 \\ 
    \hline
    2.5 & Verificación y aceptación de la arquitectura del sistema & 45,34
    & 45,34 & 46,67 & 46,77 \\ 
    \hline
    2.6 & Generación y especificación de construcción & 46,77 & 49,07 &
    52,5 & 54,8 \\ 
    \hline
    2.7 & Diseño de migración y carga inicial de datos & 46,77 & 51,93 &
    49,64 & 54,8 \\ 
    \hline
    2.8 & Especificación técnica del plan de prueba & 46,77 & 51,07 & 50,5
    & 54,8 \\ 
    \hline
    2.9 & Establecimiento de requisitos de implantación & 46,77 & 46,77
    &54,8 & 54,8 \\ 
    \hline
    2.10 & Aprobación de diseño y SI & 54,8 & 54,8 & 56,23 & 56,23\\
    \hline
  \end{tabular}
  \caption{\textbf{Calendario} de tareas de \textbf{diseño}}
  \label{Tab:CALdis}
\end{table}
    
\begin{table}[!h]
  \centering
  \begin{tabular}{|c||p{5.3cm}||c|c|c|c|}
    \hline
    \textbf{Id} & \textbf{Tarea} & \textbf{FCE} & \textbf{FCL} &
    \textbf{FFE} & \textbf{FFL}\\
    \hline \hline
    3.1 & Preparación del entorno de generación y construcción & 56,23 &
    56,23 & 59,36 & 59,36\\ 
    \hline
    3.2 & Generación del código de los componentes y los procedimientos
    &59,36 & 59,36 & 68,66 & 68,66\\ 
    \hline
    3.3 & Elaboración del manual de usuario & 56,23 & 58,16 & 66,83 & 68,66\\
    \hline
    3.4 & Definición de la formación de los usuarios finales & 56,23 &
    65,19 & 59,8 & 68,66\\ 
    \hline
    3.5 & Construcción de los componentes y procedimientos de carga inicial
    de datos & 56,23 & 63,76 & 61,23 & 68,66\\
    \hline
  \end{tabular}
  \caption{\textbf{Calendario} de tareas de \textbf{implementación}}
  \label{Tab:CALimp}
\end{table}
    
\begin{table}[!h]
  \centering
  \begin{tabular}{|c||p{5.3cm}||c|c|c|c|}
    \hline
    \textbf{Id} & \textbf{Tarea} & \textbf{FCE} & \textbf{FCL} &
    \textbf{FFE} & \textbf{FFL}\\
    \hline \hline
    4.1 & Ejecución de las pruebas unitarias & 68,66 & 69,96 & 72,66 & 73,96\\
    \hline
    4.2 & Ejecución de las pruebas de integración & 68,66 & 69,39 & 73,23 &
    73,96\\ 
    \hline
    4.3 & Ejecución de las pruebas del sistema & 68,66 & 68,66 & 73,96 &
    73,96 \\ 
    \hline
    4.4 & Aprobación del SI & 73,96 & 73,96 & 75,96 & 75,96\\
    \hline
  \end{tabular}
  \caption{\textbf{Calendario} de tareas de \textbf{pruebas}}
  \label{Tab:CALpru}
\end{table}


\section{Fechas finales}
Dadas las tablas \ref{Tab:CALana}, \ref{Tab:CALdis}, \ref{Tab:CALimp} y
\ref{Tab:CALpru} del apartado anterior, elaboramos el calendario definitivo
(ver figura\ref{Tab:calfec}) teniendo en cuenta que no se trabaja los fines
de semana y el horario laboral consta de 8 horas diarias.
 
\begin{table}[!h]
\centering
   \begin{tabular}{|c||b{2.4cm}<{\centering}|b{2.1cm}<{\centering}
       ||b{1.9cm}<{\centering}|b{1.6cm}<{\centering}|}
     \hline
     \textbf{Actividad} & \textbf{Cominezo más cercano} &\textbf{Comienzo más
     tardío} & \textbf{Fin más temprano}& \textbf{Fin más tardío}\\
     \hline \hline
     1.1   & 03/05/10 & 03/05/10 & 10/05/10 & 10/05/10 \\
     1.2   & 10/05/10 & 10/05/10 & 17/05/10 & 17/05/10 \\
     1.3   & 17/05/10 & 24/05/10 & 21/05/10 & 28/05/10 \\
     1.4.1 & 17/05/10 & 17/05/10 & 24/05/10 & 24/05/10 \\
     1.4.2 & 24/05/10 & 24/05/10 & 26/05/10 & 26/05/10 \\
     1.4.3 & 26/05/10 & 26/05/10 & 28/05/10 & 28/05/10 \\
     1.5   & 17/05/10 & 21/05/10 & 25/05/10 & 28/05/10 \\
     1.6   & 17/05/10 & 21/05/10 & 25/05/10 & 28/05/10 \\
     1.7   & 28/05/10 & 31/05/10 & 03/06/10 & 04/06/10 \\
     1.8   & 28/05/10 & 28/05/10 & 04/06/10 & 04/06/10 \\
     1.9   & 04/06/10 & 04/06/10 & 07/06/10 & 07/06/10 \\
     \hline
     2.1   & 07/06/10 & 07/06/10 & 11/06/10 & 11/06/10 \\
     2.2   & 11/06/10 & 29/06/10 & 18/06/10 & 06/07/10 \\
     2.3.1 & 11/06/10 & 11/06/10 & 24/06/10 & 24/06/10 \\
     2.3.2 & 24/06/10 & 24/06/10 & 30/06/10 & 30/06/10 \\
     2.3.3 & 30/06/10 & 30/06/10 & 06/07/10 & 06/07/10 \\
     2.4.1 & 11/06/10 & 18/06/10 & 17/06/10 & 24/06/10 \\
     2.4.2 & 17/06/10 & 24/06/10 & 23/06/10 & 30/06/10 \\
     2.4.3 & 23/06/10 & 30/06/10 & 06/07/10 & 06/07/10 \\
     2.5   & 06/07/10 & 06/07/10 & 07/07/10 & 07/06/10 \\
     2.6   & 07/07/10 & 12/07/10 & 15/07/10 & 19/07/10 \\
     2.7   & 07/07/10 & 14/07/10 & 12/07/10 & 19/07/10 \\
     2.8   & 07/07/10 & 14/07/10 & 13/07/10 & 19/07/10 \\
     2.9   & 07/07/10 & 07/07/10 & 19/07/10 & 19/07/10 \\
     2.10  & 19/07/10 & 19/07/10 & 21/07/10 & 21/07/10 \\
     \hline
     3.1   & 21/07/10 & 21/07/10 & 26/07/10 & 26/07/10 \\
     3.2   & 26/07/10 & 26/07/10 & 06/08/10 & 06/08/10 \\
     3.3   & 21/07/10 & 23/07/10 & 04/08/10 & 06/08/10 \\
     3.4   & 21/07/10 & 03/08/10 & 26/07/10 & 06/08/10 \\
     3.5   & 21/07/10 & 30/07/10 & 28/07/10 & 06/08/10 \\
     \hline
     4.1   & 06/08/10 & 09/08/10 & 12/08/10 & 13/08/10 \\
     4.2   & 06/08/10 & 09/08/10 & 13/08/10 & 13/08/10 \\
     4.3   & 06/08/10 & 06/08/10 & 13/08/10 & 13/08/10 \\
     4.4   & 13/08/10 & 13/08/10 & 17/08/10 & 17/08/10 \\
     \hline
  \end{tabular}
  \caption{Calendario de fechas definitivas} \label{Tab:calfec}
\end{table}

\end{document}
