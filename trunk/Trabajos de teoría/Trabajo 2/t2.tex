% Clase
\documentclass[11pt,a4paper,spanish,twoside]{report}

% Órdenes auxiliares
\input{inc/includes.tex}

% Encabezado y pie de página
\encabezado

\begin{document}

% Silabación extra
\hyphenation{
a-sig-na-tu-ras
au-to-ma-ti-za-rá
ca-tá-lo-go
ca-rre-ra
cons-truc-ción
co-rres-pon-de
diag-nos-tico
fi-na-li-za-ción
ge-ne-ra-ción
in-fe-rior
man-te-ni-mien-to
me-dian-te
per-so-nal
pro-ce-di-mien-tos
pro-por-cio-na-rá
pu-bli-ca-da
re-qui-si-tos
res-pecto
u-su-a-rios
vi-lla-rre-al
}


% Portada
\portada{Planificación y Gestión de\\Sistemas de Información}
{Trabajo 1}{Plan de Sistemas y Tecnologías de Información}
{Sergio de la Rubia García-Carpintero\\Miguel Millán Sánchez-Grande\\
  Luis Muñoz Villarreal\\Alicia Serrano Sánchez\\
  Juan Miguel Torres Triviño}{10 de Marzo de 2009}

% Licencia
\licencia{Sergio de la Rubia García-Carpintero, Miguel Millán Sánchez-Grande,
  Luis Muñoz Vi\-lla\-rre\-al, Alicia Serrano Sánchez, Juan Miguel Torres 
Triviño}

\chapter*{Ficha de trabajo}
\begin{description}
\item[Código] T1
\item[Fecha] 10 de Marzo de 2010
\item[Título]Plan de Sistemas y Tecnologías de Información
\end{description}

\begin{table}[!ht]
  \centering
  \begin{tabular}{lp{5cm}c}
    \multicolumn{3}{l}{\Large \textbf{Equipo} G4} \\ \\
    \multicolumn{1}{c}{\emph{Apellidos y nombre}} & 
    \multicolumn{1}{c}{\emph{Firma}} & \emph{Puntos} \\
    \hline \\
    de la Rubia García-Carpintero, Sergio & & 10 \\ \\
    Millán Sánchez-Grande, Miguel         & & 10 \\ \\
    Muñoz Villarreal, Luis                & & 10 \\ \\
    Serrano Sánchez, Alicia               & & 10 \\ \\
    Torres Triviño, Juan Miguel           & & 10 \\ \\
    \hline
  \end{tabular}
%  \caption{}\label{}
\end{table}

% Índices
\tableofcontents
% \listoffigures
% \listoftables

%% INICIO DEL DOCUMENTO %%%%%%%%%%%%%%%%%%%%%%%%%%%%%%%%%%%%%%%%%%%%%%%%%
\chapter*{Introducción}

A la hora de decidir la institución sobre la cual centrar nuestra 
investigación, empezamos analizando la posibilidad de buscar una empresa 
cercana geográficamente como podría haber sido el aeropuerto de Ciudad Real. 
Pero ante la posibilidad de encontrar dificultades a la hora de recopilar 
información nos decantamos por una entidad pública. Nuestra primera opción fue 
la ESI, pero buscando, encontramos mucha más información sobre la Universidad 
de Málaga, de ahí nuestra elección.

La universidad de Málaga es una universidad pública, joven y dinámica que ha 
apostado decididamente por la calidad en la docencia, la investigación y por el
servicio al alumno. Cuenta con más de 40.000 alumnos matriculados y 2.000 
investigadores. 

Para la realización del trabajo hemos usado como ayuda el estándar IEEE 1058, 
el PMBOK-2004, la metodología METRICA 3 y diferentes apuntes e informaciones 
recogidas a través de Internet como material de complemento.

\chapter{Integración del proyecto}
La gestión de la integración del proyecto incluye los procesos requeridos para 
asegurar que los distintos elementos del proyecto están perfectamente 
coordinados. Esto implica establecer soluciones equilibradas entre los 
objetivos y alternativas a estos para cumplir o exceder las necesidades  y 
expectativas de las partes interesadas.

Hay que recordar que se establecieron como responsables de proyecto a Dña. 
Adelaida de la Calle, jefa de proyecto y rectora de la universidad, y a Dña. 
María Valpuesta, vicerrectora de Innovación y Desarrollo Tecnológico.

Tras un exhaustivo estudio de los medios disponibles con los que cuenta la 
universidad para el tratamiento de la información, se concluyó que se 
necesitaba un software que facilite el almacenamiento, distribución y 
mantenimiento de dicha información.

\section{Desarrollo del plan de proyecto}
La propuesta del plan del proyecto ya cuenta con la aprobación oficial de la 
rectora y la vicerrectora de la UMA. En el caso de que surjan modificaciones
mínimas en el desarrollo del proyecto, estas estarán consensuadas con los
miembros de la UMA.

El actual plan de proyecto consta de los siguientes puntos:
\begin{itemize}
\item Descripción de la estrategia de gestión del proyecto.
\item Colección de entregables y defición de objetivos en el alcance del 
proyecto.
\item Descomposición estructurada de trabajos (WBS).
\item Calendario que incluye las fechas de comienzo, conclusión e hitos 
establecidos para el cumplimiento de plazos.
\item Estimación de costos, fechas y asignación de recursos hasta el nivel 
alcanzado en el WBS.
\item Documento de distribución de responsabilidades entre los implicados
en el proyecto de acuerdo al nivel de WBS.
\item Líneas de bases de realización de calendario y costes.
\item Registro del personal requerido e identificación de los individuos 
clave.
\item Descripción de posibles riesgos y especificación de las restricciones,
así como los planes de actuación previstos.
\item Planes de gestión secundarios, entre ellos los de alcance y calendario.
\item Futuras ampliaciones y mejoras del proyecto.
\item Salidas de otros procesos de planificación que no están incluidos en el 
plan de proyecto.
\item Información adicional o documentación generada durante el desarrollo del
plan del proyecto.  
\end{itemize}
\section{Ejecución del plan de proyecto}
En este proceso, el director y el equipo de gestión del proyecto deberán 
acordar distintos tipos de decisiones y compromisos, los cuales significarán 
un punto de partida para la ejecución del plan de proyecto. Además, el equipo 
de gestión del proyecto precisará de una autorización escrita por parte de los 
directivos para asegurar que el trabajo del proyecto será realizado en un 
tiempo adecuado y en una secuencia correcta.

Por otro lado, en este apartado del proyecto se deberán obtener los resultados 
de las actividades ejecutadas para llevar a cabo dicho proyecto. Esta 
información citada anteriormente será: 

\begin{itemize}
\item Objetivos completados y no completados.
\item Alcance del complimiento de las normas de calidad.
\item Costes incurridos del proyecto.
\end{itemize}

Estas informaciones se han de recopilar como parte del plan de ejecución del 
proyecto y alimentan el proceso de informe de la realización del proyecto.

\section{Control integrado de los cambios}
En el transcurso del desarrollo del proyecto software es probable que puedan 
surgir distintas desviaciones que pueden alterar el ritmo del proyecto. 
Existen desviaciones posibles respecto de:

\begin{itemize}
\item Los hitos completados.
\item El tamaño del software a realizar.
\item El esfuerzo que inicialmente se estimó.
\item El coste que nos ocasionará el proyecto.
\item El tiempo empleado en realizar el proyecto software.
\end{itemize}

Es por ello que si esto sucede deberán adoptarse ciertas acciones correctivas, 
entre las que destacan las siguientes:

\begin{itemize}
\item Añadir personal en el caso de que el previsto sea insuficiente.
\item Reducir el alcance o contenido de una entrega.
\item Alargar o retrasar el calendario previsto.
\end{itemize}

También puede suceder que existan otra serie de problemas como cambios en la 
decisión del cliente, incapacidad para controlar el progreso del proyecto, 
dificultad al valorar los riesgos, etc.

Por todo lo dicho anteriormente, se ve que se precisa de un mecanismo de 
control que regule los cambios en el proyecto y se encargue del seguimiento de 
diferentes actividades como las siguientes:

\begin{itemize}
\item Seguimiento de los costes frente al presupuesto previsto.
\item Seguimiento de los sucesos frente al calendario estimado.
\item Seguimiento de los aspectos ténicos críticos del proyecto.
\item Seguimiento del tamaño del software.
\item Seguimiento de hitos por completar.
\end{itemize}

Además, cualquier acción realizada para llevar a cabo la ejecución del 
proyecto y que esté relacionada con el plan del mismo debe ser registrada.

Otro aspecto importante a considerar es que de debe realimentar la 
documentación generada para posibles usos futuros, llevando un control de 
versiones actualizado para realizar más fácilmente la documentación del 
proyecto y de la base de datos.

\chapter{Alcance del proyecto}
El plazo estimado de entrega del proyecto se fija entre 3 y 4 meses a cumplir
en Julio de 2010, a partir del comienzo de la ejecución del proyecto. La
fecha de entrega del proyecto se establece con el objetivo de utilizar dicho
software para el curso 2010-2011. 
\section{Análisis del proyecto}
\subsection{Establecimiento de requisitos}
En esta etapa se recogen los requisitos necesarios que debe cumplir el
software teniendo en cuenta las necesidades actuales de la UMA y los usuarios
que interactúan con esta herramienta. 
El objetivo de esta actividad es obtener un catálogo detallado de los
requisitos, a partir del cual se pueda comprobar que los productos generados
en las actividades de modelización se ajustan a los requisitos del usuario.
En la definición de los requisitos hay que tener en cuenta las prioridades
que se deben tener en cuenta según los criterios de los usuarios
acerca de las funcionalidades a cubrir. 
\subsection{Identificación de subsistemas de análisis}
El objetivo de esta actividad es facilitar el análisis del sistema de
información llevando a cabo la descomposición del sistema en diversos
subsistemas. 
Cada subsistema corresponde con una de las etapas del proceso.
\subsection{Elaboración del modelo de datos}
\subsection{Elaboración del modelo de procesos}
\subsection{Definición de interfaces de usuario}
\subsection{Análisis de consistencia y especificación de requisitos}
Esta actividad se centra en garantizar la calidad de los distintos modelos
generados en el proceso de Análisis del Sistema de Información, y asegurar
que los usuarios y los analistas tienen el mismo concepto del sistema. Para
cumplir dicho objetivo, se llevan a cabo las acciones definidas abajo.  

\subsubsection{Verificación de los Modelos}
Con esta tarea se asegura calidad formal en los distintos modelos, conforme a 
la técnica seguida para la elaboración de cada producto.

\subsubsection{Análisis de Consistencia de los Modelos}
Se asegura la coherencia en los modelos, comprobando la ausencia de
ambigüedades y/o duplicación de información. 

\subsubsection{Validación de los Modelos}
Se validan los distintos modelos con los requisitos especificados para el
proyecto a través del catálogo de requisitos y a través de la validación
directa del usuario. 


\subsubsection{Elaboración de la Especificación de Requisitos Software (ERS)}
Esta tarea incluye la elaboración de la Especificación de Requisitos Software
(ERS),una vez validados los modelos. Incorpora la información necesaria para
la aprobación final del análisis, según el siguiente índice:
\begin{itemize}
\item Introdicción.
\item Ámbito y alcance.
\item Participantes.
\item Requitisitos del proyecto.
\item Visión general del proyecto.
\item Referencia de los productos a entregar.
\item Plan de acción.
\end{itemize}
 
\subsection{Especificación del plan de pruebas}
En esta actividad se inicia la definición del plan de pruebas, el cual sirve
como guía para la realización de las pruebas, y permite verificar que el
sistema de información cumple las necesidades establecidas por el usuario,
con las debidas garantías de calidad. 
\subsubsection{Definición del Alcance de las Pruebas}
En esta tarea especificaremos y justificaremos los niveles de pruebas a
realizar, así como el marco general de planificación de cada nivel de prueba,
según el siguiente esquema:
\begin{itemize}
\item Definición de los perfiles implicados en los distintos niveles de prueba.
\item Planificación temporal.
\item Criterios de verificación y aceptación de cada nivel de prueba.
\item Definición, generación y mantenimiento de verificaciones y casos de
  prueba.  
\item Análisis y evaluación de los resultados de cada nivel de prueba.
\item Productos a entregar como resultado de la ejecución de las pruebas.
\end{itemize}

\subsubsection{Definición de Requisitos del Entorno de Pruebas}
En esta sección se recopilan los requisitos relativos al entorno de pruebas,
completando el plan de pruebas. La realización de las pruebas aconseja
disponer de un entorno de pruebas separado del entorno de desarrollo y del
entorno de operación, garantizando cierta independencia y estabilidad en los
datos y elementos a probar, de modo que los resultados obtenidos sean
objetivamente representativos, punto especialmente crítico en pruebas de
rendimiento. Además, se inicia la definición de las especificaciones
necesarias para la correcta ejecución de las distintas pruebas del sistema de
información. Entre ellas podemos citar las siguientes:
\begin{itemize}
\item Requisitos básicos de hardware y software base: sistemas operativos,
  gestores de bases de datos, monitores de teleproceso, etc.
\item Requisitos de configuración de entorno: librerías, bases de datos,
  ficheros, procesos, comunicaciones, necesidades de almacenamiento,
  configuración de accesos, etc. 
\item Herramientas auxiliares. Por ejemplo, de extracción de juegos de
  ensayo, análisis de rendimiento y calidad, etc. 
\item Procedimientos para la realización de pruebas y migración de elementos
  entre entornos. 
\end{itemize}

\subsubsection{Definición de las pruebas de Aceptación del Sistema}
En esta tarea, como su propio nombre indica, se realiza la especificación
de las pruebas de aceptación del sistema. Los criterios de aceptación deben
ser definidos de forma clara, prestando especial atención a aspectos como: 
\begin{itemize}
\item Procesos críticos del sistema.
\item Rendimiento del sistema.
\item Seguridad.
\item Disponibilidad.
\end{itemize}

\subsection{Aprobación del análisis del sistema de información}
En esta tarea se realiza la presentación del análisis del sistema de
información a la Dirección, para la aprobación final del mismo.

\section{Diseño del proceso}
\subsection{Definición de la arquitectura del sistema}
\subsection{Diseño de la arquitectura de soporte}
\subsection{Diseño de la arquitectura de módulos del sistema}
\subsection{Diseño físico de datos}
\subsection{Verificación y aceptación de la arquitectura del sistema}
\subsection{Generación de especificaciones de construcción}
\subsection{Diseño de la migración y carga inicial de datos}
\subsection{Especificación técnica del plan de pruebas}
\subsection{Establecimiento de los requisitos de implantación}
\subsection{Presentación y aprobación del diseño del sistema de información}
\section{Construcción y prueba del proyecto}
\subsection{Preparación del entorno de generación y construcción}
\subsection{Generación del código de los componentes y procedimientos}
\subsection{Ejecución de las pruebas unitarias}
\subsection{Ejecución del las pruebas de integración}
\subsection{Ejecución de las pruebas del sistema}
\subsection{Elaboración de los manuales de usuario}
\subsection{Diseño de la migración y carga inicial de datos}
\subsection{Construcción de los componentes y procedimientos de carga inicial 
de datos}
\subsection{Aprobación del diseño del sistema de información}

\bibliographystyle{plain} 
\bibliography{t1}

\end{document}
