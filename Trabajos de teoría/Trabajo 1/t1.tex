% Clase
\documentclass[11pt,a4paper,spanish,twoside]{book}

% Órdenes auxiliares
% Español
\usepackage[spanish]{babel}
\usepackage{lmodern}
\usepackage[utf8]{inputenc}

% Imágenes
\usepackage[pdftex]{graphicx}
\usepackage{latexsym}
\usepackage{fancybox}

% Ruta para las imágenes
\graphicspath{{imagenes/}}

% Rotaciones
\usepackage[twoside]{rotating}

% Referencias
\usepackage[spanish]{varioref}
\usepackage[pdftex,colorlinks=true,linkcolor=black]{hyperref}

% Colores
\usepackage{color}
\usepackage{colortbl}

% Párrafos
\setlength{\parskip}{6pt}

% Entorno Listings para código fuente
\usepackage{listingsutf8}[2007/11/11]

\lstset{
  frame=Ltb, %forma del cuadro
  framerule=0pt, %ancho del borde
  aboveskip=0.5cm, %separación de los números de línea
  framexleftmargin=0.4cm, %margen externo izquierdo
  framesep=0pt,
  rulesep=.4pt,
  rulesepcolor=\color{black},
  % 
  stringstyle=\ttfamily,
  showstringspaces = false,
  basicstyle=\scriptsize,
  keywordstyle=\bfseries,
  % 
  numbers=left,
  numbersep=15pt,
  numberstyle=\tiny,
  numberfirstline= false,
  %
  inputencoding=utf8/latin1
}

% minimizar fragmentado de listados
\lstnewenvironment{listing}[1][]{
  \lstset{#1}\pagebreak[0]}{\pagebreak[0]
}


% Fancyhdr - Encabezados y pies de página
\usepackage{fancyhdr}
% Márgenes
\headsep=8mm
\footskip=14mm

% Fancy Header Style Options
\pagestyle{fancy}               % Sets fancy header and footer
\fancyfoot{}                    % Delete current footer settings

% Sin mayúsculas en la cabecera
\lhead{\nouppercase{\rightmark}}
\rhead{\nouppercase{\leftmark}}

% Capítulo
\renewcommand{\chaptermark}[1]{ % Lower Case Chapter marker style
  \markboth{\chaptername\ \thechapter.\ #1}{}} 

% Sección
\renewcommand{\sectionmark}[1]{ % Lower case Section marker style
  \markright{\thesection.\ #1}} 

% Página
\fancyhead[LE,RO]{\bfseries\thepage} % Page number (boldface) in left on even
                                     % pages and right on odd pages

% ------------------ Macro para encabezados y pies de página-------------------
%    Uso: \encabezado{pares(pag izquierda)}
% -----------------------------------------------------------------------------
\def\encabezado{
  \fancyhead[RE]{\bfseries\leftmark}     % In the right on even pages
  \fancyhead[LO]{\bfseries\rightmark}  % In the left on odd pages
  \renewcommand{\headrulewidth}{0.5pt} % Width of head rule
}
% -----------------------------------------------------------------------------


% ------------------ Macro para insertar una imagen ---------------------------
%    Uso: \imagen{nombreFichero}{Ancho(cm))}{Etiqueta}{Identificador}
% -----------------------------------------------------------------------------
\usepackage{float}
\usepackage{ifthen}
\def\imagen#1#2#3#4{
  \begin{figure}[H]
    \begin{center}
      \ifthenelse{\equal{#2}{}}
      {\includegraphics{#1}}{\resizebox{#2cm}{!}{\includegraphics{#1}}}
      \ifthenelse{\equal{#3}{}}{}{\caption{#3}}
      \label{#4}
    \end{center}
  \end{figure}
}
% -----------------------------------------------------------------------------


% ------------------ Macro para la portada ------------------------------------
%    Uso: \portada{asignatura}{titulo}{subtítulo}{autor}{fecha}
% -----------------------------------------------------------------------------
\def\portada#1#2#3#4#5{
  \thispagestyle{empty}
  \vspace*{-3.3cm}

  \begin{minipage}[t]{14cm}
    \begin{center}
      \includegraphics[scale=0.25]{logoesi}
  
      \vspace*{1.5cm}
      {\Large \textbf{Universidad de Castilla-La Mancha\\ 
          Escuela Superior de Informática}\\}
    
      \vspace*{1.2cm}
      {\huge \textbf{#1}\\}
    
      \vspace*{1.5cm}
      {\huge #2}\\{\Large #3\\}
    
      \vspace*{1.5cm}
      {\large #4\\}
      \vspace*{1.4cm}
      \large{#5}
    \end{center}
  \end{minipage}

  \newpage
  \vspace*{1cm}
  \thispagestyle{empty} 
  \newpage
}
% -----------------------------------------------------------------------------

% ------------------ Macro para la licencia -----------------------------------
%    Uso: \portada{asignatura}{titulo}{subtítulo}{autor}{fecha}
% -----------------------------------------------------------------------------
\def\licencia#1{
  \thispagestyle{empty}  % Suprime la numeración de esta página
  \vspace*{16cm}
  \begin{small}
    \copyright~ #1. Se permite la copia, distribución y/o 
    modificación de este documento bajo los términos de la licencia de
    documentación libre GNU, versión 1.1 o cualquier versión posterior publicada
    por la {\em Free Software Foundation}, sin secciones invariantes. Puede
    consultar esta licencia en http://www.gnu.org. \\[0.2cm]
    Este documento fue compuesto con \LaTeX{}. 
  \end{small}
  \newpage
  \thispagestyle{empty}
  \vspace*{0cm}
  \newpage
}
% -----------------------------------------------------------------------------

% Code for creating empty pages
% No headers on empty pages before new chapter
\makeatletter
\def\cleardoublepage{\clearpage\if@twoside \ifodd\c@page\else
    \hbox{}
    \thispagestyle{empty}
    \newpage
    \if@twocolumn\hbox{}\newpage\fi\fi\fi}
\makeatother \clearpage{\pagestyle{empty}\cleardoublepage}


% Encabezado y pie de página
\encabezado

\begin{document}

% Portada
\portada{Planificación y Gestión de Sistemas de Información}
{Trabajo 1}{Plan de Sistemas y Tecnologías de Información}
{Sergio de la Rubia García-Carpintero\\Miguel Millán Sánchez-Grande\\
  Luis Muñoz Villarreal\\Alicia Serrano Sánchez\\
  Juan Miguel Torres Triviño}{10 de Marzo de 2009}

% Licencia
\licencia{Sergio de la Rubia García-Carpintero, Miguel Millán Sánchez-Grande,
  Luis Muñoz Vi\-lla\-rre\-al, Alicia Serrano Sánchez, Juan Miguel Torres 
Triviño}

% Índices
\tableofcontents
% \listoffigures
% \listoftables

%% INICIO DEL DOCUMETO %%%%%%%%%%%%%%%%%%%%%%%%%%%%%%%%%%%%%%%%%%%%%%%%%
\chapter*{Ficha de trabajo}
\begin{description}
\item[Código] T1
\item[Fecha] 10 de Marzo de 2010
\item[Título]Plan de Sistemas y Tecnologías de Información
\end{description}

\begin{table}[!ht]
  \centering
  \begin{tabular}{lp{5cm}c}
    \multicolumn{3}{l}{\Large \textbf{Equipo} G4} \\ \\
%    \hline \hline
    \multicolumn{1}{c}{\emph{Apellidos y nombre}} & 
    \multicolumn{1}{c}{\emph{Firma}} & \emph{Puntos} \\
    \hline \\
    Mejorar y agilizar el trato con los estudiantes.         & Producción software \\ \\
    Facilitar el acceso a los datos.         & Producción software Formación del personal. \\ \\
    Adaptación al Espacio Europeo de Educación Superior.         & Formación del personal Control de Proyectos \\ \\
    Reducir el gasto en personal.         & Producción software Gestión de Personal Control de Proyectos\\ \\
    Facilitar la actualización de los planes de estudios.         & Producción software Formación del personal \\ \\
    \hline
  \end{tabular}
%  \caption{}\label{}
\end{table}

\chapter*{Introducción}
A la hora de decidir la institución sobre la cual centrar nuestra 
investigación, empezamos analizando la posibilidad de buscar una empresa 
cercana geográficamente como podría haber sido el aeropuerto de Ciudad Real. 
Pero ante la posibilidad de encontrar dificultades a la hora de recopilar 
información nos decantamos por una entidad pública. Nuestra primera opción fue
la ESI pero encontramos más facilidades para encontrar información sobre sus 
planes de futuro vía internet sobre la Universidad de Málaga, de ahí nuestra 
elección.

Universidad de Málaga es una universidad pública, joven y dinámica que ha 
apostado decididamente por la calidad en la docencia, la investigación y por el
servicio al alumno. Cuenta con más de 40.000 alumnos matriculados y 2.000 
investigadores. 

La historia de la Universidad de Málaga, en adelante UMA, empieza en 1968 con 
la creación de la Asociación de Amigos de la Universidad de Málaga. Esta 
asociación buscaba la creación de la universidad debido a las necesidades de la 
ciudad, ya que era la única ciudad de España con una población superior a 
300.000 habitantes que carecía de ella. La UMA fue finalmente 
fundada por decreto de 18 de agosto de 1972.

Desde su sitio en el Sur de Europa, cuenta con un personal docente e 
investigador altamente preparado y un entorno cultural idóneo para hacer mas 
cómoda y fructífera su vida académica.

Su oferta de titulaciones, tanto regladas como propias, es muy amplia; ya que 
cuenta con los siguentes centros
\begin{itemize}
  \item E.T.S. de Arquitectura
  \item E.T.S.I. de Telecomunicación
  \item E.T.S.I. Industrial
  \item E.T.S.I. Informática
  \item E.U. de Ciencias de la Salud
  \item E.U. de Estudios Empresariales
  \item Facultad de Estudios Sociales y del Trabajo
  \item E.U. de Turismo
  \item E.U. Politécnica
  \item Facultad de Bellas Artes
  \item Facultad de Ciencias
  \item Facultad de C.C. Educación
  \item Facultad de C.C. de la Comunicación
  \item Facultad de C.C. Económicas
  \item Facultad de Derecho
  \item Facultad de Filosofía y Letras
  \item Facultad de Medicina
  \item Facultad de Psicología
  \item E.U. de Enfermería (Dipu. Prov.)
  \item E.U. Enfermería (Ronda)
  \item E.U. Magisterio (Antequera)
  \item Existe un proyecto que se negocia con el ayuntamiento de Estepona para 
      crear centros en este municipio.
\end{itemize}

En la figura \ref{Mapa_Malaga} se puede ver la situación geográfica de los 
mismos

\imagen{PSIIntro_mapaUMA.pdf}{9}{Situación Geográfica de la UMA}{Mapa_Malaga}

\chapter{Inicio del plan de sistemas de información}
El objetivo de esta actividad es determinar la necesidad del Plan de Sistemas  
de Información y llevar a cabo el arranque formal del mismo, con el apoyo del
nivel más alto de la organización. Como resultado, se obtiene una descripción
general del Plan de Sistemas de Información que proporciona una definición
inicial del mismo, identificando los objetivos estratégicos a los que apoya,
así como el ámbito general de la organización al que afecta, lo que permite
implicar a las direcciones de las áreas afectadas por el Plan de Sistemas de
Información. 

Además, se identifican los factores críticos de éxito y los participantes en
el Plan de Sistemas de Información, nombrando a los máximos responsables.

\section{Análisis de la necesidad del PSI}
\subsection{Descripción general del PSI}\label{ss:1.1.1}
\subsubsection{Aprobación del inicio del PSI}
El siguiente Plan de Sistemas de Información tiene como finalidad crear un 
marco estratégico en el que la institución de enseñanza de la UMA pueda
mejorar y agilizar sus servicios de cara al personal administrativo, docente, 
investigador y estudiantil.

Para llevar a cabo nuestro Plan de Sistemas de Información de manera exitosa 
y conseguir que ayude a la UMA a mejorar sus servicios utilizaremos como 
herramientas las Tecnologías de la Información.

\section{Identificación del alcance del PSI}
En base a lo expuesto en la sección \vref{ss:1.1.1}, \emph{\nameref{ss:1.1.1}},
se desarrolla lo siguiente:

\subsection{Descripción general del PSI}\label{ss:1.2.1}
\subsubsection{Ámbito y objetivos del PSI}
Para el desarrollo del PSI se trabajará directamente con la sección de 
Desarrollo Tecnológico e Innovación. Esta sección afecta, de una u otra
forma, a todos los sectores de nuestra organización, por lo que el PSI se
implantará con objeto de mejorar los principales procesos internos de la
organización.

\subsubsection{Objetivos estratégicos del PSI}
Los principales objetivos estratégicos que aborda el PSI serán:
\begin{enumerate}
\item Prestar apoyo en materia de las TIC a todas las actividades relacionadas
con la investigación, la docencia y la gestión.
\item Dar soporte tecnológico a las nuevas demandas del sistema universitario.
\item Prestar servicios en la elaboración de contenidos audiovisuales.
\item Asegurar el acceso a los recursos bibliográficos y de información y 
promover su conservación, difusión e intercambio.
\end{enumerate}

\subsection{Factores críticos de éxito}
Los factores críticos de éxito serán la aceptación de
los usuarios al PSI y conseguir una rápida adaptación de éstos a las
Tecnologías de la Información.

\section{Determinación de responsables}
En base a lo expuesto en la sección \vref{ss:1.2.1}, \emph{\nameref{ss:1.2.1}},
se desarrolla lo siguiente:

\subsection{Description general del PSI}\label{ss:1.3.1}
\subsubsection{Responsables del PSI}
Para la correcta elaboración del PSI necesitaremos una persona, el Jefe de 
Proyecto, que será la figura principal. Como apoyo al Jefe de Proyecto 
tendremos un responsable. También tenemos la figura del coordinador del plan 
que será el encargado de ir dirigiendo su elaboración junto con su grupo de 
trabajo. Estas figuras serán representadas por las siguientes personas:
\begin{itemize}
\item[Jefa de Proyecto]
Dña. Adelaida de la Calle, rectora de la universidad.
\item[Responsable]
Dña. María Valpuesta, vicerrectorado de Innovación y Desarrollo Tecnológico.
\item[Coordinador]
D. Luis Muñoz Villarreal.
\item[Grupo de trabajo]
D. Sergio de la Rubia García-Carpintero.
D. Miguel Millán Sánchez-Grande.
Dña. Alicia Serrano Sánchez.
D. Juan Miguel Torres Triviño.
\end{itemize}

Dña. Adelaida de la Calle será la encargada de administrar el plan,
cumpliendo las tareas de seguimiento y control del mismo, revisión y 
estimación de resultados. El Coordinador se encargará de coordinar el
proyecto, de la gestión y resolución de incidencias que puedan aparecer 
durante el progreso del proyecto así como de la actualización de la 
planificación original.

Dña. María Valpuesta ofrecerá apoyo al Jefe de Proyecto y al Coordinador en
la actividad que necesite de tal apoyo. Las actividades donde tendrá que 
colaborar de forma activa el responsable serán determinadas por el Jefe de 
Proyecto. La buena coordinación entre el Responsable y el Jefe de Proyecto 
será esencial para el buen desarrollo del Plan.


\chapter{Definición y organización del PSI}
En esta actividad se detalla el alcance del plan, se organiza el equipo de
personas que lo va a llevar a cabo y se elabora un calendario de
ejecución. Todos los resultados o productos de esta actividad constituirán el
marco de actuación del proyecto más detallado que en PSI 1 en cuanto a
objetivos, procesos afectados, participantes, resultados y fechas de
entrega. 

\section{Especificación del ámbito y alcance}
En base a lo expuesto en la sección \vref{ss:1.3.1}, \emph{\nameref{ss:1.3.1}},
se desarrolla lo siguiente:

\subsection{Descripción general de procesos de la organización afectados}
\label{ss:2.1.1}
Dentro de la organización se han identificado los siguientes procesos afectados 
por el PSI:
\begin{enumerate}
  \item Contratación de personal.
  \item Implantación de los planes de estudios de cada carrera.
  \item Asignación del profesorado a las distintas asignaturas.
  \item Establececimiento de los horarios lectivos de cada facultad.
  \item Organización de aulas.
  \item Fijación de fechas, horarios y localizaciones de evaluación de
    asignaturas. 
\end{enumerate}

\subsection{Catálogo de objetivos del PSI} \label{ss:2.1.2}
Los objetivos que procuramos conseguir con el Plan de Sistemas de Información 
son los siguientes:
\begin{itemize}
  \item Mejorar y agilizar el trato con los estudiantes.
  \item Facilitar el acceso a los datos por parte de los integrantes de la 
    Universidad.
  \item Ayudar en la adaptación al Espacio Europeo de Educación Superior.
  \item Reducir el gasto en la contratación de personal para elaborar guías 
    docentes.
  \item Facilitar la actualización de los planes de estudio.
\end{itemize}

\section{Organización del PSI}
En base a la entrada externa \emph{Estructura organizativa} y a lo expuesto
en la sección \vref{ss:1.3.1}, \emph{\nameref{ss:1.3.1}}; en la sección
\vref{ss:2.1.1}, \emph{\nameref{ss:2.1.1}}; y en la sección \vref{ss:2.1.2},
\emph{\nameref{ss:2.1.2}}; se desarrolla lo siguiente:

\subsection{Catálogo de usuarios} \label{ss:2.2.1}
A lo largo del proceso de elaboración, implantación y gestión del Plan 
Estratégico en la Universidad de Málaga, intervienen diversos órganos expuestos 
en la figura \ref{Org_Uma}

\imagen{PSI2.2_organosUma.jpg}{9}{Órganos de la UMA}{Org_Uma}

\subsection{Equipos de trabajo} \label{ss:2.2.2}
Esta jerarquía se puede concentrar en tres grupos de decisiones:
\renewcommand{\labelenumi}{\alph{enumi}$)$ }
\begin{enumerate}
\item Políticos:
  \begin{itemize}
  \item Asamblea General del Plan Estratégico formado por el Claustro de la 
    Universidad de Málaga.
  \item Consejo de Gobierno.
  \end{itemize}
\item Ejecutivos:
  \begin{itemize}
  \item Presidenta del Plan Estratégico - Rectora.
  \item Vicepresidente del Plan Estratégico - Vicerrector de Calidad, 
    Planificación Estratégica y Responsabilidad Social. Coordinador General del 
    Plan Estratégico.
  \end{itemize}
\item De consulta: dónde entran las distintas comisiones:
  \begin{itemize}
  \item Comisión Estratégica de la Universidad de Málaga.
  \item Comisión de Planificación Estratégica de la Universidad de Málaga.
  \item Comisión Asesora Externa del Plan Estratégico de la Universidad de 
    Málaga.
  \item Comisión Asesora Interna del Plan Estratégico de la Universidad de 
    Málaga.
  \end{itemize}
\item Técnicos:
  \begin{itemize}
  \item Comisión Técnica del Plan Estratégico.
  \item Directora Técnica del Plan Estratégico.
  \item Oficina del Plan Estratégico.
  \end{itemize}
\end{enumerate}

\section{Definición del plan de trabajo}
En base a lo expuesto en la sección \vref{ss:2.2.2},
\emph{\nameref{ss:2.2.2}}; en la sección \vref{ss:1.3.1},
\emph{\nameref{ss:1.3.1}}; en la sección \vref{ss:2.1.2},
\emph{\nameref{ss:2.1.2}}; en la sección \vref{ss:2.1.1},
\emph{\nameref{ss:2.1.1}}; y en la sección \vref{ss:2.2.1},
\emph{\nameref{ss:2.2.1}}; se desarrolla lo siguiente:

\subsection{Plan de trabajo} \label{ss:2.3.1}
En cada uno de los cinco procesos tendremos un calendario de entrega de sus
respectivos sistemas. Este calendario ha sido elaborado teniendo en cuenta
las necesidades y las dimensiones de cada proceso dentro de la UMA, las
necesidades de la organización para tener funcionando cada sistema en el menor
tiempo posible y los recursos de los que se dispone para elaborar dichos
sistemas. 

\begin{itemize}
  \item Mejorar y agilizar el trato con los estudiantes.
  \item Facilitar el acceso a los datos por parte de los integrantes de la 
    Universidad.
  \item Ayudar en la adaptación al Espacio Europeo de Educación Superior.
  \item Reducir el gasto en la contratación de personal para elaborar guías 
    docentes.
  \item Facilitar la actualización de los planes de estudio.
\end{itemize}

Las fechas de los primeros sistemas se han establecido dos meses antes de que
acabe el año para que puedan ser utilizados desde el comienzo del ejercicio
2009. Se dejan esos dos meses para que una vez acabados el personal aprenda a
utilizarlo. El sistema para la inmobiliaria al ser un proceso nuevo en la
empresa se ha dejado más margen para su elaboración. 

\section{Comunicación del plan de trabajo}
En base a lo expuesto en la sección \vref{ss:2.3.1},
\emph{\nameref{ss:2.3.1}}; yen la sección \vref{ss:2.2.1},
\emph{\nameref{ss:2.2.1}}; se desarrolla lo siguiente:

\subsection{Plan de trabajo}
\subsubsection{Aceptación del Plan de Trabajo}

\chapter{Estudio de la información relevante}
El objetivo de esta actividad es recopilar y analizar todos los antecedentes
generales que puedan afectar a los procesos y a las unidades organizativas
implicadas en el Plan de Sistemas de Información, así como a los resultados
del mismo. Pueden ser de especial interés los estudios realizados con
anterioridad al Plan de Sistemas de Información, relativos a los sistemas de
información de su ámbito, o bien a su entorno tecnológico, cuyas conclusiones
deben ser conocidas por el equipo de trabajo del Plan de Sistemas de
Información. 

La información obtenida en esta actividad se tendrá en cuenta en
la elaboración de los requisitos.

\section{Selección y análisis de antecedentes}
En base a la entrada externa \emph{Información relevante} y a lo expuesto 
en la sección \vref{ss:2.1.1}, \emph{\nameref{ss:2.1.1}}; 
en la sección \vref{ss:2.1.2}, \emph{\nameref{ss:2.1.2}}; y
en la sección \vref{ss:2.2.1}, \emph{\nameref{ss:2.2.1}}; 
se desarrolla lo siguiente:

\subsection{Análisis de antecedentes} \label{ss:3.1.1}
En esta sección primeramente seleccionaremos aquellos antecedentes que son
relevantes para nuestro plan. Posteriormente analizaremos como pueden
afectar a este plan. Los antecedentes que nosotros consideraremos serán el
plan estratégico de sistemas de información actual, plan general
informático, etc. 
El plan establece renovación de equipos, aumento del grado de cualificación
del personal, renovación de herramientas de última generación para el
desarrollo de software, facilidad a la hora de obtener información, etc. 

La Universidad de Málaga se caracterizada por la excelencia en el proceso de
enseñanza-aprendizaje y es reconocida por la excelencia investigadora, la
transferencia de conocimiento y la promoción de la innovación. Una
Universidad que garantiza el desarrollo personal y profesional. Está
comprometida con su entorno tecnológico, medioambiental, económico, social,
histórico y cultural, y que incorpora en su actividad los principios de
responsabilidad social. 

En cuanto a proyectos más importantes establecidos en el plan y que se han
conseguido desde el momento en que se estableció dicho plan hasta la fecha
actual son: 
\begin{itemize}
\item Mejora de los procesos de desarrollo software a través de las
  TI. Incorporando herramientas que permitan un mejor control de riesgos,
  seguimiento de proyectos, etc.  

\item Renovación de equipos y herramientas para trabajar siempre con las
  últimas versiones disponibles en el mercado.

\item Proveer a la alta dirección de algún paquete informático, de desarrollo
  interno o adquirido, para realizar informes estadísticos, estudios de
  mercado, etc. 

\item Incorporar un Espacio Virtual de Aprendizaje intentando favorecer
  una mayor, eficiente y continua formación del personal.  
\end{itemize}
        
\section{Valoración de antecedentes} \label{3.2.1}
En base a la entrada externa \emph{Información relevante} y a lo expuesto 
en la sección \vref{ss:3.1.1}, \emph{\nameref{ss:3.1.1}}; y
se desarrolla lo siguiente:

\subsection{Catálogo de requisitos}
\subsubsection{Requisitos generales}
Según la información anterior podemos concluir que las líneas de trabajo de
la organización son según dicha información las siguientes: 
\begin{itemize}
\item Adecuar la oferta de estudios a las necesidades de formación de la
sociedad.
\item Conseguir una docencia de excelencia, potenciar el dominio de nuevas
tecnologías y mejorar los resultados académicos de los estudiantes.
\item Garantizar la formación tanto del personal de administración y servicios
como del docente e investigador.
\item Garantizar el acceso electrónico de los estudiantes a todos los servicios
públicos universitarios.
\item Resultados clave destinados a la consolidación de un sistema de
comunicación que contribuya a satisfacer la misión, la visión, los valores
y los objetivos de la institución y ayude a la consecución de resultados
sobresalientes por la Universidad y a reforzar la reputación de la
Universidad de Málaga.
\item Resultados en los usuarios que incrementen el nivel de satisfacción y
mejora de sus expectativas.
\end{itemize}
        
El análisis de la fase de utilización de las TI en la que se enmarca la 
UMA está en condición de llevarse a cabo. Este análisis sirve después
para enmarcar otros aspectos y decisiones que se tomen en este plan. Se puede
concluir que la empresa se enmarca en el nivel de formalización/control. Por
otra parte también se observa cómo cada vez son más las herramientas
informáticas y Sistemas de Información que se incorporan a la organización
con el fin de controlarla y gestionarla. 

\chapter{Identificación de requisitos}
El objetivo final de esta actividad va a ser la especificación de los
requisitos de información de la organización, así como obtener un modelo de
información que los complemente.

Para conseguir este objetivo, se estudia el proceso o procesos de la
organización incluidos en el ámbito del Plan de Sistemas de Información. Para
ello es necesario llevar a cabo sesiones de trabajo con los usuarios,
analizando cada proceso tal y como debería ser, y no según su situación
actual, ya que ésta puede estar condicionada por los sistemas de información
existentes. 

Del mismo modo, se identifican los requisitos de información, y
se elabora un modelo de información que represente las distintas entidades
implicadas en el proceso, así como las relaciones entre ellas. 

Por último, se clasifican los requisitos identificados según su prioridad,
con el objetivo de incorporarlos al catálogo de requisitos del Plan de
Sistemas de Información. 

\section{Estudio de los procesos del PSI}
En esta sección se muestran los procesos más significativos de la
organización en el ámbito del Plan de Sistemas e Información.  
\subsection{Contratación de personal}
Indica el proceso de selección e incorporación de nuevos empleados para la
realización de nuestro software.
\subsection{Implantación de los planes de estudios de cada carrera}
Proceso que se encarga de recoger la información de los planes de estudio de
cada una de las carreras asociadas a la Universidad de Málaga. Para ello, se
tienen los planes de estudio y junto con las directrices del Espacio Europeo
de Educación Superior (EEES) se realizan los nuevos Planes de Estudio
adaptados.

\subsection{Asignación del profesorado a distintas asignaturas}
Se trata de buscar profesores cualificados para impartir las distintas
asignaturas que están contenidas en los planes de estudio. Para ello, se
tiene una lista con las características de cada profesor. Tras la selección
de estos profesores, se realiza la asignación a las distintas asignaturas de
los planes de estudio.
 
\subsection{Establecimiento de horarios lectivos de cada facultad}
En este proceso tiene como entrada los planes de estudio de cada
facultad. Tras una serie de reuniones con cada facultad y siguiente sus
espectativas y directrices obtenemos el calendario final.

\subsection{Organización de aulas}
Las aulas disponibles de cada facultad se distribuyen acorde con los horarios
lectivos obtenidos en el proceso anterior. Este proceso nos proporciona una
lista de aulas asignadas. 

\subsection{Fijación de fechas, horarios y localizaciones de evaluación de
  asignaturas}
Para este proceso se necesitan los planes de estudio, así como el calendario
académico y las aulas que se necesitan para la evaluación de asignaturas. El
resultado de este proceso genera una lista que contiene el nombre de la
asignatura, fechas de evaluación, hora y emplazamiento. 

\section{Análisis de las necesidades de información}
Según los procesos identificados en la sección anterior se establecen las
necesidades de información: 

\subsection{Contratación de personal}
Información sobre posibles empleados. Establecer convenios con universidades,
etc. También sería bueno recoger toda la información relativa a procesos de
selección para poder recuperar a solicitantes que no fueron seleccionados, o
evitar a otros.  

\subsection{Implantación de los planes de estudios de cada carrera}
Se necesita un mecanismo eficaz por el cual se identifiquen los planes de
estudio públicos. También hay que mantener actualizado y estudiar qué
tecnologías se están demandando actualmente.  

\subsection{Asignación del profesorado a las distintas asignaturas}
La información necesaria para este proceso pasa por tener un buen equipo de
selección y por lo tanto la UMA debe tener información sobre capacidades,
experiencia, etc.
Se tiene una base de datos con los profesores y sus principales
características para agilizar la selección. 

\subsection{Establecimiento de horarios lectivos de cada facultad}
Es necesario tener una herramienta que elabore los horarios de clase
relacionando los créditos de las asignaturas, horas magistrales, conflictos
entre diferentes carreras de la misma facultad, etc.

\subsection{Organización de aulas}
Se tiene una base de datos con información sobre localización, estado,
capacidad, recursos, etc. 

\subsection{Fijación de  fechas, horarios y localizaciones de evaluación de
  asignaturas}
Utilización de una herramienta que complete de manera eficiente y automática
el calendario de evaluación de asignaturas teniendo en cuenta las aulas de
las facultades, asignaturas del plan de estudios, calendario académico,
tiempo de realización de la evaluación, etc.

\section{Catalogación de requisitos}
\begin{table}[!h]
\centering
  \begin{tabular}{clc}
    \textbf{Proceso} & \textbf{Requisitos} & \textbf{Prioridad} \\ \hline
    \hline \hline
     \multirow{2}{*}{Contratación de personal}
     & Servicios web para la inscripción de personal interesado en el trabajo
     & Alta\\
     & Herramienta para la selección del personal más cualificado de acuerdo
     a los objetivos de la UMA & Media-Alta\\

     \multirow{2}{*}{Implantación de los planes de estudio de cada
       carrera}
     & Servicios web para la obtención y consulta de los planes de estudio &
     Alta\\
     & Herramienta de administración de todas las características de cada
     asignatura & Media\\ 

     \multirow{3}{*}{Asignación del profesorado a las distintas asignaturas}
     & Base de datos de profesores que imparten clase en la UMA & Alta \\ 
     & Herramienta de búsqueda del profesorado, aplicando criterios de selección
     & Media\\ 
     & Herramienta de asignación de los distintos profesores a las
     asignaturas correspondientes & Media\\ 
     
     \multirow{2}{*}{Establecimiento de horarios lectivos de cada facultad}
     & Base de datos de los planes de estudio & Alta\\ 
     & Herramienta de elaboración de horarios según las características de
     las asignaturas & Media \\

     \multirow{3}{*}{Organización de aulas}
     & Base de datos de las aulas disponibles en cada facultad & Alta\\
     & Horario elaborado en el proceso anterior & Alta \\
     & Herramienta que realice automáticamente la asignación
     aulas-asignaturas teniendo en cuenta localización, estado, capacidad,
     recursos, etc & Media \\

     \multirow{4}{*}{Fijación de fechas, horarios y localizaciones de
       evaluación de asignaturas} 
     & Base de datos de los planes de estudio & Alta \\
     & Base de datos de las aulas disponibles & Alta \\
     & Calendario académico para conocer las fechas de los exámenes & Alta \\
     & Herramienta que genere de forma automática la organización de los
     exámenes, contando con la duración del examen, profesores implicados,
     facultades donde se imparte la asignatura, ... & Media-Alta \\
     
  \end{tabular}
\caption{Catálogo de requisitos de los procesos}
\end{table}


\chapter{Estudio de los sistemas de información actuales}
El objetivo de esta actividad es obtener una valoración de la situación
actual al margen de los requisitos del catálogo, apoyándose en criterios
relativos a facilidad de mantenimiento, documentación, flexibilidad,
facilidad de uso, etc. En esta actividad se debe tener en cuenta la opinión
de los usuarios, ya que aportarán elementos de valoración, como por ejemplo,
su nivel de satisfacción con cada sistema de información. 

Se seleccionan los
sistemas de información actuales que son objeto del análisis y se lleva a
cabo el estudio de los mismos con la profundidad y el detalle que se
determine conveniente en función de los objetivos definidos para el Plan de
Sistemas de Información. Este estudio permite, para cada sistema, determinar
sus carencias y valorarlos. Esta valoración se utilizará en la actividad
Diseño del Modelo de Sistemas de Información (PSI 6), donde se analizará la
cobertura de los sistemas de información actuales con respecto a los
requisitos.

\section{Alcance y objetivos del estudio de los sistemas de 
  información actuales} 
El dominio de sistemas de información de la a considerar quedará fijado por
aquellos procesos de la organización que afectan al Plan, así como por los 
objetivos definidos para y por este Plan de Sistemas de Información. En la 
sección siguiente veremos cual son esos sistemas de información y su estado y 
valoración.

\begin{table}[!ht]
  \centering
  \begin{tabular}{lp{5cm}c}
    \multicolumn{2}{l}{\Large \textbf{Tabla de relación entre Objetivos y Procesos de la organización} } \\ \\
%    \hline \hline
    \multicolumn{1}{c}{\emph{Objetivos}} & 
    \multicolumn{1}{c}{\emph{Procesos}} \\
    \hline \\
    Mejorar y agilizar el trato con los estudiantes.         & 10 \\ \\
    Facilitar el acceso a los dato.         & 10 \\ \\
    Adaptación al Espacio Europeo de Educación Superior.         & 10 \\ \\
    Reducir el gasto en personal.         & 10 \\ \\
    Facilitar la actualización de los planes de estudios.         & 10 \\ \\
    \hline
  \end{tabular}
%  \caption{}\label{}
\end{table}
\section{Análisis de los sistemas de información actuales}
Los Sistemas de Información de la organización actuales afectados con 
relevancia por el Plan son los que a continuación aparecen. Para cada sistema 
de información se recogen las características básicas, así como su utilidad.
\begin{itemize}
  \item La página web, es la forma que tiene la universidad de interactuar, de 
darse a conocer con la gente externa. También es una manera de comunicarse con
los usuarios propios.
  \item campusvirtual.uma.es es un lugar de encuentro de la comunidad 
universitaria de la UMA donde alumnado, profesorado y personal de 
administración y servicios pueden relacionarse sin que sean coincidentes en el 
espacio y en el tiempo. Las actividades se organizan en base a la herramienta 
de teleformación Campus Virtual, un entorno de virtual de enseñanza-aprendizaje 
desarrollado a partir de Moodle.
  \item Wifi Uma, es el nombre de la conexión a internet de la universidad, 
propiamente no variará sus funciones, pero al ser la base de las otras dos 
merece la pena mencionarlo. 

\end{itemize}
\section{Valoración de los sistemas de información actuales}

Una vez descritas las características de los principales sistemas de 
información actuales, se va a analizar sus problemas reales, las opiniones de 
los usuarios, etc. Finalizando el estudio con una valoración de cada sistema.

\begin{itemize}
  \item La web aunque desde el punto de la vista del usuario cumple con 
suficiencia su cometido ,también es cierto que se aprecian ciertas carencias en 
su contenido y sobre todo en su organización. No se ve necesaria ningún tipo de 
reforma a corto plazo.
  \item El campus virtual cumple con eficiencia sus funciones, además de 
obtener una alta valoración entre sus usuarios, dado que es un sistema más o 
menos "nuevo", está actualizado a las necesidades actuales del sistema. Por lo 
cual no necesita ninguna reforma.
  \item La Wifi Uma, el punto más débil de los que se han estudiado. No se ha 
sabido adaptar al incremento sucesivo de flujo de datos, y usuarios. No 
consigue una buena valoración de sus usuarios. 
\chapter{Diseño del modelo de sistemas de información}
El objetivo de esta actividad es identificar y definir los sistemas de
información que van a dar soporte a los procesos de la organización afectados
por el Plan de Sistemas de Información. Para ello, en primer lugar, se
analiza la cobertura que los sistemas de información actuales dan a los
requisitos recogidos en el catálogo elaborado en las actividades Estudio de
la Información Relevante (PSI 3) e Identificación de Requisitos (PSI 4). Esto
permitirá efectuar un diagnóstico de la situación actual, a partir del cual
se seleccionan los sistemas de información actuales considerados válidos,
identificando las mejoras a realizar en los mismos. 

Por último, se definen los nuevos sistemas de información necesarios para
cubrir los requisitos y funciones de los procesos no soportados por los
sistemas actuales seleccionados. 

Teniendo en cuenta los resultados anteriores, se elabora el modelo de
sistemas de información válido para dar soporte a los procesos de la
organización incluidos en el ámbito del Plan de Sistemas de Información. 

\section{Diagnóstico de la situación actual}
Analizados los diferentes procesos y los sistemas de información actuales que les dan cobertura, se concluye lo siguiente:

\subsection{Contratación de personal}
\begin{itemize}
\item Espacio en la web de la UMA para la solicitud de empleo $\to$ Si.
\item Herramienta para gestionar los procesos de selección de personal $\to$ No.
\end{itemize}

\subsection{Implamantación de los planes de estudios de cada carrera}
\begin{itemize}
\item Base de datos de los planes de estudio $\to$ Si, pero ésta quedará obsoleta cuando esté totalmente implantado el EEES, por lo que se necesitará una nueva para almacenar los datos de las nuevas titulaciones.
\item Servicios web para la obtención y consulta de los planes de estudio $\to$ Si, aunque debe actualizarse.
\item Herramientas de administración de todas las características de cada asignatura $\to$ No.
\end{itemize}

\subsection{Asignación del profesorado a las distintas asignaturas}
\begin{itemize}
\item Base de datos de profesores que imparten clase en la UMA $\to$ Si, pero se deben ampliar los contenidos existentes para dotarla de una mayor funcionalidad.
\item Herramientas de búsqueda de profesorado, aplicando criterios de selección $\to$ Si, aunque debe mejorarse el motor de búsqueda actual.
\item Herramientas de asignación de los distintos profesores a las asignaturas correspondientes $\to$ No.
\end{itemize}

\subsection{Establecimiento de los horarios lectivos de cada facultad}
\begin{itemize}
\item Base de datos de los planes de estudio $\to$ Si, pero ésta quedará obsoleta cuando esté totalmente implantado el EEES, por lo que se necesitará una nueva para almacenar los datos de las nuevas titulaciones.
\item Herramientas de elaboración de horarios según las características de las asignaturas $\to$ No.
\end{itemize}

\subsection{Organización de aulas}
\begin{itemize}
\item Base de datos de las aulas disponibles de cada facultad $\to$ Si, pero se deben ampliar los contenidos existentes para dotarla de una mayor funcionalidad.
\item Horario elaborado en el proceso anterior $\to$ Si.
\item  Herramienta que realice automáticamente la asignación
    aulas-asignaturas teniendo en cuenta localización, estado, capacidad,
    recursos, etc $\to$ No.
\end{itemize}

\subsection{Fijación de fechas, horarios y localizaciones de evaluación de asignaturas}
\begin{itemize}
\item Base de datos de los planes de estudio $\to$ Si, pero ésta quedará obsoleta cuando esté totalmente implantado el EEES, por lo que se necesitará una nueva para almacenar los datos de las nuevas titulaciones. 
\item Base de datos de las aulas disponibles $\to$ Si, pero se deben ampliar los contenidos existentes para dotarla de una mayor funcionalidad. 
\item Calendario académico para conocer las fechas de los exámenes $\to$ Si.
\item Herramienta que genere de forma automática la organización de los
    exámenes, contando con la duración del examen, profesores implicados,
    facultades donde se imparte la asignatura\... $\to$ No.
\end{itemize}

\section{Definición del modelo de sistemas de información}
A continuación se detalla la funcionalidad principal de cada proceso:
\begin{description}
\item[Contratación de personal] A través de la página web se publicitarán las plazas demandadas para la realización del proyecto. Un sistema de selección automatizado nos ayudará a filtrar los candidatos más acordes con el puesto.
\item[Implamantación de los planes de estudios de cada carrera] Se proporcionará un mecanismo que facilite la transición entre planes de estudio. Basándose en las bases de datos del plan antiguo y siguiendo con las directrices del plan de EEES, se crearán nuevas bases de datos que alberguen las características del nuevo plan.
\item[Asignación del profesorado a las distintas asignaturas] Un nuevo motor de búsqueda de profesores permitirá una rápida visión de su situación. Además, se contará con la funcionalidad de la modificación de sus datos, como la asignatura o asignaturas que tiene asignadas.
\item[Establecimiento de los horarios lectivos de cada facultad] Mediante un calendario y el plan actual, además de otros requisitos opcionales, se generarán unos horarios para cada facultad.
\item[Organización de aulas] A través de los horarios generados automáticamente y teniendo en cuenta las posibilidades que ofrecen las instalaciones de cada facultad, se generará una relación de asignaciones entre aulas y asignaturas, dentro de los horarios.
\item[Fijación de fechas, horarios y localizaciones de evaluación de asignaturas] A través del plan establecido, y utilizando las asignaciones de aulas y horarios generados por los procesos anteriormente descritos, se automatizará la asignación de exámenes dentro de los horarios y las aulas disponibles, siguiendo los periodos descritos en el plan.
\end{description}
 
\chapter{Definición de la arquitectura tecnológica}
En esta actividad se propone una arquitectura tecnológica que de soporte al
modelo de información y de sistemas de información incluyendo, si es
necesario, opciones. Para esta actividad se tienen en cuenta especialmente
los requisitos de carácter tecnológico, aunque es necesario considerar el
catálogo completo de requisitos para entender las necesidades de los procesos
y proponer los entornos tecnológicos que mejor se adapten a las mismas. 

\section{Identificación de las necesidades de infraestructura tecnológica}

\section{Selección de la arquitectura tecnológica}


\chapter{Definición del plan de acción}
En el Plan de Acción, que se elabora en esta actividad, se definen los
proyectos y acciones a llevar a cabo para la implantación de los modelos de
información y de sistemas de información, determinados en las actividades
Identificación de Requisitos (PSI 4) y Diseño del Modelo de Sistemas de
Información (PSI 6), con la arquitectura tecnológica propuesta en la
actividad Definición de la Arquitectura Tecnológica (PSI 7). El conjunto de
estos tres modelos constituye la arquitectura de información. 

Dentro del Plan
de Acción se incluye un calendario de proyectos, con posibles alternativas, y
una estimación de recursos, cuyo detalle será mayor para los más
inmediatos. Para la elaboración del calendario se tienen que analizar las
distintas variables que afecten a la prioridad de cada proyecto y sistema de
información. El orden definitivo de los proyectos y acciones debe pactarse
con los usuarios, para llegar a una solución de compromiso que resulte la
mejor posible para la organización. 

Por último, se propone un plan de
mantenimiento para el control y seguimiento de la ejecución de los proyectos,
así como para la actualización de los productos finales del Plan de Sistemas
de Información. 

\section{Definición de proyectos a realizar}

\section{Elaboración del plan de mantenimiento del PSI}


\bibliographystyle{plain} 
\bibliography{t1}

\end{document}
