% Clase
\documentclass[12pt,a4paper,spanish,twoside]{book}

% Órdenes auxiliares
% Español
\usepackage[spanish]{babel}
\usepackage{lmodern}
\usepackage[utf8]{inputenc}

% Imágenes
\usepackage[pdftex]{graphicx}
\usepackage{latexsym}
\usepackage{fancybox}

% Ruta para las imágenes
\graphicspath{{imagenes/}}

% Rotaciones
\usepackage[twoside]{rotating}

% Referencias
\usepackage[spanish]{varioref}
\usepackage[pdftex,colorlinks=true,linkcolor=black]{hyperref}

% Colores
\usepackage{color}
\usepackage{colortbl}

% Párrafos
\setlength{\parskip}{6pt}

% Entorno Listings para código fuente
\usepackage{listingsutf8}[2007/11/11]

\lstset{
  frame=Ltb, %forma del cuadro
  framerule=0pt, %ancho del borde
  aboveskip=0.5cm, %separación de los números de línea
  framexleftmargin=0.4cm, %margen externo izquierdo
  framesep=0pt,
  rulesep=.4pt,
  rulesepcolor=\color{black},
  % 
  stringstyle=\ttfamily,
  showstringspaces = false,
  basicstyle=\scriptsize,
  keywordstyle=\bfseries,
  % 
  numbers=left,
  numbersep=15pt,
  numberstyle=\tiny,
  numberfirstline= false,
  %
  inputencoding=utf8/latin1
}

% minimizar fragmentado de listados
\lstnewenvironment{listing}[1][]{
  \lstset{#1}\pagebreak[0]}{\pagebreak[0]
}


% Fancyhdr - Encabezados y pies de página
\usepackage{fancyhdr}
% Márgenes
\headsep=8mm
\footskip=14mm

% Fancy Header Style Options
\pagestyle{fancy}               % Sets fancy header and footer
\fancyfoot{}                    % Delete current footer settings

% Sin mayúsculas en la cabecera
\lhead{\nouppercase{\rightmark}}
\rhead{\nouppercase{\leftmark}}

% Capítulo
\renewcommand{\chaptermark}[1]{ % Lower Case Chapter marker style
  \markboth{\chaptername\ \thechapter.\ #1}{}} 

% Sección
\renewcommand{\sectionmark}[1]{ % Lower case Section marker style
  \markright{\thesection.\ #1}} 

% Página
\fancyhead[LE,RO]{\bfseries\thepage} % Page number (boldface) in left on even
                                     % pages and right on odd pages

% ------------------ Macro para encabezados y pies de página-------------------
%    Uso: \encabezado{pares(pag izquierda)}
% -----------------------------------------------------------------------------
\def\encabezado{
  \fancyhead[RE]{\bfseries\leftmark}     % In the right on even pages
  \fancyhead[LO]{\bfseries\rightmark}  % In the left on odd pages
  \renewcommand{\headrulewidth}{0.5pt} % Width of head rule
}
% -----------------------------------------------------------------------------


% ------------------ Macro para insertar una imagen ---------------------------
%    Uso: \imagen{nombreFichero}{Ancho(cm))}{Etiqueta}{Identificador}
% -----------------------------------------------------------------------------
\usepackage{float}
\usepackage{ifthen}
\def\imagen#1#2#3#4{
  \begin{figure}[H]
    \begin{center}
      \ifthenelse{\equal{#2}{}}
      {\includegraphics{#1}}{\resizebox{#2cm}{!}{\includegraphics{#1}}}
      \ifthenelse{\equal{#3}{}}{}{\caption{#3}}
      \label{#4}
    \end{center}
  \end{figure}
}
% -----------------------------------------------------------------------------


% ------------------ Macro para la portada ------------------------------------
%    Uso: \portada{asignatura}{titulo}{subtítulo}{autor}{fecha}
% -----------------------------------------------------------------------------
\def\portada#1#2#3#4#5{
  \thispagestyle{empty}
  \vspace*{-3.3cm}

  \begin{minipage}[t]{14cm}
    \begin{center}
      \includegraphics[scale=0.25]{logoesi}
  
      \vspace*{1.5cm}
      {\Large \textbf{Universidad de Castilla-La Mancha\\ 
          Escuela Superior de Informática}\\}
    
      \vspace*{1.2cm}
      {\huge \textbf{#1}\\}
    
      \vspace*{1.5cm}
      {\huge #2}\\{\Large #3\\}
    
      \vspace*{1.5cm}
      {\large #4\\}
      \vspace*{1.4cm}
      \large{#5}
    \end{center}
  \end{minipage}

  \newpage
  \vspace*{1cm}
  \thispagestyle{empty} 
  \newpage
}
% -----------------------------------------------------------------------------

% ------------------ Macro para la licencia -----------------------------------
%    Uso: \portada{asignatura}{titulo}{subtítulo}{autor}{fecha}
% -----------------------------------------------------------------------------
\def\licencia#1{
  \thispagestyle{empty}  % Suprime la numeración de esta página
  \vspace*{16cm}
  \begin{small}
    \copyright~ #1. Se permite la copia, distribución y/o 
    modificación de este documento bajo los términos de la licencia de
    documentación libre GNU, versión 1.1 o cualquier versión posterior publicada
    por la {\em Free Software Foundation}, sin secciones invariantes. Puede
    consultar esta licencia en http://www.gnu.org. \\[0.2cm]
    Este documento fue compuesto con \LaTeX{}. 
  \end{small}
  \newpage
  \thispagestyle{empty}
  \vspace*{0cm}
  \newpage
}
% -----------------------------------------------------------------------------

% Code for creating empty pages
% No headers on empty pages before new chapter
\makeatletter
\def\cleardoublepage{\clearpage\if@twoside \ifodd\c@page\else
    \hbox{}
    \thispagestyle{empty}
    \newpage
    \if@twocolumn\hbox{}\newpage\fi\fi\fi}
\makeatother \clearpage{\pagestyle{empty}\cleardoublepage}


% Encabezado y pie de página
\encabezado

\begin{document}

% Portada
\portada{Planificación y Gestión de Sistemas de Información}
{Trabajo 1}{Plan de Sistemas y Tecnologías de Información}
{Sergio de la Rubia García-Carpintero\\Miguel Millán Sánchez-Grande\\
  Luis Muñoz Villarreal\\Alicia Serrano Sánchez\\
  Juan Miguel Torres Triviño}{10 de Marzo de 2009}

% Licencia
\licencia{Sergio de la Rubia García-Carpintero, Miguel Millán Sánchez-Grande,
  Luis Muñoz Vi\-lla\-rre\-al, Alicia Serrano Sánchez, Juan Miguel Torres 
Triviño}

% Índices
\tableofcontents
% \listoffigures
% \listoftables

%% INICIO DEL DOCUMETO %%%%%%%%%%%%%%%%%%%%%%%%%%%%%%%%%%%%%%%%%%%%%%%%%
\chapter*{Ficha de trabajo}
\begin{description}
\item[Código] T1
\item[Fecha] 10 de Marzo de 2010
\item[Título]Plan de Sistemas y Tecnologías de Información
\end{description}

\begin{table}[!ht]
  \centering
  \begin{tabular}{lp{5cm}c}
    \multicolumn{3}{l}{\Large \textbf{Equipo} G4} \\ \\
%    \hline \hline
    \multicolumn{1}{c}{\emph{Apellidos y nombre}} & 
    \multicolumn{1}{c}{\emph{Firma}} & \emph{Puntos} \\
    \hline \\
    de la Rubia García-Carpintero, Sergio &  & 10 \\ \\
    Millán Sánchez-Grande, Miguel         &  & 10 \\ \\
    Muñoz Villarreal, Luis                &  & 10 \\ \\
    Serrano Sánchez, Alicia               &  & 10 \\ \\
    Torres Triviño, Juan Miguel           &  & 10 \\ \\
    \hline
  \end{tabular}
%  \caption{}\label{}
\end{table}

\chapter*{Introducción}

\chapter{Inicio del plan de sistemas de información}
El objetivo de esta actividad es determinar la necesidad del Plan de Sistemas  
de Información y llevar a cabo el arranque formal del mismo, con el apoyo del
nivel más alto de la organización. Como resultado, se obtiene una descripción
general del Plan de Sistemas de Información que proporciona una definición
inicial del mismo, identificando los objetivos estratégicos a los que apoya,
así como el ámbito general de la organización al que afecta, lo que permite
implicar a las direcciones de las áreas afectadas por el Plan de Sistemas de
Información. 

Además, se identifican los factores críticos de éxito y los participantes en
el Plan de Sistemas de Información, nombrando a los máximos responsables.

\section{Análisis de la necesidad del PSI}
\subsection{Descripción general del PSI}
El siguiente Plan de Sistemas de Información tiene como finalidad crear un 
marco estratégico en el que la institución de enseñanza de la UMA pueda
mejorar y agilizar sus servicios de cara al personal administrativo, docente, 
investigador y estudiantil.

Para llevar a cabo nuestro Plan de Sistemas de Información de manera exitosa 
y conseguir que ayude a la UMA a mejorar sus servicios utilizaremos como 
herramientas las Tecnologías de la Información.

\section{Identificación del alcance del PSI}
\subsection{Ámbito y objetivos del PSI}
Para el desarrollo del PSI se trabajará directamente con la sección de 
Desarrollo Tecnológico e Innovación. Esta sección afecta, de una u otra
forma, a todos los sectores de nuestra organización, por lo que el PSI se
implantará con objeto de mejorar los principales procesos internos de la
organización.

\subsection{Objetivos estratégicos del PSI}
Los principales objetivos estratégicos que aborda el PSI serán:
\begin{enumerate}
\item Prestar apoyo en materia de las TIC a todas las actividades relacionadas
con la investigación, la docencia y la gestión.
\item Dar soporte tecnológico a las nuevas demandas del sistema universitario.
\item Prestar servicios en la elaboración de contenidos audiovisuales.
\item Asegurar el acceso a los recursos bibliográficos y de información y 
promover su conservación, difusión e intercambio.
\end

\subsection{Factores críticos de riesgo}
Uno de los factores críticos de riesgo será conseguir evitar el rechazo de
los usuarios al PSI y conseguir una rápida adaptación de éstos a las
Tecnologías de la Información.

\section{Determinación de responsables}
\subsection{Responsables del PSI}
Para la correcta elaboración del PSI necesitaremos una persona, el Jefe de 
Proyecto, que será la figura principal. Como apoyo al Jefe de Proyecto 
tendremos un responsable. También tenemos la figura del coordinador del plan 
que será el encargado de ir dirigiendo su elaboración junto con su grupo de 
trabajo. Estas figuras serán representadas por las siguientes personas:
\begin{itemize}
\item[Jefa de Proyecto]
Dña. Adelaida de la Calle, rectora de la universidad.
\item[Responsable]
Dña. María Valpuesta, vicerrectorado de Innovación y Desarrollo Tecnológico.
\item[Coordinador]
D. Luis Muñoz.
\item[Grupo de trabajo]
D. Sergio de la Rubia.
D. Miguel Millán.
Dña. Alicia Serrano.
D. Juan Miguel Torres.
\end

Dña. Adelaida de la Calle será la encargada de administrar el plan,
cumpliendo las tareas de seguimiento y control del mismo, revisión y 
estimación de resultados. El Coordinador se encargará de coordinar el
proyecto, de la gestión y resolución de incidencias que puedan aparecer 
durante el progreso del proyecto así como de la actualización de la 
planificación original.

Dña. María Valpuesta ofrecerá apoyo al Jefe de Proyecto y al Coordinador en
la actividad que necesite de tal apoyo. Las actividades donde tendrá que 
colaborar de forma activa el responsable serán determinadas por el Jefe de 
Proyecto. La buena coordinación entre el Responsable y el Jefe de Proyecto 
será esencial para el buen desarrollo del Plan.

\chapter{Definición y organización del PSI}
En esta actividad se detalla el alcance del plan, se organiza el equipo de
personas que lo va a llevar a cabo y se elabora un calendario de
ejecución. Todos los resultados o productos de esta actividad constituirán el
marco de actuación del proyecto más detallado que en PSI 1 en cuanto a
objetivos, procesos afectados, participantes, resultados y fechas de
entrega. 

\section{Especificación del ámbito y alcance}

\section{Organización del PSI}

\section{Definición del plan de trabajo}

\section{Comunicación del plan de trabajo}


\chapter{Estudio de la información relevante}
El objetivo de esta actividad es recopilar y analizar todos los antecedentes
generales que puedan afectar a los procesos y a las unidades organizativas
implicadas en el Plan de Sistemas de Información, así como a los resultados
del mismo. Pueden ser de especial interés los estudios realizados con
anterioridad al Plan de Sistemas de Información, relativos a los sistemas de
información de su ámbito, o bien a su entorno tecnológico, cuyas conclusiones
deben ser conocidas por el equipo de trabajo del Plan de Sistemas de
Información. 

La información obtenida en esta actividad se tendrá en cuenta en
la elaboración de los requisitos.

\section{Selección y análisis de antecedentes}

\section{Valoración de antecedentes}


\chapter{Identificación de requisitos}
El objetivo final de esta actividad va a ser la especificación de los
requisitos de información de la organización, así como obtener un modelo de
información que los complemente.

Para conseguir este objetivo, se estudia el proceso o procesos de la
organización incluidos en el ámbito del Plan de Sistemas de Información. Para
ello es necesario llevar a cabo sesiones de trabajo con los usuarios,
analizando cada proceso tal y como debería ser, y no según su situación
actual, ya que ésta puede estar condicionada por los sistemas de información
existentes. 

Del mismo modo, se identifican los requisitos de información, y
se elabora un modelo de información que represente las distintas entidades
implicadas en el proceso, así como las relaciones entre ellas. 

Por último, se clasifican los requisitos identificados según su prioridad,
con el objetivo de incorporarlos al catálogo de requisitos del Plan de
Sistemas de Información. 

\section{Estudio de los procesos del PSI}

\section{Análisis de las necesidades de información}

\section{Catalogación de requisitos}


\chapter{Estudio de los sistemas de información actuales}
El objetivo de esta actividad es obtener una valoración de la situación
actual al margen de los requisitos del catálogo, apoyándose en criterios
relativos a facilidad de mantenimiento, documentación, flexibilidad,
facilidad de uso, etc. En esta actividad se debe tener en cuenta la opinión
de los usuarios, ya que aportarán elementos de valoración, como por ejemplo,
su nivel de satisfacción con cada sistema de información. 

Se seleccionan los
sistemas de información actuales que son objeto del análisis y se lleva a
cabo el estudio de los mismos con la profundidad y el detalle que se
determine conveniente en función de los objetivos definidos para el Plan de
Sistemas de Información. Este estudio permite, para cada sistema, determinar
sus carencias y valorarlos. Esta valoración se utilizará en la actividad
Diseño del Modelo de Sistemas de Información (PSI 6), donde se analizará la
cobertura de los sistemas de información actuales con respecto a los
requisitos.

\section{Alcance y objetivos del estudio de los sistemas de 
  información actuales} 

\section{Análisis de los sistemas de información actuales}

\section{Valoración de los sistemas de información actuales}


\chapter{Diseño del modelo de sistemas de información}
El objetivo de esta actividad es identificar y definir los sistemas de
información que van a dar soporte a los procesos de la organización afectados
por el Plan de Sistemas de Información. Para ello, en primer lugar, se
analiza la cobertura que los sistemas de información actuales dan a los
requisitos recogidos en el catálogo elaborado en las actividades Estudio de
la Información Relevante (PSI 3) e Identificación de Requisitos (PSI 4). Esto
permitirá efectuar un diagnóstico de la situación actual, a partir del cual
se seleccionan los sistemas de información actuales considerados válidos,
identificando las mejoras a realizar en los mismos. 

Por último, se definen los nuevos sistemas de información necesarios para
cubrir los requisitos y funciones de los procesos no soportados por los
sistemas actuales seleccionados. 

Teniendo en cuenta los resultados anteriores, se elabora el modelo de
sistemas de información válido para dar soporte a los procesos de la
organización incluidos en el ámbito del Plan de Sistemas de Información. 

\section{Diagnóstico de la situación actual}

\section{Definición del modelo de sistemas de información}


\chapter{Definición de la arquitectura tecnológica}
En esta actividad se propone una arquitectura tecnológica que de soporte al
modelo de información y de sistemas de información incluyendo, si es
necesario, opciones. Para esta actividad se tienen en cuenta especialmente
los requisitos de carácter tecnológico, aunque es necesario considerar el
catálogo completo de requisitos para entender las necesidades de los procesos
y proponer los entornos tecnológicos que mejor se adapten a las mismas. 

\section{Identificación de las necesidades de infraestructura tecnológica}

\section{Selección de la arquitectura tecnológica}


\chapter{Definición del plan de acción}
En el Plan de Acción, que se elabora en esta actividad, se definen los
proyectos y acciones a llevar a cabo para la implantación de los modelos de
información y de sistemas de información, determinados en las actividades
Identificación de Requisitos (PSI 4) y Diseño del Modelo de Sistemas de
Información (PSI 6), con la arquitectura tecnológica propuesta en la
actividad Definición de la Arquitectura Tecnológica (PSI 7). El conjunto de
estos tres modelos constituye la arquitectura de información. 

Dentro del Plan
de Acción se incluye un calendario de proyectos, con posibles alternativas, y
una estimación de recursos, cuyo detalle será mayor para los más
inmediatos. Para la elaboración del calendario se tienen que analizar las
distintas variables que afecten a la prioridad de cada proyecto y sistema de
información. El orden definitivo de los proyectos y acciones debe pactarse
con los usuarios, para llegar a una solución de compromiso que resulte la
mejor posible para la organización. 

Por último, se propone un plan de
mantenimiento para el control y seguimiento de la ejecución de los proyectos,
así como para la actualización de los productos finales del Plan de Sistemas
de Información. 

\section{Definición de proyectos a realizar}

\section{Elaboración del plan de mantenimiento del PSI}


\bibliographystyle{plain} 
\bibliography{t1}

\end{document}
