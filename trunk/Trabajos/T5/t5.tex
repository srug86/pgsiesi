% Clase
\documentclass[11pt,a4paper,spanish,twoside]{book}

% Órdenes auxiliares
\input{inc/includes.tex}

% Encabezado y pie de página
\encabezado

\usepackage{enumerate}

\begin{document}

% Silabación extra
\hyphenation{
a-sig-na-tu-ras
au-to-ma-ti-za-rá
ca-tá-lo-go
ca-rre-ra
cons-truc-ción
co-rres-pon-de
diag-nos-tico
fi-na-li-za-ción
ge-ne-ra-ción
in-fe-rior
man-te-ni-mien-to
me-dian-te
per-so-nal
pro-ce-di-mien-tos
pro-por-cio-na-rá
pu-bli-ca-da
re-qui-si-tos
res-pecto
u-su-a-rios
vi-lla-rre-al
}


% Portada
\portada{Planificación y Gestión de\\Sistemas de Información}
{Trabajo 5}{Estimación del software}
{Sergio de la Rubia García-Carpintero\\Miguel Millán Sánchez-Grande\\
  Luis Muñoz Villarreal\\Alicia Serrano Sánchez\\
  Juan Miguel Torres Triviño}{24 de Mayo de 2010}

% Licencia
\licencia{Sergio de la Rubia García-Carpintero, Miguel Millán Sánchez-Grande,
  Luis Muñoz Villarreal, Alicia Serrano Sánchez, Juan Miguel Torres Triviño}

\chapter*{Ficha de trabajo}
\begin{description}
\item[Código] T5
\item[Fecha] 24 de Mayo de 2010
\item[Título] Estimación del software
\end{description}

\begin{table}[!ht]
  \centering
  \begin{tabular}{lp{5cm}c}
    \multicolumn{3}{l}{\Large \textbf{Equipo} G4} \\ \\
    \multicolumn{1}{c}{\emph{Apellidos y nombre}} & 
    \multicolumn{1}{c}{\emph{Firma}} & \emph{Puntos} \\
    \hline \\
    de la Rubia García-Carpintero, Sergio & & 8 \\ \\
    Millán Sánchez-Grande, Miguel         & & 8 \\ \\
    Muñoz Villarreal, Luis                & & 8 \\ \\
    Serrano Sánchez, Alicia               & & 8 \\ \\
    Torres Triviño, Juan Miguel           & & 8 \\ \\
    \hline
  \end{tabular}
\end{table}

% Índices
\tableofcontents
\listoftables
%\listoffigures

%% INICIO DEL DOCUMENTO %%%%%%%%%%%%%%%%%%%%%%%%%%%%%%%%%%%%%%%%%%%%%%%%%
\chapter*{Introducción}
%breve introducción del software llamado...

\section{Visión general}

\section{Objetivos}

\chapter{Identificación de los módulos}
\section{Gestión de usuarios}
\subsection{Entradas externas}
\subsubsection{Identificación del usuario}
Cada usuario debe identificarse mediante un nombre y una contraseña.

\subsubsection{Preferencia de horarios} 
Cada profesor introduce sus preferencias de horarios de trabajo para que el
sistema lo tenga en cuenta a la hora de la realización del horario oficial
del curso.
\subsection{Salidas externas}
\subsubsection{Datos personales}
Muestra los datos personales de un determinado profesor, incluyendo
nombre, correo electrónico, teléfono, publicaciones, currículum, \dots
\subsubsection{Horario de un profesor}
El sistema muestra el horario que tiene asignado un determinado profesor.

\subsubsection{Tutorías de los profesores} 
Muestra los horarios de tutorías junto con el despacho asignado de los
profesores de cada facultad.
 
\subsubsection{Asignaturas impartidas por cada profesor}
El sistema muestra la información de cada asignatura que comprende:
\begin{itemize}
\item Profesor/es que la imparte/n.
\item Créditos.
\item Contenidos de la asignatura.
\item Asignaturas recomendadas.
\item Página web de la asignatura.
\item Planificación docente.
\item Sistema de evaluación.
\end{itemize}

\subsection{Consultas externas}
\subsubsection{Consultar currículum}

Información del currículum de un determinado profesor.

\subsubsection{Consultar datos de asignatura} 
Información sobre una determinada asignatura (créditos, profesor que la
imparte, planificación, \dots
\subsubsection{Horarios de asignaturas}
Horarios de las asignaturas de cada curso. Cada horario
también muestra información de la localización en la que cada asignatura se
imparte. 

\subsection{Archivos lógicos de interfaz externos}
\subsubsection{Datos de los profesores}
El sistema tiene una base de datos donde se guarda toda la información con
respecto a los profesores (nombre, apellidos, teléfono, correo electrónico,
asignaturas que imparte, \dots)

\subsubsection{Datos de las asignaturas}
En la base de datos de la universidad se tienen los datos de las asignaturas,
tales como nombre, créditos, guión de la asignatura, contenidos, evaluación,
\dots 


\section{Gestión de cursos}
\subsection{Entradas externas}
\subsubsection{Añadir curso}
Entrada destinada a la inclusión de los distintos cursos correspondientes al
plan de estudios en el sistema.

\subsubsection{Añadir asignaturas}
Entrada generada para la inserción de las diferentes asignaturas asociadas a
cada curso.

\subsubsection{Añadir aulas}
Entrada destinada a la introducción las aulas disponibles destinadas al
correspondiente plan de estudio.

\subsubsection{Añadir profesores}
Entrada generada para la incluir los profesores encargados de la docencia de
cada asignatura.

\subsection{Salidas externas}
\subsubsection{Relación cursos/asignaturas}
Información sobre las asginaturas que forman cada curso.

\subsubsection{Relación asignaturas/profesores}
Información de los profesores que imparten la docencia de cada asignatura.

\subsubsection{Relación asignaturas/aulas}
Información sobre la relación entre las asignaturas y las aulas donde son
impartidas. 

\subsubsection{Cuadrante de horarios}
Tabla de horarios por cursos con la información de cada asignatura, aula y
profesor que la imparte.

\subsection{Consultas externas}
\subsubsection{Consultar relacion cursos/asignaturas}
Consulta de la información necesaria para relacionar las tablas de
cursos con las tablas de asginaturas.
\subsubsection{Consultar relación asignaturas/profesores}
Consulta de la información necesaria para relacionar las tablas de
asignaturas con las tablas de profesores.
\subsubsection{Consultar relación cursos/aulas}
Consulta de la información necesaria para relacionar las tablas de
cursos con las tablas de aulas.
\subsubsection{Consultar cuadrante de horarios}
Consulta de la información requerida para realizar los cuadrantes de horarios.
\subsection{Archivos lógicos internos}
\subsubsection{Datos de las asignaturas}
Toda la información relacionada con cada asginatura, temario, créditos \dots

\subsubsection{Datos de los cursos}
Información relacionada con cada curso como las asignaturas que los forman.

\subsubsection{Datos de las aulas}
Información de las aulas disponibles y las utilizadas para la docencia de las asignaturas.

\subsection{Archivos lógicos de interfaz externos}
\subsubsection{Historial de guías docentes}
Datos externos relacionados con las guías docentes de cursos anteriores. 

\section{Interfaz}
\subsection{Entradas externas}
\subsubsection{Autentificación de los usuarios}
Recoge los datos necesarios para ver si un usuario entra al sistema,
dependiendo si este es administrador o no.

\subsubsection{Introducción de datos para la inserción/modificación/borrado 
de información de un determinado usuario}
Por su similitud se explican juntas pero equivale a tres \emph{funciones de
usuario} diferentes. El objetivo de estas funciones es recoger los datos
necesarios para la de inserción, modificación o borrado de la información de
un usuario en la aplicación.

\subsection{Salidas externas}
\subsubsection{Mostar los datos para una búsqueda de información con
posibilidad de ordenación por diferentes criterios}
El objetivo de este proceso es mostrar un listado de usuarios que cumplen las
condiciones de búsqueda solicitadas y en el orden indicado.

\subsubsection{Mostrar los datos de un determinado usuario}
El objetivo de este proceso es mostrar información detallada referente a un
determinado usuario (profesores).

\subsubsection{Mostrar los datos de un determinado curso}
El objetivo de este proceso es mostrar información detallada referente a un
curso, incluyendo horario, aulas, asignaturas, etc.

\subsection{Consultas externas}
\subsubsection{Búsqueda de información con posibilidad de ordenación por
  diferentes criterios}
El objetivo de este proceso es la introdución de unos criterios para realizar 
una búsqueda detallada tanto de cursos como de usuarios.

\subsubsection{Consulta de los datos de un determinado usuario}
El objetivo de esta función es la consulta simple de la información referente
a un usuario (profesor).

\subsubsection{Consulta de los datos de un determinado curso}
El objetivo de esta función es la consulta simple de la información referente
a un curso.

\subsection{Archivos lógicos internos}
\subsubsection{Importación de datos desde un fichero}
El objetivo de esta función es introducir los datos necesarios para importar 
inserciones, modificaciones o borrados desde un fichero de texto.

\subsubsection{Exportación de los resultados de la búsqueda a un fichero}
El objetivo de esta función es introducir los datos para obtener de la base
de datos de la aplicación el listado de usuarios que cumplen las condiciones de
búsqueda solicitadas y en el orden indicado en un fichero de texto.

\subsection{Archivos lógicos de interfaz externos}
No se ha detectado ningún archivo lógico de interfaz externo.

\chapter{Determinación de la complejidad de las funciones de los 
  módulos}
\section{Gestión de usuarios}
\subsection{Entradas externas}
\subsubsection{Identificación del usuario}
La complejidad de esta función de usuario es \textbf{baja} porque se hace
referencia a un registro para la información de usuario y el número de datos
elementales es dos. 

\subsubsection{Preferencia de horarios} 
La complejidad de esta función es \textbf{media} porque se hace
referencia a dos registros donde se tienen guardados los horarios y las
asignaturas y el número de datos elementales son que se utilizan son seis. 

\subsection{Salidas externas}
\subsubsection{Datos personales}
La complejidad de esta función es \textbf{media} porque se hace referencia a
un registro para la información de usuario y los datos elementales que se
utilizan son dieciséis.

\subsubsection{Horario de un profesor}
La complejidad de esta función es \textbf{media} porque se hace referencia a
dos registros, horarios y asignaturas; y los datos elementales que
se utilizan son seis.

\subsubsection{Tutorías de los profesores} 
La complejidad de esta función es \textbf{baja} porque se hace referencia a
dos registros, profesores y tutorías; y el número de datos elementales que se
utilizan son cuatro. 

\subsubsection{Asignaturas impartidas por cada profesor}
La complejidad de esta función es \textbf{medio} porque se hace referencia a
dos registros, profesores y asignaturas; y el número de datos elementales que se
utilizan son siete. 

\subsection{Consultas externas}
\subsubsection{Consultar currículum}
La complejidad de esta función es \textbf{baja} porque se hace referencia a
un registro y el número de datos elementales es dos.

\subsubsection{Consultar datos de asignatura} 
La complejidad de esta función es \textbf{baja} porque se hace referencia a
un registro y el número de datos elementales es nueve.

\subsubsection{Horarios de asignaturas}
La complejidad de esta función es \textbf{baja} porque se hace referencia a
tres registros y los datos elementales que se utilizan son cuatro.

\subsection{Archivos lógicos de interfaz externos}
\subsubsection{Datos de los profesores}
La complejidad de esta función es \textbf{baja} porque se hace referencia a
un registro y los datos elementales son siete.

\subsubsection{Datos de las asignaturas}
La complejidad de esta función es \textbf{baja} porque se hace referencia a
un registro y los datos elementales son catorce.

\section{Gestión de cursos}
\subsection{Entradas externas}
\subsubsection{Añadir curso}
La complejidad de esta función es \textbf{baja}, porque tiene un
registro para la información del curso y el número de tipo de datos
elementales es menor que quince.
\subsubsection{Añadir asignaturas}
La complejidad de esta función es \textbf{baja}, porque tiene un
registro para la información de las asignaturas y el número de tipo de datos
elementales es menor que quince.
\subsubsection{Añadir aulas}
La complejidad de esta función es \textbf{baja}, porque tiene un
registro para la información de las aulas y el número de tipo de datos
elementales es menor que quince.
\subsubsection{Añadir profesores}
La complejidad de esta función es \textbf{baja}, porque tiene un
registro para la información de los profesores y el número de tipo de datos
elementales es menor que quince.

\subsection{Salidas externas}
\subsubsection{Relación cursos/asignaturas}
La complejidad de esta función es \textbf{media}, porque usa dos
registros para la información del curso y las asignaturas y el número de tipo
de datos elementales se encuentra entre seis y diecinueve.
\subsubsection{Relación asignaturas/profesores}
La complejidad de esta función es \textbf{media}, porque usa dos
registros para la información de las asignaturas y los profesores y el número
de tipo de datos elementales se encuentra entre seis y diecinueve. 
\subsubsection{Relación asignaturas/aulas}
La complejidad de esta función es \textbf{media}, porque usa dos
registros para la información de las aulas y las asignaturas y el número de
tipo de datos elementales se encuentra entre seis y diecinueve.
\subsubsection{Cuadrante de horarios}
La complejidad de esta función es \textbf{alta}, porque usa todos los
registros de este módulo para mostrar el cuadrante y el número de
tipo de datos elementales es mayor que veinte.

\subsection{Consultas externas}
\subsubsection{Consultar relación cursos/asignaturas}
La complejidad de esta función es \textbf{media}, porque usa dos
registros para la información de las asignaturas y los cursos de este
módulo y el número de tipo de datos elementales se encuentra entre seis y
deicinueve.
\subsubsection{Consultar relación asignaturas/profesores}
La complejidad de esta función es \textbf{media}, porque usa dos
registros para la información de las asignaturas y los profesores de este
móduloy el número de tipo de datos elementales se encuentra entre seis y
deicinueve.
\subsubsection{Consultar relación asignaturas/aulas}
La complejidad de esta función es \textbf{media}, porque usa dos
registros para la información de las asignaturas y las aular de este
módulo y el número de tipo de datos elementales se encuentra entre seis y
deicinueve.
\subsubsection{Consultar cuadrante de horarios}
La complejidad de esta función es \textbf{alta}, porque usa todos los
registros disponibles en este módulo y el número de tipo de datos elementales
es mayor que veinte.

\subsection{Archivos lógicos internos}
\subsubsection{Datos de las asignaturas}
La complejidad de esta función es \textbf{baja}, porque usa un
registros para la información de las asignaturas de este módulo y el número
de tipo de datos elementales es menor que veinte.

\subsubsection{Datos de los cursos}
La complejidad de esta función es \textbf{baja}, porque usa un
registros para la información de los cursos de este módulo y el número
de tipo de datos elementales es menor que veinte.

\subsubsection{Datos de las aulas}
La complejidad de esta función es \textbf{baja}, porque usa un
registros para la información de las aulas de este módulo y el número
de tipo de datos elementales es menor que veinte.
\subsection{Archivos lógicos de interfaz externos}
\subsubsection{Historial de guías docentes}
La complejidad de esta función es \textbf{alta}, porque usa todos los
registros de este módulo y el número de tipo de datos elementales es mayor
que cincuenta y uno. 


\section{Interfaz}
\subsection{Entradas externas}
\subsubsection{Autentificación de los usuarios}
La complejidad de esta función de usuario es \textbf{baja}, porque tiene un
registro para la información del usuario y el número de tipo de datos
elementales es dos. 

\subsubsection{Introducción de datos para la inserción/modificación/borrado 
de información de un determinado usuario}
La complejidad para cada función de usuario (inserción, modificación y
borrado) es \textbf{media}, porque tiene un registro para la información del
usuario y el número de tipo de datos elementales es diecisiete.

\subsubsection{Introducción de datos para la inserción/modificación/borrado 
de información de un determinado curso }
La complejidad para cada función de usuario (inserción, modificación y
borrado) es \textbf{baja}, porque tiene un registro para la información del
curso y el número de tipo de datos elementales es diez.

\subsection{Salidas externas}
\subsubsection{Mostar los datos para una búsqueda de información con
posibilidad de ordenación por diferentes criterios}
La complejidad de esta función de usuario es \textbf{media}, porque tiene un
registro para la información y el número de tipo de datos elementales es veinte.

\subsubsection{Mostrar los datos de un determinado usuario}
La complejidad de esta función de usuario es \textbf{baja}, porque tiene un
registro para la información del usuario y el número de tipo de datos
elementales es nueve.

\subsubsection{Mostrar los datos de un determinado curso}
La complejidad de esta función de usuario es \textbf{baja}, porque tiene un
registro para la información del curso y el número de tipo de datos
elementales es cinco.

\subsection{Consultas externas}
\subsubsection{Búsqueda de información con posibilidad de ordenación por
  diferentes criterios}
La complejidad de esta función de usuario es \textbf{media}, porque no tiene
ningún registro para la información y el número de tipo de datos elementales
es dieciocho.

\subsubsection{Consulta de los datos de un determinado usuario}
La complejidad de esta función de usuario es \textbf{baja}, porque no tiene
ningún para la información del usuario y el número de tipo de datos
elementales es uno.

\subsubsection{Consulta de los datos de un determinado curso}
La complejidad de esta función de usuario es \textbf{baja}, porque no tiene
ningún registro para la información del curso y el número de tipo de datos
elementales es uno.

\subsection{Archivos lógicos internos}
\subsubsection{Importación de datos desde un fichero}
La complejidad de esta función de usuario es \textbf{baja}, porque tiene un
fichero referenciado para la información y el número de tipo de datos
elementales es nueve.

\subsubsection{Exportación de los resultados de la búsqueda a un fichero}
La complejidad de esta función de usuario es \textbf{alta}, porque tiene dos
fichero referenciado para la información y el número de tipo de datos
elementales es veinte.

\chapter{Tablas de puntos función sin ajustar}
\section{Gestión de usuarios}
\begin{table}[!h]
  \centering
  \begin{tabular}{|b{2.1cm}<\centering|b{2.5cm}<{\centering} |c c c c c|}
    \hline
    \textbf{Tipo de función} & \textbf{Nivel de complejidad} &
    \textbf{Número} & \textbf{*} & \textbf{Peso} & \textbf{=} & \textbf{Total}\\
    \hline \hline
    \multirow{3}{*}{Entradas} 
    & Baja  & 1 & * & 3 & = & \textbf{3} \\
    \cline{3-7}
    & Media & 1 & * & 4 & = & \textbf{4} \\
    \cline{3-7}
    & Alta  &   &   & 6  &   & \\
    \hline
    \multirow{3}{*}{Salidas}
    & Baja  & 1 & * & 4 & = & \textbf{4} \\
    \cline{3-7}
    & Media & 3 & * & 5 & = & \textbf{15} \\
    \cline{3-7}
    & Alta  & & & 7 & & \\
    \hline
    \multirow{3}{*}{Consultas}
    & Baja  & 3 & * & 3 & = & \textbf{9} \\
    \cline{3-7}
    & Media & & & 4 & & \\
    \cline{3-7}
    & Alta  & & & 6 & & \\
    \hline
    \multirow{3}{*}{Interfaces}
    & Baja  & 2 & * & 5 & = & \textbf{10}\\
    \cline{3-7} 
    & Media & & & 7 & & \\
    \cline{3-7}
    & Alta  & & & 10 & & \\
    \hline
    \multicolumn{6}{|l}{\textbf{Número de puntos función sin ajustar}} &
    \textbf{\textcolor{rojo}{45}} \\ 
    \hline \hline
  \end{tabular}
  \caption{Puntos de función sin ajustar del módulo gestión de usuarios} 
  \label{Tab:PFSAusu}
\end{table}
\section{Gestión de cursos}
\begin{table}[!h]
  \centering
  \begin{tabular}{|b{2.1cm}<\centering|b{2.5cm}<{\centering} |c c c c c|}
    \hline
    \textbf{Tipo de función} & \textbf{Nivel de complejidad} &
    \textbf{Número} & \textbf{*} & \textbf{Peso} & \textbf{=} & \textbf{Total}\\
    \hline \hline
    \multirow{3}{*}{Entradas} 
    & Baja & 4 & * & 3 & = & \textbf{12} \\
    \cline{3-7}
    & Media &  &  & 4 &  &  \\
    \cline{3-7}
    & Alta  &  &  & 6 & &  \\
    \hline
    \multirow{3}{*}{Salidas}
    & Baja  & & & 4 & & \\
    \cline{3-7}
    & Media & 3 & * & 5 & = & \textbf{15} \\
    \cline{3-7}
    & Alta  & 1 & * & 7 & = & \textbf{7} \\
    \hline
    \multirow{3}{*}{Consultas}
    & Baja  & & & 3 & & \\
    \cline{3-7}
    & Media & 3 & * & 4 & = & \textbf{12} \\
    \cline{3-7}
    & Alta  & 1 & * & 6 & = & \textbf{6} \\
    \hline
    \multirow{3}{*}{Archivos}
    & Baja  & 3 & * & 7 & = & \textbf{21} \\
    \cline{3-7}
    & Media &  &  & 10 &  & \\
    \cline{3-7}
    & Alta  &  &  & 15 &  & \\
    \hline
    \multirow{3}{*}{Interfaces}
    & Baja  & & & 5 & & \\
    \cline{3-7}
    & Media & & & 7 & & \\
    \cline{3-7}
    & Alta  & 1 & * & 10 & = & \textbf{10} \\
    \hline \hline
    \multicolumn{6}{|l}{\textbf{Número de puntos función sin ajustar}} &
    \textbf{\textcolor{rojo}{83}} \\ 
    \hline
  \end{tabular}
  \caption{Puntos de función sin ajustar del módulo gestión de cursos} 
  \label{Tab:PFSAcur}
\end{table}

\section{Interfaz}
\begin{table}[!h]
  \centering
  \begin{tabular}{|b{2.1cm}<\centering|b{2.5cm}<{\centering} |c c c c c|}
    \hline
    \textbf{Tipo de función} & \textbf{Nivel de complejidad} &
    \textbf{Número} & \textbf{*} & \textbf{Peso} & \textbf{=} & \textbf{Total}\\
        \hline \hline
    \multirow{3}{*}{Entradas} 
    & Baja & 4 & * & 3 & = & \textbf{12} \\
    \cline{3-7}
    & Media & 3 & * & 4 & = & \textbf{12} \\
    \cline{3-7}
    & Alta  & & & 6 & & \\
    \hline
    \multirow{3}{*}{Salidas}
    & Baja  & 2 & * & 4 & = & \textbf{8} \\
    \cline{3-7}
    & Media & 1 & * & 5 & = & \textbf{5} \\
    \cline{3-7}
    & Alta  & & & 7 & & \\
    \hline
    \multirow{3}{*}{Consultas}
    & Baja  & 2 & * & 3 & = & \textbf{6} \\
    \cline{3-7}
    & Media & 1 & * & 4 & = & \textbf{4} \\
    \cline{3-7}
    & Alta  & & & 6 & & \\
    \hline
    \multirow{3}{*}{Archivos}
    & Baja  & 1 & * & 7 & = & \textbf{7} \\
    \cline{3-7}
    & Media & & & 10 & & \\
    \cline{3-7}
    & Alta  & 1 & * & 15 & = & \textbf{15} \\
    \hline \hline
    \multicolumn{6}{|l}{\textbf{Número de puntos función sin ajustar}} &
    \textbf{\textcolor{rojo}{69}} \\ 
    \hline
  \end{tabular}
  \caption{Puntos de función sin ajustar del módulo interfaz} 
  \label{Tab:PFSAint}
\end{table}

\chapter{Explicación de los valores de los factores de influencia} 
En este apartado se especifican los pesos que se han asignado a cada factor
de influencia. Estos valores han sido extraídos de la información que
proporciona Métrica v.3. 

\section{Gestión de usuarios}
\begin{enumerate}[{\bf 1.}]

\item {\bf Comunicación de datos:} \emph{Peso 4} La aplicación está basada en
  un teleproceso interactivo, pero con un solo protocolo de comunicaciones.

\item {\bf Procesamiento distribuido:} \emph{Peso 4} Proceso distribuido
  con transferencia de datos on line en ambas direcciones. 

\item {\bf Objetivos de rendimiento:} \emph{Peso 1} Se establecen requisitos
  de rendimiento y diseño, pero no se requieren acciones especiales.

\item {\bf Configuración de uso intensivo:} \emph{Peso2} Se incluyen algunas
  consideraciones sobre tiempo y seguridad. 

\item {\bf Tasas de transacción rápida:} \emph{Peso 0} Las transacciones no
  están afectadas por picos de tráfico. 

\item {\bf Entradas de datos en línea:} \emph{Peso 5} Más del 30\% de las
  funciones son entradas interactivas de datos. 

\item {\bf Eficiencia de Usuario Final:} \emph{Peso 5} Además se requiere del
  uso de herramientas y procesos para demostrar que se cumplen los requisitos
  establecidos.

\item {\bf Actualización de datos en línea:} \emph{Peso 5} Además, grandes
  volúmenes implican consideraciones de coste en el proceso de
  recuperación. Se incluyen procedimientos de recuperación que requieren
  mínima intervención del operador. 

\item {\bf Procesamiento complejo:} \emph{Peso 1} Es aplicable una de las
  situaciones descritas. 

\item {\bf Reutilización:} \emph{Peso 3} El 10\% o más de la aplicación se
  considera reusable. 

\item {\bf Facilidad de instalación:} \emph{Peso 2} El usuario ha declarado
  consideraciones especiales para la conversión e instalación y se requieren
  guías probadas de conversión e instalación. El impacto de la conversión
  sobre el proyecto no se considera importante. 

\item {\bf Facilidad operacional:} \emph{Peso 5} La aplicación debe diseñarse
  para una operación totalmente automática. La recuperación automática ante
  errores es una funcionalidad de la aplicación. 

\item {\bf Múltiples localizaciones:} \emph{Peso 2} Se consideran en el diseño múltiples
  instalaciones pero para operar con similar configuración HW y/o SW.

\item {\bf Facilidad de cambio:} \emph{Peso 1} Consultas y Generación de
  Informes Flexible que manejan peticiones simples.

\end{enumerate}

\section{Gestión de cursos}
\begin{enumerate}[{\bf 1.}]

\item {\bf Comunicación de datos:} \emph{Peso 4} La aplicación está basada en
  un teleproceso interactivo, pero con un solo protocolo de comunicaciones.

\item {\bf Procesamiento distribuido:} \emph{Peso 4} Proceso distribuido
  con transferencia de datos on line en ambas direcciones. 

\item {\bf Objetivos de rendimiento:} \emph{Peso 1} Se establecen requisitos
  de rendimiento y diseño, pero no se requieren acciones especiales.

\item {\bf Configuración de uso intensivo:} \emph{Peso2} Se incluyen algunas
  consideraciones sobre tiempo y seguridad. 

\item {\bf Tasas de transacción rápida:} \emph{Peso 0} Las transacciones no
  están afectadas por picos de tráfico. 

\item {\bf Entradas de datos en línea:} \emph{Peso 5} Más del 30\% de las
  funciones son entradas interactivas de datos. 

\item {\bf Eficiencia de Usuario Final:} \emph{Peso 5} Además se requiere del
  uso de herramientas y procesos para demostrar que se cumplen los requisitos
  establecidos.

\item {\bf Actualización de datos en línea:} \emph{Peso 5} Además, grandes
  volúmenes implican consideraciones de coste en el proceso de
  recuperación. Se incluyen procedimientos de recuperación que requieren
  mínima intervención del operador. 

\item {\bf Procesamiento complejo:} \emph{Peso 5} Son aplicables todas las
  situaciones descritas.
 
\item {\bf Reutilización:} \emph{Peso 3} El 10\% o más de la aplicación se
  considera reusable. 

\item {\bf Facilidad de instalación:} \emph{Peso 3} El usuario ha declarado
  consideraciones especiales para la conversión e instalación y se requieren
  guías probadas de conversión e instalación. El impacto de la conversión
  sobre el proyecto se considera importante. 

\item {\bf Facilidad operacional:} \emph{Peso 5} La aplicación debe diseñarse
  para una operación totalmente automática. La recuperación automática ante
  errores es una funcionalidad de la aplicación. 

\item {\bf Múltiples localizaciones:} \emph{Peso 2} Se consideran en el diseño múltiples
  instalaciones pero para operar con similar configuración HW y/o SW.

\item {\bf Facilidad de cambio:} \emph{Peso 3} Consultas y Generación de
  Informes Flexible que manejan peticiones complejas.

\end{enumerate}

\section{Interfaz}
\begin{enumerate}[{\bf 1.}]

\item {\bf Comunicación de datos:} \emph{Peso 4} La aplicación está basada en
  un teleproceso interactivo, pero con un solo protocolo de comunicaciones.

\item {\bf Procesamiento distribuido:} \emph{Peso 4} Proceso distribuido
  con transferencia de datos on line en ambas direcciones. 

\item {\bf Objetivos de rendimiento:} \emph{Peso 1} Se establecen requisitos
  de rendimiento y diseño, pero no se requieren acciones especiales.

\item {\bf Configuración de uso intensivo:} \emph{Peso2} Se incluyen algunas
  consideraciones sobre tiempo y seguridad. 

\item {\bf Tasas de transacción rápida:} \emph{Peso 0} Las transacciones no
  están afectadas por picos de tráfico. 

\item {\bf Entradas de datos en línea:} \emph{Peso 5} Más del 30\% de las
  funciones son entradas interactivas de datos. 

\item {\bf Eficiencia de Usuario Final:} \emph{Peso 5} Además se requiere del
  uso de herramientas y procesos para demostrar que se cumplen los requisitos
  establecidos.

\item {\bf Actualización de datos en línea:} \emph{Peso 5} Además, grandes
  volúmenes implican consideraciones de coste en el proceso de
  recuperación. Se incluyen procedimientos de recuperación que requieren
  mínima intervención del operador. 

\item {\bf Procesamiento complejo:} \emph{Peso 1} Es aplicable una de las
  situaciones descritas. 

\item {\bf Reutilización:} \emph{Peso 3} El 10\% o más de la aplicación se
  considera reusable. 

\item {\bf Facilidad de instalación:} \emph{Peso 1} El usuario no ha
  declarado consideraciones especiales pero se requiere una configuración
  especial para la instalación 

\item {\bf Múltiples localizaciones:} \emph{Peso 2} Se consideran en el diseño múltiples
  instalaciones pero para operar con similar configuración HW y/o SW.

\item {\bf Facilidad de cambio:} \emph{Peso 1} Consultas y Generación de
  Informes Flexible que manejan peticiones simples.

\end{enumerate}
\chapter{Cálculo de los puntos función ajustados}

\end{document}
