% Clase
\documentclass[11pt,a4paper,spanish,twoside]{book}

% Órdenes auxiliares
\input{inc/includes.tex}

% Árboles de directorios
%\usepackage{dirtree}

% Encabezado y pie de página
\encabezado
\setcounter{secnumdepth}{3} 
\begin{document}

% Silabación extra
\hyphenation{
a-sig-na-tu-ras
au-to-ma-ti-za-rá
ca-tá-lo-go
ca-rre-ra
cons-truc-ción
co-rres-pon-de
diag-nos-tico
fi-na-li-za-ción
ge-ne-ra-ción
in-fe-rior
man-te-ni-mien-to
me-dian-te
per-so-nal
pro-ce-di-mien-tos
pro-por-cio-na-rá
pu-bli-ca-da
re-qui-si-tos
res-pecto
u-su-a-rios
vi-lla-rre-al
}


% Portada
\portada{Planificación y Gestión de\\Sistemas de Información}
{Práctica 1}{Elaboración de un plan de proyecto\\utilizando Microsoft Project}
{Sergio de la Rubia García-Carpintero\\Miguel Millán Sánchez-Grande\\
  Luis Muñoz Villarreal\\Alicia Serrano Sánchez\\
  Juan Miguel Torres Triviño}{26 de Mayo de 2010}

% Licencia
\licencia{Sergio de la Rubia García-Carpintero, Miguel Millán Sánchez-Grande,
  Luis Muñoz Villarreal, Alicia Serrano Sánchez, Juan Miguel Torres Triviño}

\chapter*{Ficha de trabajo}
\begin{description}
\item[Código] P2
\item[Fecha] 26 de Abril de 2010
\item[Título] Estimación del esfuerzo de desarrollo de un Software utilizando
USC COCOMO II
\end{description}

\begin{table}[!ht]
  \centering
  \begin{tabular}{lp{5cm}c}
    \multicolumn{3}{l}{\Large \textbf{Equipo} G4} \\ \\
    \multicolumn{1}{c}{\emph{Apellidos y nombre}} & 
    \multicolumn{1}{c}{\emph{Firma}} & \emph{Puntos} \\
    \hline \\
    de la Rubia García-Carpintero, Sergio & & 6 \\ \\
    Millán Sánchez-Grande, Miguel         & & 6 \\ \\
    Muñoz Villarreal, Luis                & & 6 \\ \\
    Serrano Sánchez, Alicia               & & 6 \\ \\
    Torres Triviño, Juan Miguel           & & 6 \\ \\
    \hline
  \end{tabular}
\end{table}

% Índices
\tableofcontents
\listoffigures
%\listoftables

%% INICIO DEL DOCUMENTO %%%%%%%%%%%%%%%%%%%%%%%%%%%%%%%%%%%%%%%%%%%%%%%%%
\chapter{Ajustes de tamaño aplicados}

Para la estimación del tamaño de los tres módulos de los que se compone el
proyecto, se va a utilizar el método de \emph{puntos función}. Como lenguaje
de implementación del proyecto se va a utilizar \emph{php} (un lenguaje
orientado a objetos}. Además, se ha considerado que el porcentaje de código que
se va a desechar cuando finalice la implementación va a ser de nulo
\emph{0.00}.

Para rellenar las tablas del formulario de entrada de estimaciones de tamaño
en \emph{puntos función}, se han utilizado los valores calculados en las tablas
de \emph{PFSA} del trabajo 5.

\section{Gestión de usuarios}
La \emph{figura \ref{CodGesUs}} muestra las estimaciones en puntos función sin
ajustar del módulo de gestión de usuario, extraídas de las estimaciones
realizadas en el trabajo 5, y el equivalente total en tamaño de líneas de
código calculado por COCOMO II.

\imagen{CodGesUs.png}{8}{Tamaño en líneas de código del módulo de gestión de
usuarios}{CodGesUs}

Como puede verse, el tamaño total equivalente en líneas de código del módulo
de gestión de usuarios es de \textbf{1440}.

\section{Gestión de cursos}
La \emph{figura \ref{CodGesCu}} muestra las estimaciones en puntos función sin
ajustar del módulo de gestión de usuario, extraídas de las estimaciones
realizadas en el trabajo 5, y el equivalente total en tamaño de líneas de
código calculado por COCOMO II.

\imagen{CodGesCu.png}{8}{Tamaño en líneas de código del módulo de gestión de
cursos}{CodGesCu}

Como puede verse, el tamaño total equivalente en líneas de código del módulo
de gestión de cursos es de \textbf{2656}.

\section{Interfaz}
La \emph{figura \ref{CodInt}} muestra las estimaciones en puntos función sin
ajustar del módulo de la interfaz, extraídas de las estimaciones realizadas en
el trabajo 5, y el equivalente total en tamaño de líneas de código calculado
por COCOMO II.

\imagen{CodInt.png}{8}{Tamaño en líneas de código del módulo de gestión de
cursos}{CodInt}

Como puede verse, el tamaño total equivalente en líneas de código del módulo
de la interfaz es de \textbf{2208}.

\chapter{Multiplicadores y estimaciones de esfuerzo}
\section{Gestión de usuarios}
\section{Gestión de cursos}
\section{Interfaz}

\chapter{Justificaciones del proyecto}
\section{Factores de escala}
\section{Estimaciones de las variables calculadas}
\subsection{Early Design Model}
\subsection{Post Architecture Model}

\chapter{Informes}
\section{Early Design Model}
\section{Post Architecture Model}

\end{document}
