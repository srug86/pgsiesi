% Clase
\documentclass[11pt,a4paper,spanish,twoside]{report}

% Órdenes auxiliares
\input{inc/includes.tex}

% Encabezado y pie de página
\encabezado

\begin{document}

% Silabación extra
\hyphenation{
a-sig-na-tu-ras
au-to-ma-ti-za-rá
ca-tá-lo-go
ca-rre-ra
cons-truc-ción
co-rres-pon-de
diag-nos-tico
fi-na-li-za-ción
ge-ne-ra-ción
in-fe-rior
man-te-ni-mien-to
me-dian-te
per-so-nal
pro-ce-di-mien-tos
pro-por-cio-na-rá
pu-bli-ca-da
re-qui-si-tos
res-pecto
u-su-a-rios
vi-lla-rre-al
}


% Portada
\portada{Planificación y Gestión de\\Sistemas de Información}
{Trabajo 2}{Integración y alcance}
{Sergio de la Rubia García-Carpintero\\Miguel Millán Sánchez-Grande\\
  Luis Muñoz Villarreal\\Alicia Serrano Sánchez\\
  Juan Miguel Torres Triviño}{13 de Abril de 2009}

% Licencia
\licencia{Sergio de la Rubia García-Carpintero, Miguel Millán Sánchez-Grande,
  Luis Muñoz Villarreal, Alicia Serrano Sánchez, Juan Miguel Torres Triviño}

\chapter*{Ficha de trabajo}
\begin{description}
\item[Código] T2
\item[Fecha] 13 de Abril de 2010
\item[Título] Integración y alcance
\end{description}

\begin{table}[!ht]
  \centering
  \begin{tabular}{lp{5cm}c}
    \multicolumn{3}{l}{\Large \textbf{Equipo} G4} \\ \\
    \multicolumn{1}{c}{\emph{Apellidos y nombre}} & 
    \multicolumn{1}{c}{\emph{Firma}} & \emph{Puntos} \\
    \hline \\
    de la Rubia García-Carpintero, Sergio & & 10 \\ \\
    Millán Sánchez-Grande, Miguel         & & 10 \\ \\
    Muñoz Villarreal, Luis                & & 10 \\ \\
    Serrano Sánchez, Alicia               & & 10 \\ \\
    Torres Triviño, Juan Miguel           & & 10 \\ \\
    \hline
  \end{tabular}
%  \caption{}\label{}
\end{table}

% Índices
\tableofcontents
% \listoffigures
% \listoftables

%% INICIO DEL DOCUMENTO %%%%%%%%%%%%%%%%%%%%%%%%%%%%%%%%%%%%%%%%%%%%%%%%%
\chapter*{Introducción}
A la hora de decidir la institución sobre la cual centrar nuestra 
investigación, empezamos analizando la posibilidad de buscar una empresa 
cercana geográficamente como podría haber sido el aeropuerto de Ciudad Real. 
Pero ante la posibilidad de encontrar dificultades a la hora de recopilar 
información nos decantamos por una entidad pública. Nuestra primera opción fue 
la ESI, pero buscando, encontramos mucha más información sobre la Universidad 
de Málaga, de ahí nuestra elección.

La universidad de Málaga es una universidad pública, joven y dinámica que ha 
apostado decididamente por la calidad en la docencia, la investigación y por el
servicio al alumno. Cuenta con más de 40.000 alumnos matriculados y 2.000 
investigadores. 

Para la realización del trabajo hemos usado como ayuda el estándar IEEE 1058, 
el PMBOK-2004, la metodología METRICA 3 y diferentes apuntes e informaciones 
recogidas a través de Internet como material de complemento.

\chapter{Introducción al plan de gestión del proyecto}
\section{Visión general del proyecto}
Se trata de realizar una aplicación que controle todos los aspectos 
relacionados con la generación automatizada de guías docentes para la UMA.
Esta aplicación no solo se encargará de construir una guía docente a partir de 
los datos almacenados en las distintas bases de datos con las que cuenta la 
universidad, también se encargará de la adaptación de los planes de estudio 
antiguos a los nuevos acordados por el EEES, la asignación del profesorado a 
las distintas asignaturas, el establecimiento de los horarios lectivos y los 
horarios de exámenes, así como de sus localizaciones.

La implantación de la aplicación tendrá como objetivo principal la agilización 
de la realización de las guías docentes. Este objetivo principal llevará 
consigo el cumplimiento de otros subobjetivos como, la actualización en tiempo 
real de los contenidos de la guías docentes, así como la mejora en la 
accesibilidad a dichos contenidos y la reducción en el gasto de la 
contratación de personal para la elaboración de dichas guías.

Para la realización del proyecto se contará con: Dña. Adelaida de la Calle, 
rectora de la UMA y jefa de proyecto; Dña. María Valpuesta, vicerrectora de 
Innovación y Desarrollo Tecnológico y responsable del proyecto; D. Luis 
Muñoz, coordinador; un grupo de trabajo formado por: D. Sergio de la Rubia, 
D. Miguel Millán, Dña. Alicia Serrano y D. Juan Miguel Torres; y la 
colaboración de: un analista, un programador, un usuario experto, un 
secretario y un operario para el servicio técnico.

La realización del proyecto sigue un ciclo de vida en cascada con un 
desarrollo de 88 días laborables con objeto de que esté concluido para el 
inicio del nuevo año académico 2010-2011.

\section{Entregables del proyecto}
El conjunto de entregables estará constituido por:
\begin{itemize}
\item El software desarrollado, que incluirá no solo la posibilidad de
  consultar y estructurar el contenido de las guías docentes de cada uno de
  los estudios que se imparten en la UMA, sino que también contendrá las
  funcionalidades de generación de automática de horarios, asignación
  automática de aulas y la adaptación automática entre planes de estudios. El
  software se integrará en los equipos que se encargan de la gestión de la
  UMA. La guía docente resultante será consultable desde la página web de la
  universidad.
\item Una completa documentación para el soporte y mantenimiento del sistema,
  así como un manual de usuario para el personal que gestiona el sistema. 
\end{itemize}

\section{Material de referencia}
El material de referencia usado como apoyo en el proyecto ha sido el siguiente:
\begin{itemize}
\item Listado de requisitos facilitados por la universidad con las 
  características y especificaciones que el software debe cumplir.
\item Recomendaciones de la rectora y la vicerrectora de Innovación y 
  Desarrollo Tecnológico acerca del desarrollo del plan de proyecto.
\item Estándar IEEE 1058.
\item Modelo Métrica 3.
\item Documentación acerca de las redes, SSOO, metodologías, etc. con los que 
  cuenta la universidad y con los que debería contar.
\end{itemize}

\section{Definiciones y acrónimos}
\begin{description}
\item[EEES] Espacio Europeo de Educación Superior.
\item[IEEE] Instituto de Ingenieros Electricistas y Electrónicos.
\item[SI] Sistema de Información.
\item[SSOO] Sistemas Operativos.
\item[TI] Tecnología de la Información.
\item[UMA] Universidad de Málaga.
\item[WAN] Wide Area Network.
\end{description}

\chapter{Organización del proyecto}
\section{Modelo de procesos}
El ciclo de vida a seguir para este proyecto será el de "cascada". Dicho 
modelo marcará en cada momento las acciones a realizar. Hay que puntualizar 
que si se necesita repasar alguna fase, supondría pérdidas de tiempo y dinero.
El esquema del modelo es el siguiente:

Operaciones
      Necesidades de usuarios.
            Requerimientos de usuarios.
                   Desarrollo de requerimientos.
                         Diseño de alto nivel.
                               Diseño de bajo nivel.
                                     Cierre del proyecto.

Al tratarse de un ciclo de vida en cascada, cada fase requiere que la anterior 
este completada, lo que implica que todas las actividades deben de ser 
realizadas en función al orden establecida. Al final de cada fase se produce 
como resultado de salida uno o varios entregables, según la etapa.

\section{Estructura organizativa}
El personal requerido para este proyecto organizado por su jerarquía es el 
siguiente:
\begin{itemize}
\item Jefe de proyecto:- Dña. Adelaida de la Calle, rectora de la uma.
  Responsable: Dña. María Valpuesta, vicerrectorado de Innovación y Desarrollo 
  Tecnológico.
\item Coordinador D. Luis Muñoz Villarreal.
\item Grupo de trabajo D. Sergio de la Rubia García-Carpintero, D. Miguel 
  Millán Sánchez-Grande, Dña. Alicia Serrano Sánchez, D. Juan Miguel Torres 
  Triviño. 
\item A estos hay que añadir los empleados contratados: un analista, un 
  programador, un usuario experto, un secretario y un operario de servicio 
  técnico.
\end{itemize}

\section{Fronteras e interfaces organizativas}
Para realizar, probar y configurar el sistema resultante del proyecto se 
necesitará una estrecha colaboración con todos los departamentos que componen 
la UMA. 

\section{Responsabilidades}
En la tabla \ref{Tab:Respon} se muestran las competencias de cada miembro del
equipo.
\begin{table}[!h]
  \centering
  \small
  \begin{tabular}{p{2.3cm}|c|c|c|c|c|c|c}
    & \textbf{Coordinador} & \textbf{Grupo de trabajo} &
    \textbf{Analista} & \textbf{Programador} & \textbf{Secretario} &
    \textbf{Usuario experto} & \textbf{Operario servicio técnico}\\
    \hline \hline
    Gestión                     & x &   & x &   &   &   &    \\ 
    \hline
    Gestión de la configuración &   & x &   &   &   &   &    \\
    \hline
    Planificación               & x & x & x &   &   &   &    \\
    \hline
    Requisitos                  &   & x & x &   &   &   &    \\
    \hline
    Diseño                      &   & x &   &   &   &   &    \\
    \hline
    Programación                &   &   &   & x &   &   &    \\
    \hline
    Pruebas                     &   &   &   &   &   & x &    \\
    \hline
    Training                    &   &   &   &   &   & x &    \\
    \hline
    Instalación                 &   &   &   &   &   &   & x  \\
    \hline
    Documentación               &   &   &   &   & x &   &    \\
  \end{tabular}
  \caption{Responsabilidades de los miembros} \label{Tab:Respon}
\end{table}

\chapter{Procesos de gestión}
\section{Objetivos y prioridades de gestión}
Las prioridades del proyecto es mejorar y ampliar el uso y la calidad de la
herramienta software. Los objetivos son los siguientes:
\begin{itemize}
\item Prestar apoyo en materia de las TI a todas las actividades relacionadas
  con la investigación, la docencia y la gestión.
\item Agilidad en las consultas de los planes de estudio.
\item Rapidez y fiabilidad en la recogida de datos para completar la
  herramienta software.
\item Entrega rápida de cara al comienzo del curso 2010-2011. La consulta de
  los planes de estudio tiene una gran demanda en el mes de septiembre debido
  a las matriculaciones de las distintas carreras.
\item Se mejora considerablemente la facilidad de trabajo de todos los
  usuarios que utilizan la herramienta.
\item Reducir el gasto en la contratación de personal para elaborar guías
  docentes. 
\end{itemize}

\section{Supuestos, dependencias y restricciones}
El proyecto depende de tres planteamientos principales:
\begin{itemize}
\item La herramienta software.
\item Pruebas del software utilizado para ver su efectividad.
\item Habilidad de los usuarios tanto administradores como consultores para
  utilizar la herramienta con facilidad.
\end{itemize}
Pueden surgir problemas con alguno de estos plantemientos, en cuyo caso, la
ejecución del proyecto se vería afectada de cara a los plazos y condiciones
de entrega.

Las suposiciones que se tienen en cuenta son las siguientes:
\begin{itemize}
\item Los usuarios no tienen la experiencia necesaria para manejar el SI, por
  lo cual se necesita una adaptación al SI. 
\item Todos los usuarios tendrán la habilidad necesaria para manejar
  adecuadamente el sistema, ya que el hecho de que un usuario no pudiese
  utilizar al 100\% la efectividad del SI, decrementaría considerablemente su
  eficacia y no podría en ningún momento satisfacer los requisitos exigidos.
\item El proyecto debe de estar íntegramente terminado antes del comienzo del
  curso 2010-2011.
\end{itemize}

\section{Gestión de riesgos}
En la tabla \ref{Tab:GestRi} se muestran los riesgos de mayor impacto del
proyecto. 
\begin{table}[!h]
  \centering
  \begin{tabular}{c|c|c|c}
    \cline{2-4}
    & \textbf{Probabilidad} & \textbf{Impacto} &
    \textbf{Exposición al riesgo} \\
    \hline \hline
    Planificación optimista & 0,70 & 7 & 3,90 \\
    Cambios de los requisitos durante la ejecución del proyecto & 0,25 & 2 &
    0,50 \\
    Valorar la calidad & 0,35 & 2 & 0,70 \\
    Diferencias entre administradores y usuarios & 0,2 & 2 & 0,40 \\
    Retrasos en la entrega de la herramienta & 0,20 & 2 & 0,50 \\
    Valorar la implantación de la herramienta & 0,25 & 3 & 0,6 \\
    Personal no competente & 0,20 & 3 & 0,80 \\
  \end{tabular}
  \caption{Gestión de Riesgos} \label{Tab:GestRi}
\end{table}

\section{Mecanismos de supervisión y control}
El grupo de trabajo y el analista realizan la planificación del proyecto
junto con el coordinador Luis Muñoz Villarreal que también se encargará de
supervisar las tareas que realizan ambos.
Mediante las pruebas de la herramienta software, que realiza el usuario
experto, se realiza la supervisión del buen funcionamiento de ésta,
encontrando los posibles fallos y realizando los cambios oportunos. 
El coordinador lleva un control de las fechas de la entrega del proyecto

\section{Plan del personal}
El número de personas requerido para llevar a cabo el proyecto son 10:
coordinador, grupo de trabajo que lo forman 4 personas, analista,
programador, secretario, usuario experto y operario servicio técnico. 
El proyecto consta de varias etapas en las que intervienen distinto tipo de
personal. En un principio se necesita el trabajo del coordinador, el grupo
de trabajo y el analista para realizar la planificación del proyecto. Se
especifican los requisitos y diseño del sistema. A partir de lo anterior, se
progra el software y se realizan las pruebas pertienentes que llevarán a
tener una herramienta eficiente. Por último, se instala la herramienta.

\chapter{Procesos técnicos}
\section{Métodos, herramientas y técnicas}
Los métodos, herramientas y técnicas usados por el GACGD son los comúnmentes 
utilizados por los proyectos habituales de la UMA. El ciclo de vida básico es 
el de cascada. Tanto el análisis de requesitos como la programación se hacen 
de acuerdo a los métodos estructurados conocidos. Dado que el alcance de la 
UMA abarca más de una ciudad, se utiliza una WAN por sus posibilidades y 
porque cubre áreas muy amplias. Las herramientas utilizadas para la 
construcción del proyecto deben funcionar para la WAN y sus respectivos SSOO. 
El lenguaje de programación a utilizar va a ser Python, ya que proporciona 
facilidades a entornos como el del proyecto. El sistema operativo elegido es 
Debian GNU/Linux Lenny.

\section{Documentación del software}
Ya que se sigue un ciclo de vida en cascada, la documentación final del 
proyecto esta formada por cada uno de los siguientes documentos, generados en 
cada una de las etapas del modelo:

\begin{itemize}
\item Necesidades de usuario:
\item Requisitos de usuario:
\item Requisitos de desarrolladores:
\item Diseño de alto nivel:
\item Diseño de bajo nivel:
\item Desarrollo de unidades SW:
\item Prueba de unidades:
\item Integración y prueba de unidades:
\item Verificación de requisitos de desarrolladores:
\item Verificación de requisitos de usuario:
\item Demostración a usuarios:
\item Opciones:
\end{itemize}

\section{Funciones de soporte a proyectos}
El departamento de SI y los consultores proveen al proyecto de las debidas 
funciones de soporte. Estos se encargan de proponer las pruebas y sus 
procedimientos asociados. Estas actividades son las de validación y 
verificación. Además se encargan de aportar soporte técnico a aquellas áreas 
donde los miembros no tengan experiencia, éstas áreas son el análisis y diseño 
de la WAN y documentación del análisis de requerimientos.

\chapter{Paquetes de trabajo}
\section{Determinación del alcance del sistema}
\begin{itemize}
\item{Número} 1.1
\begin{itemize}
\item{Nombre} Definición del sistema
\item{Descripción} 
\item{Duración estimada} 
\end{itemize}
\item{Número} 1.2
\begin{itemize}
\item{Nombre} Establecimiento de requisitos
\item{Descripción} 
\item{Duración estimada} 
\end{itemize}
\item{Número} 1.3
\begin{itemize}
\item{Nombre} Identificación de subsistemas
\item{Descripción} 
\item{Duración estimada} 
\end{itemize}
\item{Número} 1.4.1
\begin{itemize}
\item{Nombre} Elaboración del modelo conceptual y lógica de datos
\item{Descripción}
\item{Duración estimada} 
\end{itemize}
\item{Número} 1.4.2
\begin{itemize}
\item{Nombre} Normalización
\item{Descripción}
\item{Duración estimada} 
\end{itemize}
\item{Número} 1.4.3
\begin{itemize}
\item{Nombre} Especificación de necesidades de carga inicial
\item{Descripción} 
\item{Duración estimada} 
\end{itemize}
\item{Número} 1.5
\begin{itemize}
\item{Nombre} Elaboración del modelo de procesos
\item{Descripción}
\item{Duración estimada} 
\end{itemize}
\item{Número} 1.6
\begin{itemize}
\item{Nombre} Definición de interfaz de usuario
\item{Descripción}
\item{Duración estimada} 
\end{itemize}
\item{Número} 1.7
\begin{itemize}
\item{Nombre} Análisis de consistencia y especificación de requisitos
\item{Descripción}
\item{Duración estimada} 
\end{itemize}
\item{Número} 1.8
\begin{itemize}
\item{Nombre} Especificación de plan de pruebas
\item{Descripción}
\item{Duración estimada} 
\end{itemize}
\item{Número} 1.9
\begin{itemize}
\item{Nombre} Aprobación del análisis del SI
\item{Descripción}
\item{Duración estimada} 
\end{itemize}\begin{itemize}
\item{Número} 1.1
\begin{itemize}
\item{Nombre} Definición del sistema
\item{Descripción} Determinación del alcance del sistema, de la tecnología
  que se va a usar, los estándares que se van a seguir para su  construcción
  teniendo en cuenta los usuarios a quienes va destinado.
\item{Duración estimada} 
\end{itemize}
\item{Número} 1.2
\begin{itemize}
\item{Nombre} Establecimiento de requisitos
\item{Descripción} Obtención, análisis y validación de los requisitos
  valiéndose de herramientas como los diagramas de casos de uso.
\item{Duración estimada} 
\end{itemize}
\item{Número} 1.3
\begin{itemize}
\item{Nombre} Identificación de subsistemas
\item{Descripción} Incluye la determinación de los distintos subsitemas y su
  posterior integración.
\item{Duración estimada} 
\end{itemize}
\item{Número} 1.4.1
\begin{itemize}
\item{Nombre} Elaboración del modelo conceptual y lógica de datos
\item{Descripción} Identifica y define las entidades que quedan dentro del
  SI, posteriormente se preparan las relaciones complejas y se eliminan
  redundancias y ambigüedades.
\item{Duración estimada} 
\end{itemize}
\item{Número} 1.4.2
\begin{itemize}
\item{Nombre} Normalización
\item{Descripción} Se revisa el modelo lógico de datos para eliminar
  redundancias e inconsistencias en las entidades de datos.
\item{Duración estimada} 
\end{itemize}
\item{Número} 1.4.3
\begin{itemize}
\item{Nombre} Especificación de necesidades de carga inicial
\item{Descripción} Incluye las necesidades hardware y estimaciones de
  capacidades.
\item{Duración estimada} 
\end{itemize}
\item{Número} 1.5
\begin{itemize}
\item{Nombre} Elaboración del modelo de procesos
\item{Descripción} Consiste en un análisis de las necesidades del usuario
  para establecer el conjunto de procesos del SI.
\item{Duración estimada} 
\end{itemize}
\item{Número} 1.6
\begin{itemize}
\item{Nombre} Definición de interfaz de usuario
\item{Descripción} Aquí se defininen las interfaces entre el sistema y el 
usuario: formatos de pantallas, diálogos, e informes, principalmente.

\item{Duración estimada} 
\end{itemize}
\item{Número} 1.7
\begin{itemize}
\item{Nombre} Análisis de consistencia y especificación de requisitos
\item{Descripción} Consiste en verificar la calidad técinca del modelo,
  acerciorandose de la coherencia entre modelos y del cumplimiento de los requisitos.
\item{Duración estimada} 
\end{itemize}
\item{Número} 1.8
\begin{itemize}
\item{Nombre} Especificación de plan de pruebas
\item{Descripción} Incluye la definición de el alcance y los requisitos del
  plan de pruebas.
\item{Duración estimada} 
\end{itemize}
\item{Número} 1.9
\begin{itemize}
\item{Nombre} Aprobación del análisis del SI
\item{Descripción} Consiste en la presentación y posterior aceptación del
  análisis del SI.
\item{Duración estimada} 
\end{itemize}
\item{Número} 2.1
\begin{itemize}
\item{Nombre} Definición de la arquitectura del sistema
\item{Descripción}
\item{Duración estimada} 
\end{itemize}
\item{Número} 2.2
\begin{itemize}
\item{Nombre} Diseño de la arquitectura de soporte
\item{Descripción}
\item{Duración estimada} 
\end{itemize}
\item{Número} 2.3.1
\begin{itemize}
\item{Nombre} Diseño de módulos del sistema
\item{Descripción}
\item{Duración estimada} 
\end{itemize}
\item{Número} 2.3.2
\begin{itemize}
\item{Nombre} Diseño de comunicación entre módulos
\item{Descripción}
\item{Duración estimada} 
\end{itemize}
\item{Número} 2.3.3
\begin{itemize}
\item{Nombre} Revisión de la interfaz de usuario
\item{Descripción}
\item{Duración estimada} 
\end{itemize}
\item{Número} 2.4.1
\begin{itemize}
\item{Nombre} Diseño del modelo físico de datos
\item{Descripción}
\item{Duración estimada} 
\end{itemize}
\item{Número} 2.4.2
\begin{itemize}
\item{Nombre} Especificación de los caminos de acceso a los datos
\item{Descripción}
\item{Duración estimada} 
\end{itemize}
\item{Número} 2.4.3
\begin{itemize}
\item{Nombre} Especificación de la distribución de datos
\item{Descripción}
\item{Duración estimada} 
\end{itemize}
\item{Número} 2.5
\begin{itemize}
\item{Nombre} Verificación y aceptación de la arquitectura del sistema
\item{Descripción}
\item{Duración estimada} 
\end{itemize}
\item{Número} 2.6
\begin{itemize}
\item{Nombre} Generación y especificación de construcción
\item{Descripción}
\item{Duración estimada} 
\end{itemize}
\item{Número} 2.7
\begin{itemize}
\item{Nombre} Diseño de migración y carga incial de datos
\item{Descripción}
\item{Duración estimada} 
\end{itemize}
\item{Número} 2.8
\begin{itemize}
\item{Nombre} Especificación técnica del plan de prueba
\item{Descripción}
\item{Duración estimada} 
\end{itemize}
\item{Número} 2.9
\begin{itemize}
\item{Nombre} Establecimiento de requisitos de implantación
\item{Descripción}
\item{Duración estimada} 
\end{itemize}
\item{Número} 2.10
\begin{itemize}
\item{Nombre} Aprobación del diseño y SI
\item{Descripción}
\item{Duración estimada} 
\end{itemize}
\item{Número} 3.1
\begin{itemize}
\item{Nombre} Preparación del entorno de generación y construcción
\item{Descripción} El objetivo es asegurar la disponibilidad de todos los
  medios y facilidades para que se pueda llevar a cabo la construcción del SI.
\item{Duración estimada} 
\end{itemize}
\item{Número} 3.2
\begin{itemize}
\item{Nombre} Generación del código de los componentes y los procedimientos
\item{Descripción} El objetvio es la codificación de los componentes del SI a
  partir de las especificaciones de contrucción en el proceso de diseño del SI.
\item{Duración estimada} 
\end{itemize}
\item{Número} 3.3
\begin{itemize}
\item{Nombre} Elaboración del manual de usuario
\item{Descripción} Elaboración de la documentación de usuario, tanto usuario
  final como de explotación, de acuerdo a los requisitos recogidos en el
  catálogo de requisitos.
\item{Duración estimada} 
\end{itemize}
\item{Número} 3.4
\begin{itemize}
\item{Nombre} Definición de la formación de los usuarios finales
\item{Descripción} Se establecen las necesidades de formación del usuario
  final, con el objetivo de conseguir la explotación eficaz del nuevo sistema.
\item{Duración estimada} 
\end{itemize}
\item{Número} 3.5
\begin{itemize}
\item{Nombre} Construcción de los componentes y procedimientos de carga 
inicial de datos
\item{Descripción} Codificación y prueba de los componentes y procedimientos
  de migración y carga inicial de datos, a partir de las especificaciones
  recogidas en el plan de migración y carga inicial de datos obtenido en el
  proceso de Diseño del SI.
\item{Duración estimada} 
\end{itemize}
\item{Número} 4.1
\begin{itemize}
\item{Nombre} Ejecución de las pruebas unitarias
\item{Descripción} Se realizan las pruebas unitarias de cada uno de los
  compomentes del SI, una vez codificados, con el objeto de comprobar que la
  estructura es correcta y se ajusta a su funcionalidad.
\item{Duración estimada} 
\end{itemize}
\item{Número} 4.2
\begin{itemize}
\item{Nombre} Ejecución de las pruebas de integración
\item{Descripción} Verificar si los componentes o subsistemas interactúan
  correctamente a través de sus interfaces, cubren la funcionalidad y se
  ajustan a los requisitos especificados.
\item{Duración estimada} 
\end{itemize}
\item{Número} 4.3
\begin{itemize}
\item{Nombre} Ejecución de las pruebas del sistema
\item{Descripción} Comprobación de la integración del sistema de información
  globalmente, verificando el funcionamiento correcto de las interfaces entre
  los distintos subsistemas que lo componen y con el resto de SI con los que
  se comunica.
\item{Duración estimada} 
\end{itemize}
\item{Número} 4.4
\begin{itemize}
\item{Nombre} Aprobación del SI
\item{Descripción} Se recopilan los productos de SI y se presentan al Jefe de
  Proyecto para su aprobación.
\item{Duración estimada} 
\end{itemize}
\end{itemize}

\bibliographystyle{plain} 
\bibliography{t1}

\end{document}
