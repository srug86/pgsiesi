% Clase
\documentclass[11pt,a4paper,spanish,twoside]{book}

% Órdenes auxiliares
\input{inc/includes.tex}

% Árboles de directorios
%\usepackage{dirtree}

% Encabezado y pie de página
\encabezado
\setcounter{secnumdepth}{3} 
\begin{document}

% Silabación extra
\hyphenation{
a-sig-na-tu-ras
au-to-ma-ti-za-rá
ca-tá-lo-go
ca-rre-ra
cons-truc-ción
co-rres-pon-de
diag-nos-tico
fi-na-li-za-ción
ge-ne-ra-ción
in-fe-rior
man-te-ni-mien-to
me-dian-te
per-so-nal
pro-ce-di-mien-tos
pro-por-cio-na-rá
pu-bli-ca-da
re-qui-si-tos
res-pecto
u-su-a-rios
vi-lla-rre-al
}


% Portada
\portada{Planificación y Gestión de\\Sistemas de Información}
{Práctica 1}{Elaboración de un plan de proyecto\\utilizando Microsoft Project}
{Sergio de la Rubia García-Carpintero\\Miguel Millán Sánchez-Grande\\
  Luis Muñoz Villarreal\\Alicia Serrano Sánchez\\
  Juan Miguel Torres Triviño}{30 de Abril de 2009}

% Licencia
\licencia{Sergio de la Rubia García-Carpintero, Miguel Millán Sánchez-Grande,
  Luis Muñoz Villarreal, Alicia Serrano Sánchez, Juan Miguel Torres Triviño}

\chapter*{Ficha de trabajo}
\begin{description}
\item[Código] P1
\item[Fecha] 30 de Abril de 2010
\item[Título] Elaboración de un plan de proyecto utilizando Microsoft Project
\end{description}

\begin{table}[!ht]
  \centering
  \begin{tabular}{lp{5cm}c}
    \multicolumn{3}{l}{\Large \textbf{Equipo} G4} \\ \\
    \multicolumn{1}{c}{\emph{Apellidos y nombre}} & 
    \multicolumn{1}{c}{\emph{Firma}} & \emph{Puntos} \\
    \hline \\
    de la Rubia García-Carpintero, Sergio & & 14 \\ \\
    Millán Sánchez-Grande, Miguel         & & 14 \\ \\
    Muñoz Villarreal, Luis                & & 14 \\ \\
    Serrano Sánchez, Alicia               & & 14 \\ \\
    Torres Triviño, Juan Miguel           & & 14 \\ \\
    \hline
  \end{tabular}
\end{table}

% Índices
\tableofcontents
\listoffigures
%\listoftables

%% INICIO DEL DOCUMENTO %%%%%%%%%%%%%%%%%%%%%%%%%%%%%%%%%%%%%%%%%%%%%%%%%
\chapter{Alcance}
En este capítulo se definen los diferentes procesos que hay que seguir y
cumplir para asegurar el éxito del proyecto. Las distintas tareas que
componen el proyecto deben realizarse satisfactoriamente en un plazo
definido.

\section{Esquemas de actividades y tareas}

\subsection{Análisis}
El objetivo principal de esta fase es la obtención de una especificación
detallada del proyecto que cumpla con los requisitos demandados por la 
universidad y sea la base del diseño posterior.

\subsubsection{Definición del sistema}
En esta tarea se realiza un exhaustivo análisis de las necesidades que se
acordaron en las reuniones con los directivos de la UMA. Además se realiza una
identificación del entorno tecnológico; uno de los requisitos vitales para el
desarrollo de este proyecto software.

\subsubsection{Establecimiento de requisitos}
Mediante sesiones de trabajo, se recoge información de los requisitos que debe
cumplir el software. A partir de estos requisitos y junto con la opinión del 
usuario experto se confirman cuáles son válidos, consistentes y completos.

\subsubsection{Identificación de subsistemas}
La descomposición del sistema en subsistemas está orientada a los procesos de
negocio. Estos subsistemas coinciden con el primer nivel de descomposición
del diagrama de flujo de datos. Se analizan los distintos modelos, para tener
una visión global y unificada de estos.

\subsubsection{Elaboración del modelo de datos}
Para realizar esta tarea, en primer lugar, se elabora un modelo conceptual y 
lógico de datos. Para el modelo conceptual, se identifican y definen las 
entidades del SI, los atributos de cada entidad, el dominio de cada atributo y 
las relaciones existentes entre las entidades. Seguidamente, se realiza una 
normalización del modelo lógico de datos. La técnica de normalización puede 
exigir la modificación de entidades, la creación de nuevas entidades y la 
reorganización de atributos. Por último, se realiza una especificación de las 
necesidades iniciales de información que requerirá la creación del proyecto.

\subsubsection{Elaboración del modelo de procesos}
Se analizan las necesidades del usuario para establecer el conjunto de procesos
que conforma el SI para cada uno de los subsistemas identificados.

\subsubsection{Definición de interfaz de usuario}
Esta actividad acomete el puente de información entre el sistema y el usuario, 
ocupándose de trabajos como el formato de pantalla o cuadros de diálogo. Se 
define el formato y contenido de cada una de las interfaces de pantalla, 
especificando su comportamiento dinámico.

\subsubsection{Análisis de consistencia y especificación de requisitos}
El objetivo de esta actividad es garantizar la calidad de los distintos modelos
generados a lo largo de todo el proceso de análisis.

\subsubsection{Especificación de plan de pruebas}
Se especifican y justifican los niveles de pruebas a realizar, así como el 
marco general de planificación de cada nivel de prueba. Además, se recopilan 
los requisitos relativos al entorno de pruebas y se inicia la definición de las
especificaciones necesarias para la correcta ejecución de las distintas
pruebas del sistema de información. Los criterios de aceptación deben
ser definidos de forma clara, prestando especial atención a aspectos como
procesos críticos del sistema, rendimiento del sistema, seguridad y 
disponibilidad.

\subsubsection{Aprobación del análisis del SI}
En esta tarea se realiza la presentación del análisis del sistema de
información a la Dirección, para la aprobación final del mismo.

\subsection{Diseño}
El Diseño de Sistema de Información (DSI) tiene como objetivo la definición
de la arquitectura del sistema y del entorno tecnológico que lo soportará. 
Además se buscará dar una especificación detallada de los componentes del 
sistema de información.

\subsubsection{Definición de la arquitectura del sistema}
La decisión del software y hardware que se utiliza es fundamental. Se ha
seleccionado como software Red Hat Enterprise Linux 5 de acuerdo a las
expectativas de crecimiento y a los servicios que se quieren ofrecer. Se
van a utilizar 6 ordenadores con el objetivo de que todos los que participan
en el desarrollo del software puedan realizar su trabajo de manera 
independiente. 
 
\subsubsection{Diseño de la arquitectura de soporte}
En esta actividad se llevará a cabo la especificación de la arquitectura de 
soporte, que comprende el diseño de los subsistemas de soporte y la 
determinación de los mecanismos genéricos de diseño. Una de las prioridades 
será intentar aprovechar al máximo los posibles sistemas de apoyo ya existentes
aunque, no obstante, no habrá más remedio que conseguir, por ejemplo, un 
servidor; ya que este constituye el material de soporte básico que se utilizará
para la implementación y el funcionamiento del sistema.

\subsubsection{Diseño de la arquitectura de módulos del sistema}
En esta tarea se definen los módulos del SI y la manera en la que van a 
interactuar unos con otros, intentando que cada módulo trate total o 
parcialmente un proceso específico y tenga una interfaz sencilla y amigable.

\subsubsection{Diseño físico de datos}
A partir del modelo lógico de datos normalizado, esta tarea
obtiene como salida el diseño del modelo físico de datos, teniendo en cuenta 
las características específicas de nuestro SGBD. Además, también se 
determinarán los caminos de acceso a dichos datos por parte de cada uno de los 
módulos de la aplicación, de acuerdo al modelo físico de datos, con el fin de 
conseguir optimizar el tiempo de respuesta y el consumo de recursos. Por 
último, se establecerá la ubicación del gestor de las bases de datos, en este
caso el servidor central, así como los elementos de la estructura física de 
datos de acuerdo a la organización del SI.

\subsubsection{Verificación y aceptación de la arquitectura del sistema}
Esta tarea consiste en garantizar la calidad de las especificaciones del diseño
del SI, así como su viabilidad.

\subsubsection{Generación y especificación de construcción}
En esta tarea se generan, a partir del diseño detallado del sistema, las 
especificaciones que definen las unidades básicas de construcción que 
compondrán el SI.

\subsubsection{Diseño de migración y carga inicial de datos}
Esta tarea consta de una especificación del entorno, un diseño de 
procedimientos y un diseño detallado de los componentes de la carga inicial. 
Primero, se realiza una estimación de la capacidad necesaria para albergar 
toda la información del sistema. Con ello podremos evaluar las necesidades a 
nivel de infraestructuras que serán demandadas para tal fin. Seguidamente, se 
procede a la definición de los procedimientos necesarios para sacar adelante 
la carga inicial de datos del sistema. Estos procedimientos estan relacionados 
con la preparación, realización y verificación de la carga inicial de datos. 
Por último, se detalla la jerarquía y orden de la ejecución de cada uno de los 
módulos de la carga inicial.

\subsubsection{Especificación técnica del plan de prueba}
En esta tarea se especifican en detalle los planes de prueba del SI para cada 
uno de los niveles de prueba establecidos.

\subsubsection{Establecimiento de requisitos de implantación}
En esta tarea se recopila toda la información necesaria para que en la etapa de 
implementación se elaboren los manuales de usuario y otros apartados en la 
documentación. También, se determinan los conocimientos y/o aptitudes
adicionales que necesitan dominar los usuarios finales para trabajar con el
sistema desarrollado. Además, se recogen requisitos de equipamiento hardware,
software y comunicaciones que requiera el nuevo sistema.  

\subsubsection{Aprobación del diseño y SI}
La última tarea de esta fase consiste en una presentación del diseño de este SI
al jefe de proyecto para que este realice la revisión pertinente y lo apruebe.

\subsection{Implementación}
En esta fase se genera el código de los componentes software del SI y se 
habilitan las infraestructuras necesarias para la gestión y mantenimiento de 
dichos componentes. Además, también se elaboran los manuales de usuario y se 
especifican las necesidades de formación que dichos usuarios deben poseer 
para conseguir una explotación eficaz del nuevo sistema.

\subsubsection{Preparación del entorno de generación y construcción}
En esta etapa se comprueba la disponibilidad de las infraestructuras físicas 
y lógicas necesarias para llevar a cabo la construcción del SI. Además, también
se recopilan los recursos necesarios que aún no se poseen, como el servidor 
central y su software (Red Hat Enterprise Linux 5).

\subsubsection{Generación del código de los componentes y los 
procedimientos}
En esta tarea se genera, de acuerdo con el diseño acordado en la fase 
anterior, el código de los componentes y los procedimientos que componen el SI.
Posteriormente, se depuran dichos códigos, con el objetivo de obtener una 
herramienta robusta y eficiente.

\subsubsection{Elaboración del manual de usuario}
Esta tarea consiste en la elaboración de una completa documentación que
posibilite a los usuarios, primero conocer y más tarde dominar, las principales
funcionalidades de la herramienta software; y que describa los procedimientos 
de actuación necesarios para realizar un correcto mantenimiento del sistema.

\subsubsection{Definición de la formación de los usuarios finales}
En esta tarea se establecen las necesidades de formación del usuario final, con
objeto de conseguir una explotación eficaz del nuevo sistema. Con esto, se 
generará una lista de prerrequisitos que deben cumplir los nuevos usuarios y 
se redactará un pequeño manual y una presentación que sirva para dar a conocer 
el sistema a dichos usuarios.

\subsubsection{Construcción de los componentes y procedimientos de carga
  inicial de datos}
En esta tarea se codifican y prueban los componentes y procedimientos que 
intervienen en la migración de los datos que se van a reutilizar (como los 
referidos a los de las bases de datos actuales) a las nuevas bases de datos 
que se han diseñado, no solo para realmacenar la información ya existente en
unas bases de datos más potentes acondicionadas a las nuevas necesidades, sino 
también para completarlas con nuevos campos (entre otras cosas para adaptar las
asignaturas al nuevo plan de estudios). Posteriormente, se procederá a la carga 
de datos según lo estipulado en la fase de diseño.

\subsection{Pruebas}
En esta fase se realiza la preparación del entorno, y la realización y 
evaluación de las pruebas unitarias, de integración y del sistema. En esta 
parte será de vital importancia la tarea del usuario experto, pues es él, 
ayudado por el coordinador, el que se encargará de la ejecución y evaluación de
dichas pruebas.

\subsubsection{Ejecución de las pruebas unitarias}
En esta tarea se realizan las pruebas unitarias de cada uno de los componentes 
del SI, con objeto de comprobar que su estructura es la correcta y que se 
ajusta a la funcionalidad establecida. Para ello, primero se realiza una 
preparación del entorno, donde quedarán aprobados todos los recursos necesarios
para la realización de dichas pruebas, tales como librerías, casos de uso, etc.
Una vez conseguidos dichos recursos, se procederá a la realización y evaluación
de las pruebas. Si los resultados no fueran los esperados, habría que proceder 
a efectuar las correcciones necesarias.

\subsubsection{Ejecución de las pruebas de integración}
En este caso, el objetivo se centra en verificar si los componentes o 
subsistemas del SI interactúan correctamente a través de sus interfaces, 
cubriendo la funcionalidad establecida y ajustándose a los requisitos 
especificados en las verificaciones correspondientes. Al igual que ocurre en la
tarea anterior, primero se procede a la preparación del entorno de pruebas, 
consiguiendo los recursos necesarios para llevar a cabo dichas pruebas; y 
posteriormente se procede a la realización y evaluación de las mismas.

\subsubsection{Ejecución de las pruebas del sistema}
Estas pruebas tienen como objetivo comprobar la integración del SI globalmente,
verificando el funcionamiento correcto de las interfaces entre los distintos 
subsistemas y, entre estos y el resto del SI con los que se comunica. Como 
ocurre en la ejecución de las pruebas anteriores, aquí también se realizará 
primero una preparación del entorno de pruebas, y después se procederá a la 
realización y evaluación de las pruebas.

\subsubsection{Aprobación del SI}
Para realizar esta tarea, a parte del coordinador y el usuario experto, se une 
la labor del secretario. Esta tarea consiste en la recopilación de los 
productos que componen el SI y la presentación al jefe de proyecto del mismo, 
con el objetivo de obtener la aprobación definitiva.

\subsection{Tareas repetitivas}
  \begin{itemize}
  \item Reunión del grupo de trabajo. Esta tarea es semanal y comenzará desde
    el inicio hasta el fin del proyecto.
  \end{itemize}

\section{Hitos}
Se considera que cada fase del proyecto no puede empezar sin que haya
terminado la anterior. Para ello, los hitos tienen lugar en la terminación de
cada fase. La implantación de estos hitos ayudan al cumplimiento de los
plazos establecidos.

Los hitos del proyecto son los siguientes:
\begin{itemize}
\item Tarea 14: Análisis aprobado.
\item Tarea 32: Diseño aprobado.
\item Tarea 39: Implementación aprobada.
\item Tarea 45: Proyecto finalizado.
\end{itemize}

\section{Vinculaciones de tareas}
En la tabla \ref{Tab:Dep} se pueden ver las vinculaciones entre las tareas.
\begin{table}[!h]
  \centering
  \small
  \begin{tabular}{l|p{5cm}|l}
    \textbf{Tarea predecesora} & \textbf{Tarea actual} & \textbf{Dependencia}\\
    \hline
    Definición del Sistema & Establecimiento de requisitos & FC \\
    \hline
    Establecimiento de requisitos & Identificación de subsistemas & FC \\
    \hline
    \multirow{3}{*}{Identificación de subsistemas} & Elaboración del modelo
    conceptual y logíca de datos & \multirow{3}{*}{FC} \\
    & Normalización \\
    & Especificación de necesidades de carga inicial \\
  \end{tabular}
  \caption{Dependencias entre tareas} \label{Tab:Dep}
\end{table}

\section{Tiempos de posposición y adelanto}
En nuestro proyecto la mayor parte de paquetes de trabajo poseen dependencias
con tiempo de posposición y adelanto nulo, por circunstancia de que nuestro
proyecto se desarrolla dentro de un marco principalmente secuencial.

Las tareas con vinculación de tiempo de adelanto son:
\begin{description}
\item[Tarea 2.4.1] Diseño del modelo físico de datos $\to$ +2 días
\item[Tarea 2.4.2] Especificación de los caminos de acceso a los datos $\to$
  +2 días
\end{description}

Las tareas con vinculación de tiempo de posposición son:
\begin{description}
\item[Tarea 2.4.2] Especificación de los caminos de acceso a los datos $\to$
  -1 días
\item[Tarea 2.4.3] Especificación de la distribución de datos $\to$ -1 días
\end{description}

\section{Visualización del camino crítico}
El camino crítico que sigue el proyecto es el recorrido formado por las
siguientes tareas:

\begin{description}
\item[Tarea 1.1] Definición del sistema.
\item[Tarea 1.2] Establecimiento de requisitos.
\item[Tarea 1.3] Identificación de subsistemas.
\item[Tarea 1.4.1] Elaboración del modelo conceptual y lógica de datos.
\item[Tarea 1.4.2] Normalización.
\item[Tarea 1.4.3] Especificación de necesidades de carga inicial.
\item[Tarea 1.8] Especificación del plan de pruebas.
\item[Tarea 1.9] Aprobación del análisis del SI.
\item[Tarea 2.1] Definición de la arquitectura del sistema.
\item[Tarea 2.2] Diseño de la arquitectura de soporte.
\item[Tarea 2.3.1] Diseño de módulos del sistema.
\item[Tarea 2.3.2] Diseño de comunicación entre módulos.
\item[Tarea 2.3.3] Revisión de la interfaz de usuario.
\item[Tarea 2.5] Verificación y aceptación de la arquitectura del sistema.
\item[Tarea 2.6] Generación y especificación de construcción.
\item[Tarea 2.7] Diseño de migración y carga inicial de datos.
\item[Tarea 2.8] Especificación técnica del plan de prueba.
\item[Tarea 2.9] Establecimiento de requisitos de implantación.
\item[Tarea 2.10] Aprobación del diseño y SI.
\item[Tarea 3.1] Preparación del entorno de generación y construcción.
\item[Tarea 3.2] Generación del código de los componentes y procedimientos.
\item[Tarea 4.1] Ejecución de las pruebas unitarias.
\item[Tarea 4.2] Ejecución de las pruebas de integración.
\item[Tarea 4.3] Ejecución de las pruebas del sistema.
\item[Tarea 4.4] Aprobación del SI.
\end{description}


\chapter{Recursos y costes}
Los recursos, tanto materiales como humanos, se utilizan para completar las
tareas de las que se compone el proyecto.


\section{Lista de recursos humanos y materiales y asignaciones a 
  tareas}

\subsection{Recursos humanos}
\begin{description}
\item[Coordinador] es aquella persona responsable de un proyecto. Supervisa y
  controla el trabajo de las personas del proyecto, así como el cumplimiento
  de los plazos de entrega de las distintas tareas. 

\item[Analista] es aquel individuo responsable de investigar y
  recomendar opciones de software y sistemas para cumplir los requerimientos
  de una empresa de negocios.  

\item[Programador] es aquel que escribe, depura y mantiene el código fuente
  de la aplicación.  

\item[Secretario] es aquella persona que redacta los informes, organiza la
  información relacionada con el proyecto, planifica las reuniones, \dots

\item[Operario de servicio técnico] realiza labores de instalación y
  mantenimiento de los recursos materiales disponibles.

\item[Usuario experto] realiza pruebas a un alto nivel de especificación.

\item[Miembro del grupo de trabajo] se encarga de recopilar los requisitos
  iniciales de la aplicación, realizar entrevistas, supervisar el trabajo del
  personal  contratado, conseguir los recursos necesarios y la contabilidad.  

\end{description}

\subsection{Recursos materiales}
\begin{description}
\item[Sala de juntas] Se utiliza para las reuniones semanales, así como para 
las juntas extraordinarias donde se tratan las aprobaciones de las distintas 
etapas del proyecto.
\item[Proyector] Se utiliza en las reuniones de la sala de juntas.
\item[Pizarra interactiva] Se utilizan en las etapas de análisis y diseño, 
donde cada miembro del equipo de trabajo puede aportar sus ideas y facilita la 
asimilación de conceptos.
\item[Impresora láser color]
\item[Servidor central] Los datos del proyecto están centralizados para que 
todos los miembros del equipo de trabajo tengan acceso de manera rápida y 
eficiente.
\item[Red Hat Enterprise Linux 5] Software para el servidor.
\item[Equipo informático] Se cuenta con 6 unidades de ordenadores para todo el 
desarrollo del proyecto.  
\end{description}

\section{Definición de costes por uso}
El único recurso que tiene costes por uso es el Operario de Servicio Técnico,
que se le pagará 100\euro\ cada vez que se requieran sus servicios.

\section{Definición de costes fijos de actividad}
Se estipula que los gastos se iban justificando a medida que avanzaba el
proyecto. Por lo tanto, no se incluyeron en la planificación inicial del
proyecto costes fijos.

\section{Tablas variables de costo}
Estas tablas se atienden en la \autoref{sec:tasas}

\section{Disponibilidad variable de un recurso}
En la planificación existe el recurso ``Operador de servicio técnico'' que se
contrata de manera temporal.

\section{Tablas variables de tasas de costos} \label{sec:tasas}
Para el recurso \emph{Analista} existen dos tablas de costo, una primera en
la que tiene un sueldo como \emph{analista} y una segunda en la que tiene un
sueldo como \emph{diseñador}, con los siguientes costes:

\begin{enumerate}
\item \textbf{Tabla A} (Analista)
  \\Tasa estándar $\to$ 38.534\euro/año
  \\Hora extra $\to$ 30,58\euro/hora
\item \textbf{Tabla B} (Diseñador) 
  \\Tasa estándar $\to$ 32.047\euro/año
  \\Hora extra $\to$ 27\euro/hora
\end{enumerate}

\section{Aplicación de distintas tablas de tasas de costo en tareas}
La distribución del apartado \ref{sec:tasas} se aplican del siguiente modo:
\begin{itemize}
\item El analista trabajará como ``Analista'' y, por lo tanto, se aplicará la
  \emph{Tabla A} del apartado anterior en las tareas en las que éste
  participe y pertenezcan al grupo de tareas del análisis.
\item El analista trabajará como ``Diseñador'' y, por lo tanto, se aplicará
  la \emph{Tabla B} del apartado anterior en las tareas en las que éste
  participe y pertenezcan al grupo de tareas del diseño.
\end{itemize}

\chapter{Calendarios}
\section{Calendarios generales de recursos humanos}
Por el siguiente calendario general se rigen los recursos \emph{analista},
\emph{programador}, \emph{secretario}, \emph{usuario experto} y
\emph{miembros del grupo de trabajo}:

\begin{description}
\item[Lunes - Viernes] \hfill
  \begin{itemize}
  \item Mañanas: 9h - 13h
  \item Tardes: 15h - 19h
  \end{itemize}
\end{description}

\subsection{Días festivos}
El calendario general descrito en el apartado anterior se ve afectado por los
siguientes días festivos:

\begin{description}
\item[Virgen de Alarcos] 24 de Mayo
\item[Comunidad de Castilla-La Mancha] 31 de Mayo
\item[Día de la Asunción de la Virgen] 15 de Agosto
\end{description}

\section{Calendarios específicos de recursos humanos}
El recurso \emph{coordinador} tiene un calendario específico que se rige por
el calendario general con la excepción que se muestra a continuación:

\begin{description}
\item[Viernes] \hfill
  \begin{itemize}
  \item Mañanas: 9h - 14h
  \end{itemize}
\end{description}


\chapter{Redistribución del proyecto}
\newpage
\section[Informe de sobreasignaciones]
{Informe de sobreasignaciones de recursos y de su resolución}
A continuación, las imagenes \ref{Rec_sob1}, \ref{Rec_sob2} incluye el
informe que Microsoft Project genera sobre los recursos sobreasignados.

\imagen{Recursosobre1}{13}{Informe de Recursos Sobreasignados 1}{Rec_sob1}
\imagen{Recursosobre2}{13}{Informe de Recursos Sobreasignados 2}{Rec_sob2}

\subsection{Gantt de redistribución}
La redistribución ha sido totalmente automática a través de los algoritmos
que proporciona \emph{Microsoft Project 2003}.

\section{Delimitaciones de tareas}
% Incluir al menos 2 tipos de delimitaciones de Tareas y analizar sus efectos
% en la programación de proyecto.
Se han introducido dos ejemplos para probar los efectos en la programación
del proyecto. Las tareas seleccionadas para ello son: 
\begin{description}
\item [Tarea 2.3.3 Revisión de la interfaz de usuario].
 Se ha aplicado la delimitación inflexible de \textbf{No
   finalizar antes del 15 de Julio del 2010}. El resultado de esta acción
 produce una postposición tal que la tarea finaliza dicho día. Esto ocasiona
 un retraso del camino crítico del proyecto.

\item [Tarea 2.4.3 Especificación de la distribución de datos].
 Se aplica una distribución flexible consistente en que la tarea
 termine \textbf{Lo más tarde posible.} Debido a la vinculación fin-comienzo
 con la tarea 2.3.3 y su grupo de tareas, la tarea 2.4.3 termina a la vez que
 la 2.3.3.
\end{description}

\chapter{Alternativas al plan de proyecto}
% * Nota: Establecer una fecha de fin y considerar un determinado coste por día
% de retraso.
\newpage
Se ha programado el proyecto a partir de una fecha de fin, 17 de Agosto de
2010, la cual provocará una serie de repercusiones que modifican nuestra
planificación actual: 
\begin{itemize}
\item Fecha de comienzo del proyecto: 16 de Abril de 2010.
\item No permite resolver las sobreasignaciones.
\item El coste no es afectado.
\item La duración del proyecto completo, sin redistribuir, se ve aumentada
  levemente.
\end{itemize}

En la imagen \ref{ProyFe} se reflejan las estadísticas más importantes del
proyecto reprogramado a partir de dicha fecha de fin.
\imagen{resumenproyectofin}{12}{Resumen del proyecto a partir de una fecha de
  fin}{ProyFe}

\chapter{Seguimiento simulado del proyecto}
Para realizar el seguimiento simulado de nuestro proyecto, se guarda una línea 
de base, a partir de la cual obtienen las estadísticas reflejadas en la figura 
\ref{ResPL}.

\imagen{resumenproyectolinbase.pdf}{12.5}{Resumen del proyecto a partir de la 
línea base}{ResPL}

\section{Introducción de duraciones reales y restantes}
Se han realizado dos ejemplos: uno sobre la introducción de una duración
real, que al ser la tarea actualizada nos proporciona la duración restante; y
un segundo sobre la introducción de una duracción real y restante.

\subsection{Primer Ejemplo}
\begin{description}
\item[Tarea 1.1] Definición del sistema \hfill
  Duración real: 3 días
\end{description}
\subsection{Segundo Ejemplo}
\begin{description}
\item[Tarea 1.4.2] Normalización \hfill
  Duración real: 1'43 días
  Duración restante: 1 día
\end{description}

\section{Introducción de un porcentaje completado}
Se ha realizado el ejemplo siguiente: la introducción de una porcentaje que 
recalculará su duración real y restante.

\begin{description}
\item[Tarea 2.3.3] Revisión de la interfaz de usuario
  Porcentaje completado: 85\%
  Efectos que produce: duración real (4'02 días) y duración restante 
(0'71 días).
\end{description}

\section{Introducción del trabajo real}
El ejemplo de esta sección se ha realizado sobre la \emph{tarea 2.3.3}, 
revisión de la interfaz de usuario. Para ver los efectos del trabajo real, se 
observa la vista "uso de tareas'', en la cual se verán celdas vacías (21 de 
julio del 2010) y celdas con valores de trabajo incompleto (20 de julio del 
2010). En el segundo caso, se ve una diferencia de 0'20 horas por realizar para
conseguir completar el trabajo. Esta diferencia puede verse también en los 
tiempos de los diferentes recursos de la tarea.

\section{Actualizar el resto del proyecto según la programación.}
Se ha realizado una actualización del proyecto sobre el día 20 de julio del 
2010. Se observan los efectos producidos en la figura \ref{ResPA}, que contiene
las estadísticas más importantes del proyecto hasta ese momento.

\imagen{resumenproyectoactualizado.pdf}{12.5}{Resumen del plan de proyecto 
actualizado}{ResPA}

\chapter{Informes}
\section{Vista resumen del plan del proyecto}
Esta sección incluye la imagen \ref{Res} con un resumen del proyecto, dónde
se pueden encontrar diversos datos como por ejemplo: la duración, el coste,
las horas de trabajo, \...

\imagen{Resumen.pdf}{12.5}{Resumen del Plan del Proyecto}{Res}

\section{Diagrama de Gantt}
Las imágenes \ref{ga1}, \ref{ga2}, \ref{ga3} y \ref{ga4} contienen el
diagrama de gant del proyecto.

\begin{sidewaystable}
\imagen{gant1.pdf}{20}{Primera Parte del Diagrama de Gantt}{ga1}
\end{sidewaystable}

\begin{sidewaystable}
\imagen{gant2.pdf}{20}{Segunda Parte del Diagrama de Gantt}{ga2}
\end{sidewaystable}

\begin{sidewaystable}
\imagen{gant3.pdf}{20}{Tercera Parte del Diagrama de Gantt}{ga3}
\end{sidewaystable}

\begin{sidewaystable}
\imagen{gant4.pdf}{20}{Cuarta Parte del Diagrama de Gantt}{ga4}
\end{sidewaystable}

\section{Diagrama de Gant con camino crítico}
Las imágenes \ref{CC1}, \ref{CC2}, \ref{CC3} y \ref{CC4} corresponden con el
camino crítico de las tareas a seguir en el proyecto.

\begin{sidewaystable}
\imagen{caminocritico1.pdf}{20}{Primera Parte del camino Crítico}{CC1}
\end{sidewaystable}

\begin{sidewaystable}
\imagen{caminocritico2.pdf}{20}{Segunda Parte del camino Crítico}{CC2}
\end{sidewaystable}

\begin{sidewaystable}
\imagen{caminocritico3.pdf}{20}{Tercera Parte del camino Crítico}{CC3}
\end{sidewaystable}

\begin{sidewaystable}
\imagen{caminocritico4.pdf}{20}{Cuarta Parte del camino Crítico}{CC4}
\end{sidewaystable}

\section{Informe general de recursos}
La imagen \ref{Rec} incluye una tabla con todos los recursos utilizados en el
proyecto, así como sus características y costos.
\begin{sidewaystable}
\imagen{Recursos.pdf}{20}{Informe General de Recursos del Sistema}{Rec}
\end{sidewaystable}

\section{Costes}

\subsection{Costes por actividades}
La imagen \ref{Cos_Ac} incluye los respectivos costes de cada una de las
actividades básicas del proyecto. 

\begin{sidewaystable}
\imagen{costeactividades.pdf}{20}{Informe de los Costes de las
  Actividades}{Cos_Ac}
\end{sidewaystable}

\section{Informe de redistribución}
Las imágenes \ref{Gan_re1}, \ref{Gan_re2}, \ref{Gan_re3} y \ref{Gan_re4}
incluyen los diagramas Gant del proyecto una vez redistribuido.

\begin{sidewaystable}
\imagen{gantredistribuido1.pdf}{20}{Primera Parte del Informe de
  Redistribución}{Gan_re1} 
\end{sidewaystable}

\begin{sidewaystable}
\imagen{gantredistribuido2.pdf}{20}{Segunda Parte del Informe de
  Redistribución}{Gan_re2} 
\end{sidewaystable}

\begin{sidewaystable}
\imagen{gantredistribuido3.pdf}{20}{Tercera Parte del Informe de
  Redistribución}{Gan_re3} 
\end{sidewaystable}

\begin{sidewaystable}
\imagen{gantredistribuido4.pdf}{20}{Cuarta Parte del Informe de
  Redistribución}{Gan_re4} 
\end{sidewaystable}

\section{Informe de seguimiento}


\bibliographystyle{plain} 
\bibliography{p1}

\end{document}

