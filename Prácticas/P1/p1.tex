% Clase
\documentclass[11pt,a4paper,spanish,twoside]{report}

% Órdenes auxiliares
\input{inc/includes.tex}

% Encabezado y pie de página
\encabezado

\begin{document}

% Silabación extra
\hyphenation{
a-sig-na-tu-ras
au-to-ma-ti-za-rá
ca-tá-lo-go
ca-rre-ra
cons-truc-ción
co-rres-pon-de
diag-nos-tico
fi-na-li-za-ción
ge-ne-ra-ción
in-fe-rior
man-te-ni-mien-to
me-dian-te
per-so-nal
pro-ce-di-mien-tos
pro-por-cio-na-rá
pu-bli-ca-da
re-qui-si-tos
res-pecto
u-su-a-rios
vi-lla-rre-al
}


% Portada
\portada{Planificación y Gestión de\\Sistemas de Información}
{Práctica 1}{Elaboración de un plan de proyecto\\utilizando Microsoft Project}
{Sergio de la Rubia García-Carpintero\\Miguel Millán Sánchez-Grande\\
  Luis Muñoz Villarreal\\Alicia Serrano Sánchez\\
  Juan Miguel Torres Triviño}{30 de Abril de 2009}

% Licencia
\licencia{Sergio de la Rubia García-Carpintero, Miguel Millán Sánchez-Grande,
  Luis Muñoz Villarreal, Alicia Serrano Sánchez, Juan Miguel Torres Triviño}

\chapter*{Ficha de trabajo}
\begin{description}
\item[Código] P1
\item[Fecha] 30 de Abril de 2010
\item[Título] Elaboración de un plan de proyecto utilizando Microsoft Project
\end{description}

\begin{table}[!ht]
  \centering
  \begin{tabular}{lp{5cm}c}
    \multicolumn{3}{l}{\Large \textbf{Equipo} G4} \\ \\
    \multicolumn{1}{c}{\emph{Apellidos y nombre}} & 
    \multicolumn{1}{c}{\emph{Firma}} & \emph{Puntos} \\
    \hline \\
    de la Rubia García-Carpintero, Sergio & & 14 \\ \\
    Millán Sánchez-Grande, Miguel         & & 14 \\ \\
    Muñoz Villarreal, Luis                & & 14 \\ \\
    Serrano Sánchez, Alicia               & & 14 \\ \\
    Torres Triviño, Juan Miguel           & & 14 \\ \\
    \hline
  \end{tabular}
%  \caption{}\label{}
\end{table}

% Índices
\tableofcontents
% \listoffigures
% \listoftables

%% INICIO DEL DOCUMENTO %%%%%%%%%%%%%%%%%%%%%%%%%%%%%%%%%%%%%%%%%%%%%%%%%
\chapter{Alcance}
\section{Esquemas de actividades y tareas}

\section{Hitos}

\section{Vinculaciones de tareas (FC, CC, FF)}

\section{Tiempos de posposición y adelanto}

\section{Visualización del camino crítico}

\chapter{Recursos y costes}
\section{Lista de recursos humanos y materiales y asignaciones a 
  tareas}

\section{Definición de costes por uso}

\section{Definición de costes fijos de actividad}

\section{Tablas variables de costo}

\section{Disponibilidad variable de un recurso}

\section{Tablas variables de tasas de costos}

\section{Aplicación de distintas tablas de tasas de costo en tareas}

\chapter{Calendarios}
\section{Calendarios generales de recursos humanos}

\section{Calendarios específicos de recursos humanos}

\chapter{Redistribución del proyecto}
\section{Informe de sobre-asignaciones de recursos y de su 
  resolución}
\subsection{Gantt de redistribución}

\section{Incluir al menos 2 tipos de delimitaciones de tareas y 
  analizar sus efectos en la programación de proyecto}

\chapter{Alternativas al plan evaluando su repercusión en coste y 
  calendario}
* Nota: Establecer una fecha de fin y considerar un determinado coste por día
de retraso.

\chapter{Seguimiento simulado del proyecto incluyendo como 
  mínimo un ejemplo de las siguientes acciones}
\section{Introducción de duraciones reales y restantes}

\section{Introducción de un porcentaje completado}

\section{Introducción del trabajo real}

\section{Actualizar el resto del proyecto según la programación.}

\chapter{Informes}
\section{Vista resumen del plan del proyecto}

\section{Diagrama de Gantt}

\section{Diagrama de Gant con camino crítico}

\section{Informe general de recursos}

\section{Costes}
\subsection{Costes por recursos}
\subsection{Costes por actividades}

\section{Informe de redistribución}

\section{Informe de seguimiento}


\bibliographystyle{plain} 
\bibliography{p1}

\end{document}
