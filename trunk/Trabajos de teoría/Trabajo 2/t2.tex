% Clase
\documentclass[11pt,a4paper,spanish,twoside]{report}

% Órdenes auxiliares
\input{inc/includes.tex}

% Encabezado y pie de página
\encabezado

\begin{document}

% Silabación extra
\hyphenation{
a-sig-na-tu-ras
au-to-ma-ti-za-rá
ca-tá-lo-go
ca-rre-ra
cons-truc-ción
co-rres-pon-de
diag-nos-tico
fi-na-li-za-ción
ge-ne-ra-ción
in-fe-rior
man-te-ni-mien-to
me-dian-te
per-so-nal
pro-ce-di-mien-tos
pro-por-cio-na-rá
pu-bli-ca-da
re-qui-si-tos
res-pecto
u-su-a-rios
vi-lla-rre-al
}


% Portada
\portada{Planificación y Gestión de\\Sistemas de Información}
{Trabajo 1}{Plan de Sistemas y Tecnologías de Información}
{Sergio de la Rubia García-Carpintero\\Miguel Millán Sánchez-Grande\\
  Luis Muñoz Villarreal\\Alicia Serrano Sánchez\\
  Juan Miguel Torres Triviño}{10 de Marzo de 2009}

% Licencia
\licencia{Sergio de la Rubia García-Carpintero, Miguel Millán Sánchez-Grande,
  Luis Muñoz Vi\-lla\-rre\-al, Alicia Serrano Sánchez, Juan Miguel Torres 
Triviño}

\chapter*{Ficha de trabajo}
\begin{description}
\item[Código] T1
\item[Fecha] 10 de Marzo de 2010
\item[Título]Plan de Sistemas y Tecnologías de Información
\end{description}

\begin{table}[!ht]
  \centering
  \begin{tabular}{lp{5cm}c}
    \multicolumn{3}{l}{\Large \textbf{Equipo} G4} \\ \\
    \multicolumn{1}{c}{\emph{Apellidos y nombre}} & 
    \multicolumn{1}{c}{\emph{Firma}} & \emph{Puntos} \\
    \hline \\
    de la Rubia García-Carpintero, Sergio & & 10 \\ \\
    Millán Sánchez-Grande, Miguel         & & 10 \\ \\
    Muñoz Villarreal, Luis                & & 10 \\ \\
    Serrano Sánchez, Alicia               & & 10 \\ \\
    Torres Triviño, Juan Miguel           & & 10 \\ \\
    \hline
  \end{tabular}
%  \caption{}\label{}
\end{table}

% Índices
\tableofcontents
% \listoffigures
% \listoftables

%% INICIO DEL DOCUMETO %%%%%%%%%%%%%%%%%%%%%%%%%%%%%%%%%%%%%%%%%%%%%%%%%
\chapter*{Introducción}
\chapter{Integración del proyecto}
\section{Desarrollo del plan de proyecto}
\section{Ejecución del plan de proyecto}
\section{Control integrado de los cambios}
\chapter{Alcance del proyecto}
El plazo estimado de entrega del proyecto se fija entre 3 y 4 meses a cumplir
en Julio de 2010, a partir del comienzo de la ejecución del proyecto. La
fecha de entrega del proyecto se establece con el objetivo de utilizar dicho
software para el curso 2010-2011. 
\section{Análisis del proyecto}
\subsection{Establecimiento de requisitos}
En esta etapa se recogen los requisitos necesarios que debe cumplir el
software teniendo en cuenta las necesidades actuales de la UMA y los usuarios
que interactúan con esta herramienta. 
El objetivo de esta actividad es obtener un catálogo detallado de los
requisitos, a partir del cual se pueda comprobar que los productos generados
en las actividades de modelización se ajustan a los requisitos del usuario.
En la definición de los requisitos hay que tener en cuenta las prioridades
que se deben tener en cuenta según los criterios de los usuarios
acerca de las funcionalidades a cubrir. 
\subsection{Identificación de subsistemas de análisis}
El objetivo de esta actividad es facilitar el análisis del sistema de
información llevando a cabo la descomposición del sistema en diversos
subsistemas. 
Cada subsistema corresponde con una de las etapas del proceso.
\subsection{Elaboración del modelo de datos}

\subsection{Elaboración del modelo de procesos}
\subsection{Definición de interfaces de usuario}
\subsection{Análisis de consistencia y especificación de requisitos}
\subsection{Especificación del plan de pruebas}
\subsection{Aprobación del análisis del sistema de información}
\section{Diseño del proceso}
\subsection{Definición de la arquitectura del sistema}
\subsection{Diseño de la arquitectura de soporte}
\subsection{Diseño de la arquitectura de módulos del sistema}
\subsection{Diseño físico de datos}
\subsection{Verificación y aceptación de la arquitectura del sistema}
\subsection{Generación de especificaciones de construcción}
\subsection{Diseño de la migración y carga inicial de datos}
\subsection{Especificación técnica del plan de pruebas}
\subsection{Establecimiento de los requisitos de implantación}
\subsection{Presentación y aprobación del diseño del sistema de información}
\section{Construcción y prueba del proyecto}
\subsection{Preparación del entorno de generación y construcción}
\subsection{Generación del código de los componentes y procedimientos}
\subsection{Ejecución de las pruebas unitarias}
\subsection{Ejecución del las pruebas de integración}
\subsection{Ejecución de las pruebas del sistema}
\subsection{Elaboración de los manuales de usuario}
\subsection{Diseño de la migración y carga inicial de datos}
\subsection{Construcción de los componentes y procedimientos de carga inicial 
de datos}
\subsection{Aprobación del diseño del sistema de información}

\bibliographystyle{plain} 
\bibliography{t1}

\end{document}
