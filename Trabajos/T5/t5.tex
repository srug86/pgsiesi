% Clase
\documentclass[11pt,a4paper,spanish,twoside]{book}

% Órdenes auxiliares
\input{inc/includes.tex}

% Encabezado y pie de página
\encabezado

\begin{document}

% Silabación extra
\hyphenation{
a-sig-na-tu-ras
au-to-ma-ti-za-rá
ca-tá-lo-go
ca-rre-ra
cons-truc-ción
co-rres-pon-de
diag-nos-tico
fi-na-li-za-ción
ge-ne-ra-ción
in-fe-rior
man-te-ni-mien-to
me-dian-te
per-so-nal
pro-ce-di-mien-tos
pro-por-cio-na-rá
pu-bli-ca-da
re-qui-si-tos
res-pecto
u-su-a-rios
vi-lla-rre-al
}


% Portada
\portada{Planificación y Gestión de\\Sistemas de Información}
{Trabajo 5}{Estimación del software}
{Sergio de la Rubia García-Carpintero\\Miguel Millán Sánchez-Grande\\
  Luis Muñoz Villarreal\\Alicia Serrano Sánchez\\
  Juan Miguel Torres Triviño}{24 de Mayo de 2010}

% Licencia
\licencia{Sergio de la Rubia García-Carpintero, Miguel Millán Sánchez-Grande,
  Luis Muñoz Villarreal, Alicia Serrano Sánchez, Juan Miguel Torres Triviño}

\chapter*{Ficha de trabajo}
\begin{description}
\item[Código] T5
\item[Fecha] 24 de Mayo de 2010
\item[Título] Estimación del software
\end{description}

\begin{table}[!ht]
  \centering
  \begin{tabular}{lp{5cm}c}
    \multicolumn{3}{l}{\Large \textbf{Equipo} G4} \\ \\
    \multicolumn{1}{c}{\emph{Apellidos y nombre}} & 
    \multicolumn{1}{c}{\emph{Firma}} & \emph{Puntos} \\
    \hline \\
    de la Rubia García-Carpintero, Sergio & & 8 \\ \\
    Millán Sánchez-Grande, Miguel         & & 8 \\ \\
    Muñoz Villarreal, Luis                & & 8 \\ \\
    Serrano Sánchez, Alicia               & & 8 \\ \\
    Torres Triviño, Juan Miguel           & & 8 \\ \\
    \hline
  \end{tabular}
\end{table}

% Índices
\tableofcontents
\listoftables
%\listoffigures

%% INICIO DEL DOCUMENTO %%%%%%%%%%%%%%%%%%%%%%%%%%%%%%%%%%%%%%%%%%%%%%%%%
\chapter*{Introducción}
%breve introducción del software llamado...

\section{Visión general}

\section{Objetivos}

\chapter{Identificación de los módulos}
\section{Gestión de usuarios}
\subsection{Entradas externas}
\subsubsection{Identificación del usuario}
Cada usuario debe identificarse mediante un nombre y una contraseña.

\subsubsection{Preferencia de horarios} 
Cada profesor introduce sus preferencias de horarios de trabajo para que el
sistema lo tenga en cuenta a la hora de la realización del horario oficial
del curso.
\subsection{Salidas externas}
\subsubsection{Datos personales}
Muestra los datos personales de un determinado profesor, incluyendo
nombre, correo electrónico, teléfono, publicaciones, currículum, \dots
\subsubsection{Horario de un profesor}
El sistema muestra el horario que tiene asignado un determinado profesor.

\subsubsection{Tutorías de los profesores} 
Muestra los horarios de tutorías junto con el despacho asignado de los
profesores de cada facultad.
 
\subsubsection{Asignaturas impartidas por cada profesor}
El sistema muestra la información de cada asignatura que comprende:
\begin{itemize}
\item Profesor/es que la imparte/n.
\item Créditos.
\item Contenidos de la asignatura.
\item Asignaturas recomendadas.
\item Página web de la asignatura.
\item Planificación docente.
\item Sistema de evaluación.
\end{itemize}

\subsection{Consultas externas}
\subsubsection{Consultar currículum}
Información del currículum de un determinado profesor.

\subsubsection{Consultar datos de asignatura} 
Información sobre una determinada asignatura (créditos, profesor que la
imparte, planificación, \dots
\subsubsection{Horarios de asignaturas}
Horarios de las asignaturas de cada curso. Cada horario
también muestra información de la localización en la que cada asignatura se
imparte. 

\subsection{Archivos lógicos de interfaz externos}
\subsubsection{Datos de los profesores}
El sistema tiene una base de datos donde se guarda toda la información con
respecto a los profesores (nombre, apellidos, teléfono, correo electrónico,
asignaturas que imparte, \dots)

\subsubsection{Datos de las asignaturas}
En la base de datos de la universidad se tienen los datos de las asignaturas,
tales como nombre, créditos, guión de la asignatura, contenidos, evaluación,
\dots 

\section{Gestión de cursos}
\subsection{Entradas externas}
\subsubsection{Añadir curso}
Entrada destinada a la inclusión de los distintos cursos correspondientes al
plan de estudios en el sistema.

\subsubsection{Añadir asignaturas}
Entrada generada para la inserción de las diferentes asignaturas asociadas a
cada curso.

\subsubsection{Añadir aulas}
Entrada destinada a la introducción las aulas disponibles destinadas al
correspondiente plan de estudio.

\subsubsection{Añadir profesores}
Entrada generada para la incluir los profesores encargados de la docencia de
cada asignatura.

\subsection{Salidas externas}
\subsubsection{Relación asignaturas/profesores}
Información 
\subsubsection{Relación cursos/aulas}
\subsubsection{Cuadrante de horarios}

\subsection{Consultas externas}
\subsubsection{Consultar relación asignaturas/profesores}
\subsubsection{Consultar relación cursos/aulas}
\subsubsection{Consultar cuadrante de horarios}

\subsection{Archivos lógicos internos}
\subsubsection{Datos de las asignaturas}
\subsubsection{Datos de las cursos}
\subsubsection{Datos de las aulas}

\subsection{Archivos lógicos de interfaz externos}
\subsubsection{Historial de guías docentes}

\section{Interfaz}
\subsection{Entradas externas}
\subsubsection{Autentificación de los usuarios}
Recoge los datos necesarios para ver si un usuario entra al sistema,
dependiendo si éste es administrador o no.

\subsubsection{Introducción de datos para la inserción/modificación/borrado 
de información de un determinado usuario}
Por su similitud se explican juntas pero equivale a tres \emph{funciones de
usuario} diferentes. El objetivo de estas funciones es recoger los datos
necesarios para la de inserción, modificación o borrado de la información de
un usuario en la aplicación.

\subsubsection{Introducción de datos para la inserción/modificación/borrado 
de información de un determinado curso }
Por su similitud se explican juntas pero equivale a tres \emph{funciones de
usuario} diferentes. El objetivo de estas funciones es recoger los datos
necesarios para la de inserción, modificación o borrado de la información de
un curso en la aplicación.

\subsection{Salidas externas}
\subsubsection{Mostar los datos para una búsqueda de información con
posibilidad de ordenación por diferentes criterios}
El objetivo de este proceso es mostrar un listado de usuarios que cumplen las
condiciones de búsqueda solicitadas y en el orden indicado.

\subsubsection{Mostrar los datos de un determinado usuario}
El objetivo de este proceso es mostrar información detallada referente a un
determinado usuario (profesores).

\subsubsection{Mostrar los datos de un determinado curso}
El objetivo de este proceso es mostrar información detallada referente a un
curso, incluyendo horario, aulas, asignaturas, etc.

\subsection{Consultas externas}
\subsubsection{Búsqueda de información con posibilidad de ordenación por
  diferentes criterios}
El objetivo de este proceso es la introdución de unos criterios para realizar 
una búsqueda detallada tanto de cursos como de usuarios.

\subsubsection{Consulta de los datos de un determinado usuario}
El objetivo de esta función es la consulta simple de la información referente
a un usuario (profesor).

\subsubsection{Consulta de los datos de un determinado curso}
El objetivo de esta función es la consulta simple de la información referente
a un curso.

\subsection{Archivos lógicos externos}
\subsubsection{Importación de datos desde un fichero}
El objetivo de esta función es introducir los datos necesarios para importar 
inserciones, modificaciones o borrados desde un fichero de texto.

\subsubsection{Exportación de los resultados de la búsqueda a un fichero}
El objetivo de esta función es introducir los datos para obtener de la base
de datos de la aplicación el listado de usuarios que cumplen las condiciones de
búsqueda solicitadas y en el orden indicado en un fichero de texto.

\subsection{Archivos lógicos de interfaz externos}
No se ha detectado ningún archivo lógico de interfaz externo.

\chapter{Determinación de la complejidad de las funciones de los 
  módulos}
\chapter{Tablas de puntos función sin ajustar}
\chapter{Explicación de los valores de los factores de influencia}
\chapter{Cálculo de los puntos función ajustados}

\end{document}
