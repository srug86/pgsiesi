% Clase
\documentclass[11pt,a4paper,spanish,twoside]{book}

% Órdenes auxiliares
\input{inc/includes.tex}

% Encabezado y pie de página
\encabezado

\begin{document}

% Silabación extra
\hyphenation{
a-sig-na-tu-ras
au-to-ma-ti-za-rá
ca-tá-lo-go
ca-rre-ra
cons-truc-ción
co-rres-pon-de
diag-nos-tico
fi-na-li-za-ción
ge-ne-ra-ción
in-fe-rior
man-te-ni-mien-to
me-dian-te
per-so-nal
pro-ce-di-mien-tos
pro-por-cio-na-rá
pu-bli-ca-da
re-qui-si-tos
res-pecto
u-su-a-rios
vi-lla-rre-al
}


% Portada
\portada{Planificación y Gestión de\\Sistemas de Información}
{Trabajo 4}{Calendario del proyecto}
{Sergio de la Rubia García-Carpintero\\Miguel Millán Sánchez-Grande\\
  Luis Muñoz Villarreal\\Alicia Serrano Sánchez\\
  Juan Miguel Torres Triviño}{10 de Mayo de 2010}

% Licencia
\licencia{Sergio de la Rubia García-Carpintero, Miguel Millán Sánchez-Grande,
  Luis Muñoz Villarreal, Alicia Serrano Sánchez, Juan Miguel Torres Triviño}

\chapter*{Ficha de trabajo}
\begin{description}
\item[Código] T4
\item[Fecha] 10 de Mayo de 2010
\item[Título] Calendario del proyecto
\end{description}

\begin{table}[!ht]
  \centering
  \begin{tabular}{lp{5cm}c}
    \multicolumn{3}{l}{\Large \textbf{Equipo} G4} \\ \\
    \multicolumn{1}{c}{\emph{Apellidos y nombre}} & 
    \multicolumn{1}{c}{\emph{Firma}} & \emph{Puntos} \\
    \hline \\
    de la Rubia García-Carpintero, Sergio & & 4 \\ \\
    Millán Sánchez-Grande, Miguel         & & 4 \\ \\
    Muñoz Villarreal, Luis                & & 4 \\ \\
    Serrano Sánchez, Alicia               & & 4 \\ \\
    Torres Triviño, Juan Miguel           & & 4 \\ \\
    \hline
  \end{tabular}
\end{table}

% Índices
\tableofcontents
\listoftables
%\listoffigures

%% INICIO DEL DOCUMENTO %%%%%%%%%%%%%%%%%%%%%%%%%%%%%%%%%%%%%%%%%%%%%%%%%
\chapter*{Introducción}
En este documento se dan a conocer los riesgos que el equipo de trabajo ha
considerado más significativos entre los que se pueden dar en este proyecto y
algunos métodos para disminuir sus efectos negativos.

Un \emph{riesgo} es un evento que, en caso de ocurrir, tiene un efecto
positivo o negativo sobre los objetivos de un proyecto. Por tanto, este
documento incluye un plan de riesgos con los más probables de este
proyecto, clasificándolos según su importancia.

De dicha lista se  seleccionan los cinco riesgos más importantes, para los
que se ha realizado un plan de respuestas particularizado, incluyendo su 
descripción, los aspectos del proyecto afectados, las responsabilidades 
asignadas, los resultados del análisis del riesgo, el plan de contingencia,
el nivel de riesgos residual, las acciones específicas y el presupuesto y 
tiempos para las respuestas.

\chapter{Selección de riesgos}
Para determinar la lista de riesgos que se tratarán se ha utilizado una
adaptación de la técnica \emph{Delphi}. Los riesgos han sido seleccionados de
la lista de comprobación de riesgos (\emph{checklist}) publicada en el libro 
\emph{Connell, S. Desarrollo y Gestión de Proyectos Informáticos. McGraw-Hill
Iberoamericana, 1997}.

\section{Técnica de selección de riesgos}
La técnica de selección de riesgos utilizada consta en los siguientes pasos:
\begin{enumerate}
\item Cada experto selecciona inicialmente veinte riesgos de entre todos los que
  forman la lista.
\item Se hace una comparativa de todos los riesgos escogidos y se seleccionan
  aquellos que hayan sido elegidos por al menos la mitad de los expertos.
\item Los riesgos que no alcanzaron la mitad de votos, pero obtuvieron alguno se
  incluyen en una lista de posibles riesgos elegibles. 
\item Debatir entre todos los expertos qué riesgos de la lista de posibles se 
han de incluir en los escogidos. 
\item Cada experto realiza votación secreta eligiendo si dicho posible riesgo 
debe estar en la lista de escogidos.
\item Añadir a la lista de escogidos los riesgos que más votos reciban sin 
llegar a sobrepasar el máximo de riesgos a elegir.
\item Los riesgos escogidos son los que forman la lista definitiva.
\end{enumerate}

\section{Riesgos resultantes}
Después de aplicar la técnica \emph{Delphi} adaptada, la lista de riesgos
seleccionada corresponde con la tabla \ref{Tab:tar_sel}.
\begin{table}[!ht]
  \centering
  \begin{tabular}{|p{3.5cm}|c|p{6.5cm}|}
    \hline
    \textbf{Categoría} & \textbf{Riesgo} & \textbf{Descripción} \\
    \hline\hline
    \multirow{2}{3.5cm}{Elaboración de la planificación}
    & A.7 & El esfuerzo es mayor que el estimado (por líneas de código,
    número de puntos función, módulos, etc). \\  
    \cline{2-3}
    & A.11 & Un retraso en una tarea produce retrasos en cascada en las
    tareas dependientes. \\
    \hline \hline
    \multirow{2}{3cm}{Usuarios finales}
    & D.1 & Los usuarios finales insisten en nuevos requisitos\\
    \cline{2-3}
    & D.4 & No se ha solicitado información al usuario, por lo que el producto
    al final  no se ajusta a las necesidades del usuario, y hay que volver a
    crear el producto.\\
    \hline \hline
    Cliente & E.1 & El cliente insiste en nuevos requisitos.\\
    \hline \hline
    Personal contratado & F.1 & El personal contratado no suministra los
    componentes en el periodo establecido.\\
    \hline \hline
    Requisitos & G.3 & Se añaden requisitos extra.\\
    \hline \hline
    Producto & H.9 &  Los requisitos para crear interfaces con otros
    sistemas, otros sistemas complejos, u otros sistemas que no están bajo el
    control del equipo de desarrollo suponen un diseño, implementación y
    prueba no previstos.\\ 
    \hline \hline
    \multirow{2}{3.5cm}{Personal}
    & J.12 & La incorporación de nuevo personal de desarrollo al proyecto ya
    avanzado, y el aprendizaje y comunicaciones extra imprevistas reducen la
    eficiencia de los miembros del equipo existentes.\\ 
    \cline{2-3}
    & J.22 & El personal trabaja más lento de lo esperado\\
    \hline \hline
    \multirow{1}{3.5cm}{Diseño e implementación}
    & K.3 & Un mal diseño implica volver a diseñar e implementar.\\
   \hline
  \end{tabular}
  \caption{Tareas seleccionadas}
  \label{Tab:tar_sel}
\end{table}

\chapter{Probabilidad de ocurrencia y magnitud de pérdida} 
En las tablas \ref{Tab:DELPHImag} y \ref{Tab:DELPHIpro}, de las páginas
\pageref{Tab:DELPHImag} y \pageref{Tab:DELPHIpro}, respectivamente; se  
puede observar la evolución seguida a la hora de calcular la probabilidad de
ocurrencia y la magnitud de pérdida con la técnica \emph{Delphi}. Además, en
las tablas \ref{Tab:ResDELPHImag} y \ref{Tab:ResDELPHIpro}, de la página
\pageref{Tab:ResDELPHIpro}, se puede encontrar los resúmenes de estos procesos.

Como leyenda, cada tabla contiene el identificador de cada riesgo (R), la
iteración (I) en la que se encuentra, las valoraciones de los distintos
expertos, así como los valores mínimo (m), medio ($\bar{x}$) (que corresponde
con la media aritmética) y máximo (M), y finalmente, si se cumple el criterio
de convergencia o no.

\begin{table}[!h]
  \centering
  \begin{tabular}{|c|c||c|c|c|c|c||c|c|c||c|}
    \hline
    \textbf{R} & \textbf{I} & \textbf{1} & \textbf{2} & 
    \textbf{3} & \textbf{4} & \textbf{5} & \textbf{m} &
    \textbf{$\bar{x}$} & \textbf{M} & \textbf{C} \\
    \hline \hline

    \hline \multirow{4}{*}{A.7} 
    & 1 & 5  & 12 & 8  & 17 & 23 & 5  & 13   & 23 & N \\
    & 2 & 9  & 13 & 10 & 18 & 19 & 9  & 13.8 & 19 & N \\
    & 3 & 11 & 13 & 12 & 19 & 15 & 11 & 14   & 19 & N \\
    & \textbf{4} & \textbf{12} & \textbf{13} & \textbf{13} & \textbf{15} &
    \textbf{14} & \textbf{12} & \textbf{13.4} & \textbf{15} & \textbf{S} \\ 

    \hline \multirow{3}{*}{A.11} 
    & 1 & 31 & 10 & 15 & 6  & 14 & 6  & 15.2 & 31 & N \\
    & 2 & 15 & 12 & 19 & 25 & 20 & 12 & 18.2 & 25 & N \\
    & \textbf{3} & \textbf{15} & \textbf{15} & \textbf{20} & \textbf{22} &
    \textbf{20} & \textbf{15} & \textbf{18.4} & \textbf{22} & \textbf{S} \\ 
    
    \hline \multirow{3}{*}{D.1} 
    & 1 & 10 & 10 & 9  & 6  & 18 & 6  & 10.6 & 18 & N \\
    & 2 & 10 & 12 & 10 & 12 & 15 & 10 & 11.8 & 15 & N \\
    & \textbf{3} & \textbf{10} & \textbf{12} & \textbf{10} & \textbf{13} &
    \textbf{14} & \textbf{10} & \textbf{11.8} & \textbf{14} & \textbf{S} \\ 
    
    \hline \multirow{5}{*}{D.4} 
    & 1 & 15 & 18 & 16 & 23 & 33 & 15 & 21   & 33 & N \\
    & 2 & 15 & 25 & 20 & 27 & 15 & 15 & 20.4 & 27 & N \\
    & 3 & 16 & 23 & 22 & 24 & 18 & 16 & 20.6 & 24 & N \\
    & 4 & 16 & 22 & 23 & 23 & 17 & 16 & 20.2 & 23 & N \\
    & \textbf{5} & \textbf{15} & \textbf{20} & \textbf{20} & \textbf{20} &
    \textbf{15} & \textbf{15} & \textbf{18} & \textbf{20} & \textbf{S} \\
    
    \hline \multirow{5}{*}{E.1} 
    & 1 & 12 & 10 & 6  & 5  & 19 & 5  & 10.4 & 19 & N \\
    & 2 & 12 & 8  & 13 & 7  & 17 & 7  & 11.4 & 17 & N \\
    & 3 & 10 & 10 & 10 & 10 & 15 & 10 & 11   & 15 & N \\
    & 4 & 10 & 10 & 12 & 11 & 14 & 10 & 11.4 & 14 & N \\
    & \textbf{5} & \textbf{10} & \textbf{11} & \textbf{12} & \textbf{12} &
    \textbf{13} & \textbf{10} & \textbf{11.6} & \textbf{13} & \textbf{S} \\ 
    
    \hline \multirow{3}{*}{F.1} 
    & 1 & 8 & 6  & 4 & 14 & 5 & 4 & 7.4 & 14 & N \\
    & 2 & 8 & 11 & 9 & 6  & 7 & 6 & 8.2 & 11 & N \\
    & \textbf{3} & \textbf{8} & \textbf{10} & \textbf{9} & \textbf{8} &
    \textbf{8} & \textbf{8} & \textbf{8.6} & \textbf{10} & \textbf{S} \\ 
    
    \hline \multirow{3}{*}{G.3} 
    & 1 & 7  & 10 & 7 & 13 & 7 & 7 & 8.8 & 13 & N \\
    & 2 & 10 & 9  & 7 & 8  & 7 & 7 & 8.2 & 10 & N \\
    & \textbf{3} & \textbf{10} & \textbf{9} & \textbf{7} & \textbf{9} &
    \textbf{8} & \textbf{7} & \textbf{8.6} & \textbf{10} & \textbf{S} \\ 
    
    \hline \multirow{2}{*}{H.9} 
    & 1 & 13 & 14 & 11 & 11 & 4 & 4 & 10.6 & 14 & N \\
    & \textbf{2} & \textbf{12} & \textbf{12} & \textbf{10} & \textbf{12} &
    \textbf{9} & \textbf{9} & \textbf{11} & \textbf{12} & \textbf{S} \\ 

    \hline \multirow{3}{*}{J.12} 
    & 1 & 6 & 9 & 5 & 7 & 8 & 5 & 7   & 9 & N \\
    & 2 & 7 & 6 & 7 & 8 & 5 & 5 & 6.6 & 8 & N \\
    & \textbf{3} & \textbf{7} & \textbf{6} & \textbf{7} & \textbf{8} &
    \textbf{6} & \textbf{6} & \textbf{6.8} & \textbf{8} & \textbf{S} \\ 

    \hline \multirow{2}{*}{J.22} 
    & 1 & 10 & 8 & 14 & 8 & 10 & 8 & 10 & 14 & N \\
    & \textbf{2} & \textbf{11} & \textbf{8} & \textbf{8} & \textbf{12} &
    \textbf{11} & \textbf{8} & \textbf{10} & \textbf{12} & \textbf{S} \\ 

    \hline \multirow{3}{*}{K.3} 
    & 1 & 20 & 19 & 23 & 36 & 19 & 19 & 23.4 & 36 & N \\
    & 2 & 23 & 21 & 30 & 22 & 25 & 21 & 24.2 & 30 & N \\
    & \textbf{3} & \textbf{23} & \textbf{22} & \textbf{22} & \textbf{29} &
    \textbf{25} & \textbf{22} & \textbf{24.2} & \textbf{29} & \textbf{S} \\ 
    \hline
  \end{tabular}
  \caption{\emph{Delphi} de la magnitud de pérdida (impacto)}
  \label{Tab:DELPHImag}
\end{table}

\begin{table}[!h]
  \centering
  \begin{tabular}{|c|c||c|c|c|c|c||c|c|c||c|}
    \hline
    \textbf{R} & \textbf{I} & \textbf{1} & \textbf{2} & 
    \textbf{3} & \textbf{4} & \textbf{5} & \textbf{m} &
    \textbf{$\bar{x}$} & \textbf{M} & \textbf{C} \\
    \hline \hline

    \hline \multirow{3}{*}{A.7} 
    & 1 & 70 & 50 & 65 & 25 & 98 & 25 & 61.6 & 98 & N \\
    & 2 & 50 & 50 & 70 & 65 & 75 & 50 & 62   & 75 & N \\
    & \textbf{3} & \textbf{50} & \textbf{50} & \textbf{70} & \textbf{60} &
    \textbf{70} & \textbf{50} & \textbf{60} & \textbf{70} & \textbf{S} \\ 
    
    \hline \multirow{4}{*}{A.10} 
    & 1 & 80 & 40 & 90 & 25 & 80 & 25 & 63 & 90 & N \\
    & 2 & 65 & 45 & 70 & 70 & 75 & 45 & 65 & 75 & N \\
    & 3 & 55 & 50 & 70 & 65 & 75 & 50 & 63 & 75 & N \\
    & \textbf{4} & \textbf{55} & \textbf{55} & \textbf{70} & \textbf{65} &
    \textbf{75} & \textbf{55} & \textbf{64} & \textbf{75} & \textbf{S} \\ 

    \hline \multirow{4}{*}{D.1} 
    & 1 & 20 & 20 & 35 & 15 & 35 & 15 & 25 & 35 & N \\
    & 2 & 35 & 20 & 20 & 30 & 20 & 20 & 25 & 35 & N \\
    & 3 & 20 & 35 & 20 & 30 & 20 & 20 & 25 & 35 & N \\
    & \textbf{4} & \textbf{20} & \textbf{30} & \textbf{20} & \textbf{30} &
    \textbf{25} & \textbf{20} & \textbf{25} & \textbf{30} & \textbf{S} \\ 

    \hline \multirow{3}{*}{D.4} 
    & 1 & 25 & 5  & 55 & 5  & 15 & 5  & 21 & 55 & N \\
    & 2 & 25 & 15 & 15 & 20 & 35 & 15 & 22 & 35 & N \\
    & \textbf{3} & \textbf{25} & \textbf{20} & \textbf{25} & \textbf{20} &
    \textbf{25} & \textbf{20} & \textbf{23} & \textbf{25} & \textbf{S} \\ 
    
    \hline \multirow{4}{*}{E.1} 
    & 1 & 20 & 15 & 70 & 40 & 50 & 15 & 39 & 70 & N \\
    & 2 & 50 & 65 & 20 & 35 & 40 & 20 & 42 & 65 & N \\
    & 3 & 55 & 55 & 30 & 35 & 40 & 30 & 43 & 55 & N \\
    & \textbf{4} & \textbf{55} & \textbf{55} & \textbf{40} & \textbf{40} &
    \textbf{50} & \textbf{40} & \textbf{48} & \textbf{55} & \textbf{S} \\ 

    \hline \multirow{3}{*}{F.1} 
    & 1 & 35 & 40 & 20 & 10 & 32 & 10 & 27.4 & 40 & N \\
    & 2 & 30 & 30 & 25 & 17 & 20 & 17 & 24.4 & 30 & N \\
    & \textbf{3} & \textbf{25} & \textbf{30} & \textbf{30} & \textbf{20} &
    \textbf{20} & \textbf{20} & \textbf{25} & \textbf{30} & \textbf{S} \\ 

    \hline \multirow{3}{*}{G.3} 
    & 1 & 20 & 65 & 65 & 30 & 25 & 20 & 41 & 65 & N \\
    & 2 & 40 & 30 & 50 & 55 & 35 & 30 & 42 & 55 & N \\
    & \textbf{3} & \textbf{45} & \textbf{40} & \textbf{45} & \textbf{50} &
    \textbf{35} & \textbf{35} & \textbf{43} & \textbf{50} & \textbf{S} \\ 

    \hline \multirow{5}{*}{H.9} 
    & 1 & 30 & 5  & 40 & 10 & 3  & 3  & 17.6 & 40 & N \\
    & 2 & 6  & 25 & 10 & 8  & 20 & 6  & 13.8 & 25 & N \\
    & 3 & 10 & 20 & 11 & 13 & 17 & 10 & 14.2 & 20 & N \\
    & 4 & 10 & 15 & 11 & 12 & 17 & 10 & 13   & 17 & N \\
    & \textbf{5} & \textbf{12} & \textbf{15} & \textbf{13} & \textbf{15} &
    \textbf{17} & \textbf{12} & \textbf{14.4} & \textbf{17} & \textbf{S} \\ 
    
    \hline \multirow{5}{*}{J.12} 
    & 1 & 15 & 15 & 5  & 5  & 34 & 5  & 14.8 & 34 & N \\
    & 2 & 14 & 9  & 25 & 13 & 10 & 9  & 14.2 & 25 & N \\
    & 3 & 12 & 13 & 20 & 15 & 12 & 12 & 14.4 & 20 & N \\
    & 4 & 13 & 17 & 20 & 15 & 14 & 13 & 15.8 & 20 & N \\
    & \textbf{5} & \textbf{16} & \textbf{17} & \textbf{18} & \textbf{15} &
    \textbf{14} & \textbf{14} & \textbf{16} & \textbf{18} & \textbf{S} \\ 
    
    \hline \multirow{5}{*}{J.22} 
    & 1 & 60 & 10 & 30 & 50 & 75 & 10 & 45   & 75 & N \\
    & 2 & 50 & 40 & 50 & 60 & 25 & 25 & 45   & 60 & N \\
    & 3 & 45 & 40 & 50 & 55 & 35 & 35 & 45   & 55 & N \\
    & 4 & 45 & 40 & 50 & 52 & 35 & 35 & 44.4 & 52 & N \\
    & \textbf{5} & \textbf{45} & \textbf{40} & \textbf{50} & \textbf{50} &
    \textbf{40} & \textbf{40} & \textbf{45} & \textbf{50} & \textbf{S} \\ 
    
    \hline \multirow{4}{*}{K.3} 
    & 1 & 20 & 5  & 40 & 1  & 50 & 1  & 23.2 & 50 & N \\
    & 2 & 40 & 18 & 20 & 25 & 10 & 10 & 22.6 & 40 & N \\
    & 3 & 30 & 19 & 22 & 23 & 20 & 19 & 22.8 & 30 & N \\
    & \textbf{4} & \textbf{21} & \textbf{20} & \textbf{22} & \textbf{23} &
    \textbf{20} & \textbf{20} & \textbf{21.2} & \textbf{23} & \textbf{S} \\ 
    \hline
   \end{tabular}
  \caption{\emph{Delphi} de la probabilidad de ocurrencia}
  \label{Tab:DELPHIpro}
\end{table}

\begin{table}[!h]
\centering
  \begin{tabular}{|c||c|c|c|c|c||c|c|c|}
    \hline
    \textbf{R} & \textbf{1} & \textbf{2} & \textbf{3} & \textbf{4} &
    \textbf{5} & \textbf{m} &\textbf{$\bar{x}$} &\textbf{M} \\
    \hline \hline
    A.7  & 12 & 13 & 13 & 15 & 14 & 12 & 13.4 & 15 \\ \hline
    A.11 & 15 & 15 & 20 & 22 & 20 & 15 & 18.4 & 22 \\ \hline
    D.1  & 10 & 12 & 10 & 13 & 14 & 10 & 11.8 & 14 \\ \hline
    D.4  & 15 & 20 & 20 & 20 & 15 & 15 & 18   & 20 \\ \hline
    E.1  & 10 & 11 & 12 & 12 & 13 & 10 & 11.6 & 13 \\ \hline
    F.1  & 8  & 10 & 9  & 8  & 8  & 8  & 8.6  & 10 \\ \hline
    G.3  & 10 & 9  & 7  & 9  & 8  & 7  & 8.6  & 10 \\ \hline
    H.9  & 12 & 12 & 10 & 12 & 9  & 9  & 11   & 12 \\ \hline
    J.12 & 7  & 6  & 7  & 8  & 6  & 6  & 6.8  & 8  \\ \hline
    J.22 & 11 & 8  & 8  & 12 & 11 & 8  & 10   & 12 \\ \hline
    K.3  & 23 & 22 & 22 & 29 & 25 & 22 & 24.2 & 29 \\ \hline
  \end{tabular}
  \caption{Resumen: \emph{Delphi} de la magnitud de pérdida (impacto)}
  \label{Tab:ResDELPHImag}
\end{table}

\begin{table}[!h]
  \centering
  \begin{tabular}{|c||c|c|c|c|c||c|c|c|}
    \hline
    \textbf{R} & \textbf{1} & \textbf{2} & \textbf{3} & \textbf{4} &
    \textbf{5} & \textbf{m} &\textbf{$\bar{x}$} &\textbf{M} \\
    \hline \hline
    A.7  & 50 & 50 & 70 & 60 & 70 & 50 & 60   & 70 \\ \hline
    A.11 & 55 & 55 & 70 & 65 & 75 & 55 & 64   & 75 \\ \hline
    D.1  & 20 & 30 & 20 & 30 & 25 & 20 & 25   & 30 \\ \hline
    D.4  & 25 & 20 & 25 & 20 & 25 & 20 & 23   & 25 \\ \hline
    E.1  & 55 & 55 & 40 & 40 & 50 & 40 & 48   & 55 \\ \hline
    F.1  & 25 & 30 & 30 & 20 & 20 & 20 & 25   & 30 \\ \hline
    G.3  & 45 & 40 & 45 & 50 & 35 & 35 & 43   & 50 \\ \hline
    H.9  & 12 & 15 & 13 & 15 & 17 & 12 & 14.4 & 17 \\ \hline
    J.12 & 16 & 17 & 18 & 15 & 14 & 14 & 16   & 18 \\ \hline
    J.22 & 45 & 40 & 50 & 50 & 40 & 40 & 45   & 50 \\ \hline
    K.3  & 21 & 20 & 22 & 23 & 20 & 20 & 21.2 & 23 \\ \hline
  \end{tabular}
  \caption{Resumen: \emph{Delphi} de la probabilidad de ocurrencia}
\label{Tab:ResDELPHIpro}
\end{table}

\chapter{Priorización de exposición a riesgos}
Tomando de la última iteración de la técnica \emph{Delphi} los datos de
probabilidad de ocurrencia y magnitud de pérdida de los riesgos, se ha
procedido a calcular la exposición a riesgos. Este valor se obtiene al
multiplicar la probabilidad de ocurrencia (en tanto por uno) y la magnitud de
pérdida, realizando esta operación para cada riesgo.

Una vez calculado este conjunto de valores, se ha realizado la separación de
los riesgos por su importancia, considerando como criterio base de
clasificación los siguientes puntos:
\begin{itemize}
\item Si la exposición a riesgos es mayor que 5 días, el riesgo se considera de
  importancia \textbf{alta}.
\item Si la exposición a riesgos es menor que 5 y mayor que 2 días, el riesgo se
  considera de importancia \textbf{media}.
\item Si la exposición a riesgos es menor que 2 días, el riesgo se considera
  de importancia \textbf{baja}.
\end{itemize}

El listado de exposición a riesgos se puede ver en la tabla \ref{Tab:Expri}.

\begin{table}[!h]
  \centering
  \begin{tabular}{|c|b{2.1cm}<{\centering}|c|b{2cm}<{\centering}|c|}
    \hline
    \textbf{Riesgo} & \textbf{Magnitud de pérdida} & \textbf{Probabilidad} & 
    \textbf{Exposición al riesgo} & \textbf{Importancia} \\
    \hline \hline
    A.7  & 13.4 & 60   & 8.04  & Alta  \\ \hline
    A.11 & 18.4 & 64   & 11.77 & Alta  \\ \hline 
    D.1  & 11.8 & 25   & 2.95  & Media \\ \hline
    D.4  & 18   & 23   & 4.14  & Media \\ \hline
    E.1  & 11.6 & 48   & 5.56  & Alta  \\ \hline
    F.1  & 8.6  & 25   & 2.15  & Media \\ \hline
    G.3  & 8.6  & 43   & 3.69  & Media \\ \hline
    H.9  & 11   & 14.4 & 1.58  & Baja  \\ \hline
    J.12 & 6.8  & 16   & 1.08  & Baja  \\ \hline
    J.22 & 10   & 45   & 4.5   & Media \\ \hline
    K.3  & 24.2 & 21.2 & 5.08  & Alta  \\ \hline
  \end{tabular}
  \caption{Exposición a los riesgos} 
  \label{Tab:Expri}
\end{table}

Si se ordena la tabla \ref{Tab:Expri} por importancia, de alta a
baja prioridad, el listado resultante se puede observar en la tabla
\ref{Tab:Expor}.

\begin{table}[!h]
  \centering
  \begin{tabular}{|c|b{2.1cm}<{\centering}|c|b{2cm}<{\centering}|c|}
    \hline
    \textbf{Riesgo} & \textbf{Magnitud de pérdida} & \textbf{Probabilidad} & 
    \textbf{Exposición al riesgo} & \textbf{Importancia} \\
    \hline \hline
    A.11 & 18.4 & 64   & 11.77 & Alta  \\ \hline
    A.7  & 13.4 & 60   & 8.04  & Alta  \\ \hline 
    E.1  & 11.6 & 48   & 5.56  & Alta  \\ \hline
    K.3  & 24.2 & 21.2 & 5.08  & Alta  \\ \hline
    J.22 & 10   & 45   & 4.5   & Media \\ \hline
    D.4  & 18   & 23   & 4.14  & Media \\ \hline
    G.3  & 8.6  & 43   & 3.69  & Media \\ \hline
    D.1  & 11.8 & 25   & 2.95  & Media \\ \hline
    F.1  & 8.6  & 25   & 2.15  & Media \\ \hline
    H.9  & 11   & 14.4 & 1.58  & Baja  \\ \hline
    J.12 & 6.8  & 16   & 1.08  & Baja  \\ \hline
  \end{tabular}
  \caption{Exposición a los riesgos (ordenado por importancia)} 
  \label{Tab:Expor}
\end{table}

\chapter{Planes de contingencia}
En este capítulo se exponen los riesgos elegidos, ordenador por importancia,
para analizarlos y plantear las contingencias correspondientes que permitan
al proyecto subsanar las posibles consecuencias de que estos riesgos se den.

\section{Riesgo A.11}
\begin{quote}
  \emph{Un retraso en una tarea produce retrasos en cascada en las tareas
    dependientes.} 
\end{quote}

\subsection{Descripción}
Este riesgo ocurre cuando una tarea retrasa su finalización respecto del plazo
planificado y otras tareas dependen de esta, bien sea para comenzar, 
completarse o terminarse. Si además la tarea que se retrasa pertenece al camino
crítico del proyecto, el causará un mayor impacto, ya que afectará
a la duración total del proyecto.

\subsection{Riesgos identificados}
\subsubsection{Aspectos del proyecto afectados}
El proyecto sigue un ciclo de vida en cascada, por lo que un retraso en 
cualquiera de sus etapas se arrastrará a las siguientes etapas.

\subsubsection{Causas}
Las causas de un retraso en la realización de una tarea pueden ser muy
diversas: la mala planificación de los tiempos de una tarea, la acumulación del trabajo por la ineficiencia de alguno de los trabajadores, la no disponibilidad
de un recurso material a tiempo, etc. En cualquier caso, estos retrasos serán
los causantes de dicho riesgo.

\subsubsection{Efectos en los objetivos del proyecto}
Un retraso en una tarea afecta a otras tareas cuando hay dependencias entre
ambas. Las dependencias pueden ser de distintos tipos: \emph{fin-comienzo}, 
\emph{comienzo-comienzo}, etc. En cada una de estas dependencias, si una de
las tareas 
que intervienen se retrasa, retrasará a las relacionadas. Esto provocará un
incumplimiento del calendario del proyecto, pudiendo provocar varios efectos
indeseados, como incumplimiento de contratos, gastos extras en el alquiler de
los equipos, trastornos en el funcionamiento normal de la gestión de datos de
la universidad, etc.

\subsection{Responsabilidades asignadas}
El responsable del retraso en la realización y finalización a tiempo de una
tarea es el coordinador. El coordinador debe preocuparse de gestionar los
recursos para poder completar de forma exitosa la realización de las
tareas, por lo que, cualquier retraso, puede, en mayor o menor medida, ser
responsabilidad de una mala gestión por su parte.

\subsection{Resultados del análisis del riesgo}
Después de haber aplicado la técnica \emph{Delphi}, la \textbf{probabilidad de
ocurrencia es del 64\%}, la \textbf{magnitud de pérdida o impacto es de 18.4
días } y, por tanto, la \textbf{exposición al riesgo es de 11.77 días}.

\subsection{Respuestas previstas}
Si una tarea se retrasa, pero ésta no tiene dependencias con otras, o bien
causa un impacto mínimo en ellas, no hará falta planificar ninguna respuesta,
siempre y cuando la tarea acabe antes de finalizar su fase del ciclo de vida;
ya que este proyecto posee un ciclo de vida en cascada y para pasar de fase
las tareas de la anterior deben estar finalizadas. Si no es este el caso,
deberán tomarse otras medidas.

\subsubsection{Acumulación de trabajo}
El retraso en la finalización de una tarea puede estar causado por una
acumulación de trabajo. Una posible solución sería contratar a más
personal o plantear al actual la realización de horas extras. Esta
solución repercutiría en el presupuesto final, por lo que primero se tendría
que contar con la aprobación de los responsables de la institución
universitaria.

Otra solución, aunque más discutible, sería la de disminuir en cantidad o
calidad la realización de la tarea. Esto repercutiría en el producto final,
pero si el peso de estas tareas no es muy significativo y la ganancia es
aceptable, puede ser conveniente llevar a cabo esta respuesta. En tareas que
afecten a la confiabilidad o la robustez del producto, deberán tomarse otras
medidas.

\subsubsection{Falta de material}
Otra posible causa puede ser la falta de material. En este caso se
trataría de 
terminales, pizarras o salas de reuniones. Como se trata de recursos
alquilados, la solución vendría por aumentar el número de estos recursos.
Esta solución también repercutirá en el presupuesto, por lo que también
deberá contar con la aprobación de los responsables de la universidad.

\subsection{Nivel del riesgo residual}
El retraso de una tarea afectará al calendario global del proyecto. Un retraso
superior a las dos semanas no podrá ser asumible en nuestro sistema. Por lo
que, si después de aplicar una respuesta no obtenemos retrasos menores a los
asumibles, deberemos aumentar el nivel de las respuestas; esto es, aumentar
el número de personal contratado o el número de recursos materiales según sea
conveniente. Si esto supone un coste inabordable para la universidad, se
tendría que optar por la opción de reducir la calidad del software.

\subsection{Acciones específicas para implementar la estrategia de respuesta
  a cambios}
Sería conveniente contar con un listado de posibles empleados para, en caso
de necesitar personal, tener accesible una serie de candidatos a ser
contratados. También puede tenerse un control de nuevos recursos a alquilar
para poder disponer de ellos en el caso de que fueran necesarios. Además,
puede hacerse una lista con las funcionalidades mínimas, recomendables y
optativas del software para, en caso de tener que reducir su calidad, tener
claro qué es y qué no es imprescindible desarrollar.

\subsection{Presupuesto y tiempos de respuesta}
El presupuesto y los tiempos para solventar este riesgo dependerán de la
gravedad del retraso. Cuanto mayor sea el retraso, mayor será el presupuesto
a invertir, claro está. En cualquier caso, se tendrá que estudiar si la
inversión a realizar aporta o no unos beneficios significativos.

\section{Riesgo A.7}
\begin{quote}
\emph{El esfuerzo es mayor que el estimado (por líneas de código, número de
  puntos función, módulos, etc.).}
\end{quote}

\subsection{Descripción}
Se han podido cometer errores en la planificación que a la hora de la
implementación del software den lugar a mayores esfuerzos de construcción
porque algunos paquetes de trabajo tengan más código o funciones de las
estimadas al principio.

\subsection{Riesgos identificados}
\subsubsection{Aspectos del proyecto afectados}
Cualquier parte del proyecto puede verse afectada por este riesgo ya que
cualquier etapa del proceso de desarrollo puede no haberse especificado
adecuadamente y por tanto acarreará errores.

\subsubsection{Causas}
Las causas que pueden dar lugar a este riesgo son variadas: pueden deberse a
una mala planificación del proyecto, como por ejemplo, tener un escaso
conocimiento de la implementación a utilizar a la hora del desarrollo;
también, a una especificación poco clara de los requisitos, material de
trabajo poco eficiente, etc.

\subsubsection{Efectos en los objetivos del proyecto}
Este riesgo puede afectar al calendario del proyecto, ya que los trabajadores
pueden realizar sus tareas en un período de tiempo más extendido, lo que
conlleva a la acumulación de retrasos que afectarían a la entrega
de producto final.

\subsection{Responsabilidad asignadas}
El principal responsable del retraso de las tareas es el coordinador, ya que
debe seguir el desarrollo del proyecto, asegurarse que los recursos estén
disponibles cuando sean necesarios y de diversas acciones de coordinación
que pueden afectar a la duración de las tareas y por consiguiente al retraso
del proyecto. En segunda instancia, los responsables serían el equipo de
trabajo, el analista, el programador y el usuario experto, que son los
responsables del retardo de su trabajo.

\subsection{Resultados del análisis del riesgo}
Después de haber aplicado la técnica \emph{Delphi}, la \textbf{probabilidad de
ocurrencia es del 60\%}, la \textbf{magnitud de pérdida o impacto son 13.5
días}, y por tanto, la \textbf{exposición al riesgo es de 8.04 días}.

\subsection{Respuestas previstas}
\begin{itemize}
\item Hacer un estudio de cómo afectará este riesgo
  al calendario del proyecto para ver si es necesario llevar a cabo alguna
  medida correctiva. Este estudio se entregará a la dirección, junto con las
  posibles acciones que se puedan llevar a cabo para mitigar el problema. 
\item Revisar la planificación del proyecto, comparándola con los resultados
  reales de las revisiones anteriores. Con esto se puede observar si la
  planificación de las tareas con retraso no ha sido realistas y, en tal caso,
  modificar el calendario del proyecto con datos más realistas para prevenir
  la ocurrencia de este riesgo.
\item Si la causa del riesgo es la deficiencia del personal, se podría
  impartir cursos de formación para que los trabajadores puedan mejorar y
  acelerar su trabajo. 
\end{itemize}

\subsection{Nivel del riesgo residual}
El nivel de riesgo residual deberá ser como máximo de una o dos semanas. No
deberían suceder más riesgos de este tipo ya que cuando se aplica el plan de 
contingencia se resuelve el problema y se revisa la planificación para
corregir los posibles cambios que causa la aplicación de la solución. 

\subsection{Acciones específicas para implementar la estrategia de respuesta
  a cambios}
Las acciones a llevar a cabo sería informar al coordinador
del proyecto. Además, se realizarían las acciones oportunas comunicando a
todos los miembros afectados para tomar las medidas que estimen necesarias. 

\subsection{Presupuesto y tiempos de respuesta}
El presupuesto para afrontar este riesgo incrementará el presupuesto estimado
inicialmente, ya que puede que haya que volver a contratar algún miembro del
equipo de trabajo. También hay que afrontar los gastos de nuevas
contrataciones de personal, si éstas fueran necesarias, o incluir presupuesto
para pagar las horas extras realizadas por los trabajadores. 

El tiempo de respuesta sería el plazo para tener disponible al nuevo
personal contratado o, en su defecto, tener bien formados al personal para
realizar más rápidamente el desarrollo del software.

\section{Riesgo E.1}
\begin{quote}
  \emph{El cliente insiste en nuevos requisitos.}
\end{quote}

\subsection{Descripción}
Este riesgo consiste en que el cliente, la empresa en este caso, al que se le 
realiza el software, solicita nuevos requisitos para incluir en el proyecto 
una vez que se definieron. La toma de requisitos normalmente se realiza en la 
etapa de análisis, pero este riesgo implica que en cualquier otra etapa 
posterior del proyecto la empresa añade más características al proyecto, lo que 
obliga a volver a etapas anteriores e incluir estos requisitos en el
proyecto.

\subsection{Riesgos identificados}
\subsubsection{Aspectos del proyecto afectados}
El proyecto se ve afectado en el momento en que la empresa incluye nuevos
requisitos. El impacto depende de la etapa en la que se produce, sabiendo que 
si ocurre en la etapa de \emph{pruebas}, el impacto será mucho mayor porque
el número de etapas a revisar será mayor. Sin embargo, en la etapa de
\emph{análisis} el impacto será menor. 

Además, se deben tener en cuenta si los requisitos que quiere añadir la
empresa contradicen alguno de los que ya existen. 

\subsubsection{Causas}
\begin{itemize}
\item El cliente no sabe lo que necesita o no sabe cómo expresarlo.
\item Falta de comunicación entre el equipo de proyecto y el cliente, lo cual 
  repercute en una captura de requisitos deficiente.
\item Los miembros del equipo de proyecto imponen su criterio e intentan
  decirle al cliente qué requisitos debe proponer en lugar de escucharlo.
\end{itemize}

\subsubsection{Efectos en los objetivos del proyecto}
\begin{itemize}
\item Se puede alargar el proyecto como resultado de tener que revisar 
  actividades ya completadas.
\item El coste del proyecto puede aumentar, ya que el cliente puede pedir 
  nuevos requisitos para los que no se tengan las herramientas adecuadas o el 
  personal especializado.
\item El cliente no acabe satisfecho con la calidad del producto final
  entregado.
\end{itemize}

\subsection{Responsabilidades asignadas}
Este riesgo puede ser debido, tanto a que la empresa no definió correctamente
los requisitos que querían en un principio, como debido a una mala labor de
consulta de requisitos por parte de los analistas. Por tanto, la
responsabilidad recaerá en el grupo de analistas y el jefe del proyecto, los
cuáles no supieron conseguir correctamente los requisitos de la herramienta
que el cliente requiere.

\subsection{Resultados del análisis del riesgo}
Después de haber aplicado la técnica \emph{Delphi}, la \textbf{probabilidad de
ocurrencia es del 48\%}, la \textbf{magnitud de pérdida o impacto son 11.6
días}, y por tanto, la \textbf{exposición al riesgo es de 5.56 días}.

\subsection{Respuestas previstas}
Algunas de las respuestas previstas para este riesgo son:
\begin{itemize}
\item Contratar a más personal para paliar los efectos producidos por el
  riesgo siempre y cuando el presupuesto lo permitiera.
\item Asignar a todo el personal que no estuviera trabajando en otra tarea y
  el personal del cual se pudiera prescindir en otras tareas a trabajar en los 
  nuevos requisitos.
\end{itemize}

\subsection{Nivel del riesgo residual}
La posibilidad de que el riesgo vuelva a suceder es baja, pero dependerá de 
cómo se haya realizado y con qué eficacia se haya llevado a cabo el plan de 
contingencia.

\subsection{Acciones específicas para implementar la estrategia de respuesta
  a cambios}
Las acciones que habría que realizar dentro de la estrategia de respuesta a
cambios serían las siguientes:
\begin{itemize}
\item Hacer una recogida de requisitos más exhaustiva añadiendo los nuevos 
  requisitos.
\item Validar los nuevos requisitos para evitar contradicciones con los ya 
  realizados.
\item Analizar qué actividades del proyecto se verán afectadas por los nuevos 
  requisitos.
\item Realizar de nuevo las tareas del proyecto que estén afectadas por los 
  nuevos requisitos.
\end{itemize}

\subsection{Presupuesto y tiempos de respuesta}
Dependiendo de la importancia y el número de requisitos a incluir, el tiempo y 
presupuesto para realizar una adecuada respuesta al riesgo variará. Así, si los 
requisitos afectan a un módulo importante, y del que dependen muchos otros, el 
presupuesto y el tiempo serían considerables. Por el contrario, si el
requisito afecta a una parte concreta del proyecto, sólo tendríamos que 
modificar dicha parte y el incremento en tiempo o costo sería más pequeño.

\section{Riesgo K.3}
\begin{quote}
  Un mal diseño implica volver a diseñar e implementar.
\end{quote}

\subsection{Descripción}
El cumplimiento de este riesgo conlleva casi volver al punto de partida y
diseñar e implementar el sistema de nuevo, con lo que esto acarrea.

\subsection{Riesgos identificados}
\subsubsection{Aspectos del proyecto afectados}
Es un riesgo bastante importante, no ya por la probabilidad de que aparezca,
si no más bien por su penalización, ya que este riesgo paralizaría toda la
producción y obligaría a reiniciar todas las actividades.

\subsubsection{Causas}
Este riesgo puede ser causado por diversos motivos, desde la incompetencia de
el/los encargado/s de diseñar el sistema, diferencias no resueltas con el
cliente, así como un mal entendido de las funciones del sistema.

\subsubsection{Efectos en los objetivos del proyecto}
Las consecuencias de este riesgo serían catastróficas para los objetivos del
proyecto; sería tarea imposible entregar el producto a tiempo si en una etapa
mínimamente avanzada se descubre que hay que rehacer todo el diseño y la
implementación. 

\subsection{Responsabilidades asignadas}
Dada la magnitud de este riesgo, la responsabilidad recaería sobre el
responsable del proyecto, en este caso el coordinador del grupo. También
recaería sobre el analista, ya que, junto con el coordinador, son los
principales encargados de realizar las reuniones con el cliente para aclarar
los requisitos del sistema y plantear un diseño correcto y apropiado del
proyecto.

\subsection{Resultados del análisis del riesgo}
Después de haber aplicado la técnica \emph{Delphi}, la \textbf{probabilidad de
ocurrencia es de 21.2\%}, la \textbf{magnitud de pérdida o impacto son 24.2
días} y, por tanto, la \textbf{exposición al riesgo es de 5.08 días}.

\subsection{Respuestas previstas}
Dada la penalización de esta tarea, no hay que escatimar en esfuerzos, así
que se proponen varías soluciones:
\begin{itemize}
\item La contratación de otro analista, para ayudar en la etapa de diseño del
  sistema. Así, los dos analistas se encargan del diseño y el coordinador se
  encargaría íntegramente de la supervisión del proceso.
\item Aumentar la duración de las reuniones semanales durante la etapa de
  diseño entre el coordinador y el grupo de trabajo para asignarle una
  importancia correspondiente a la que se merece.
\item Igualmente en la etapa de implementación del sistema, centrar el
  contenido de las reuniones semanales en la revisión de la implementación de
  programador.
\end{itemize}

\subsection{Nivel del riesgo residual}
Aún siguiendo las respuestas estipuladas a dicho problema, siempre puede
suceder que el personal contratado no cubra las expectativas, en especial los
analistas y el programador. De todas formas, estos problemas se suponen
cubiertos. Entonces, se concluye que la probabilidad de que este problema
suceda es escaso.

\subsection{Acciones específicas para implementar la estrategia de respuesta
  a cambios}
\begin{itemize}
\item Contratar al personal especificado.
\item Cambiar en el calendario la planificación de las reuniones semanales
\end{itemize}

\subsection{Presupuesto y tiempos de respuesta}
En caso de suceder dicho problema sería fatal tanto para el presupuesto como
para el tiempo de respuesta, llegando a duplicar e incluso triplicar los
valores. En cuanto al plan para evitar el problema, repercutiría de forma
ínfima comparándolo con repetir el proyecto. 

\end{document}
