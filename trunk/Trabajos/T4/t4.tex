% Clase
\documentclass[11pt,a4paper,spanish,twoside]{report}

% Órdenes auxiliares
\input{inc/includes.tex}

% Encabezado y pie de página
\encabezado

\begin{document}

% Silabación extra
\hyphenation{
a-sig-na-tu-ras
au-to-ma-ti-za-rá
ca-tá-lo-go
ca-rre-ra
cons-truc-ción
co-rres-pon-de
diag-nos-tico
fi-na-li-za-ción
ge-ne-ra-ción
in-fe-rior
man-te-ni-mien-to
me-dian-te
per-so-nal
pro-ce-di-mien-tos
pro-por-cio-na-rá
pu-bli-ca-da
re-qui-si-tos
res-pecto
u-su-a-rios
vi-lla-rre-al
}


% Portada
\portada{Planificación y Gestión de\\Sistemas de Información}
{Trabajo 3}{Calendario del proyecto}
{Sergio de la Rubia García-Carpintero\\Miguel Millán Sánchez-Grande\\
  Luis Muñoz Villarreal\\Alicia Serrano Sánchez\\
  Juan Miguel Torres Triviño}{26 de Abril de 2010}

% Licencia
\licencia{Sergio de la Rubia García-Carpintero, Miguel Millán Sánchez-Grande,
  Luis Muñoz Villarreal, Alicia Serrano Sánchez, Juan Miguel Torres Triviño}

\chapter*{Ficha de trabajo}
\begin{description}
\item[Código] T3
\item[Fecha] 26 de Abril de 2010
\item[Título] Calendario del proyecto
\end{description}

\begin{table}[!ht]
  \centering
  \begin{tabular}{lp{5cm}c}
    \multicolumn{3}{l}{\Large \textbf{Equipo} G4} \\ \\
    \multicolumn{1}{c}{\emph{Apellidos y nombre}} & 
    \multicolumn{1}{c}{\emph{Firma}} & \emph{Puntos} \\
    \hline \\
    de la Rubia García-Carpintero, Sergio & & 4 \\ \\
    Millán Sánchez-Grande, Miguel         & & 4 \\ \\
    Muñoz Villarreal, Luis                & & 4 \\ \\
    Serrano Sánchez, Alicia               & & 4 \\ \\
    Torres Triviño, Juan Miguel           & & 4 \\ \\
    \hline
  \end{tabular}
\end{table}

% Índices
\tableofcontents
\listoftables
\listoffigures

%% INICIO DEL DOCUMENTO %%%%%%%%%%%%%%%%%%%%%%%%%%%%%%%%%%%%%%%%%%%%%%%%%

\chapter*{Introducción}

En este trabajo se pretende conocer los riesgos más habituales que se pueden 
dar en el proyecto y algunos métodos para disminuir sus efectos negativos.

Un \emph{riesgo} es un evento que, en caso de ocurrir, tiene un efecto
positivo o negativo sobre los objetivos de un proyecto. Por tanto, en este
trabajo se incluye un plan de riesgos con los más probables de nuestro
proyecto, clasificándolos según su importancia.

De esa lista se han seleccionado los cinco riesgos más importantes, para los
que se ha realizado su plan de respuestas particularizado, incluyendo su 
descripción, los aspectos del proyecto afectados, las responsabilidades 
asignadas, los resultados del análisis del riesgo, el plan de contingencia,
el nivel de riesgos residual, las acciones específicas y el presupuesto y 
tiempos para las respuestas.

\chapter{Selección de riesgos}
Para determinar la lista de riesgos que se tratarán, se ha utilizado una
adaptación de la técnica \emph{Delphi}. Los riesgos han sido seleccionados de
la lista de comprobación de riesgos (\emph{checklist}) publicada en el 
\emph{Connell, S. Desarrollo y Gestión de Proyectos Informáticos. McGraw-Hill
Iberoamericana, 1997}.

\section{Técnica de selección de riesgos}
La técnica de selección de riesgos utilizada consta en los siguientes pasos:

\begin{enumerate}
\item Cada experto selecciona, de entre todos los riesgos, aproximadamente 
20 riesgos como mínimo.
\item Los riesgos elegidos por más de la mitad de los expertos se incluyen 
directamente en la lista de riesgos escogidos, los que no se incluyen en una 
lista de posibles riesgos elegibles.
\item Debatir entre todos los expertos qué riesgos de la lista de posibles se 
han de incluir en los escogidos. 
\item Cada experto realiza votación secreta eligiendo si dicho posible riesgo 
debe estar en la lista de escogidos.
\item Añadir a la lista de escogidos los riesgos que más votos reciban sin 
llegar a sobrepasar el máximo de riesgos a elegir.
\item Los riesgos escogidos son los que forman la lista definitiva.
\end{enumerate}

\section{Riesgos resultantes}

A. Elaboración de la planificación
A.7 El esfuerzo es mayor que el estimado (por líneas de código, número de
puntos función, módulos, etc).
A.10 Un retraso en una tarea produce retrasos en cascada en las tareas
dependientes.

D. Usuarios finales
D.1 Los usuarios finales insisten en nuevos requisitos.
D.4 No se ha solicitado información al usuario, por lo que el producto al
final no se ajusta a las necesidades del usuario, y hay que volver a crear el
producto.

E. Cliente
E.1 El cliente insiste en nuevos requisitos.

F. Personal contratado
F.1 El personal contratado no suministra los componentes en el periodo
establecido.

G. Requisitos
G.3 Se añaden requisitos extra.

H. Producto
H.9 Los requisitos para crear interfaces con otros sistemas, otros sistemas
complejos, u otros sistemas que no están bajo el control del equipo de
desarrollo suponen un diseño, implementación y prueba no previstos.

J. Personal
J.12. La incorporación de nuevo personal de desarrollo al proyecto ya
avanzado, y el aprendizaje y comunicaciones extra imprevistas reducen la
eficiencia de los miembros del equipo existentes.
J.22. El personal trabaja más lento de lo esperado.

K. Diseño e implementación
K.3. Un mal diseño implica volver a diseñar e implementar.

\chapter{Probabilidad de ocurrencia y magnitud de pérdida} 


\chapter{Priorización de exposición a riesgos}

\chapter{Planes de contingencia}
\section{Riesgo tal}

\section{Riesgo pascual}

\section{y tal...}

\end{document}
