% Clase
\documentclass[11pt,a4paper,spanish,twoside]{report}

% Órdenes auxiliares
\input{inc/includes.tex}

% Encabezado y pie de página
\encabezado

\begin{document}

% Silabación extra
\hyphenation{
a-sig-na-tu-ras
au-to-ma-ti-za-rá
ca-tá-lo-go
ca-rre-ra
cons-truc-ción
co-rres-pon-de
diag-nos-tico
fi-na-li-za-ción
ge-ne-ra-ción
in-fe-rior
man-te-ni-mien-to
me-dian-te
per-so-nal
pro-ce-di-mien-tos
pro-por-cio-na-rá
pu-bli-ca-da
re-qui-si-tos
res-pecto
u-su-a-rios
vi-lla-rre-al
}


% Portada
\portada{Planificación y Gestión de\\Sistemas de Información}
{Trabajo 3}{Calendario del proyecto}
{Sergio de la Rubia García-Carpintero\\Miguel Millán Sánchez-Grande\\
  Luis Muñoz Villarreal\\Alicia Serrano Sánchez\\
  Juan Miguel Torres Triviño}{26 de Abril de 2010}

% Licencia
\licencia{Sergio de la Rubia García-Carpintero, Miguel Millán Sánchez-Grande,
  Luis Muñoz Villarreal, Alicia Serrano Sánchez, Juan Miguel Torres Triviño}

\chapter*{Ficha de trabajo}
\begin{description}
\item[Código] T3
\item[Fecha] 26 de Abril de 2010
\item[Título] Calendario del proyecto
\end{description}

\begin{table}[!ht]
  \centering
  \begin{tabular}{lp{5cm}c}
    \multicolumn{3}{l}{\Large \textbf{Equipo} G4} \\ \\
    \multicolumn{1}{c}{\emph{Apellidos y nombre}} & 
    \multicolumn{1}{c}{\emph{Firma}} & \emph{Puntos} \\
    \hline \\
    de la Rubia García-Carpintero, Sergio & & 6 \\ \\
    Millán Sánchez-Grande, Miguel         & & 6 \\ \\
    Muñoz Villarreal, Luis                & & 6 \\ \\
    Serrano Sánchez, Alicia               & & 6 \\ \\
    Torres Triviño, Juan Miguel           & & 6 \\ \\
    \hline
  \end{tabular}
\end{table}

% Índices
\tableofcontents
\listoftables
\listoffigures

%% INICIO DEL DOCUMENTO %%%%%%%%%%%%%%%%%%%%%%%%%%%%%%%%%%%%%%%%%%%%%%%%%

\chapter*{Introducción}

\chapter{Selección de riesgos}
A. Elaboración de la planificación
A.7 El esfuerzo es mayor que el estimado (por líneas de código, número de
puntos función, módulos, etc).
A.10 Un retraso en una tarea produce retrasos en cascada en las tareas
dependientes.

D. Usuarios finales
D.1 Los usuarios finales insisten en nuevos requisitos.
D.4 No se ha solicitado información al usuario, por lo que el producto al
final no se ajusta a las necesidades del usuario, y hay que volver a crear el
producto.

E. Cliente
E.1 El cliente insiste en nuevos requisitos.

F. Personal contratado
F.1 El personal contratado no suministra los componentes en el periodo
establecido.

G. Requisitos
G.3 Se añaden requisitos extra.

H. Producto
H.9 Los requisitos para crear interfaces con otros sistemas, otros sistemas
complejos, u otros sistemas que no están bajo el control del equipo de
desarrollo suponen un diseño, implementación y prueba no previstos.

J. Personal
J.12. La incorporación de nuevo personal de desarrollo al proyecto ya
avanzado, y el aprendizaje y comunicaciones extra imprevistas reducen la
eficiencia de los miembros del equipo existentes.
J.22. El personal trabaja más lento de lo esperado.

K. Diseño e implementación
K.3. Un mal diseño implica volver a diseñar e implementar.

\section{Técnica de selección de riesgos}

\section{Riesgos resultantes}


\chapter{Probabilidad de ocurrencia y magnitud de pérdida} 


\chapter{Priorización de exposición a riesgos}

\chapter{Planes de contingencia}
\section{Riesgo tal}

\section{Riesgo pascual}

\section{y tal...}

\end{document}
