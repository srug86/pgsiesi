% Clase
\documentclass[12pt,a4paper,titlepage,spanish,twoside]{book}

% Órdenes auxiliares
% Español
\usepackage[spanish]{babel}
\usepackage{lmodern}
\usepackage[utf8]{inputenc}

% Imágenes
\usepackage[pdftex]{graphicx}
\usepackage{latexsym}
\usepackage{fancybox}

% Ruta para las imágenes
\graphicspath{{imagenes/}}

% Rotaciones
\usepackage[twoside]{rotating}

% Referencias
\usepackage[spanish]{varioref}
\usepackage[pdftex,colorlinks=true,linkcolor=black]{hyperref}

% Colores
\usepackage{color}
\usepackage{colortbl}

% Párrafos
\setlength{\parskip}{6pt}

% Entorno Listings para código fuente
\usepackage{listingsutf8}[2007/11/11]

\lstset{
  frame=Ltb, %forma del cuadro
  framerule=0pt, %ancho del borde
  aboveskip=0.5cm, %separación de los números de línea
  framexleftmargin=0.4cm, %margen externo izquierdo
  framesep=0pt,
  rulesep=.4pt,
  rulesepcolor=\color{black},
  % 
  stringstyle=\ttfamily,
  showstringspaces = false,
  basicstyle=\scriptsize,
  keywordstyle=\bfseries,
  % 
  numbers=left,
  numbersep=15pt,
  numberstyle=\tiny,
  numberfirstline= false,
  %
  inputencoding=utf8/latin1
}

% minimizar fragmentado de listados
\lstnewenvironment{listing}[1][]{
  \lstset{#1}\pagebreak[0]}{\pagebreak[0]
}


% Fancyhdr - Encabezados y pies de página
\usepackage{fancyhdr}
% Márgenes
\headsep=8mm
\footskip=14mm

% Fancy Header Style Options
\pagestyle{fancy}               % Sets fancy header and footer
\fancyfoot{}                    % Delete current footer settings

% Sin mayúsculas en la cabecera
\lhead{\nouppercase{\rightmark}}
\rhead{\nouppercase{\leftmark}}

% Capítulo
\renewcommand{\chaptermark}[1]{ % Lower Case Chapter marker style
  \markboth{\chaptername\ \thechapter.\ #1}{}} 

% Sección
\renewcommand{\sectionmark}[1]{ % Lower case Section marker style
  \markright{\thesection.\ #1}} 

% Página
\fancyhead[LE,RO]{\bfseries\thepage} % Page number (boldface) in left on even
                                     % pages and right on odd pages

% ------------------ Macro para encabezados y pies de página-------------------
%    Uso: \encabezado{pares(pag izquierda)}
% -----------------------------------------------------------------------------
\def\encabezado{
  \fancyhead[RE]{\bfseries\leftmark}     % In the right on even pages
  \fancyhead[LO]{\bfseries\rightmark}  % In the left on odd pages
  \renewcommand{\headrulewidth}{0.5pt} % Width of head rule
}
% -----------------------------------------------------------------------------


% ------------------ Macro para insertar una imagen ---------------------------
%    Uso: \imagen{nombreFichero}{Ancho(cm))}{Etiqueta}{Identificador}
% -----------------------------------------------------------------------------
\usepackage{float}
\usepackage{ifthen}
\def\imagen#1#2#3#4{
  \begin{figure}[H]
    \begin{center}
      \ifthenelse{\equal{#2}{}}
      {\includegraphics{#1}}{\resizebox{#2cm}{!}{\includegraphics{#1}}}
      \ifthenelse{\equal{#3}{}}{}{\caption{#3}}
      \label{#4}
    \end{center}
  \end{figure}
}
% -----------------------------------------------------------------------------


% ------------------ Macro para la portada ------------------------------------
%    Uso: \portada{asignatura}{titulo}{subtítulo}{autor}{fecha}
% -----------------------------------------------------------------------------
\def\portada#1#2#3#4#5{
  \thispagestyle{empty}
  \vspace*{-3.3cm}

  \begin{minipage}[t]{14cm}
    \begin{center}
      \includegraphics[scale=0.25]{logoesi}
  
      \vspace*{1.5cm}
      {\Large \textbf{Universidad de Castilla-La Mancha\\ 
          Escuela Superior de Informática}\\}
    
      \vspace*{1.2cm}
      {\huge \textbf{#1}\\}
    
      \vspace*{1.5cm}
      {\huge #2}\\{\Large #3\\}
    
      \vspace*{1.5cm}
      {\large #4\\}
      \vspace*{1.4cm}
      \large{#5}
    \end{center}
  \end{minipage}

  \newpage
  \vspace*{1cm}
  \thispagestyle{empty} 
  \newpage
}
% -----------------------------------------------------------------------------

% ------------------ Macro para la licencia -----------------------------------
%    Uso: \portada{asignatura}{titulo}{subtítulo}{autor}{fecha}
% -----------------------------------------------------------------------------
\def\licencia#1{
  \thispagestyle{empty}  % Suprime la numeración de esta página
  \vspace*{16cm}
  \begin{small}
    \copyright~ #1. Se permite la copia, distribución y/o 
    modificación de este documento bajo los términos de la licencia de
    documentación libre GNU, versión 1.1 o cualquier versión posterior publicada
    por la {\em Free Software Foundation}, sin secciones invariantes. Puede
    consultar esta licencia en http://www.gnu.org. \\[0.2cm]
    Este documento fue compuesto con \LaTeX{}. 
  \end{small}
  \newpage
  \thispagestyle{empty}
  \vspace*{0cm}
  \newpage
}
% -----------------------------------------------------------------------------

% Code for creating empty pages
% No headers on empty pages before new chapter
\makeatletter
\def\cleardoublepage{\clearpage\if@twoside \ifodd\c@page\else
    \hbox{}
    \thispagestyle{empty}
    \newpage
    \if@twocolumn\hbox{}\newpage\fi\fi\fi}
\makeatother \clearpage{\pagestyle{empty}\cleardoublepage}



\begin{document}
\section{INTRODUCCION}

\section{Inicio del plan de sistemas de información}
El objetivo de esta actividad es determinar la necesidad del Plan de Sistemas de
Información y llevar a cabo el arranque formal del mismo, con el apoyo del nivel
más alto de la organización. Como resultado, se obtiene una descripción general 
del Plan de Sistemas de Información que proporciona una definición inicial del 
mismo, identificando los objetivos estratégicos a los que apoya, así como el 
ámbito general de la organización al que afecta, lo que permite implicar a las 
direcciones de las áreas afectadas por el Plan de Sistemas de Información.

Además, se identifican los factores críticos de éxito y los participantes en el 
Plan de Sistemas de Información, nombrando a los máximos responsables.

\subsection{Análisis de la necesidad del PSI}
\subsection{Identificación del alcance del PSI}
\subsection{Determinación de responsables}

\section{Definición y organización del PSI)
En esta actividad se detalla el alcance del plan, se organiza el equipo de 
personas que lo va a llevar a cabo y se elabora un calendario de ejecución. 
Todos los resultados o productos de esta actividad constituirán el marco de
actuación del proyecto más detallado que en PSI 1 en cuanto a objetivos, 
procesos afectados, participantes, resultados y fechas de entrega.

\subsection{Especificación del ámbito y alcance}
\subsection{Organización del PSI}
\subsection{Definición del plan de trabajo}
\subsection{Comunicación del plan de trabajo}

\section{Estudio de la información relevante}
El objetivo de esta actividad es recopilar y analizar todos los antecedentes 
generales que puedan afectar a los procesos y a las unidades organizativas 
implicadas en el Plan de Sistemas de Información, así como a los resultados
del mismo. Pueden ser de especial interés los estudios realizados con 
anterioridad al Plan de Sistemas de Información, relativos a los sistemas de 
información de su ámbito, o bien a su entorno tecnológico, cuyas
conclusiones deben ser conocidas por el equipo de trabajo del Plan de Sistemas 
de Información.
La información obtenida en esta actividad se tendrá en cuenta en la elaboración 
de los requisitos.

\subsection{Selección y análisis de antecedentes}
\subsection{Valoración de antecedentes}


\section{Identificación de requisitos}
El objetivo final de esta actividad va a ser la especificación de los requisitos
de información de la organización, así como obtener un modelo de información que
los complemente.
Para conseguir este objetivo, se estudia el proceso o procesos de la 
organización incluidos en el ámbito del Plan de Sistemas de Información. Para 
ello es necesario llevar a cabo sesiones de trabajo con los usuarios, analizando
cada proceso tal y como debería ser, y no según su situación actual, ya que ésta
puede estar condicionada por los sistemas de información existentes.
Del mismo modo, se identifican los requisitos de información, y se elabora un 
modelo de información que represente las distintas entidades implicadas en el 
proceso, así como las relaciones entre ellas.
Por último, se clasifican los requisitos identificados según su prioridad, con 
el objetivo de incorporarlos al catálogo de requisitos del Plan de Sistemas de 
Información.


\subsection{Estudio de los procesos del PSI}
\subsection{Análisis de las necesidades de información}
\subsection{Catalogación de requisitos}

\section{Estudio de los sistemas de información actuales}
El objetivo de esta actividad es obtener una valoración de la situación actual al margen de los requisitos del
catálogo, apoyándose en criterios relativos a facilidad de mantenimiento, documentación, flexibilidad, facilidad de
uso, etc. En esta actividad se debe tener en cuenta la opinión de los usuarios, ya que aportarán elementos de
valoración, como por ejemplo, su nivel de satisfacción con cada sistema de información.
Se seleccionan los sistemas de información actuales que son objeto del análisis y se lleva a cabo el estudio de los
mismos con la profundidad y el detalle que se determine conveniente en función de los objetivos definidos para el
Plan de Sistemas de Información. Este estudio permite, para cada sistema, determinar sus carencias y valorarlos.
Esta valoración se utilizará en la actividad Diseño del Modelo de Sistemas de Información (PSI 6), donde se
analizará la cobertura de los sistemas de información actuales con respecto a los requisitos.

\subsection{Alcance y objetivos del estudio de los sistemas de información 
actuales}
\subsection{Análisis de los sistemas de información actuales}
\subsection{Valoración de los sistemas de información actuales}

\section{Diseño del modelo de sistemas de información}
\subsection{Diagnóstico de la situación actual}
\subsection{Definición del modelo de sistemas de información}


\section{Definición de la arquitectura tecnológica}
\subsection{Identificación de las necesidades de infraestructura tecnológica}
\subsection{Selección de la arquitectura tecnológica}

\section{Definición del plan de acción}
\subsection{Definición de proyectos a realizar}
\subsection{Elaboración del plan de mantenimiento del PSI}

\section{Bibliografía}

\end{document}
