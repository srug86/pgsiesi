% Clase
\documentclass[11pt,a4paper,spanish,twoside]{report}

% Órdenes auxiliares
\input{inc/includes.tex}

% Encabezado y pie de página
\encabezado

\begin{document}

% Silabación extra
\hyphenation{
a-sig-na-tu-ras
au-to-ma-ti-za-rá
ca-tá-lo-go
ca-rre-ra
cons-truc-ción
co-rres-pon-de
diag-nos-tico
fi-na-li-za-ción
ge-ne-ra-ción
in-fe-rior
man-te-ni-mien-to
me-dian-te
per-so-nal
pro-ce-di-mien-tos
pro-por-cio-na-rá
pu-bli-ca-da
re-qui-si-tos
res-pecto
u-su-a-rios
vi-lla-rre-al
}


% Portada
\portada{Planificación y Gestión de\\Sistemas de Información}
{Trabajo 3}{Calendario del proyecto}
{Sergio de la Rubia García-Carpintero\\Miguel Millán Sánchez-Grande\\
  Luis Muñoz Villarreal\\Alicia Serrano Sánchez\\
  Juan Miguel Torres Triviño}{26 de Abril de 2010}

% Licencia
\licencia{Sergio de la Rubia García-Carpintero, Miguel Millán Sánchez-Grande,
  Luis Muñoz Villarreal, Alicia Serrano Sánchez, Juan Miguel Torres Triviño}

\chapter*{Ficha de trabajo}
\begin{description}
\item[Código] T3
\item[Fecha] 26 de Abril de 2010
\item[Título] Calendario del proyecto
\end{description}

\begin{table}[!ht]
  \centering
  \begin{tabular}{lp{5cm}c}
    \multicolumn{3}{l}{\Large \textbf{Equipo} G4} \\ \\
    \multicolumn{1}{c}{\emph{Apellidos y nombre}} & 
    \multicolumn{1}{c}{\emph{Firma}} & \emph{Puntos} \\
    \hline \\
    de la Rubia García-Carpintero, Sergio & & 4 \\ \\
    Millán Sánchez-Grande, Miguel         & & 4 \\ \\
    Muñoz Villarreal, Luis                & & 4 \\ \\
    Serrano Sánchez, Alicia               & & 4 \\ \\
    Torres Triviño, Juan Miguel           & & 4 \\ \\
    \hline
  \end{tabular}
\end{table}

% Índices
\tableofcontents
\listoftables
\listoffigures

%% INICIO DEL DOCUMENTO %%%%%%%%%%%%%%%%%%%%%%%%%%%%%%%%%%%%%%%%%%%%%%%%%

\chapter*{Introducción}

En este trabajo se pretende conocer los riesgos más habituales que se pueden 
dar en el proyecto y algunos métodos para disminuir sus efectos negativos.

Un \emph{riesgo} es un evento que, en caso de ocurrir, tiene un efecto
positivo o negativo sobre los objetivos de un proyecto. Por tanto, en este
trabajo se incluye un plan de riesgos con los más probables de nuestro
proyecto, clasificándolos según su importancia.

De esa lista se han seleccionado los cinco riesgos más importantes, para los
que se ha realizado su plan de respuestas particularizado, incluyendo su 
descripción, los aspectos del proyecto afectados, las responsabilidades 
asignadas, los resultados del análisis del riesgo, el plan de contingencia,
el nivel de riesgos residual, las acciones específicas y el presupuesto y 
tiempos para las respuestas.

\chapter{Selección de riesgos}
Para determinar la lista de riesgos que se tratarán, se ha utilizado una
adaptación de la técnica \emph{Delphi}. Los riesgos han sido seleccionados de
la lista de comprobación de riesgos (\emph{checklist}) publicada en el 
\emph{Connell, S. Desarrollo y Gestión de Proyectos Informáticos. McGraw-Hill
Iberoamericana, 1997}.

\section{Técnica de selección de riesgos}
La técnica de selección de riesgos utilizada consta en los siguientes pasos:

\begin{enumerate}
\item Cada experto selecciona inicialmente veinte riesgos de entre todos los que
  forman la lista.
\item Se hace una comparativa de todos los riesgos elegidos y se seleccionan
  aquellos que hayan sido elegidos por al menos tres expertos.
\item Los riesgos que no alcanzaron los tres votos pero obtuvieron alguno se
  incluyen en una lista de posibles riesgos elegibles. 
\item Debatir entre todos los expertos qué riesgos de la lista de posibles se 
han de incluir en los escogidos. 
\item Cada experto realiza votación secreta eligiendo si dicho posible riesgo 
debe estar en la lista de escogidos.
\item Añadir a la lista de escogidos los riesgos que más votos reciban sin 
llegar a sobrepasar el máximo de riesgos a elegir.
\item Los riesgos escogidos son los que forman la lista definitiva.
\end{enumerate}

\section{Riesgos resultantes}
Después de aplicar la técnica DELPHI adaptada, la lista de riesgos
seleccionada corresponde con la tabla \ref{Tab:tar_sel}.

\begin{table}[!ht]
  \centering
  \begin{tabular}{|p{3.5cm}||p{1.1cm}|p{6.3cm}|}
    \hline
    \textbf{Categoría} & \textbf{Riesgo} & \textbf{Descripción} \\
    \hline\hline
    \multirow{2}{3.5cm}{Elaboración de la planificación}
    & A.7 & El esfuerzo es mayor que el estimado (por líneas de código,
    número de puntos función, módulos, etc). \\  
    \cline{2-3}
    & A.11 & Un retraso en una tarea produce retrasos en cascada en las
    tareas dependientes. \\
    \hline
    \multirow{2}{3cm}{Usuarios finales}
    & D.1 & Los usuarios finales insisten en nuevos requisitos\\
    \cline{2-3}
    & D.4 & No se ha solicitado información al usuario, por lo que el prducto
    al final  no se ajusta a las necesidades del usuario, y hay que volver a
    crear el producto.\\
    \hline
    Cliente& E.1 & El cliente insiste en nuevos requisitos.\\
    \hline
    Personal contratado & F.1 & El personal contratado no suministra los
    componentes en el periodo establecido.\\
    \hline
    Requisitos& G.3 & Se añaden requisitos extra.\\
    \hline
    Producto & H.9 &  Los requisitos para crear interfaces con otros
    sistemas, otros sistemas complejos, u otros sistemas que no están bajo el
    control del equipo de desarrollo suponen un diseño, implementación y
    prueba no previstos.\\ 
    \hline
    \multirow{2}{3.5cm}{Personal}
    & J.12 & La incorporación de nuevo personal de desarrollo al proyecto ya
    avanzado, y el aprendizaje y comunicaciones extra imprevistas reducen la
    eficiencia de los miembros del equipo existentes.\\ 
    \cline{2-3}
    & J.22 & El personal trabaja más lento de lo esperado\\
    \hline
    \multirow{1}{3.5cm}{Diseño e implementación}
    & K.3 & Un mal diseño implica volver a diseñar e implementar.\\
   \hline
  \end{tabular}
  \caption{Tareas seleccionadas}
  \label{Tab:tar_sel}
\end{table}

\chapter{Probabilidad de ocurrencia y magnitud de pérdida} 
En las tablas \ref{Tab:DELPHImag} y \ref{Tab:DELPHIpro}, de las páginas
\pageref{Tab:DELPHImag} y \pageref{Tab:DELPHIpro}; se  
puede observar la evolución seguida a la hora de calcular la probabilidad de
ocurrencia y la magnitud de pérdida con la técnica de DELPHI. Cada tabla
contiene el identificador de cada riesgo, la iteración en 
la que se encuentra, las valoraciones de los distintos expertos, así como los
valores mínimo, medio (que
corresponde con la media aritmética) y máximo, y finalmente, si se cumple el
criterio de convergencia o no. 

\begin{table}[!h]
\centering
  \begin{tabular}{|c|c||c|c|c|c|c||c|c|c||c|}
    \hline
    \textbf{R} & \textbf{I} & \textbf{1} &
    \textbf{2} & \textbf{3} & \textbf{4} & \textbf{5} & \textbf{m}
    &\textbf{$\bar{x}$} &\textbf{M} & \textbf{C}\\
    \hline
    \multirow{4}{*}{A.7} & 1 & 5 & 12 & 8 & 17 & 23 & 5 & 13 & 23 & N \\
    & 2 & 9 & 13 & 10 & 18 & 19 & 9 & 13.8 & 19 & N \\
    & 3 & 11 & 13 & 12 & 19 & 15 & 11 & 14 & 19 & N \\
    & \textbf{4} & \textbf{12} & \textbf{13} & \textbf{13} & \textbf{15} &
    \textbf{14} & \textbf{12} & \textbf{13.4} & \textbf{15} & \textbf{S} \\ 
    \hline
    \multirow{3}{*}{A.10} & 1 & 31 & 10 & 15 & 6 & 14 & 6 & 15.2 & 31 & N \\
    & 2 & 15 & 12 & 19 & 25 & 20 & 12 & 18.2 & 25 & N \\
    & \textbf{3} & \textbf{15} & \textbf{15} & \textbf{20} & \textbf{22} &
    \textbf{20} & \textbf{15} & \textbf{18.4} & \textbf{22} & \textbf{S} \\ 
    \hline
    \multirow{3}{*}{D.1} & 1 & 10 & 10 & 9 & 6 & 18 & 6 & 10.6 & 18 & N \\
    & 2 & 10 & 12 & 10 & 12 & 15 & 10 & 11.8 & 15 & N \\
    & \textbf{3} & \textbf{10} & \textbf{12} & \textbf{10} & \textbf{13} &
    \textbf{14} & \textbf{10} & \textbf{11.8} & \textbf{14} & \textbf{S} \\ 
    \hline
    \multirow{5}{*}{D.4} & 1 & 15 & 18 & 16 & 23 & 33 & 15 & 21 & 33 & N \\
    & 2 & 15 & 25 & 20 & 27 & 15 & 15 & 20.4 & 27 & N \\
    & 3 & 16 & 23 & 22 & 24 & 18 & 16 & 20.6 & 24 & N \\
    & 4 & 16 & 22 & 23 & 23 & 17 & 16 & 20.2 & 23 & N \\
    & \textbf{5} & \textbf{15} & \textbf{20} & \textbf{20} & \textbf{20} &
    \textbf{15} & \textbf{15} & \textbf{18} & \textbf{20} & \textbf{S} \\
    \hline
    \multirow{5}{*}{E.1} & 1 & 12 & 10 & 6 & 5 & 19 & 5 & 10.4 & 19 & N \\
    & 2 & 12 & 8 & 13 & 7 & 17 & 7 & 11.4 & 17 & N \\
    & 3 & 10 & 10 & 10 & 10 & 15 & 10 & 11 & 15 & N \\
    & 4 & 10 & 10 & 12 & 11 & 14 & 10 & 11.4 & 14 & N \\
    & \textbf{5} & \textbf{10} & \textbf{11} & \textbf{12} & \textbf{12} &
    \textbf{13} & \textbf{10} & \textbf{11.6} & \textbf{13} & \textbf{S} \\ 
    \hline
    \multirow{3}{*}{F.1} & 1 & 8 & 6 & 4 & 14 & 5 & 4 & 7.4 & 14 & N \\
    & 2 & 8 & 11 & 9 & 6 & 7 & 6 & 8.2 & 11 & N \\
    & \textbf{3} & \textbf{8} & \textbf{10} & \textbf{9} & \textbf{8} &
    \textbf{8} & \textbf{8} & \textbf{8.6} & \textbf{10} & \textbf{S} \\ 
    \hline
    \multirow{3}{*}{G.3} & 1 & 7 & 10 & 7 & 13 & 7 & 7 & 8.8 & 13 & N \\
    & 2 & 10 & 9 & 7 & 8 & 7 & 7 & 8.2 & 10 & N \\
    & \textbf{3} & \textbf{10} & \textbf{9} & \textbf{7} & \textbf{9} &
    \textbf{8} & \textbf{7} & \textbf{8.6} & \textbf{10} & \textbf{S} \\ 
    \hline
    \multirow{2}{*}{H.9} & 1 & 13 & 14 & 11 & 11 & 4 & 4 & 10.6 & 14 & N \\
    & \textbf{2} & \textbf{12} & \textbf{12} & \textbf{10} & \textbf{12} &
    \textbf{9} & \textbf{9} & \textbf{11} & \textbf{12} & \textbf{S} \\ 
    \hline
    \multirow{3}{*}{J.12} & 1 & 6 & 9 & 5 & 7 & 8 & 5 & 7 & 9 & N \\
    & 2 & 7 & 6 & 7 & 8 & 5 & 5 & 6.6 & 8 & N \\
    & \textbf{3} & \textbf{7} & \textbf{6} & \textbf{7} & \textbf{8} &
    \textbf{6} & \textbf{6} & \textbf{6.8} & \textbf{8} & \textbf{S} \\ 
    \hline
    \multirow{2}{*}{J.22} & 1 & 10 & 8 & 14 & 8 & 10 & 8 & 10 & 14 & N \\
    & \textbf{2} & \textbf{11} & \textbf{8} & \textbf{8} & \textbf{12} &
    \textbf{11} & \textbf{8} & \textbf{10} & \textbf{12} & \textbf{S} \\ 
    \hline
    \multirow{3}{*}{K.3} & 1 & 20 & 19 & 23 & 36 & 19 & 19 & 23.4 & 36 & N \\
    & 2 & 23 & 21 & 30 & 22 & 25 & 21 & 24.2 & 30 & N \\
    & \textbf{3} & \textbf{23} & \textbf{22} & \textbf{22} & \textbf{29} &
    \textbf{25} & \textbf{22} & \textbf{24.2} & \textbf{29} & \textbf{S} \\ 

   \end{tabular}
  \caption{\textbf{DELPHI} de la magnitud de pérdida (impacto)}
  \label{Tab:DELPHImag}
\end{table}

\begin{table}[!h]
\centering
  \begin{tabular}{|c|c||c|c|c|c|c||c|c|c||c|}
    \hline
    \textbf{R} & \textbf{I} & \textbf{1} &
    \textbf{2} & \textbf{3} & \textbf{4} & \textbf{5} & \textbf{m}
    &\textbf{$\bar{x}$} &\textbf{M} & \textbf{C}\\
    \hline
    \multirow{3}{*}{A.7} & 1 & 70 & 50 & 65 & 25 & 98 & 25 & 61.6 & 98 & N \\
    & 2 & 50 & 50 & 70 & 65 & 75 & 50 & 62 & 75 & N \\
    & \textbf{3} & \textbf{50} & \textbf{50} & \textbf{70} & \textbf{60} &
    \textbf{70} & \textbf{50} & \textbf{60} & \textbf{70} & \textbf{S} \\ 
    \hline
    \multirow{4}{*}{A.10} & 1 & 80 & 40 & 90 & 25 & 80 & 25 & 63 & 90 & N \\
    & 2 & 65 & 45 & 70 & 70 & 75 & 45 & 65 & 75 & N \\
    & 3 & 55 & 50 & 70 & 65 & 75 & 50 & 63 & 75 & N \\
    & \textbf{4} & \textbf{55} & \textbf{55} & \textbf{70} & \textbf{65} &
    \textbf{75} & \textbf{55} & \textbf{64} & \textbf{75} & \textbf{S} \\ 
    \hline
    \multirow{4}{*}{D.1} & 1 & 20 & 20 & 35 & 15 & 35 & 15 & 25 & 35 & N \\
    & 2 & 35 & 20 & 20 & 30 & 20 & 20 & 25 & 35 & N \\
    & 3 & 20 & 35 & 20 & 30 & 20 & 20 & 25 & 35 & N \\
    & \textbf{4} & \textbf{20} & \textbf{30} & \textbf{20} & \textbf{30} &
    \textbf{25} & \textbf{20} & \textbf{25} & \textbf{30} & \textbf{S} \\ 
    \hline
    \multirow{3}{*}{D.4} & 1 & 25 & 5 & 55 & 5 & 15 & 5 & 21 & 55 & N \\
    & 2 & 25 & 15 & 15 & 20 & 35 & 15 & 22 & 35 & N \\
    & \textbf{3} & \textbf{25} & \textbf{20} & \textbf{25} & \textbf{20} &
    \textbf{25} & \textbf{20} & \textbf{23} & \textbf{25} & \textbf{S} \\ 
    \hline
    \multirow{4}{*}{E.1} & 1 & 20 & 15 & 70 & 40 & 50 & 15 & 39 & 70 & N \\
    & 2 & 50 & 65 & 20 & 35 & 40 & 20 & 42 & 65 & N \\
    & 3 & 55 & 55 & 30 & 35 & 40 & 30 & 43 & 55 & N \\
    & \textbf{4} & \textbf{55} & \textbf{55} & \textbf{40} & \textbf{40} &
    \textbf{50} & \textbf{40} & \textbf{48} & \textbf{55} & \textbf{S} \\ 
    \hline
    \multirow{3}{*}{F.1} & 1 & 35 & 40 & 20 & 10 & 32 & 10 & 27.4 & 40 & N \\
    & 2 & 30 & 30 & 25 & 17 & 20 & 17 & 24.4 & 30 & N \\
    & \textbf{3} & \textbf{25} & \textbf{30} & \textbf{30} & \textbf{20} &
    \textbf{20} & \textbf{20} & \textbf{25} & \textbf{30} & \textbf{S} \\ 
    \hline
    \multirow{3}{*}{G.3} & 1 & 20 & 65 & 65 & 30 & 25 & 20 & 41 & 65 & N \\
    & 2 & 40 & 30 & 50 & 55 & 35 & 30 & 42 & 55 & N \\
    & \textbf{3} & \textbf{45} & \textbf{40} & \textbf{45} & \textbf{50} &
    \textbf{35} & \textbf{35} & \textbf{43} & \textbf{50} & \textbf{S} \\ 
    \hline
    \multirow{5}{*}{H.9} & 1 & 30 & 5 & 40 & 10 & 3 & 3 & 17.6 & 40 & N \\
    & 2 & 6 & 25 & 10 & 8 & 20 & 6 & 13.8 & 25 & N \\
    & 3 & 10 & 20 & 11 & 13 & 17 & 10 & 14.2 & 20 & N \\
    & 4 & 10 & 15 & 11 & 12 & 17 & 10 & 13 & 17 & N \\
    & \textbf{5} & \textbf{12} & \textbf{15} & \textbf{13} & \textbf{15} &
    \textbf{17} & \textbf{12} & \textbf{14.4} & \textbf{17} & \textbf{S} \\ 
    \hline
    \multirow{5}{*}{J.12} & 1 & 15 & 15 & 5 & 5 & 34 & 5 & 14.8 & 34 & N \\
    & 2 & 14 & 9 & 25 & 13 & 10 & 9 & 14.2 & 25 & N \\
    & 3 & 12 & 13 & 20 & 15 & 12 & 12 & 14.4 & 20 & N \\
    & 4 & 13 & 17 & 20 & 15 & 14 & 13 & 15.8 & 20 & N \\
    & \textbf{5} & \textbf{16} & \textbf{17} & \textbf{18} & \textbf{15} &
    \textbf{14} & \textbf{14} & \textbf{16} & \textbf{18} & \textbf{S} \\ 
    \hline
    \multirow{5}{*}{J.22} & 1 & 60 & 10 & 30 & 50 & 75 & 10 & 45 & 75 & N \\
    & 2 & 50 & 40 & 50 & 60 & 25 & 25 & 45 & 60 & N \\
    & 3 & 45 & 40 & 50 & 55 & 35 & 35 & 45 & 55 & N \\
    & 4 & 45 & 40 & 50 & 52 & 35 & 35 & 44.4 & 52 & N \\
    & \textbf{5} & \textbf{45} & \textbf{40} & \textbf{50} & \textbf{50} &
    \textbf{40} & \textbf{40} & \textbf{45} & \textbf{50} & \textbf{S} \\ 
    \hline
    \multirow{4}{*}{K.3} & 1 & 20 & 5 & 40 & 1 & 50 & 1 & 23.2 & 50 & N \\
    & 2 & 40 & 18 & 20 & 25 & 10 & 10 & 22.6 & 40 & N \\
    & 3 & 30 & 19 & 22 & 23 & 20 & 19 & 22.8 & 30 & N \\
    & \textbf{4} & \textbf{21} & \textbf{20} & \textbf{22} & \textbf{23} &
    \textbf{20} & \textbf{20} & \textbf{21.2} & \textbf{23} & \textbf{S} \\ 

   \end{tabular}
  \caption{\textbf{DELPHI} de la probabilidad de ocurrencia}
  \label{Tab:DELPHIpro}
\end{table}


\chapter{Priorización de exposición a riesgos}
Tomando de la última iteración de la técnica Delphi los datos de probabilidad
de ocurrencia y magnitud de pérdida de los riesgos, se ha procedido a
calcular la exposición a riesgos. Este valor se obtiene al multiplicar la
probabilidad de ocurrencia (en tanto por uno) y la magnitud de pérdida,
realizando esta operación para cada riesgo.

Una vez calculado este conjunto de valores, se ha realizado la separación de
los riesgos por su importancia, considerando como criterio base de
clasificación:

\begin{itemize}
\item Si la exposición a riesgos es mayor de 5 días, el riesgo se considera de
importancia alta.
\item Si la exposición a riesgos es menor de 5 y mayor de 2 días, el riesgo se
considera de importancia media.
\item Si la exposición a riesgos es menor de 2 días, el riesgo se considera
de importancia baja.
\end{itemize}

La lista de exposición a riesgos se puede ver en la tabla \ref{Tab:Expri}.

\begin{table}[!h]
  \centering
  \begin{tabular}{|c||p{2,5cm}||c||p{2,5cm}||c|}
    \hline
    \textbf{Riesgo} & \textbf{Magnitud de pérdida} & \textbf{Propabilidad} & 
    \textbf{Exposición al riesgo} & \textbf{Importancia} \\
    \hline \hline
    A.7 & 13.4 & 60 & 8.04 & Alta \\ 
    \hline
    A.11 & 18.4 & 64 & 11.77 & Alta \\
    \hline 
    D.1 & 11.8 & 25 & 2.95 & Media \\
    \hline
    D.4 & 18 & 23 & 4.14 & Media \\
    \hline
    E.1 & 11.6 & 48 & 5.56 & Alta \\
    \hline
    F.1 & 8.6 & 25 & 2.15 & Media \\
    \hline
    G.3 & 8.6 & 43 & 3.69 & Media \\
    \hline
    H.9 & 11 & 14.4 & 1.58 & Baja \\
    \hline
    J.12 & 6.8 & 16 & 1.08 & Baja \\
    \hline
    J.22 & 10 & 45 & 4.5 & Media \\
    \hline
    K.3 & 24.2 & 21.2 & 5.08 & Alta \\
    \hline
  \end{tabular}
  \caption{Exposición al riesgo} 
  \label{Tab:Expri}
\end{table}

Por tanto, si se ordena la tabla \ref{Tab:Expri} por importancia, de alta a
baja prioridad, queda una tabla que se puede observar en el cuadro
\ref{Tab:Expor}.


\begin{table}[!h]
  \centering
  \begin{tabular}{|c||p{2,5cm}||c||p{2,5cm}||c|}
    \hline
    \textbf{Riesgo} & \textbf{Magnitud de pérdida} & \textbf{Propabilidad} & 
    \textbf{Exposición al riesgo} & \textbf{Importancia} \\
    \hline \hline
    A.11 & 18.4 & 64 & 11.77 & Alta \\ 
    \hline
    A.7 & 13.4 & 60 & 8.04 & Alta \\
    \hline 
    E.1 & 11.6 & 48 & 5.56 & Alta \\
    \hline
    K.3 & 24.2 & 21.2 & 5.08 & Alta \\
    \hline
    J.22 & 10 & 45 & 4.5 & Media \\
    \hline
    D.4 & 18 & 23 & 4.14 & Media \\
    \hline
    G.3 & 8.6 & 43 & 3.69 & Media \\
    \hline
    D.1 & 11.8 & 25 & 2.95 & Media \\
    \hline
    F.1 & 8.6 & 25 & 2.15 & Media \\
    \hline
    H.9 & 11 & 14.4 & 1.58 & Baja \\
    \hline
    J.12 & 6.8 & 16 & 1.08 & Baja \\
    \hline
  \end{tabular}
  \caption{Exposición al riesgo ordenada por importancia} 
  \label{Tab:Expor}
\end{table}
\chapter{Planes de contingencia}
\section{Plan de contiengencia para el riesgo A.11}
\section{Riesgo pascual}
\subsection{Riesgo}
\subsection{Descripción}
\subsection{Aspectos del proyecto afectados}
\subsection{Causas}
\subsection{Efectos de los objetivos del proyecto}
\subsection{Responsabilidad asignadas}
\subsection{Resultados del análisis del riesgo}
\subsection{Respuestas previstas}
\subsection{Nivel del riesgo residual}
\subsection{Acciones específicas para implementar la estrategia de respuesta
a cambios}
\subsection{Presupuesto y tiempos de respuesta}

\section{Plan de contiengencia para el riesgo A.7}
\subsection{Descripción}
\subsection{Aspectos del proyecto afectados}
\subsection{Causas}
\subsection{Efectos de los objetivos del proyecto}
\subsection{Responsabilidad asignadas}
\subsection{Resultados del análisis del riesgo}
\subsection{Respuestas previstas}
\subsection{Nivel del riesgo residual}
\subsection{Acciones específicas para implementar la estrategia de respuesta
a cambios}
\subsection{Presupuesto y tiempos de respuesta}

\section{Plan de contiengencia para el riesgo E.1}
\subsection{Descripción}
\subsection{Aspectos del proyecto afectados}
\subsection{Causas}
\subsection{Efectos de los objetivos del proyecto}
\subsection{Responsabilidad asignadas}
\subsection{Resultados del análisis del riesgo}
\subsection{Respuestas previstas}
\subsection{Nivel del riesgo residual}
\subsection{Acciones específicas para implementar la estrategia de respuesta
a cambios}
\subsection{Presupuesto y tiempos de respuesta}

\section{Plan de contiengencia para el riesgo K.3}
Un mal diseño implica volver a diseñar e implementar
\subsection{Descripción}
El cumplimiento este riesgo conlleva casi volver al punto de partida y
diseñar e implementar el sistema, con lo que esto acarrea.
\subsection{Riesgos identificados}
\subsubsection{Aspectos del proyecto afectados}
Es un riesgo bastante importante, no ya por la probabilidad de que aparezca
si no más bien por su penalización, ya que este riesgo paralizaría toda la
producción y obligaría a iniciar todas las actividades.
\subsubsection{Causas}
Este riesgo puede ser causado por diversos motivos, desde la incopetencia de
el/los encargados de diseñar el sistema, diferencias no resueltas con el
cliente, así como un mal entendido de las funciones del sistema.
\subsubsection{Efectos en los objetivos del proyecto}
Las consecuencias de este riesgo serían catastróficas para los obejtivos del
proyecto, sería tarea imposible entregar el producto a tiempo si en una etapa
mínimamente avanzada se descubre que hay que rehacer todo el diseño e
implementación. 
\subsection{Responsabilidades asignadas}
Dada la magnitud de este riesgo, la responsabilidad recaería sobre el
responsable del proyecto, en este caso el coordinador del grupo. También
recaería sobre el analista, ya que junto con el coordinador son los
principales encargados de realizar las reuniones con el cliente para aclarar
los requisitos del sistema y plantear un diseño correcto y apropiado del
proyecto.
\subsection{Resultados del análisis del riesgo}
Después de haber aplicado la técnica \emph{Delphi}, la probabilidad de
ocurrencia es de 21.2\%, la magnitud de pérdida o impacto son 24.2 días y, por
tanto, la exposición al riesgo es de 5.08 días.
\subsection{Respuestas previstas}
Dada la penalización de esta tarea, no hay que escatimar en esfuerzos, así
que se proponen varías soluciones:
\begin{itemize}
\item La contratación de otro analista, para ayudar en la etapa de diseño del
  sistema, así los dos analistas se encargan del diseño y el coordinador se
  encargaría integramente de la supervisión del proceso.
\item Aumentar la duración de las reuniones semanales durante la etapa de
  diseño entre el coordinador y el grupo de trabajo, para asignarle una
  importancia correspondiente a la que se merece.
\item Igualmente en la etapa de implementación del sistema, centrar el
  contenido de las reuniones semanales en la revisión de la implementación de
  programador.
\end{itemize}
\subsection{Nivel del riesgo residual}
Aún siguiendo las respuestas estipuladas a dicho problema, siempre puede
suceder que el personal contratado no cubra las espectativas, en especial los
analistas y el programador. De todas formas estos problemas se suponen
cubiertos. Entonces se concluye que la probabilidad de que este problema
suceda es escaso.
\subsection{Acciones específicas para implementar la estrategia de respuesta
a cambios}
\begin{itemize}
\item Contratar al personal especificado.
\item Cambiar en el calendario la planificación de las reuniones semanales
\subsection{Presupuesto y tiempos de respuesta}
En caso de suceder dicho problema sería fatal tanto para el presupuesto como
para el tiempo de respuesta, llegando a duplicar e incluso triplicar los
valores. En cuanto al plan para evitar el problema, repercutiría de forma
ínfima comparándolo con repetir el proyecto. 

\end{document}
