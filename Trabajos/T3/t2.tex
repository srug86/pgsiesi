% Clase
\documentclass[11pt,a4paper,spanish,twoside]{report}

% Órdenes auxiliares
\input{inc/includes.tex}

% Encabezado y pie de página
\encabezado

\setcounter{secnumdepth}{3}
\setcounter{tocdepth}{3}

\begin{document}

% Silabación extra
\hyphenation{
a-sig-na-tu-ras
au-to-ma-ti-za-rá
ca-tá-lo-go
ca-rre-ra
cons-truc-ción
co-rres-pon-de
diag-nos-tico
fi-na-li-za-ción
ge-ne-ra-ción
in-fe-rior
man-te-ni-mien-to
me-dian-te
per-so-nal
pro-ce-di-mien-tos
pro-por-cio-na-rá
pu-bli-ca-da
re-qui-si-tos
res-pecto
u-su-a-rios
vi-lla-rre-al
}


% Portada
\portada{Planificación y Gestión de\\Sistemas de Información}
{Trabajo 3}{Calendario del proyecto}
{Sergio de la Rubia García-Carpintero\\Miguel Millán Sánchez-Grande\\
  Luis Muñoz Villarreal\\Alicia Serrano Sánchez\\
  Juan Miguel Torres Triviño}{26 de Abril de 2010}

% Licencia
\licencia{Sergio de la Rubia García-Carpintero, Miguel Millán Sánchez-Grande,
  Luis Muñoz Villarreal, Alicia Serrano Sánchez, Juan Miguel Torres Triviño}

\chapter*{Ficha de trabajo}
\begin{description}
\item[Código] T3
\item[Fecha] 26 de Abril de 2010
\item[Título] Calendario del proyecto
\end{description}

\begin{table}[!ht]
  \centering
  \begin{tabular}{lp{5cm}c}
    \multicolumn{3}{l}{\Large \textbf{Equipo} G4} \\ \\
    \multicolumn{1}{c}{\emph{Apellidos y nombre}} & 
    \multicolumn{1}{c}{\emph{Firma}} & \emph{Puntos} \\
    \hline \\
    de la Rubia García-Carpintero, Sergio & & 6 \\ \\
    Millán Sánchez-Grande, Miguel         & & 6 \\ \\
    Muñoz Villarreal, Luis                & & 6 \\ \\
    Serrano Sánchez, Alicia               & & 6 \\ \\
    Torres Triviño, Juan Miguel           & & 6 \\ \\
    \hline
  \end{tabular}
%  \caption{}\label{}
\end{table}

% Índices
\tableofcontents
% \listoffigures
% \listoftables

%% INICIO DEL DOCUMENTO %%%%%%%%%%%%%%%%%%%%%%%%%%%%%%%%%%%%%%%%%%%%%%%%%

\chapter*{Introducción}
En este trabajo se realiza la elaboración del calendario del proyecto
mediante el uso de la técnica PERT (Program Evaluation and Review Technique),
la cual proporciona un método para realizar una estimación de la duración
total del proyecto a partir de las actividades, su secuencia y la estimación
ponderada de la duración media de estas. 

Para la estimación de los tiempos PERT se aplica la técnica DELPHI, reuniendo
a 5 expertos en el tema, de tal modo que por su nivel de formación y grado de
conocimiento puedan aportar ideas y puntos de vistas diferentes al problema
en cuestión, con el fin de obtener juicios coherentes y enriquecidos con
respecto al problema. 
\chapter{Técnica DELPHI}
Mediante la técnica DELPHI se intenta obtener un consenso lo más fiable
posible del grupo de expertos
\section{Tabla de tareas}
\section{Tablas de tiempos DELPHI}

\chapter{Técnica PERT}
\section{Cálculos de los tiempos PERT}
\section{Cálculos de tiempos early y late}
\section{Cálculo de las holguras}
\section{Determinación de los tiempos críticos}

\chapter{Calendario}
\section{Tabla de tiempos de comienzo y finalizacion}
\section{Tabla de dependencias}
\section{Grafo de actividades}


\bibliographystyle{plain} 
\bibliography{t2}

\end{document}
