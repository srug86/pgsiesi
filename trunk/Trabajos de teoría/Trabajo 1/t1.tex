% Clase
\documentclass[11pt,a4paper,spanish,twoside]{book}

% Órdenes auxiliares
% Español
\usepackage[spanish]{babel}
\usepackage{lmodern}
\usepackage[utf8]{inputenc}

% Imágenes
\usepackage[pdftex]{graphicx}
\usepackage{latexsym}
\usepackage{fancybox}

% Ruta para las imágenes
\graphicspath{{imagenes/}}

% Rotaciones
\usepackage[twoside]{rotating}

% Referencias
\usepackage[spanish]{varioref}
\usepackage[pdftex,colorlinks=true,linkcolor=black]{hyperref}

% Colores
\usepackage{color}
\usepackage{colortbl}

% Párrafos
\setlength{\parskip}{6pt}

% Entorno Listings para código fuente
\usepackage{listingsutf8}[2007/11/11]

\lstset{
  frame=Ltb, %forma del cuadro
  framerule=0pt, %ancho del borde
  aboveskip=0.5cm, %separación de los números de línea
  framexleftmargin=0.4cm, %margen externo izquierdo
  framesep=0pt,
  rulesep=.4pt,
  rulesepcolor=\color{black},
  % 
  stringstyle=\ttfamily,
  showstringspaces = false,
  basicstyle=\scriptsize,
  keywordstyle=\bfseries,
  % 
  numbers=left,
  numbersep=15pt,
  numberstyle=\tiny,
  numberfirstline= false,
  %
  inputencoding=utf8/latin1
}

% minimizar fragmentado de listados
\lstnewenvironment{listing}[1][]{
  \lstset{#1}\pagebreak[0]}{\pagebreak[0]
}


% Fancyhdr - Encabezados y pies de página
\usepackage{fancyhdr}
% Márgenes
\headsep=8mm
\footskip=14mm

% Fancy Header Style Options
\pagestyle{fancy}               % Sets fancy header and footer
\fancyfoot{}                    % Delete current footer settings

% Sin mayúsculas en la cabecera
\lhead{\nouppercase{\rightmark}}
\rhead{\nouppercase{\leftmark}}

% Capítulo
\renewcommand{\chaptermark}[1]{ % Lower Case Chapter marker style
  \markboth{\chaptername\ \thechapter.\ #1}{}} 

% Sección
\renewcommand{\sectionmark}[1]{ % Lower case Section marker style
  \markright{\thesection.\ #1}} 

% Página
\fancyhead[LE,RO]{\bfseries\thepage} % Page number (boldface) in left on even
                                     % pages and right on odd pages

% ------------------ Macro para encabezados y pies de página-------------------
%    Uso: \encabezado{pares(pag izquierda)}
% -----------------------------------------------------------------------------
\def\encabezado{
  \fancyhead[RE]{\bfseries\leftmark}     % In the right on even pages
  \fancyhead[LO]{\bfseries\rightmark}  % In the left on odd pages
  \renewcommand{\headrulewidth}{0.5pt} % Width of head rule
}
% -----------------------------------------------------------------------------


% ------------------ Macro para insertar una imagen ---------------------------
%    Uso: \imagen{nombreFichero}{Ancho(cm))}{Etiqueta}{Identificador}
% -----------------------------------------------------------------------------
\usepackage{float}
\usepackage{ifthen}
\def\imagen#1#2#3#4{
  \begin{figure}[H]
    \begin{center}
      \ifthenelse{\equal{#2}{}}
      {\includegraphics{#1}}{\resizebox{#2cm}{!}{\includegraphics{#1}}}
      \ifthenelse{\equal{#3}{}}{}{\caption{#3}}
      \label{#4}
    \end{center}
  \end{figure}
}
% -----------------------------------------------------------------------------


% ------------------ Macro para la portada ------------------------------------
%    Uso: \portada{asignatura}{titulo}{subtítulo}{autor}{fecha}
% -----------------------------------------------------------------------------
\def\portada#1#2#3#4#5{
  \thispagestyle{empty}
  \vspace*{-3.3cm}

  \begin{minipage}[t]{14cm}
    \begin{center}
      \includegraphics[scale=0.25]{logoesi}
  
      \vspace*{1.5cm}
      {\Large \textbf{Universidad de Castilla-La Mancha\\ 
          Escuela Superior de Informática}\\}
    
      \vspace*{1.2cm}
      {\huge \textbf{#1}\\}
    
      \vspace*{1.5cm}
      {\huge #2}\\{\Large #3\\}
    
      \vspace*{1.5cm}
      {\large #4\\}
      \vspace*{1.4cm}
      \large{#5}
    \end{center}
  \end{minipage}

  \newpage
  \vspace*{1cm}
  \thispagestyle{empty} 
  \newpage
}
% -----------------------------------------------------------------------------

% ------------------ Macro para la licencia -----------------------------------
%    Uso: \portada{asignatura}{titulo}{subtítulo}{autor}{fecha}
% -----------------------------------------------------------------------------
\def\licencia#1{
  \thispagestyle{empty}  % Suprime la numeración de esta página
  \vspace*{16cm}
  \begin{small}
    \copyright~ #1. Se permite la copia, distribución y/o 
    modificación de este documento bajo los términos de la licencia de
    documentación libre GNU, versión 1.1 o cualquier versión posterior publicada
    por la {\em Free Software Foundation}, sin secciones invariantes. Puede
    consultar esta licencia en http://www.gnu.org. \\[0.2cm]
    Este documento fue compuesto con \LaTeX{}. 
  \end{small}
  \newpage
  \thispagestyle{empty}
  \vspace*{0cm}
  \newpage
}
% -----------------------------------------------------------------------------

% Code for creating empty pages
% No headers on empty pages before new chapter
\makeatletter
\def\cleardoublepage{\clearpage\if@twoside \ifodd\c@page\else
    \hbox{}
    \thispagestyle{empty}
    \newpage
    \if@twocolumn\hbox{}\newpage\fi\fi\fi}
\makeatother \clearpage{\pagestyle{empty}\cleardoublepage}


% Encabezado y pie de página
\encabezado

\begin{document}

% Silabación extra
\hyphenation{
a-sig-na-tu-ras
au-to-ma-ti-za-rá
ca-tá-lo-go
ca-rre-ra
cons-truc-ción
co-rres-pon-de
diag-nos-tico
fi-na-li-za-ción
ge-ne-ra-ción
in-fe-rior
man-te-ni-mien-to
me-dian-te
per-so-nal
pro-ce-di-mien-tos
pro-por-cio-na-rá
pu-bli-ca-da
re-qui-si-tos
res-pecto
u-su-a-rios
vi-lla-rre-al
}


% Portada
\portada{Planificación y Gestión de\\Sistemas de Información}
{Trabajo 1}{Plan de Sistemas y Tecnologías de Información}
{Sergio de la Rubia García-Carpintero\\Miguel Millán Sánchez-Grande\\
  Luis Muñoz Villarreal\\Alicia Serrano Sánchez\\
  Juan Miguel Torres Triviño}{10 de Marzo de 2009}

% Licencia
\licencia{Sergio de la Rubia García-Carpintero, Miguel Millán Sánchez-Grande,
  Luis Muñoz Vi\-lla\-rre\-al, Alicia Serrano Sánchez, Juan Miguel Torres 
Triviño}

% Índices
\tableofcontents
% \listoffigures
% \listoftables

%% INICIO DEL DOCUMETO %%%%%%%%%%%%%%%%%%%%%%%%%%%%%%%%%%%%%%%%%%%%%%%%%
\chapter*{Ficha de trabajo}
\begin{description}
\item[Código] T1
\item[Fecha] 10 de Marzo de 2010
\item[Título]Plan de Sistemas y Tecnologías de Información
\end{description}

\begin{table}[!ht]
  \centering
  \begin{tabular}{lp{5cm}c}
    \multicolumn{3}{l}{\Large \textbf{Equipo} G4} \\ \\
    \multicolumn{1}{c}{\emph{Apellidos y nombre}} & 
    \multicolumn{1}{c}{\emph{Firma}} & \emph{Puntos} \\
    \hline \\
    de la Rubia García-Carpintero, Sergio & & 10 \\ \\
    Millán Sánchez-Grande, Miguel         & & 10 \\ \\
    Muñoz Villarreal, Luis                & & 10 \\ \\
    Serrano Sánchez, Alicia               & & 10 \\ \\
    Torres Triviño, Juan Miguel           & & 10 \\ \\
    \hline
  \end{tabular}
%  \caption{}\label{}
\end{table}

\chapter*{Introducción}
A la hora de decidir la institución sobre la cual centrar nuestra 
investigación, empezamos analizando la posibilidad de buscar una empresa 
cercana geográficamente como podría haber sido el aeropuerto de Ciudad Real. 
Pero ante la posibilidad de encontrar dificultades a la hora de recopilar 
información nos decantamos por una entidad pública. Nuestra primera opción fue
la ESI pero encontramos más facilidades para encontrar información sobre sus 
planes de futuro vía internet sobre la Universidad de Málaga, de ahí nuestra 
elección.

La universidad de Málaga es una universidad pública, joven y dinámica que ha 
apostado decididamente por la calidad en la docencia, la investigación y por el
servicio al alumno. Cuenta con más de 40.000 alumnos matriculados y 2.000 
investigadores. 

La historia de la Universidad de Málaga, en adelante UMA, empieza en 1968 con 
la creación de la Asociación de Amigos de la Universidad de Málaga. Esta 
asociación buscaba la creación de la universidad debido a las necesidades de la 
ciudad, ya que era la única ciudad de España con una población superior a 
300.000 habitantes que carecía de ella. La UMA fue finalmente 
fundada por decreto de 18 de agosto de 1972.

Desde su sitio en el Sur de Europa, cuenta con un personal docente e 
investigador altamente preparado y un entorno cultural idóneo para hacer mas 
cómoda y fructífera su vida académica.

Su oferta de titulaciones, tanto regladas como propias, es muy amplia; ya que 
cuenta con los siguentes centros
\begin{itemize}
  \item E.T.S. de Arquitectura
  \item E.T.S.I. de Telecomunicación
  \item E.T.S.I. Industrial
  \item E.T.S.I. Informática
  \item E.U. de Ciencias de la Salud
  \item E.U. de Estudios Empresariales
  \item Facultad de Estudios Sociales y del Trabajo
  \item E.U. de Turismo
  \item E.U. Politécnica
  \item Facultad de Bellas Artes
  \item Facultad de Ciencias
  \item Facultad de C.C. Educación
  \item Facultad de C.C. de la Comunicación
  \item Facultad de C.C. Económicas
  \item Facultad de Derecho
  \item Facultad de Filosofía y Letras
  \item Facultad de Medicina
  \item Facultad de Psicología
  \item E.U. de Enfermería (Dipu. Prov.)
  \item E.U. Enfermería (Ronda)
  \item E.U. Magisterio (Antequera)
  \item Existe un proyecto que se negocia con el ayuntamiento de Estepona para 
      crear centros en este municipio.
\end{itemize}

En la figura \ref{Mapa_Malaga} se puede ver la situación geográfica de los 
mismos.

\imagen{PSIIntro_mapaUMA.pdf}{12.5}{Situación Geográfica de la UMA}{Mapa_Malaga}

\chapter{Inicio del plan de sistemas de información}
El objetivo de esta actividad es determinar la necesidad del Plan de Sistemas  
de Información y llevar a cabo el arranque formal del mismo, con el apoyo del
nivel más alto de la organización. Como resultado, se obtiene una descripción
general del Plan de Sistemas de Información que proporciona una definición
inicial del mismo, identificando los objetivos estratégicos a los que apoya,
así como el ámbito general de la organización al que afecta, lo que permite
implicar a las direcciones de las áreas afectadas por el Plan de Sistemas de
Información. 

Además, se identifican los factores críticos de éxito y los participantes en
el Plan de Sistemas de Información, nombrando a los máximos responsables.

\section{Análisis de la necesidad del PSI}
\subsection{Descripción general del PSI}\label{ss:1.1.1}
\subsubsection{Aprobación del inicio del PSI}
El siguiente Plan de Sistemas de Información tiene como finalidad crear un 
marco estratégico en el que la institución de enseñanza de la UMA pueda
mejorar y agilizar sus servicios de cara al personal administrativo, docente, 
investigador y estudiantil.

Para llevar a cabo el Plan de Sistemas de Información de manera exitosa 
y conseguir que ayude a la UMA a mejorar sus servicios utilizaremos como 
herramientas las Tecnologías de la Información.

\section{Identificación del alcance del PSI}
En base a lo expuesto en la sección \vref{ss:1.1.1}, \emph{\nameref{ss:1.1.1}},
se desarrolla lo siguiente:

\subsection{Descripción general del PSI}\label{ss:1.2.1}
\subsubsection{Ámbito y objetivos del PSI}
Para el desarrollo del PSI se trabajará directamente con la sección de 
Desarrollo Tecnológico e Innovación. Esta sección afecta, de una u otra
forma, a todos los sectores de nuestra organización, por lo que el PSI se
implantará con objeto de mejorar los principales procesos internos de la
organización.

\subsubsection{Objetivos estratégicos del PSI}
Los principales objetivos estratégicos que aborda el PSI serán:
\begin{enumerate}
\item Prestar apoyo en materia de las TI a todas las actividades relacionadas
con la investigación, la docencia y la gestión.
\item Dar soporte tecnológico a las nuevas demandas del sistema universitario.
\item Prestar servicios en la elaboración de contenidos audiovisuales.
\item Asegurar el acceso a los recursos bibliográficos y de información y 
promover su conservación, difusión e intercambio.
\end{enumerate}

\subsection{Factores críticos de éxito}
Los factores críticos de éxito serán la aceptación de
los usuarios al PSI y conseguir una rápida adaptación de éstos a las
Tecnologías de la Información.

\section{Determinación de responsables}
En base a lo expuesto en la sección \vref{ss:1.2.1}, \emph{\nameref{ss:1.2.1}},
se desarrolla lo siguiente:

\subsection{Descripción general del PSI}\label{ss:1.3.1}
\subsubsection{Responsables del PSI}
Para la correcta elaboración del PSI se requiere de una persona, el Jefe de 
Proyecto, que será la figura principal. Como apoyo al Jefe de Proyecto 
existe un responsable. También está la figura del coordinador del plan 
que será el encargado de ir dirigiendo su elaboración junto con su grupo de 
trabajo. Estas figuras serán representadas por las siguientes personas:
\begin{description}
\item[Jefa de Proyecto]
Dña. Adelaida de la Calle, rectora de la universidad.
\item[Responsable]
Dña. María Valpuesta, vicerrectorado de Innovación y Desarrollo Tecnológico.
\item[Coordinador]
D. Luis Muñoz Villarreal.
\item[Grupo de trabajo]
D. Sergio de la Rubia García-Carpintero.
D. Miguel Millán Sánchez-Grande.
Dña. Alicia Serrano Sánchez.
D. Juan Miguel Torres Triviño.
\end{description}

Dña. Adelaida de la Calle será la encargada de administrar el plan,
cumpliendo las tareas de seguimiento y control del mismo, revisión y 
estimación de resultados. El Coordinador se encargará de coordinar el
proyecto, de la gestión y resolución de incidencias que puedan aparecer 
durante el progreso del proyecto así como de la actualización de la 
planificación original.

Dña. María Valpuesta ofrecerá apoyo al Jefe de Proyecto y al Coordinador en
la actividad que necesite de tal apoyo. Las actividades donde tendrá que 
colaborar de forma activa el responsable serán determinadas por el Jefe de 
Proyecto. La buena coordinación entre el Responsable y el Jefe de Proyecto 
será esencial para el buen desarrollo del Plan.


\chapter{Definición y organización del PSI}
En esta actividad se detalla el alcance del plan, se organiza el equipo de
personas que lo va a llevar a cabo y se elabora un calendario de
ejecución. Todos los resultados o productos de esta actividad constituirán el
marco de actuación del proyecto más detallado que en PSI 1 en cuanto a
objetivos, procesos afectados, participantes, resultados y fechas de
entrega. 

\section{Especificación del ámbito y alcance}
En base a lo expuesto en la sección \vref{ss:1.3.1}, \emph{\nameref{ss:1.3.1}},
se desarrolla lo siguiente:

\subsection{Descripción general de procesos de la organización afectados}
\label{ss:2.1.1}
Dentro de la organización se han identificado los siguientes procesos afectados 
por el PSI:
\begin{enumerate}
  \item Contratación de personal.
  \item Implantación de los planes de estudios de cada carrera.
  \item Asignación del profesorado a las distintas asignaturas.
  \item Establececimiento de los horarios lectivos de cada facultad.
  \item Organización de aulas.
  \item Fijación de fechas, horarios y localizaciones de evaluación de
    asignaturas. 
\end{enumerate}

\subsection{Catálogo de objetivos del PSI} \label{ss:2.1.2}
Los objetivos que procuramos conseguir con el Plan de Sistemas de Información 
son los siguientes:
\begin{itemize}
  \item Mejorar y agilizar el trato con los estudiantes.
  \item Facilitar el acceso a los datos por parte de los integrantes de la 
    Universidad.
  \item Ayudar en la adaptación al Espacio Europeo de Educación Superior.
  \item Reducir el gasto en la contratación de personal para elaborar guías 
    docentes.
  \item Facilitar la actualización de los planes de estudio.
\end{itemize}

\section{Organización del PSI}
En base a la entrada externa \emph{Estructura organizativa} y a lo expuesto
en la sección \vref{ss:1.3.1}, \emph{\nameref{ss:1.3.1}}; en la sección
\vref{ss:2.1.1}, \emph{\nameref{ss:2.1.1}}; y en la sección \vref{ss:2.1.2},
\emph{\nameref{ss:2.1.2}}; se desarrolla lo siguiente:

\subsection{Catálogo de usuarios} \label{ss:2.2.1}
A lo largo del proceso de elaboración, implantación y gestión del Plan 
Estratégico en la Universidad de Málaga, intervienen diversos órganos expuestos 
en la figura \ref{Org_Uma}

\imagen{PSI2-2_organosUMA.pdf}{12.5}{Órganos de la UMA}{Org_Uma}

\subsection{Equipos de trabajo} \label{ss:2.2.2}
Esta jerarquía se puede concentrar en tres grupos de decisiones:
\renewcommand{\labelenumi}{\alph{enumi}$)$ }
\begin{enumerate}
\item Políticos:
  \begin{itemize}
  \item Asamblea General del Plan Estratégico formado por el Claustro de la 
    Universidad de Málaga.
  \item Consejo de Gobierno.
  \end{itemize}
\item Ejecutivos:
  \begin{itemize}
  \item Presidenta del Plan Estratégico - Rectora.
  \item Vicepresidente del Plan Estratégico - Vicerrector de Calidad, 
    Planificación Estratégica y Responsabilidad Social. Coordinador General del 
    Plan Estratégico.
  \end{itemize}
\item De consulta: dónde entran las distintas comisiones:
  \begin{itemize}
  \item Comisión Estratégica de la Universidad de Málaga.
  \item Comisión de Planificación Estratégica de la Universidad de Málaga.
  \item Comisión Asesora Externa del Plan Estratégico de la Universidad de 
    Málaga.
  \item Comisión Asesora Interna del Plan Estratégico de la Universidad de 
    Málaga.
  \end{itemize}
\item Técnicos:
  \begin{itemize}
  \item Comisión Técnica del Plan Estratégico.
  \item Directora Técnica del Plan Estratégico.
  \item Oficina del Plan Estratégico.
  \end{itemize}
\end{enumerate}

\section{Definición del plan de trabajo}
En base a lo expuesto en la sección \vref{ss:2.2.2},
\emph{\nameref{ss:2.2.2}}; en la sección \vref{ss:1.3.1},
\emph{\nameref{ss:1.3.1}}; en la sección \vref{ss:2.1.2},
\emph{\nameref{ss:2.1.2}}; en la sección \vref{ss:2.1.1},
\emph{\nameref{ss:2.1.1}}; y en la sección \vref{ss:2.2.1},
\emph{\nameref{ss:2.2.1}}; se desarrolla lo siguiente:

\subsection{Plan de trabajo} \label{ss:2.3.1}
En cada uno de los cinco procesos, como se refleja en la tabla
\ref{Tab:NecTI}, se tiene un calendario de entrega de sus 
respectivos sistemas. Este calendario ha sido elaborado teniendo en cuenta
las necesidades y las dimensiones de cada proceso dentro de la UMA, las
necesidades de la organización para tener funcionando cada sistema en el menor
tiempo posible y los recursos de los que se dispone para elaborar dichos
sistemas. 

\begin{table}[!h]
\centering
  \begin{tabular}{p{4cm}p{5cm}c}
    \textbf{Proceso} & \textbf{Participantes} & \textbf{Fecha} \\
    \hline \hline
    Contratación de personal & Sección de recursos humanos & 10 Abril 2010\\ 
    \hline
    Implantación de los planes de estudios de cada carrera & Equipo de
    desarrollo del software & 20 Junio 2010\\
    \hline
    Asignación del profesorado a las distintas asignaturas & Jefe de estudios
    cada centro  & 4 Julio 2010\\
    \hline
    Establecimiento de los horarios lectivos de cada facultad & Herramienta
    automática con prioridades del profesor & 18 Julio 2010\\
    \hline
    Organización de aulas & Herramienta automática & 18 Julio 2010\\
    \hline
    Fijación de fechas, horarios y localizaciones de evaluación de
    asignaturas & Herramienta automática & 18 Julio 2010\\
    \hline
  \end{tabular}
  \caption{Tabla de necesidades del TI} \label{Tab:NecTI}
\end{table}

\section{Comunicación del plan de trabajo}
En base a lo expuesto en la sección \vref{ss:2.3.1},
\emph{\nameref{ss:2.3.1}}; y en la sección \vref{ss:2.2.1},
\emph{\nameref{ss:2.2.1}}; se desarrolla lo siguiente:

\subsection{Plan de trabajo}
Para la aceptación del Plan de Trabajo se convoca una junta extraordinaria en
la cual el equipo de desarrollo del software presenta la herramienta al
cuerpo directivo de la UMA.
\subsubsection{Aceptación del Plan de Trabajo}
Tras la aceptación del Plan de Trabajo, se comunica al resto de la
universidad mediante un comunicado oficial, además de informar por correo
electrónico y una publicación en la web.


\chapter{Estudio de la información relevante}
El objetivo de esta actividad es recopilar y analizar todos los antecedentes
generales que puedan afectar a los procesos y a las unidades organizativas
implicadas en el Plan de Sistemas de Información, así como a los resultados
del mismo. Pueden ser de especial interés los estudios realizados con
anterioridad al Plan de Sistemas de Información, relativos a los sistemas de
información de su ámbito, o bien a su entorno tecnológico, cuyas conclusiones
deben ser conocidas por el equipo de trabajo del Plan de Sistemas de
Información. 

La información obtenida en esta actividad se tendrá en cuenta en
la elaboración de los requisitos.

\section{Selección y análisis de antecedentes}
En base a la entrada externa \emph{Información relevante} y a lo expuesto 
en la sección \vref{ss:2.1.1}, \emph{\nameref{ss:2.1.1}}; 
en la sección \vref{ss:2.1.2}, \emph{\nameref{ss:2.1.2}}; y
en la sección \vref{ss:2.2.1}, \emph{\nameref{ss:2.2.1}}; 
se desarrolla lo siguiente:

\subsection{Análisis de antecedentes} \label{ss:3.1.1}
En esta sección primeramente seleccionaremos aquellos antecedentes que son
relevantes para nuestro plan. Posteriormente analizaremos como pueden
afectar a este plan. Los antecedentes que nosotros consideraremos serán el
plan estratégico de sistemas de información actual, plan general
informático, etc. 
El plan establece renovación de equipos, aumento del grado de cualificación
del personal, renovación de herramientas de última generación para el
desarrollo de software, facilidad a la hora de obtener información, etc. 

La Universidad de Málaga se caracterizada por la excelencia en el proceso de
enseñanza-aprendizaje y es reconocida por la excelencia investigadora, la
transferencia de conocimiento y la promoción de la innovación. Una
Universidad que garantiza el desarrollo personal y profesional. Está
comprometida con su entorno tecnológico, medioambiental, económico, social,
histórico y cultural, y que incorpora en su actividad los principios de
responsabilidad social. 

En cuanto a proyectos más importantes establecidos en el plan y que se han
conseguido desde el momento en que se estableció dicho plan hasta la fecha
actual son: 
\begin{itemize}
\item Mejora de los procesos de desarrollo software a través de las
  TI. Incorporando herramientas que permitan un mejor control de riesgos,
  seguimiento de proyectos, etc.  

\item Renovación de equipos y herramientas para trabajar siempre con las
  últimas versiones disponibles en el mercado.

\item Proveer a la alta dirección de algún paquete informático, de desarrollo
  interno o adquirido, para realizar informes estadísticos, estudios de
  mercado, etc. 

\item Incorporar un Espacio Virtual de Aprendizaje intentando favorecer
  una mayor, eficiente y continua formación del personal.  
\end{itemize}
        
\section{Valoración de antecedentes} 
En base a la entrada externa \emph{Información relevante} y a lo expuesto 
en la sección \vref{ss:3.1.1}, \emph{\nameref{ss:3.1.1}}; y
se desarrolla lo siguiente:

\subsection{Catálogo de requisitos} \label{ss:3.2.1}
\subsubsection{Requisitos generales}
Según la información anterior se concluye que las líneas de trabajo de
la organización son según dicha información las siguientes: 
\begin{itemize}
\item Adecuar la oferta de estudios a las necesidades de formación de la
  sociedad.
\item Conseguir una docencia de excelencia, potenciar el dominio de nuevas
  tecnologías y mejorar los resultados académicos de los estudiantes.
\item Garantizar la formación tanto del personal de administración y servicios
  como del docente e investigador.
\item Garantizar el acceso electrónico de los estudiantes a todos los servicios
  públicos universitarios.
\item Resultados clave destinados a la consolidación de un sistema de
  comunicación que contribuya a satisfacer la misión, la visión, los valores
  y los objetivos de la institución y ayude a la consecución de resultados
  sobresalientes por la Universidad y a reforzar la reputación de la
  Universidad de Málaga.
\item Resultados en los usuarios que incrementen el nivel de satisfacción y
  mejora de sus expectativas.
\end{itemize}
        
El análisis de la fase de utilización de las TI en la que se enmarca la 
UMA está en condición de llevarse a cabo. Este análisis sirve después
para enmarcar otros aspectos y decisiones que se tomen en este plan. Se puede
concluir que la empresa se enmarca en el nivel de formalización/control. Por
otra parte también se observa cómo cada vez son más las herramientas
informáticas y Sistemas de Información que se incorporan a la organización
con el fin de controlarla y gestionarla. 

\chapter{Identificación de requisitos}
El objetivo final de esta actividad va a ser la especificación de los
requisitos de información de la organización, así como obtener un modelo de
información que los complemente.

Para conseguir este objetivo, se estudia el proceso o procesos de la
organización incluidos en el ámbito del Plan de Sistemas de Información. Para
ello es necesario llevar a cabo sesiones de trabajo con los usuarios,
analizando cada proceso tal y como debería ser, y no según su situación
actual, ya que ésta puede estar condicionada por los sistemas de información
existentes. 

Del mismo modo, se identifican los requisitos de información, y
se elabora un modelo de información que represente las distintas entidades
implicadas en el proceso, así como las relaciones entre ellas. 

Por último, se clasifican los requisitos identificados según su prioridad,
con el objetivo de incorporarlos al catálogo de requisitos del Plan de
Sistemas de Información. 

\section{Estudio de los procesos del PSI}
En base a lo expuesto 
en la sección \vref{ss:2.1.1}, \emph{\nameref{ss:2.1.1}}; 
en la sección \vref{ss:2.1.2}, \emph{\nameref{ss:2.1.2}}; 
en la sección \vref{ss:2.2.1}, \emph{\nameref{ss:2.2.1}}; y
en la sección \vref{ss:2.2.2}, \emph{\nameref{ss:2.2.2}}; 
se desarrolla lo siguiente:

\subsection{Modelo de procesos de la organización} \label{ss:4.1.1}
En esta sección se muestran los procesos más significativos de la
organización en el ámbito del Plan de Sistemas e Información.  

\subsubsection{Contratación de personal}
Indica el proceso de selección e incorporación de nuevos empleados para la
realización de nuestro software.

\subsubsection{Implantación de los planes de estudios de cada carrera}
Proceso que se encarga de recoger la información de los planes de estudio de
cada una de las carreras asociadas a la Universidad de Málaga. Para ello, se
tienen los planes de estudio y junto con las directrices del Espacio Europeo
de Educación Superior (EEES) se realizan los nuevos Planes de Estudio
adaptados.

\subsubsection{Asignación del profesorado a distintas asignaturas}
Se trata de buscar profesores cualificados para impartir las distintas
asignaturas que están contenidas en los planes de estudio. Para ello, se
tiene una lista con las características de cada profesor. Tras la selección
de estos profesores, se realiza la asignación a las distintas asignaturas de
los planes de estudio.
 
\subsubsection{Establecimiento de horarios lectivos de cada facultad}
En este proceso tiene como entrada los planes de estudio de cada
facultad. Tras una serie de reuniones con cada facultad y siguiente sus
espectativas y directrices obtenemos el calendario final.

\subsubsection{Organización de aulas}
Las aulas disponibles de cada facultad se distribuyen acorde con los horarios
lectivos obtenidos en el proceso anterior. Este proceso proporciona una
lista de aulas asignadas. 

\subsubsection{Fijación de fechas, horarios y localizaciones de evaluación de
  asignaturas}
Para este proceso se necesitan los planes de estudio, así como el calendario
académico y las aulas que se necesitan para la evaluación de asignaturas. El
resultado de este proceso genera una lista que contiene el nombre de la
asignatura, fechas de evaluación, hora y emplazamiento. 

\section{Análisis de las necesidades de información} 
En base a lo expuesto 
en la sección \vref{ss:4.1.1}, \emph{\nameref{ss:4.1.1}}; 
se desarrolla lo siguiente:

\subsection{Modelos y necesidades de información} \label{ss:4.2.1}
Según los procesos identificados en la sección anterior se establecen las
necesidades de información: 

\subsubsection{Contratación de personal}
Información sobre posibles empleados. Establecer convenios con universidades,
etc. También se recoge toda la información relativa a procesos de
selección para poder recuperar a solicitantes que no fueron seleccionados, o
evitar a otros.  

\subsubsection{Implantación de los planes de estudios de cada carrera}
Se necesita un mecanismo eficaz por el cual se identifiquen los planes de
estudio públicos. También hay que mantener actualizado y estudiar qué
tecnologías se están demandando actualmente.  

\subsubsection{Asignación del profesorado a las distintas asignaturas}
La información necesaria para este proceso debe tener un buen equipo de
selección y por lo tanto la UMA debe tener información sobre capacidades,
experiencia, etc.
Se tiene una base de datos con los profesores y sus principales
características para agilizar la selección. 

\subsubsection{Establecimiento de horarios lectivos de cada facultad}
Es necesario tener una herramienta que elabore los horarios de clase
relacionando los créditos de las asignaturas, horas magistrales, conflictos
entre diferentes carreras de la misma facultad, etc.

\subsubsection{Organización de aulas}
Se tiene una base de datos con información sobre localización, estado,
capacidad, recursos, etc. 

\subsubsection{Fijación de  fechas, horarios y localizaciones de evaluación de
  asignaturas}
Se realiza mediante la utilización de una herramienta que complete de manera
eficiente y automática 
el calendario de evaluación de asignaturas teniendo en cuenta las aulas de
las facultades, asignaturas del plan de estudios, calendario académico,
tiempo de realización de la evaluación, etc.

\section{Catalogación de requisitos}
En base a lo expuesto 
en la sección \vref{ss:2.1.1}, \emph{\nameref{ss:2.1.1}}; 
en la sección \vref{ss:3.2.1}, \emph{\nameref{ss:3.2.1}}; 
en la sección \vref{ss:4.1.1}, \emph{\nameref{ss:4.1.1}}; y
en la sección \vref{ss:4.2.1}, \emph{\nameref{ss:4.2.1}};
se desarrolla lo siguiente:

\subsection{Catálogo de requisitos} \label{ss:4.3.1}
En la tabla \ref{Tab:CatReq} se pueden ver los requisitos de los procesos y
su prioridad.

\begin{table}[!h]
\centering
\small
  \begin{tabular}{c|p{5cm}|c}
    \textbf{Proceso} & \textbf{Requisitos} & \textbf{Prioridad} \\
    \hline \hline
    \multirow{2}{4cm}{Contratación de personal}
    & Espacio en la web de la UMA para la solicitud de empleo & Alta \\
    \cline{2-3}
    & Herramienta para gestionar los procesos de selección de personal &
    Media-Alta\\ \hline

    \multirow{2}{4cm}{Implantación de los planes de estudio de cada
      carrera}
    & Servicios web para la obtención y consulta de los planes de estudio &
    Alta\\ \cline{2-3}
    & Herramienta de administración de todas las características de cada
    asignatura & Media\\  \hline

    \multirow{3}{4cm}{Asignación del profesorado a las distintas asignaturas}
    & Base de datos de profesores que imparten clase en la UMA & Alta \\ 
    \cline{2-3}
    & Herramienta de búsqueda del profesorado, aplicando criterios de selección
    & Media\\ \cline{2-3}
    & Herramienta de asignación de los distintos profesores a las
    asignaturas correspondientes & Media\\ \hline
     
    \multirow{2}{4cm}{Establecimiento de horarios lectivos de cada facultad}
    & Base de datos de los planes de estudio & Alta\\ \cline{2-3}
    & Herramienta de elaboración de horarios según las características de
    las asignaturas & Media \\ \hline
    
    \multirow{3}{4cm}{Organización de aulas}
    & Base de datos de las aulas disponibles en cada facultad & Alta\\
    \cline{2-3}
    & Horario elaborado en el proceso anterior & Alta \\
    \cline{2-3}
    & Herramienta que realice automáticamente la asignación
    aulas-asignaturas teniendo en cuenta localización, estado, capacidad,
    recursos, etc & Media \\ \hline
    
    \multirow{4}{4cm}{Fijación de fechas, horarios y localizaciones de
      evaluación de asignaturas} 
    & Base de datos de los planes de estudio & Alta \\
    \cline{2-3}
    & Base de datos de las aulas disponibles & Alta \\
    \cline{2-3}
    & Calendario académico para conocer las fechas de los exámenes & Alta \\
    \cline{2-3}
    & Herramienta que genere de forma automática la organización de los
    exámenes, contando con la duración del examen, profesores implicados,
    facultades donde se imparte la asignatura, ... & Media-Alta \\ 
  \end{tabular}
  \caption{Catálogo de requisitos de los procesos} \label{Tab:CatReq}
\end{table}


\chapter{Estudio de los sistemas de información actuales}
El objetivo de esta actividad es obtener una valoración de la situación
actual al margen de los requisitos del catálogo, apoyándose en criterios
relativos a facilidad de mantenimiento, documentación, flexibilidad,
facilidad de uso, etc. En esta actividad se debe tener en cuenta la opinión
de los usuarios, ya que aportan elementos de valoración, como por ejemplo,
su nivel de satisfacción con cada sistema de información. 

Se seleccionan los
sistemas de información actuales que son objeto del análisis y se lleva a
cabo el estudio de los mismos con la profundidad y el detalle que se
determine conveniente en función de los objetivos definidos para el Plan de
Sistemas de Información. Este estudio permite, para cada sistema, determinar
sus carencias y valorarlos. Esta valoración se utiliza en la actividad
Diseño del Modelo de Sistemas de Información (PSI 6), donde se analiza la
cobertura de los sistemas de información actuales con respecto a los
requisitos.

\section{Alcance y objetivos del estudio de los sistemas de 
  información actuales} 
En base a lo expuesto 
en la sección \vref{ss:2.1.1}, \emph{\nameref{ss:2.1.1}}; 
en la sección \vref{ss:2.1.2}, \emph{\nameref{ss:2.1.2}}; 
en la sección \vref{ss:2.2.1}, \emph{\nameref{ss:2.2.1}}; y
en la sección \vref{ss:2.2.2}, \emph{\nameref{ss:2.2.2}}; 
se desarrolla lo siguiente:

\subsection{Catálogo de objetivos de PSI} \label{ss:5.1.1}
El dominio de sistemas de información a considerar queda fijado por
aquellos procesos de la organización que afectan al Plan, así como por los 
objetivos definidos para y por este Plan de Sistemas de Información, como se
muestra en la tabla \ref{Tab:ObjPro}. En la siguiente sección se puede ver
cuáles son esos sistemas de información, su estado y valoración.

\begin{table}[!ht]
  \centering
  \begin{tabular}{p{5cm}|p{5cm}}
    \textbf{Objetivos} & \textbf{Procesos} \\
    \hline \hline
    Mejorar y agilizar el trato con los estudiantes & Producción software \\
    \hline
    Facilitar el acceso a los datos & 
    Producción software Formación del personal. \\
    \hline
    Adaptación al Espacio Europeo de Educación Superior & 
    Formación del personal Control de Proyectos \\
    \hline
    Reducir el gasto en personal & 
    Producción software Gestión de Personal Control de Proyectos\\
    \hline
    Facilitar la actualización de los planes de estudios & 
    Producción software Formación del personal \\
    \hline
  \end{tabular}
  \caption{Tabla de relación entre Objetivos y Procesos de la organización}
  \label{Tab:ObjPro}
\end{table}

\subsection{Identificación de sistemas de información actuales afectados por
  el PSI} \label{ss:5.1.2}
Todos los objetivos del PSI afectan únicamente al campus virtual, ya que la
herramienta software que se va a desarrollar está integrada dentro de este.

\section{Análisis de los sistemas de información actuales}
En base a lo expuesto 
en la sección \vref{ss:2.1.1}, \emph{\nameref{ss:2.1.1}}; 
en la sección \vref{ss:5.1.1}, \emph{\nameref{ss:5.1.1}}; y
en la sección \vref{ss:5.1.2}, \emph{\nameref{ss:5.1.2}}; 
se desarrolla lo siguiente:

\subsection{Descripción general de sistemas de información actuales} 
\label{ss:5.2.1}
Los Sistemas de Información de la organización actuales afectados con 
relevancia por el Plan son los que a continuación aparecen. Para cada sistema 
de información se recogen las características básicas, así como su utilidad.
\begin{itemize}
\item La página web, es la forma que tiene la universidad de interactuar,
  de darse a conocer con la gente externa. También es una manera de
  comunicarse con los usuarios propios.
\item campusvirtual.uma.es es un lugar de encuentro de la comunidad
  universitaria de la UMA donde alumnado, profesorado y personal de
  administración y servicios pueden relacionarse sin que sean coincidentes en
  el espacio y en el tiempo. Las actividades se organizan en base a la
  herramienta de teleformación Campus Virtual, un entorno virtual de
  enseñanza-aprendizaje desarrollado a partir de Moodle. 
\item Wifi UMA, es la base de los puntos anteriores y no varía sus funciones.
\end{itemize}

\section{Valoración de los sistemas de información actuales}
En base a lo expuesto 
en la sección \vref{ss:5.1.1}, \emph{\nameref{ss:5.1.1}}; y
en la sección \vref{ss:5.2.1}, \emph{\nameref{ss:5.2.1}}; 
se desarrolla lo siguiente:

\subsection{Valoración de la situación actual} \label{ss:5.3.1}
Una vez descritas las características de los principales sistemas de 
información actuales, se va a analizar sus problemas reales, las opiniones de 
los usuarios, etc. Se finaliza el estudio con una valoración de cada sistema.

\begin{itemize}
  \item La web, aunque desde el punto de vista del usuario cumple con
    eficiencia su cometido, se aprecian ciertas carencias tanto en su
    contenido como en su organización. 
  \item El campus virtual cumple con eficiencia sus funciones, además de
    obtener una alta valoración entre sus usuarios, dado que es un sistema
    relativamente ``nuevo''; está al día en las necesidades actuales del
    sistema. Por lo cual no necesita ninguna reforma. 
  \item La Wifi UMA, el punto más débil de los que se han estudiado. No se ha
    adaptado al sucesivo incremento de flujo de datos y usuarios. No
    consigue una buena valoración de sus usuarios.  
\end{itemize}

\chapter{Diseño del modelo de sistemas de información}
El objetivo de esta actividad es identificar y definir los sistemas de
información que van a dar soporte a los procesos de la organización afectados
por el Plan de Sistemas de Información. Para ello, en primer lugar, se
analiza la cobertura que los sistemas de información actuales dan a los
requisitos recogidos en el catálogo elaborado en las actividades Estudio de
la Información Relevante (PSI 3) e Identificación de Requisitos (PSI 4). Esto
permitirá efectuar un diagnóstico de la situación actual, a partir del cual
se seleccionan los sistemas de información actuales considerados válidos,
identificando las mejoras a realizar en los mismos. 

Por último, se definen los nuevos sistemas de información necesarios para
cubrir los requisitos y funciones de los procesos no soportados por los
sistemas actuales seleccionados. 

Teniendo en cuenta los resultados anteriores, se elabora el modelo de
sistemas de información válido para dar soporte a los procesos de la
organización incluidos en el ámbito del Plan de Sistemas de Información. 

\section{Diagnóstico de la situación actual}
Analizados los diferentes procesos y los sistemas de información actuales que
les dan cobertura, se concluye lo siguiente:

En base a lo expuesto 
en la sección \vref{ss:4.2.1}, \emph{\nameref{ss:4.2.1}}; 
en la sección \vref{ss:4.3.1}, \emph{\nameref{ss:4.3.1}}; y 
en la sección \vref{ss:5.3.1}, \emph{\nameref{ss:5.3.1}};
se desarrolla lo siguiente:

\subsection{Diagnóstico de la situación actual} \label{ss:6.1.1}
\subsubsection{Contratación de personal}
\begin{itemize}
\item Espacio en la web de la UMA para la solicitud de empleo $\to$ Sí.
\item Herramienta para gestionar los procesos de selección de personal $\to$ No.
\end{itemize}

\subsubsection{Implamantación de los planes de estudios cada carrera}
\begin{itemize}
\item Base de datos de los planes de estudio $\to$ Si, pero ésta quedará
  obsoleta cuando esté totalmente implantado el EEES, por lo que se
  necesitará una nueva para almacenar los datos de las nuevas titulaciones. 
\item Servicios web para la obtención y consulta de los planes de estudio
  $\to$ Sí, aunque debe actualizarse. 
\item Herramientas de administración de todas las características de cada
  asignatura $\to$ No. 
\end{itemize}

\subsubsection{Asignación del profesorado a las distintas asignaturas}
\begin{itemize}
\item Base de datos de profesores que imparten clase en la UMA $\to$ Sí, pero
  se deben ampliar los contenidos existentes para dotarla de una mayor
  funcionalidad. 
\item Herramientas de búsqueda de profesorado, aplicando criterios de
  selección $\to$ Sí, aunque debe mejorarse el motor de búsqueda actual. 
\item Herramientas de asignación de los distintos profesores a las
  asignaturas correspondientes $\to$ No.
\end{itemize}

\subsubsection{Establecimiento de los horarios lectivos de cada facultad}
\begin{itemize}
\item Base de datos de los planes de estudio $\to$ Sí, pero ésta quedará
  obsoleta cuando esté totalmente implantado el EEES, por lo que se
  necesitará una nueva para almacenar los datos de las nuevas titulaciones. 
\item Herramientas de elaboración de horarios según las características de
  las asignaturas $\to$ No. 
\end{itemize}

\subsubsection{Organización de aulas}
\begin{itemize}
\item Base de datos de las aulas disponibles de cada facultad $\to$ Sí, pero
  se deben ampliar los contenidos existentes para dotarla de una mayor
  funcionalidad. 
\item Horario elaborado en el proceso anterior $\to$ Sí.
\item  Herramienta que realice automáticamente la asignación
  aulas-asignaturas teniendo en cuenta localización, estado, capacidad,
  recursos, etc $\to$ No.
\end{itemize}

\subsubsection{Fijación de fechas, horarios y localizaciones de evaluación de
  asignaturas} 
\begin{itemize}
\item Base de datos de los planes de estudio $\to$ Sí, pero ésta quedará
  obsoleta cuando esté totalmente implantado el EEES, por lo que se
  necesitará una nueva para almacenar los datos de las nuevas titulaciones.  
\item Base de datos de las aulas disponibles $\to$ Si, pero se deben ampliar
  los contenidos existentes para dotarla de una mayor funcionalidad.
\item Calendario académico para conocer las fechas de los exámenes $\to$ Sí.
\item Herramienta que genere de forma automática la organización de los
    exámenes, contando con la duración del examen, profesores implicados,
    facultades donde se imparte la asignatura\... $\to$ No.
\end{itemize}

\section{Definición del modelo de sistemas de información}
En base a lo expuesto 
en la sección \vref{ss:6.1.1}, \emph{\nameref{ss:6.1.1}}; 
en la sección \vref{ss:4.1.1}, \emph{\nameref{ss:4.1.1}}; 
en la sección \vref{ss:4.2.1}, \emph{\nameref{ss:4.2.1}}; y
en la sección \vref{ss:4.3.1}, \emph{\nameref{ss:4.3.1}}; 
se desarrolla lo siguiente:

\subsection{Modelo de sistemas de información} \label{ss:6.2.1}
A continuación se detalla la funcionalidad principal de cada proceso:

\begin{description}
\item[Contratación de personal] A través de la página web se publicitarán las
  plazas demandadas para la realización del proyecto. Un sistema de selección
  automatizado nos ayudará a filtrar los candidatos más acordes con el
  puesto. 
\item[Implamantación de los planes de estudios de cada carrera] Se
  proporcionará un mecanismo que facilite la transición entre planes de
  estudio. Basándose en las bases de datos del plan antiguo y siguiendo con
  las directrices del plan de EEES, se crearán nuevas bases de datos que
  alberguen las características del nuevo plan. 
\item[Asignación del profesorado a las distintas asignaturas] Un nuevo motor
  de búsqueda de profesores permitirá una rápida visión de su
  situación. Además, se contará con la funcionalidad de la modificación de
  sus datos, como la asignatura o asignaturas que tiene asignadas. 
\item[Establecimiento de los horarios lectivos de cada facultad] Mediante un
  calendario y el plan actual, además de otros requisitos opcionales, se
  generarán unos horarios para cada facultad. 
\item[Organización de aulas] A través de los horarios generados
  automáticamente y teniendo en cuenta las posibilidades que ofrecen las
  instalaciones de cada facultad, se generará una relación de asignaciones
  entre aulas y asignaturas, dentro de los horarios. 
\item[Fijación de fechas, horarios y localizaciones de evaluación de
  asignaturas] A través del plan establecido, y utilizando las asignaciones
  de aulas y horarios generados por los procesos anteriormente descritos, se
  automatizará la asignación de exámenes dentro de los horarios y las aulas
  disponibles, siguiendo los periodos descritos en el plan. 
\end{description}
 
\chapter{Definición de la arquitectura tecnológica}
En esta actividad se propone una arquitectura tecnológica que de soporte al
modelo de información y de sistemas de información incluyendo, si es
necesario, opciones. Para esta actividad se tienen en cuenta especialmente
los requisitos de carácter tecnológico, aunque es necesario considerar el
catálogo completo de requisitos para entender las necesidades de los procesos
y proponer los entornos tecnológicos que mejor se adapten a las mismas. 

\section{Identificación de las necesidades de infraestructura 
  tecnológica}
En base a la entrada externa \emph{Entorno tecnológico actual y estándares} y
a lo expuesto 
en la sección \vref{ss:6.2.1}, \emph{\nameref{ss:6.2.1}}; 
en la sección \vref{ss:4.2.1}, \emph{\nameref{ss:4.2.1}}; 
en la sección \vref{ss:4.3.1}, \emph{\nameref{ss:4.3.1}};
en la sección \vref{ss:5.2.1}, \emph{\nameref{ss:5.2.1}}; y
en la sección \vref{ss:5.3.1}, \emph{\nameref{ss:5.3.1}}; 
se desarrolla lo siguiente:

Lo que se pretende analizar en esta tarea es la infraestructura tecnológica
que necesita la UMA, así como proponer algunas alternativas viables desde
el punto de vista tecnológico. Para ello, primero se identifican las
necesidades y a continuación se proponen posibles alternativas de
infraestructura tecnológica. 

\subsection{Alternativas de arquitectura tecnológica} \label{ss:7.1.1}
En la tabla \ref{Tab:TabNec} se pueden observar las necesidades 
tecnológicas con respecto a lo actual.
\begin{table}[!h]
\centering
  \begin{tabular}{p{2.7cm}|p{5cm}|c}
    \textbf{Sistemas de información} & \textbf{Necesidades de TI} &
    \textbf{¿Cubierta?} \\
    \hline \hline
    \multirow{3}{4cm}{Página Web}
    & Servidores & Sí \\ \cline{2-3}
    & SGBD & Sí, mejorable \\ \cline{2-3}
    & Equipo de diseño propio en cuanto a infraestructuras de TI & Sí\\ 
    \cline{2-3}
    & Herramientas de desarrollo y mantenimiento web & Sí\\ \hline
    
    \multirow{3}{4cm}{Campus virtual}
    & Servidores & Sí\\ \cline{2-3}
    & Herramienta de publicación de documentos & Sí\\ \cline{2-3}
    & Herramientas de traducción de documentos & Mejorable\\ \cline{2-3}
    & Equipo de diseño propio en cuanto a infraestructurasde TI & Sí\\
    \cline{2-3}
    & Correo interno & Sí, mejorable\\ \hline
    
    \multirow{3}{4cm}{Wifi UMA}
    & Sistema de acceso a la wifi & Sí\\ \cline{2-3}
    & Puntos de acceso inalámbricos suficientes & No\\ \cline{2-3}
    & Cobertura de red inalámbrica suficiente & Sí, mejorable\\ \cline{2-3}
    & Calidad de la señal dentro del alcance wifi & Mejorable\\ \hline
    
  \end{tabular}
  \caption{Tabla de necesidades del TI} \label{Tab:TabNec}
\end{table}

\section{Selección de la arquitectura tecnológica}
En base a la entrada externa \emph{Entorno tecnológico actual y estándares} y
a lo expuesto 
en la sección \vref{ss:7.1.1}, \emph{\nameref{ss:7.1.1}}; 

En este apartado se decide la mejor alternativa tecnológica
dentro de los aspectos inexistentes o mejorables recogidos en el punto
anterior.

\subsection{Arquitectura tecnológica} \label{ss:7.2.1}
\subsubsection{SGBD}
En la tabla \ref{Tab:TabNecTI} se especifican los beneficios y los costes que
conllevan los sistemas de gestión de base de datos.
\begin{table}[!h]
\centering
  \begin{tabular}{p{3cm}p{4.7cm}p{2.3cm}}
    \textbf{Alternativas TI} & \textbf{Beneficios} &
    \textbf{Costes} \\
    \hline \hline
    Oracle\cite{ora} & Gran capacidad y potencia & Coste elevado\\ \hline
    SQL Server\cite{sql} & Potencia suficiente para la mayoría de SI y fácil
    integración al pertenecer a Microsoft, la plataforma utilizada & Coste
    inferior a otras alterativas más potentes\\ \hline
  \end{tabular}
\caption{Tabla de necesidades del TI} \label{Tab:TabNecTI}
\end{table}

\subsubsection{Herramientas de traducción de documentos}
Se utilizan programas como \emph{Textanalyser}, \emph{Stylewriter},
\emph{Antidoto} y \emph{Duden Korrektor Plus} que permiten la corrección y el
etiquetado de los trabajos en lengua inglesa, francesa y alemana,
respectivamente\cite{tra}. 

\subsubsection{Correo interno}
El sistema ADAMAIL\cite{ada} y la gama de soluciones de comunicación privada,
son software basados en tecnologías libres, que tienen como finalidad generar un
gestor de correos, envíos de archivos comunicación y datos completos. El
sistema basa su funcionamiento en el uso de una base de datos, lo que permite
llevar un completo control del los mensajes enviados y recibidos, así mismo
que un control de mensajes no deseados, y la documentación enviada por el
mismo. 

\subsubsection{Equipamiento de la red inalámbrica}
Aunque la UMA tiene cobertura wifi en todas sus bibliotecas y centros desde
el 2006, el continuo crecimiento de la universidad tanto en infraestructuras
como en alumnos, hace necesario un crecimiento proporcional de los servicios
wifi. En función a esto, se propone la implantación de repetidores de señal
con un coste asequible con respecto a la mejora de calidad de la señal.


\chapter{Definición del plan de acción}
En el Plan de Acción, que se elabora en esta actividad, se definen los
proyectos y acciones a llevar a cabo para la implantación de los modelos de
información y de sistemas de información, determinados en las actividades
Identificación de Requisitos (PSI 4) y Diseño del Modelo de Sistemas de
Información (PSI 6), con la arquitectura tecnológica propuesta en la
actividad Definición de la Arquitectura Tecnológica (PSI 7). El conjunto de
estos tres modelos constituye la arquitectura de información. 

Dentro del Plan de Acción se incluye un calendario de proyectos, con posibles 
alternativas, y una estimación de recursos, cuyo detalle será mayor para los 
más inmediatos. Para la elaboración del calendario se tienen que analizar las
distintas variables que afecten a la prioridad de cada proyecto y sistema de
información. El orden definitivo de los proyectos y acciones debe pactarse
con los usuarios, para llegar a una solución de compromiso que resulte la
mejor posible para la organización. 

Por último, se propone un plan de mantenimiento para el control y seguimiento 
de la ejecución de los proyectos, así como para la actualización de los 
productos finales del Plan de Sistemas de Información. 

\section{Definición de proyectos a realizar}
En base a lo expuesto 
en la sección \vref{ss:1.2.1}, \emph{\nameref{ss:1.2.1}}; 
en la sección \vref{ss:5.1.1}, \emph{\nameref{ss:5.1.1}}; 
en la sección \vref{ss:4.2.1}, \emph{\nameref{ss:4.2.1}}; 
en la sección \vref{ss:6.2.1}, \emph{\nameref{ss:6.2.1}}; y
en la sección \vref{ss:7.2.1}, \emph{\nameref{ss:7.2.1}}; 
se desarrolla lo siguiente:

\subsection{Plan de proyectos} \label{ss:8.1.1}
\subsubsection{Proyecto a corto plazo}
Este proyecto, llamado SIGGD (Sistema Integrado de Generación de Guías 
Docentes), consistirá en la generación automática de guías docentes a partir de
la información disponible.

Como se ha explicado en apartados anteriores, los procesos a seguir son:
contratación de personal, implantación de los planes de estudios de cada 
carrera, asignación del profesorado a las distintas asignaturas, 
establecimiento de los horarios lectivos de cada facultad y fijación de 
fechas, horarios y localizaciones de evaluación de asignaturas.

El grupo de personas que trabajará en el proyecto se constituirá por la
jefa de proyecto, Dña. Adelaida de la Calle; el coordinador del plan D. Luis
Muñoz; y su grupo de trabajo, que está formado por: D. Sergio de la Rubia,
D. Miguel Millán, Dña. Alicia Serrano y D. Juan Miguel Torres.

Este proyecto es a corto plazo puesto que se pretende que esté acabado para 
finales de agosto de 2010. De este modo se quiere conseguir que las guías 
docentes estén disponibles para el inicio del nuevo año académico.

Para la elaboración del proyecto se ha establecido un calendario provisional
(tabla \ref{tab:cal}), en el que se muestran los diferentes plazos de los 
entregables del proyecto.

\begin{table}[!h]
\centering
  \begin{tabular}{p{4cm}cp{3.7cm}}
    \textbf{Entregable} & \textbf{Fecha de entrega} &
    \textbf{Acciones Asociadas} \\
    \hline \hline
    Informe con el nuevo personal contratado & 10/04/2010 & Aprobación de la 
    jefa de proyecto\\ \hline
    Implantación de los planes de estudio & 20/06/2010 & Aprobación de la jefa 
    de proyecto\\ \hline
    Asignación del profesorado & 04/07/2010 & Aprobación de la jefa de 
    proyecto\\ \hline
    Establecimiento de horarios & 18/07/2010 & Aprobación de la jefa de 
    proyecto\\ \hline
    Establecimiento de exámenes & 18/07/2010 & Aprobación de la jefa de 
    proyecto\\ \hline
  \end{tabular}
\caption{Calendario de entregas}\label{tab:cal}
\end{table}

Esta planificación puede estar sujeta a variaciones.

\subsubsection{Proyecto a medio/largo plazo}
La ampliación del campus de la UMA con el objetivo de albergar nuevas 
titulaciones.

El proyecto constará de un estudio sobre las titulaciones más demandadas 
inexistentes en la UMA y sobre los emplazamientos de los centros; así
como de su posterior construcción y acondicionamiento.

\subsubsection{Proyecto a largo plazo}
La incorporación de un CGA (Centro de Gestión Avanzado) que se encargará de la
atención a los usuarios, soporte a los centros, desarrollo de nuevos servicios 
y gestión remota; en todo lo referido a las TI.

El CGA estará formado por un equipo técnico multidisciplinar de personal 
cualificado con experiencia en el uso de las TI.

Su implantación tendrá como objetivo liberar al personal docente de las tareas 
de administración, mantenimiento y configuración de servidores, electrónica de 
red y estaciones de trabajo, definir políticas de seguridad, etc.

\section{Elaboración del plan de mantenimiento del PSI}
En base a lo expuesto
en la sección \vref{ss:8.1.1}, \emph{\nameref{ss:8.1.1}}; 

A continuación vamos a especificar el plan de mantenimiento de nuestro PSI.
\subsection{Plan de mantenimiento del PSI}
\subsubsection{Ciclos mensuales}
Cada mes se revisará el estado del proyecto a corto plazo para confirmar si 
éste se desarrolla según las especificaciones originales del plan o hay que 
hacer modificaciones sobre la marcha.

De ello se encargará la jefa de proyecto que determinará si el proyecto se está
desarrollando según los requisitos y especificaciones iniciales y sobre si se 
va desarrollando cumpliendo las necesidades y objetivos de la UMA.

La jefa de proyecto junto con el coordinador se encargarán de evaluar parte por
parte el desarrollo del proyecto.

\subsubsection{Ciclos de tres años}
Los proyectos a medio o largo plazo son los susceptibles del mantenimiento y 
revisión cada tres años.

En reunión, el Jefe de Proyecto y el Coordinador tratarán de ver cómo 
evoluciona el proyecto y evaluarán posibles cambios en el desarrollo de éste 
debido al incesante avance de la tecnología que puede que nuestro plan original
quede en parte obsoleto.

\bibliographystyle{plain} 
\bibliography{t1}

\end{document}
