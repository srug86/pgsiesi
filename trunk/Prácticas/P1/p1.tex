% Clase
\documentclass[11pt,a4paper,spanish,twoside]{book}

% Órdenes auxiliares
\input{inc/includes.tex}

% Árboles de directorios
\usepackage{dirtree}

% Encabezado y pie de página
\encabezado

\begin{document}

% Silabación extra
\hyphenation{
a-sig-na-tu-ras
au-to-ma-ti-za-rá
ca-tá-lo-go
ca-rre-ra
cons-truc-ción
co-rres-pon-de
diag-nos-tico
fi-na-li-za-ción
ge-ne-ra-ción
in-fe-rior
man-te-ni-mien-to
me-dian-te
per-so-nal
pro-ce-di-mien-tos
pro-por-cio-na-rá
pu-bli-ca-da
re-qui-si-tos
res-pecto
u-su-a-rios
vi-lla-rre-al
}


% Portada
\portada{Planificación y Gestión de\\Sistemas de Información}
{Práctica 1}{Elaboración de un plan de proyecto\\utilizando Microsoft Project}
{Sergio de la Rubia García-Carpintero\\Miguel Millán Sánchez-Grande\\
  Luis Muñoz Villarreal\\Alicia Serrano Sánchez\\
  Juan Miguel Torres Triviño}{30 de Abril de 2009}

% Licencia
\licencia{Sergio de la Rubia García-Carpintero, Miguel Millán Sánchez-Grande,
  Luis Muñoz Villarreal, Alicia Serrano Sánchez, Juan Miguel Torres Triviño}

\chapter*{Ficha de trabajo}
\begin{description}
\item[Código] P1
\item[Fecha] 30 de Abril de 2010
\item[Título] Elaboración de un plan de proyecto utilizando Microsoft Project
\end{description}

\begin{table}[!ht]
  \centering
  \begin{tabular}{lp{5cm}c}
    \multicolumn{3}{l}{\Large \textbf{Equipo} G4} \\ \\
    \multicolumn{1}{c}{\emph{Apellidos y nombre}} & 
    \multicolumn{1}{c}{\emph{Firma}} & \emph{Puntos} \\
    \hline \\
    de la Rubia García-Carpintero, Sergio & & 14 \\ \\
    Millán Sánchez-Grande, Miguel         & & 14 \\ \\
    Muñoz Villarreal, Luis                & & 14 \\ \\
    Serrano Sánchez, Alicia               & & 14 \\ \\
    Torres Triviño, Juan Miguel           & & 14 \\ \\
    \hline
  \end{tabular}
\end{table}

% Índices
\tableofcontents
% \listoffigures
% \listoftables

%% INICIO DEL DOCUMENTO %%%%%%%%%%%%%%%%%%%%%%%%%%%%%%%%%%%%%%%%%%%%%%%%%
\chapter{Alcance}
En este capítulo se definen los diferentes procesos que hay que seguir y
cumplir para asegurar el éxito del proyecto. Las distintas tareas que
componen el proyecto deben realizarse satisfactoriamente en un plazo
definido.

\section{Esquemas de actividades y tareas}

\subsection{Análisis}
El objetivo principal de esta fase es la obtención de una especificación
detallada delproyecto que cumpla con los requisitos de la empresa y sea la
base del diseño posterior.

\subsubsection{Definición del sistema}
En esta tarea se realiza un exhaustivo análisis de las necesidades que se
acordaron en las reuniones con los directivos de la UMA . Además se realiza una
identificación del entorno tecnológico, uno de los reqisitos vitales para el
desarrollo de nuestro proyecto software.

\subsubsection{Establecimiento de requisitos}
Mediante sesiones de trabajo se recoge información de los requisitos que debe
cumplir el software. A partir de estos requisitos especificados y junto con la
opinión de la empresa se confirman cuales son válidos, consistentes y completos.

\subsubsection{Identificación de subsistemas}
La descomposición del sistema en subsistemas está orientada a los procesos de
negocio, estos subsistemas coinciden con el primer nivel de descomposición
del diagrama de flujo de datos. Se analizan los distintos modelos, para tener
una visión global y unificada de estos.

\subsubsection{Elaboración del modelo de datos}
    \begin{itemize}
    \item Elaboración del modelo conceptual y lógica de datos.
    \item Normalización.
    \item Especificación de necesidades de carga inicial.
    \end{itemize}

\subsubsection{Elaboración del modelo de procesos}
Se analizan las necesidades del usuario para establecer el conjunto de
procesos que conforma el SI para cada uno de los subsistemas
identificados.

\subsubsection{Definición de interfaz de usuario}
Esta actividad acomete el puente a la información ente el sistema y el
usuario, ocupándose de trabajos como el formato de pantalla o cuadros de
diálogo. Se define el formato y contenido de cada una de las interfaces de 
pantalla, especificando su comportamiento dinámico.

\subsubsection{Análisis de consistencia y especificación de requisitos}
El objetivo de esta actividad es garantizar la calidad de los distintos
modelos generados a lo largo de todo el proceso de análisis.

\subsubsection{Especificación de plan de pruebas}
Se especifican y justifican los niveles de pruebas a realizar, así como el marco
general de planificación de cada nivel de prueba. Además, se recopilan los
requisitos relativos al entorno de pruebas y se inicia la definición de las
especificaciones necesarias para la correcta ejecución de las distintas
pruebas del sistema de información. Los criterios de aceptación deben
ser definidos de forma clara, prestando especial atención a aspectos como:
procesos críticos del sistema, rendimiento del sistema, seguridad y 
disponibilidad.

\subsubsection{Aprobación del análisis del SI}
En esta tarea se realiza la presentación del análisis del sistema de
información a la Dirección, para la aprobación final del mismo.


\subsection{Diseño}
El Diseño de Sistema de Información (DSI) tiene como objetivo la definición
de la arquitectura del sistema y del entorno tecnológico que lo
soportará. Además se buscará dar una especificación detallada de los
componentes del sistema de información.

\subsubsection{Definición de la arquitectura del sistema}
\subsubsection{Diseño de la arquitectura de soporte}
\subsubsection{Diseño de la arquitectura de módulos del sistema}
    \begin{itemize}
    \item Diseño de módulos del sistema.
    \item Diseño de comunicación entre módulos.
    \item Revisión de la interfaz de usuario.
    \end{itemize}
\subsubsection{Diseño físico de datos}
    \begin{itemize}
    \item Diseño del modelo físico de datos.
    \item Especificación de los caminos de acceso a los datos.
    \item Especificación de la distribución de datos.
    \end{itemize}
\subsubsection{Verificación y aceptación de la arquitectura del sistema}
\subsubsection{Generación y especificación de construcción}
\subsubsection{Diseño de migración y carga inicial de datos}
\subsubsection{Especificación técnica del plan de prueba}
\subsubsection{Establecimiento de requisitos de implantación}
\subsubsection{Aprobación del diseño y SI}

\subsection{Implementación}
\subsubsection{Preparación del entorno de generación y construcción}
\subsubsection{Generación del código de los componentes y los 
procedimientos}
\subsubsection{Elaboración del manual de usuario}
\subsubsection{Definición de la formación de los usuarios finales}
\subsubsection{Construcción de los componentes y procedimientos de carga
  inicial de datos} 

\subsection{Pruebas}
\subsubsection{Ejecución de las pruebas unitarias}
\subsubsection{Ejecución de las pruebas de integración}
\subsubsection{Ejecución de las pruebas del sistema}
\subsubsection{Aprobación del SI}

\subsection{Tareas repetitivas}
  \begin{itemize}
  \item Reunión del grupo de trabajo. Esta tarea es semanal y comenzará desde
    el inicio hasta el fin del proyecto.
  \end{itemize}
\section{Hitos}
Se considera que cada fase del proyecto no puede empezar sin que haya
terminado la anterior. Para ello, los hitos tienen lugar en la terminación de
cada fase. La implantación de estos hitos ayudan al cumplimiento de los
plazos establecidos.

Los hitos del proyecto son los siguientes:
\begin{itemize}
\item Tarea 14: Análisis aprobado.
\item Tarea 32: Diseño aprobado.
\item Tarea 39: Implementación aprobada.
\item Tarea 45: Proyecto finalizado.
\end{itemize}

\section{Vinculaciones de tareas}
En la tabla \ref{Tab:Dep} se pueden ver las vinculaciones entre las tareas.
\begin{table}[!h]
  \centering
  \small
  \begin{tabular}{l|p{5cm}|l}
    \textbf{Tarea predecesora} & \textbf{Tarea actual} & \textbf{Dependencia} \\
    \hline
    Definición del Sistema & Establecimiento de requisitos & FC \\
    \hline
    Establecimiento de requisitos & Identificación de subsistemas & FC \\
    \hline
    \multirow{3}{*}{Identificación de subsistemas} & Elaboración del modelo conceptual y logíca de datos & \multirow{3}{*}{FC} \\
    & Normalización \\
    & Especificación de necesidades de carga inicial \\
    
  \end{tabular}
  \caption{Dependencias entre tareas} \label{Tab:Dep}
\end{table}

\section{Tiempos de posposición y adelanto}


\section{Visualización del camino crítico}
\begin{description}
\item[Tarea 1.1] Definición del sistema.
\item[Tarea 1.2] Establecimiento de requisitos.
\item[Tarea 1.3] Identificación de subsistemas.
\item[Tarea 1.4.1] Elaboración del modelo conceptual y lógica de datos.
\item[Tarea 1.4.2] Normalización.
\item[Tarea 1.4.3] Especificación de necesidades de carga inicial.
\item[Tarea 1.8] Especificación del plan de pruebas.
\item[Tarea 1.9] Aprobación del análisis del SI.
\item[Tarea 2.1] Definición de la arquitectura del sistema.
\item[Tarea 2.2] Diseño de la arquitectura de soporte.
\item[Tarea 2.3.1] Diseño de módulos del sistema.
\item[Tarea 2.3.2] Diseño de comunicación entre módulos.
\item[Tarea 2.3.3] Revisión de la interfaz de usuario.
\item[Tarea 2.5] Verificación y aceptación de la arquitectura del sistema.
\item[Tarea 2.6] Generación y especificación de construcción.
\item[Tarea 2.7] Diseño de migración y carga inicial de datos.
\item[Tarea 2.8] Especificación técnica del plan de prueba.
\item[Tarea 2.9] Establecimiento de requisitos de implantación.
\item[Tarea 2.10] Aprobación del diseño y SI.
\item[Tarea 3.1] Preparación del entorno de generación y construcción.
\item[Tarea 3.2] Generación del código de los componentes y procedimientos.
\item[Tarea 4.1] Ejecución de las pruebas unitarias.
\item[Tarea 4.2] Ejecución de las pruebas de integración.
\item[Tarea 4.3] Ejecución de las pruebas del sistema.
\item[Tarea 4.4] Aprobación del SI.
\end{description}


\chapter{Recursos y costes}
Los recursos, tanto materiales como humanos, se utilizan para completar las
tareas de las que se compone el proyecto.


\section{Lista de recursos humanos y materiales y asignaciones a 
  tareas}

\subsection{Recursos humanos}
\begin{description}
\item[Coordinador] es aquella persona responsable de un proyecto. Supervisa y
  controla el trabajo de las personas del proyecto, así como el cumplimiento
  de los plazos de entrega de las distintas tareas. 

\item[Analista] es aquel individuo responsable de investigar y
  recomendar opciones de software y sistemas para cumplir los requerimientos
  de una empresa de negocios.  

\item[Programador] es aquel que escribe, depura y mantiene el código fuente
  de la aplicación.  

\item[Secretario] es aquella persona que redacta los informes, organiza la
  información relacionada con el proyecto, planifica las reuniones, \dots

\item[Operario de servicio técnico] realiza labores de instalación y
  mantenimiento de los recursos materiales disponibles.

\item[Usuario experto] realiza pruebas a un alto nivel de especificación.

\item[Miembro del grupo de trabajo] recopilar los requisitos iniciales de la
  aplicación, realizar entrevistas, supervisar el trabajo del personal
  contratado, conseguir los recursos necesarios y la contabilidad.

\end{description}

\subsection{Recursos materiales}
\begin{description}
\item[Sala de juntas] Se utiliza para las reuniones semanales, así como para las juntas extraordinarias donde se tratan las aprobaciones de las distintas etapas del proyecto.
\item[Proyector] Se utiliza en las reuniones de la sala de juntas.
\item[Pizarra interactiva] Se utilizan en las etapas de análisis y diseño, donde cada miembro del equipo de trabajo puede aportar sus ideas y facilita la asimilación de conceptos.
\item[Impresora láser color]
\item[Servidor central] Los datos del proyecto están centralizados para que todos los miembros del equipo de trabajo tengan acceso de manera rápida y eficiente.
\item[Red Hat Enterprise Linux 5] Software para el servidor.
\item[Equipo informático] Se cuenta con 6 unidades de ordenadores para todo el desarrollo del proyecto.  
\end{description}

\section{Definición de costes por uso}

\begin{table}[!h]
\centering
  \begin{tabular}{|c|c|}
    \hline
    \textbf{Recursos} & \textbf{Costes/Uso} \\
    \hline \hline
    Sala de juntas & \\
    \hline
    Proyector & \\
    \hline
    Pizarra interactiva & \\
    \hline
    Impresora láser color & \\
    \hline
    Servidor central & \\
    \hline
    Red Hat Enterprise Linux 5 & \\
    \hline
    Equipo informático & \\
    \hline

  \end{tabular}
  \caption{Costes fijos por actividad}
  \label{Tab:costefijo}
\end{table}

\section{Definición de costes fijos de actividad}

\section{Tablas variables de costo}

\section{Disponibilidad variable de un recurso}

\section{Tablas variables de tasas de costos}

\section{Aplicación de distintas tablas de tasas de costo en tareas}

\chapter{Calendarios}
\section{Calendarios generales de recursos humanos}

\section{Calendarios específicos de recursos humanos}

\chapter{Redistribución del proyecto}
\section{Informe de sobre-asignaciones de recursos y de su 
  resolución}
\subsection{Gantt de redistribución}

\section{Incluir al menos 2 tipos de delimitaciones de tareas y 
  analizar sus efectos en la programación de proyecto}

\chapter{Alternativas al plan evaluando su repercusión en 
coste y  calendario}
* Nota: Establecer una fecha de fin y considerar un determinado coste por día
de retraso.

\chapter{Seguimiento simulado del proyecto incluyendo como 
  mínimo un ejemplo de las siguientes acciones}
\section{Introducción de duraciones reales y restantes}

\section{Introducción de un porcentaje completado}

\section{Introducción del trabajo real}

\section{Actualizar el resto del proyecto según la programación.}

\chapter{Informes}
\section{Vista resumen del plan del proyecto}

\section{Diagrama de Gantt}

\section{Diagrama de Gant con camino crítico}

\section{Informe general de recursos}

\section{Costes}
\subsection{Costes por recursos}
\subsection{Costes por actividades}

\section{Informe de redistribución}

\section{Informe de seguimiento}


\bibliographystyle{plain} 
\bibliography{p1}

\end{document}

