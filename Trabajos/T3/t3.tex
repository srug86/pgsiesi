% Clase
\documentclass[11pt,a4paper,spanish,twoside]{report}

% Órdenes auxiliares
\input{inc/includes.tex}


% Encabezado y pie de página
\encabezado

\setcounter{secnumdepth}{3}
\setcounter{tocdepth}{3}

\begin{document}

% Silabación extra
\hyphenation{
a-sig-na-tu-ras
au-to-ma-ti-za-rá
ca-tá-lo-go
ca-rre-ra
cons-truc-ción
co-rres-pon-de
diag-nos-tico
fi-na-li-za-ción
ge-ne-ra-ción
in-fe-rior
man-te-ni-mien-to
me-dian-te
per-so-nal
pro-ce-di-mien-tos
pro-por-cio-na-rá
pu-bli-ca-da
re-qui-si-tos
res-pecto
u-su-a-rios
vi-lla-rre-al
}


% Portada
\portada{Planificación y Gestión de\\Sistemas de Información}
{Trabajo 3}{Calendario del proyecto}
{Sergio de la Rubia García-Carpintero\\Miguel Millán Sánchez-Grande\\
  Luis Muñoz Villarreal\\Alicia Serrano Sánchez\\
  Juan Miguel Torres Triviño}{26 de Abril de 2010}

% Licencia
\licencia{Sergio de la Rubia García-Carpintero, Miguel Millán Sánchez-Grande,
  Luis Muñoz Villarreal, Alicia Serrano Sánchez, Juan Miguel Torres Triviño}

\chapter*{Ficha de trabajo}
\begin{description}
\item[Código] T3
\item[Fecha] 26 de Abril de 2010
\item[Título] Calendario del proyecto
\end{description}

\begin{table}[!ht]
  \centering
  \begin{tabular}{lp{5cm}c}
    \multicolumn{3}{l}{\Large \textbf{Equipo} G4} \\ \\
    \multicolumn{1}{c}{\emph{Apellidos y nombre}} & 
    \multicolumn{1}{c}{\emph{Firma}} & \emph{Puntos} \\
    \hline \\
    de la Rubia García-Carpintero, Sergio & & 6 \\ \\
    Millán Sánchez-Grande, Miguel         & & 6 \\ \\
    Muñoz Villarreal, Luis                & & 6 \\ \\
    Serrano Sánchez, Alicia               & & 6 \\ \\
    Torres Triviño, Juan Miguel           & & 6 \\ \\
    \hline
  \end{tabular}
%  \caption{}\label{}
\end{table}

% Índices
\tableofcontents
% \listoffigures
% \listoftables

%% INICIO DEL DOCUMENTO %%%%%%%%%%%%%%%%%%%%%%%%%%%%%%%%%%%%%%%%%%%%%%%%%

\chapter*{Introducción}
En este trabajo se realiza la elaboración del calendario del proyecto
mediante el uso de la técnica PERT (Program Evaluation and Review Technique),
la cual proporciona un método para realizar una estimación de la duración
total del proyecto a partir de las actividades, su secuencia y la estimación
ponderada de la duración media de estas. 

Para la estimación de los tiempos PERT se aplica la técnica DELPHI, reuniendo
a 5 expertos en el tema, de tal modo que por su nivel de formación y grado de
conocimiento puedan aportar ideas y puntos de vistas diferentes al problema
en cuestión, con el fin de obtener juicios coherentes y enriquecidos con
respecto al problema. 
\chapter{Técnica DELPHI}
Mediante la técnica DELPHI se intenta obtener un consenso lo más fiable
posible del grupo de expertos. Antes de empezar comenzar a fondo con la
técnica DELPHI, en la tabla \ref{Tab:corr} se muestra la etapa a la que
corresponde cada tarea, su nombre y su identificador.
%\begin{table}[!h]
%\centering
 % \begin{tabular}{|c|c|c|}
  %  \hline
   % \textbf{Etapa} & \textbf{Tarea} & \textbf{Identificador}
   % \hline \hline
   % \multirow{}{*}{} & & 1.1
    
\section{Tablas de tiempos DELPHI}

En las tablas \ref{Tab:anal}, \ref{Tab:dis},\ref{Tab:imp} y \ref{Tab:pru}  se
pueden observar la evolución seguida a la hora de calcular los tiempos
DELPHI. Cada tabla contiene el identificador de cada tarea, la iteración en
la que se encuentra, las valoraciones de los distintos expertos, así como los
valores mínimo (que corresponde con el tiempo optimista), medio (que
correspondeo con la media ponderada) y máximo( que es el tiempo pesimista de
cada tarea), y finalmente, si se cumple el criterio de convergencia o no.

En la tabla \ref{Tab:val} se muestran el tiempo más probable (la media
ponderada), el tiempo optimista (mínimo significativo), y el tiempo pesimista
(máximo significativo) de cada tarea.

\begin{table}[!h]
\centering
  \begin{tabular}{|c|c||c|c|c|c|c||c|c|c||c|}
    \hline
    \textbf{T} & \textbf{I} & \textbf{1} &
    \textbf{2} & \textbf{3} & \textbf{4} & \textbf{5} & \textbf{m}
    &\textbf{$\bar{x}$} &\textbf{M} & \textbf{C}\\
    \hline \hline
    \multirow{3}{*}{1.1} & 1 & 4 & 5 & 3 & 6 & 4 & 3 & 4.4 & 6 & N \\
    & 2 & 4 & 5 & 3 & 4 & 5 & 3 & 4.2 & 5 & N\\
    &\textbf{3} &\textbf {5} &\textbf {5} &\textbf {4} &\textbf {4} &\textbf{5} & \textbf{4} & \textbf{4.6} &\textbf{5} & \textbf{S}\\
    \hline

    \multirow{3}{*}{1.2} & 1 & 5 & 5 & 4 & 7 & 5 & 4 & 5,2 & 7 & N\\
    & 2 & 6 & 5 & 4 & 6 & 5 & 4 & 5.2 & 6 & N\\
    &\textbf{3} & \textbf{6} & \textbf{5} & \textbf{4} & \textbf{6} &
    \textbf{4} & \textbf{4} & \textbf{5} & \textbf{6} & \textbf{S} \\    
    \hline

    \multirow{5}{*}{1.3} & 1 & 2 & 3 & 10 & 3 & 5 & 2 & 4,6 & 10& N\\
    & 2 & 2 & 3 & 8 & 3 & 5 & 2 & 4.2 & 8 & N \\
    & 3 & 3 & 3 & 6 & 3 & 5 & 3 & 4 & 6 & N \\
    & 4 & 3 & 3 & 6 & 4 & 5 & 3 & 4.2 & 6 & N \\
    & \textbf{5} & \textbf{3} & \textbf{3} & \textbf{5} & \textbf{4} & \textbf{5} & \textbf{3} & \textbf{4} & \textbf{5} & \textbf{S} \\
    \hline

    \multirow{7}{*}{1.4.1} & 1 & 2 & 4 & 7 & 4 & 3 & 2 & 4 & 7 & N\\
    & 2 & 4 & 4 & 3 & 7 & 4 & 3 & 4.4 & 7 & N \\
    & 3 & 4 & 4 & 4 & 7 & 5 & 4 & 4.8 & 7 & N \\
    & 4 & 4 & 4 & 4 & 5 & 6 & 4 & 4.6 & 6 & N \\
    & 5 & 4 & 5 & 4 & 5 & 6 & 4 & 4.8 & 6 & N \\
    & 6 & 4 & 6 & 5 & 5 & 5 & 4 & 5 & 6 & N \\
    & \textbf{7} & \textbf{5} & \textbf{5} & \textbf{5} & \textbf{5} & \textbf{5} & \textbf{5} & \textbf{5} & \textbf{5} & \textbf{S} \\
    \hline

    \multirow{2}{*}{1.4.2} & 1 & 2 & 2 & 4 & 3 & 2 & 2 & 2,6 & 4 & N\\
    & \textbf{2} & \textbf{2} & \textbf{2} & \textbf{3} & \textbf{3} & \textbf{2} & \textbf{2} & \textbf{2.4} & \textbf{3} & \textbf{S} \\
    \hline

    \multirow{3}{*}{1.4.3} & 1 & 2 & 2 & 2 & 2 & 1 & 1 & 1,8 & 2 & N\\
    & 2 & 2 & 2 & 2 & 2 & 3 & 2 & 2.2 & 3 & N \\
    & \textbf{3} & \textbf{2} & \textbf{2} & \textbf{2} & \textbf{2} & \textbf{2} & \textbf{2} & \textbf{2} & \textbf{2} & \textbf{S} \\
    \hline

    \multirow{5}{*}{1.5} & 1 & 9 & 5 & 4 & 6 & 5 & 4 & 5.8 & 9 & N \\
    & 2 & 8 & 5.5 & 4.5 & 6 & 5.5 & 4.5 & 5.9 & 8 & N \\
    & 3 & 7.5 & 5.5 & 4.5 & 6 & 5.5 & 4.5 & 5.8 & 7.5 & N \\
    & 4 & 5.5 & 6 & 5.5 & 6 & 5 & 5 & 5.6 & 6 & N \\
    & \textbf{5} & \textbf{5.5} & \textbf{6} & \textbf{5.5} & \textbf{6} & \textbf{6} & \textbf{5.5} & \textbf{5.8} & \textbf{6} & \textbf{S} \\
    \hline

    \multirow{5}{*}{1.6} & 1 & 4 & 10 & 2 & 3 & 5 & 2 & 4.8 & 10 & N \\
    & 2 & 6 & 8 & 3 & 5 & 6 & 3 & 5.6 & 8 & N \\
    & 3 & 5 & 8 & 3.5 & 5 & 6 & 3.5 & 5.5 & 8 & N \\
    & 4 & 5 & 8 & 6 & 4 & 5.5 & 4 & 5.7 & 8 & N \\
    & \textbf{5} & \textbf{5} & \textbf{7} & \textbf{6} & \textbf{4} & \textbf{5.5} & \textbf{4} & \textbf{5.5} & \textbf{7} & \textbf{S} \\
    \hline

    \multirow{2}{*}{1.7} & 1 & 4 & 5 & 3 & 2 & 2 & 2 & 3.2 & 5 & N \\
    & \textbf{2} & \textbf{4} & \textbf{4.5} & \textbf{4} & \textbf{2} & \textbf{2} & \textbf{2} & \textbf{3.3} & \textbf{4.5} & \textbf{S} \\
    \hline

    \multirow{11}{*}{1.8} & 1 & 4 & 3 & 6 & 3 & 2 & 2 & 3.6 & 6   & N \\
    & 2 & 4 & 3   & 5.5 & 3   & 2.5 & 2.5 & 3.6 & 5.5 & N \\
    & 3 & 4 & 3.5 & 5.5 & 3   & 2.5 & 2.5 & 3.7 & 5.5 & N \\
    & 3 & 4 & 3.5 & 5.5 & 2.5 & 3   & 2.5 & 3.7 & 5.5 & N \\
    & 5 & 4 & 3.5 & 5   & 2.5 & 3   & 2.5 & 3.6 & 5   & N \\
    & 6 & 4 & 3.5 & 5   & 3   & 3   & 3   & 3.7 & 5   & N \\
    & 7 & 4 & 3   & 5   & 4   & 4   & 3   & 4   & 5   & N \\
    & 8 & 4 & 3.5 & 5   & 4   & 4   & 3.5 & 4.1 & 5   & N \\
    & 9 & 4.5 & 4 & 5 & 4.5 & 4 & 4 & 4.4 & 5 & N \\
    & 10 & 4 & 4 & 4.5 & 4 & 4 & 4 & 4.1 & 4.5 & N \\
    & \textbf{11} & \textbf{4.5} & \textbf{4} & \textbf{4.5} & \textbf{4} & \textbf{4} & \textbf{4} & \textbf{4.2} & \textbf{4.5} & \textbf{S} \\
    \hline

    \textbf{1.9} & \textbf{1} & \textbf{1} & \textbf{1} & \textbf{1} & \textbf{1} & \textbf{1} & \textbf{1} & \textbf{1} & \textbf{1} & \textbf{S} \\
    \hline
  \end{tabular}
  \caption{Resultados de la técnica DELPHI para las tareas de
    Análisis} \label{Tab:anal}
\end{table}

\begin{table}[!h]
\centering
  \begin{tabular}{|c|c||c|c|c|c|c||c|c|c||c|}
    \hline
    \textbf{T} & \textbf{I} & \textbf{1} &
    \textbf{2} & \textbf{3} & \textbf{4} & \textbf{5} & \textbf{m}
    &\textbf{$\bar{x}$} &\textbf{M} & \textbf{C}\\
    \hline \hline
    \textbf{2.1} & \textbf{1} & \textbf{6} & \textbf{3} & \textbf{6} & \textbf{5} & \textbf{3} & \textbf{3} & \textbf{4.6} & \textbf{6} & \textbf{S}\\
    \hline

    \textbf{2.2} & \textbf{1} & \textbf{6} & \textbf{6} & \textbf{7} & \textbf{2} & \textbf{2} & \textbf{2} & \textbf{4.6} & \textbf{7} & \textbf{S}\\
    \hline

    \multirow{3}{*}{2.3.1}& 1 & 3 & 11 & 15 & 8 & 6 & 3 & 8.6 & 15 & N \\
    & 2 & 3 & 11 & 14 & 8 & 8 & 3 & 8.8 & 14 & N \\
    & \textbf{3} & \textbf{3} & \textbf{11} & \textbf{14} & \textbf{7} & \textbf{7} & \textbf{3} & \textbf{8.4} & \textbf{14} & \textbf{S} \\
    \hline

    \multirow{2}{*}{2.3.2} & 1 & 4 & 7 & 3 & 5 & 2 & 2 & 4.2 & 7 & N \\
    & \textbf{2} & \textbf{5} & \textbf{6} & \textbf{3} & \textbf{6} & \textbf{2} & \textbf{2} & \textbf{4.4} & \textbf{6} & \textbf{S} \\
    \hline

    \multirow{3}{*}{2.3.3}& 1 & 4 & 5 & 4 & 5 & 2 & 2 & 4 & 5 & N \\
    & 2 & 3 & 5 & 4 & 5 & 2 & 2 & 3.8 & 5 & N \\
    & \textbf{3} & \textbf{3} & \textbf{5} & \textbf{4} & \textbf{5} &
    \textbf{2.5} & \textbf{2.5} & \textbf{3.9} & \textbf{5} & \textbf{S} \\
    \hline

    \textbf{2.4.1} & \textbf{1} & \textbf{5} & \textbf{2} & \textbf{5} & \textbf{4} & \textbf{2} & \textbf{2} & \textbf{3.6} & \textbf{5} & \textbf{S} \\
    \hline

    \multirow{3}{*}{2.4.2}& 1 & 4 & 8 & 4 & 3 & 1 & 1 & 4 & 8 & N \\
    & 2 & 5 & 7 & 5 & 3 & 1 & 1 & 4.2 & 7 & N \\
    & \textbf{3} & \textbf{5} & \textbf{7} & \textbf{5} & \textbf{3} & \textbf{2} & \textbf{2} & \textbf{4.4} & \textbf{7} & \textbf{S} \\
    \hline

    \multirow{4}{*}{2.4.3}& 1 & 5 & 3 & 4 & 1 & 3 & 1 & 3.2 & 5 & N \\
    & 2 & 5 & 4 & 5 & 2 & 3 & 2 & 3.8 & 5 & N \\
    & 3 & 4 & 4 & 5 & 2 & 3 & 2 & 3.6 & 5 & N \\
    & \textbf{4} & \textbf{4} & \textbf{4} & \textbf{5} & \textbf{2.5} & \textbf{3} & \textbf{2.5} & \textbf{3.7} & \textbf{5} & \textbf{S} \\
    \hline

    \multirow{2}{*}{2.5}& 1 & 1 & 2 & 1 & 1 & 3 & 1 & 1.6 & 3 & N \\
    & \textbf{2} & \textbf{1} & \textbf{2} & \textbf{1} & \textbf{1} & \textbf{2} & \textbf{1} & \textbf{1.4} & \textbf{2} & \textbf{S} \\
    \hline

    \multirow{3}{*}{2.6}& 1 & 4 & 12 & 5 & 4 & 3 & 3 & 5.6 & 12 & N \\
    & 2 & 4 & 11 & 8 & 4 & 3 & 3 & 6 & 11 & N \\
    & \textbf{3} & \textbf{4} & \textbf{9} & \textbf{8} & \textbf{4} & \textbf{3} & \textbf{3} & \textbf{5.6} & \textbf{9} & \textbf{S} \\
    \hline

    \multirow{2}{*}{2.7}& 1 & 2 & 4 & 2 & 2 & 5 & 2 & 3 & 5 & N \\
    & \textbf{2} & \textbf{2} & \textbf{4} & \textbf{2} & \textbf{2} & \textbf{4} & \textbf{2} & \textbf{2.8} & \textbf{4} & \textbf{S} \\
    \hline

    \textbf{2.8} & \textbf{1} & \textbf{3} & \textbf{6} & \textbf{5} & \textbf{2} & \textbf{2} & \textbf{2} & \textbf{3.6} & \textbf{6} & \textbf{S} \\
    \hline

    \multirow{4}{*}{2.9}& 1 & 4 & 20 & 3 & 4 & 3 & 3 & 6.8 & 20 & N \\
    & 2 & 6 & 15 & 4 & 4 & 5 & 4 & 6.8 & 15 & N \\
    & 3 & 6 & 13 & 6 & 6 & 8 & 6 & 7.8 & 13 & N \\
    & \textbf{4} & \textbf{6} & \textbf{11} & \textbf{6} & \textbf{6} & \textbf{10} & \textbf{6} & \textbf{7.8} & \textbf{11} & \textbf{S} \\
    \hline

    \textbf{2.10} & \textbf{1} & \textbf{2} & \textbf{1} & \textbf{1} & \textbf{2} & \textbf{1} & \textbf{1} & \textbf{1.4} & \textbf{2} & \textbf{S} \\
    \hline
  \end{tabular}
  \caption{Resultados de la técnica DELPHI para las tareas de
    Diseño} \label{Tab:dis}
\end{table}

\begin{table}[!h]
\centering
  \begin{tabular}{|c|c||c|c|c|c|c||c|c|c||c|}
    \hline
    \textbf{T} & \textbf{I} & \textbf{1} &
    \textbf{2} & \textbf{3} & \textbf{4} & \textbf{5} & \textbf{m}
    &\textbf{$\bar{x}$} &\textbf{M} & \textbf{C}\\    
    \hline \hline
    \multirow{2}{*}{3.1}& 1 & 2 & 3 & 3 & 2 & 4 & 2 & 2.8 & 4 & N \\
    & \textbf{2} & \textbf{2} & \textbf{4} & \textbf{4} & \textbf{2} & \textbf{4} & \textbf{2} & \textbf{3.2} & \textbf{4} & \textbf{S} \\
    \hline

    \textbf{3.2} & \textbf{1} & \textbf{14} & \textbf{8} & \textbf{7} & \textbf{5} & \textbf{12} & \textbf{5} & \textbf{9.2} & \textbf{14} & \textbf{S} \\
    \hline

    \multirow{3}{*}{3.3}& 1 & 18 & 5 & 10 & 3 & 6 & 3 & 8.4 & 18 & N \\
    & 2 & 15 & 7 & 11 & 5 & 7 & 5 & 9 & 15 & N \\
    & \textbf{3} & \textbf{15} & \textbf{9} & \textbf{13} & \textbf{7} & \textbf{8} & \textbf{7} & \textbf{10.4} & \textbf{15} & \textbf{S} \\
    \hline

    \multirow{3}{*}{3.4}& 1 & 7 & 3 & 2 & 2 & 3 & 2 & 3.4 & 7 & N \\
    & 2 & 5 & 3 & 3 & 3 & 4 & 3 & 3.6 & 5 & N \\
    & \textbf{3} & \textbf{4} & \textbf{3} & \textbf{4} & \textbf{3} & \textbf{4} & \textbf{3} & \textbf{3.6} & \textbf{4} & \textbf{S} \\
    \hline

    \textbf{3.5} & \textbf{1} & \textbf{7} & \textbf{3} & \textbf{6} & \textbf{5} & \textbf{4} & \textbf{3} & \textbf{5} &\textbf{7} & \textbf{S} \\
    \hline

  \end{tabular}
  \caption{Resultados de la técnica DELPHI para las tareas de
    Implementación} \label{Tab:imp}
\end{table}

\begin{table}[!h]
\centering
  \begin{tabular}{|c|c||c|c|c|c|c||c|c|c||c|}
    \hline
    \textbf{T} & \textbf{I} & \textbf{1} &
    \textbf{2} & \textbf{3} & \textbf{4} & \textbf{5} & \textbf{m}
    &\textbf{$\bar{x}$} &\textbf{M} & \textbf{C}\\    
    \hline \hline
    \multirow{4}{*}{4.1}& 1 & 5 & 3 & 4 & 2 & 4 & 2 & 3.6 & 5 & N \\
    & 2 & 5 & 4 & 4 & 2.5 & 4 & 2.5 & 3.9 & 5 & N \\
    & 3 & 5 & 4 & 4 & 3 & 4 & 3 & 4 & 5 & N \\
    & \textbf{5} & \textbf{4.5} & \textbf{4.5} & \textbf{3.5} & \textbf{3.5} & \textbf{4} & \textbf{3.5} & \textbf{4} & \textbf{4.5} & \textbf{S} \\    
    \hline
    
    \multirow{2}{*}{4.2}& 1 & 5 & 7 & 5 & 3 & 3 & 3 & 4.6 & 7 & N \\
    & \textbf{2} & \textbf{5} & \textbf{6} & \textbf{6} & \textbf{3} & \textbf{3} & \textbf{3} & \textbf{4.6} & \textbf{6} & \textbf{S} \\
    \hline
    \multirow{3}{*}{4.3}& 1 & 6 & 4 & 10 & 3 & 3 & 3 & 5.2 & 10 & N \\
    & 2 & 6 & 4 & 8 & 3 & 4 & 3 & 5 & 8 & N \\
    & \textbf{3} & \textbf{7} & \textbf{4} & \textbf{8} & \textbf{3} & \textbf{4} & \textbf{3} & \textbf{5.2} & \textbf{8} & \textbf{S} \\
    \hline
    \textbf{4.4} & \textbf{1} & \textbf{3} & \textbf{2} & \textbf{1} & \textbf{3} & \textbf{1} & \textbf{1} & \textbf{2} & \textbf{3} & \textbf{S} \\
    \hline
  \end{tabular}
  \caption{Resultados de la técnica DELPHI para las tareas de Pruebas} x\label{Tab:pru}
\end{table} 
\section{Tabla resumen de DELPHI}
En las tablas \ref{Tab:tareas1}, \ref{Tab:tareas2}, \ref{Tab:tareas3} y
\ref{Tab:tareas4} reprensentan los resultados abreviados de aplicar la
técnica DELPHI, cada tabla se corresponde con las tareas de cada etapa de
elaboración del software. Cada incidencia de la tabla contiene las tareas, el
tiempo mínimo, el medio y el máximo.

\begin{table}[!h]
\centering
  \begin{tabular}{|p{0.8cm}||p{8cm}||p{0.75cm}|p{0.75cm}|p{0.75cm}|}
    \hline
    \textbf{Id.} & \textbf{Tarea} & \textbf{Min} &
    \textbf{Med} & \textbf{Max}\\
    \hline \hline
    1.1 & Definición del sistema & 4 & 4.6 & 5\\ 
    \hline
    1.2 & Establecimiento de requisitos & 4 & 5 & 6\\
    \hline
    1.3 & Identificación de subsistemas & 3 & 4 & 5\\
    \hline
    1.4.1 & Elaboración del modelo conceptual y lógica de datos & 5 & 5 & 5\\
    \hline
    1.4.2 & Normalización & 2 & 2.4 & 3 \\
    \hline
    1.4.3 & Especificación de necesidades de carga inicial & 2 & 2 & 2\\
    \hline
    1.5 & Elaboración del modelo de procesos & 5.5 & 5.8 & 6\\
    \hline
    1.6 & Definición de interfaz de usuario & 4 & 5.5 & 7\\
    \hline
    1.7 & Análisis de consistencia y especificación de requisitos & 2 & 3.3 & 4.5\\
    \hline
    1.8 & Especificación del plan de pruebas & 4 & 4.3 & 4.5\\
    \hline
    1.9 & Aprobación del análisis del SI & 1 & 1 & 1\\
    \hline
  \end{tabular}
  \caption{Tabla resumen de tareas de Análisis} \label{Tab:tareas1}
\end{table}

\begin{table}[!h]
\centering
  \begin{tabular}{|p{0.8cm}||p{8cm}||p{0.75cm}|p{0.75cm}|p{0.75cm}|}
    \hline
    \textbf{Id.} & \textbf{Tarea} & \textbf{Min} &
    \textbf{Med} & \textbf{Max}\\
    \hline
    2.1 & Definición de la arquitectura del sistema & 3 & 4.6 & 6 \\
    \hline
    2.2 & Diseño de arquitectura de soporte & 2 & 4.6 & 7 \\
    \hline
    2.3.1 & Diseño de módulos del sistema & 3 & 8.4 & 14 \\
    \hline
    2.3.2 & Diseño de comunicación entre módulos & 2 & 4.4 & 6 \\
    \hline
    2.3.3 & Revisión de la interfaz de usuario &2.5 & 3.9 & 5 \\
    \hline
    2.4.1 & Diseño del modelo físico de datos & 2 & 3.6 & 5 \\
    \hline
    2.4.2 & Especificación de los caminos de acceso a los datos & 2 & 4.4 &
    7\\
    \hline
    2.4.3 & Especificación de la distribución de datos & 2.5 & 3.7 & 5 \\
    \hline
    2.5 & Verificación y aceptación de la arquitectura del sistema & 1 & 1.4
    & 2 \\
    \hline
    2.6 & Generación y especificación de construcción & 3 & 5.6 & 9 \\
    \hline
    2.7 & Diseño de migración y carga inicial de datos & 2 & 2.8 & 4 \\
    \hline
    2.8 & Especificación técnica del plan de prueba & 2 & 3.6 & 6 \\
    \hline
    2.9 & Establecimiento de requisitos de implantación & 6 & 7.8 & 11 \\
    \hline
    2.10 & Aprobación de diseño y SI & 1 & 1.4 & 2 \\
    \hline
  \end{tabular}
  \caption{Tabla resumen de tareas de diseño} \label{Tab:tareas2}
\end{table}

\begin{table}[!h]
\centering
  \begin{tabular}{|p{0.8cm}||p{8cm}||p{0.75cm}|p{0.75cm}|p{0.75cm}|}
    \hline
    \textbf{Id.} & \textbf{Tarea} & \textbf{Min} &
    \textbf{Med} & \textbf{Max}\\
    \hline
    3.1 & Preparación del entorno de generación y construcción & 2 & 3.2  & 4 \\
    \hline
    3.2 & Generación del código de los componentes y los procedimientos & 5 &
    9.2 & 14\\
    \hline
    3.3 & Elaboración del manual de usuario & 7 & 10.4 & 15\\
    \hline
    3.4 & Definición de la formación de los usuarios finales & 3 & 3.6 & 4 \\
    \hline
    3.5 & Construcción de los componentes y procedimientos de carga inicial
    de datos & 3 & 5 & 7\\
    \hline
  \end{tabular}
  \caption{Tabla resumen de tareas de Implementación} \label{Tab:tareas3}
\end{table}

\begin{table}[!h]
\centering
  \begin{tabular}{|p{0.8cm}||p{8cm}||p{0.75cm}|p{0.75cm}|p{0.75cm}|}
    \hline
    \textbf{Id.} & \textbf{Tarea} & \textbf{Min} &
    \textbf{Med} & \textbf{Max}\\
    \hline
    4.1 & Ejecución de las pruebas unitarias & 3.5 & 4 & 4.5\\
    \hline
    4.2 & Ejecución de las pruebas de integración & 3 & 4.6 & 6\\
    \hline
    4.3 & Ejecución de las pruebas del sistema & 3 & 5.2 & 8\\
    \hline
    4.4 & Aprobación del SI & 3 & 5.2 & 8 \\
    \hline
  \end{tabular}
  \caption{Tabla resumen de tareas de Pruebas} \label{Tab:tareas4}
\end{table}


\chapter{Técnica PERT}
\section{Cálculos de los tiempos PERT}
\section{Cálculos de tiempos early y late}
\section{Cálculo de las holguras}
\section{Determinación de los tiempos críticos}

\chapter{Calendario}
\section{Tabla de tiempos de comienzo y finalizacion}
\section{Tabla de dependencias}
\section{Grafo de actividades}


\bibliographystyle{plain} 
\bibliography{t2}

\end{document}
