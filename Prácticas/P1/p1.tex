% Clase
\documentclass[11pt,a4paper,spanish,twoside]{report}

% Órdenes auxiliares
\input{inc/includes.tex}

% Encabezado y pie de página
\encabezado

\begin{document}

% Silabación extra
\hyphenation{
a-sig-na-tu-ras
au-to-ma-ti-za-rá
ca-tá-lo-go
ca-rre-ra
cons-truc-ción
co-rres-pon-de
diag-nos-tico
fi-na-li-za-ción
ge-ne-ra-ción
in-fe-rior
man-te-ni-mien-to
me-dian-te
per-so-nal
pro-ce-di-mien-tos
pro-por-cio-na-rá
pu-bli-ca-da
re-qui-si-tos
res-pecto
u-su-a-rios
vi-lla-rre-al
}


% Portada
\portada{Planificación y Gestión de\\Sistemas de Información}
{Práctica 1}{Elaboración de un plan de proyecto\\utilizando Microsoft Project}
{Sergio de la Rubia García-Carpintero\\Miguel Millán Sánchez-Grande\\
  Luis Muñoz Villarreal\\Alicia Serrano Sánchez\\
  Juan Miguel Torres Triviño}{30 de Abril de 2009}

% Licencia
\licencia{Sergio de la Rubia García-Carpintero, Miguel Millán Sánchez-Grande,
  Luis Muñoz Villarreal, Alicia Serrano Sánchez, Juan Miguel Torres Triviño}

\chapter*{Ficha de trabajo}
\begin{description}
\item[Código] P1
\item[Fecha] 30 de Abril de 2010
\item[Título] Elaboración de un plan de proyecto utilizando Microsoft Project
\end{description}

\begin{table}[!ht]
  \centering
  \begin{tabular}{lp{5cm}c}
    \multicolumn{3}{l}{\Large \textbf{Equipo} G4} \\ \\
    \multicolumn{1}{c}{\emph{Apellidos y nombre}} & 
    \multicolumn{1}{c}{\emph{Firma}} & \emph{Puntos} \\
    \hline \\
    de la Rubia García-Carpintero, Sergio & & 14 \\ \\
    Millán Sánchez-Grande, Miguel         & & 14 \\ \\
    Muñoz Villarreal, Luis                & & 14 \\ \\
    Serrano Sánchez, Alicia               & & 14 \\ \\
    Torres Triviño, Juan Miguel           & & 14 \\ \\
    \hline
  \end{tabular}
%  \caption{}\label{}
\end{table}

% Índices
\tableofcontents
% \listoffigures
% \listoftables

%% INICIO DEL DOCUMENTO %%%%%%%%%%%%%%%%%%%%%%%%%%%%%%%%%%%%%%%%%%%%%%%%%
\chapter{Alcance}
En esta sección se definen los diferentes procesos que se hay que seguir y
cumplir para asegurar el éxito del proyecto. Las distintas tareas que
componen el proyecto deben realizarse satisfactoriamente en un plazo
definido.
Los requisitos del proyecto deben estar bien definidos y tener un control de
cambios eficaz para evitar cambios no controlados que pueden ocasionar que el
proyecto se desvíe de su propósito inicial.

\section{Esquemas de actividades y tareas}
\begin{description}
\item[Análisis] \hfill
  \begin{itemize}
  \item Definición del sistema.
  \item Establecimiento de requisitos.
  \item Identificación de subsistemas.
  \item Elaboración del modelo de datos.
    \begin{itemize}
    \item Elaboración del modelo conceptual y lógica de datos.
    \item Normalización.
    \item Especificación de necesidades de carga inicial.
    \end{itemize}
  \item Elaboración del modelo de procesos.
  \item Definición de interfaz de usuario.
  \item Análisis de consistencia y especificación de requisitos.
  \item Especificación de plan de pruebas.
  \item Aprobación del análisis del SI.
  \end{itemize}
\item[Diseño] \hfill
  \begin{itemize}
  \item Definición de la arquitectura del sistema.
  \item Diseño de la arquitectura de soporte.
  \item Diseño de la arquitectura de módulos del sistema.
    \begin{itemize}
    \item Diseño de módulos del sistema.
    \item Diseño de comunicación entre módulos.
    \item Revisión de la interfaz de usuario.
    \end{itemize}
  \item Diseño físico de datos.
    \begin{itemize}
    \item Diseño del modelo físico de datos.
    \item Especificación de los caminos de acceso a los datos.
    \item Especificación de la distribución de datos.
    \end{itemize}
  \item Verificación y aceptación de la arquitectura del sistema.
  \item Generación y especificación de construcción.
  \item Diseño de migración y carga inicial de datos.
  \item Especificación técnica del plan de prueba.
  \item Establecimiento de requisitos de implantación.
  \item Aprobación del diseño y SI.
  \end{itemize}
\item[Implementación]\hfill
  \begin{itemize}
  \item Preparación del entorno de generación y construcción.
  \item Generación del código de los componentes y los procedimientos.
  \item Elaboración del manual de usuario.
  \item Definición de la formación de los usuarios finales.
  \item Construcción de los componentes y procedimientos de carga inicial de
    datos. 
  \end{itemize}
\item[Pruebas] \hfill
  \begin{itemize}
  \item Ejecución de las pruebas unitarias.
  \item Ejecución de las pruebas de integración.
  \item Ejecución de las pruebas del sistema.
  \item Aprobación del SI.
  \end{itemize}
\item[Tareas repetitivas] \hfill
  \begin{itemize}
  \item Reunión del grupo de trabajo. Esta tarea es semanal y comenzará desde
    el inicio hasta el fin del proyecto.
  \end{itemize}
\end{description}
\section{Hitos}
Se consideran que cada fase del proyecto no puede empezar sin que haya
terminado la anterior. Para ello, los hitos tienen lugar en la terminación de
cada fase. La implantación de estos hitos ayudan al cumplimiento de los
plazos establecidos.

Los hitos del proyecto son los siguientes:
\begin{itemize}
\item Aprobación del análisis del SI.
\item Aprobación del diseño y SI.
\item Aprobación del SI.
\end{itemize}

\section{Vinculaciones de tareas (FC, CC, FF)}
\begin{table}[!h]
\centering
\small
  \begin{tabular}{l|p{5cm}|l}
    \textbf{Tarea predecesora} & \textbf{Tarea actual} & \textbf{Dependencia} \\
    \hline \hline
    Definición del Sistema & Establecimiento de requisitos & FC \\
    \hline
    Establecimiento de requisitos & Identificación de subsistemas & FC \\
    \hline
    \multirow{3}{*}{Identificación de subsistemas} & Elaboración del modelo conceptual y logíca de datos & \multirow{3}{*}{FC} \\
    & Normalización \\
    & Especificación de necesidades de carga inicial \\
    
    
  \end{tabular}
  \caption{Dependencias entre tareas} \label{Tab:Dep}
\end{table}

\section{Tiempos de posposición y adelanto}

\section{Visualización del camino crítico}

\chapter{Recursos y costes}
Los recursos, tanto materiales como personal contratado, se utilizan para
completar las tareas de las que se compone un proyecto. 


\section{Lista de recursos humanos y materiales y asignaciones a 
  tareas}

\subsection{Recursos humanos}
\begin{description}
\item[Coordinador] es aquella persona responsable de un proyecto. Supervisa y
  controla el trabajo de las personas del proyecto, así como el cumplimiento
  de los plazos de entrega de las distintas tareas. 

\item[Analista] es aquel individuo responsable de investigar y
  recomendar opciones de software y sistemas para cumplir los requerimientos
  de una empresa de negocios.  

\item[Programador] es aquel que escribe, depura y mantiene el código fuente
  de la aplicación.  

\item[Secretario] es aquella persona que redacta los informes, organiza la
  información relacionada con el proyecto, planifica las reuniones, \dots

\item[Operario de servicio técnico] realiza labores de instalación y
  mantenimiento de los recursos materiales disponibles.

\item[Usuario experto] realiza pruebas a un alto nivel de especificación.

\item[Miembro del grupo de trabajo] recopilar los requisitos iniciales de la
  aplicación, realizar entrevistas, supervisar el trabajo del personal
  contratado, conseguir los recursos necesarios y la contabilidad.

\end{description}

\subsection{Recursos materiales}
\begin{description}
\item[Sala de juntas] Se utiliza para las reuniones semanales, así como para las juntas extraordinarias donde se tratan las aprobaciones de las distintas etapas del proyecto.
\item[Proyector] Se utiliza en las reuniones de la sala de juntas.
\item[Pizarra interactiva] Se utilizan en las etapas de análisis y diseño, donde cada miembro del equipo de trabajo puede aportar sus ideas y facilita la asimilación de conceptos.
\item[Impresora láser color]
\item[Servidor central] Los datos del proyecto están centralizados para que todos los miembros del equipo de trabajo tengan acceso de manera rápida y eficiente.
\item[Red Hat Enterprise Linux 5] Software para el servidor.
\item[Equipo informático] Se cuenta con 6 unidades de ordenadores para todo el desarrollo del proyecto.  
\end{description}

\section{Definición de costes por uso}

\begin{table}[!h]
\centering
  \begin{tabular}{|c|c|}
    \hline
    \textbf{Recursos} & \textbf{Costes/Uso} \\
    \hline \hline
    Sala de juntas & \\
    \hline
    Proyector & \\
    \hline
    Pizarra interactiva & \\
    \hline
    Impresora láser color & \\
    \hline
    Servidor central & \\
    \hline
    Red Hat Enterprise Linux 5 & \\
    \hline
    Equipo informático & \\
    \hline

  \end{tabular}
  \caption{Costes fijos por actividad}
  \label{Tab:costefijo}
\end{table}

\section{Definición de costes fijos de actividad}

\section{Tablas variables de costo}

\section{Disponibilidad variable de un recurso}

\section{Tablas variables de tasas de costos}

\section{Aplicación de distintas tablas de tasas de costo en tareas}

\chapter{Calendarios}
\section{Calendarios generales de recursos humanos}

\section{Calendarios específicos de recursos humanos}

\chapter{Redistribución del proyecto}
\section{Informe de sobre-asignaciones de recursos y de su 
  resolución}
\subsection{Gantt de redistribución}

\section{Incluir al menos 2 tipos de delimitaciones de tareas y 
  analizar sus efectos en la programación de proyecto}

\chapter{Alternativas al plan evaluando su repercusión en coste y 
  calendario}
* Nota: Establecer una fecha de fin y considerar un determinado coste por día
de retraso.

\chapter{Seguimiento simulado del proyecto incluyendo como 
  mínimo un ejemplo de las siguientes acciones}
\section{Introducción de duraciones reales y restantes}

\section{Introducción de un porcentaje completado}

\section{Introducción del trabajo real}

\section{Actualizar el resto del proyecto según la programación.}

\chapter{Informes}
\section{Vista resumen del plan del proyecto}

\section{Diagrama de Gantt}

\section{Diagrama de Gant con camino crítico}

\section{Informe general de recursos}

\section{Costes}
\subsection{Costes por recursos}
\subsection{Costes por actividades}

\section{Informe de redistribución}

\section{Informe de seguimiento}


\bibliographystyle{plain} 
\bibliography{p1}

\end{document}
